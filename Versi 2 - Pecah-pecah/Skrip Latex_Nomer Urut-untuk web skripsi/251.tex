\documentclass[a4paper,12pt]{article}
\usepackage{arabtex}
\usepackage[bahasa] {babel}
\usepackage[top=2cm,left=3cm,right=3cm,bottom=3cm]{geometry}
\title{\Large Doa Bersin dan Menguap}
\author{\small Hanifah Atiya Budianto\\
		\small contact.us@latex-dailyprayers.com}
\begin{document}
\sffamily
\maketitle
\fullvocalize
\setcode{arabtex}
\begin{arabtext}
\noindent
'inna al-ll_aha yu.hibbu al-`u.tAsa wayakrahu al-tta_tA'uba, fa'i-_dA
`a.tasa 'a.hadukum wa.hamida al-ll_aha kAna .haqqaN `alY kulli muslimiN
sami`ahu 'an yaqu-wla lahu :  yar.hamuka al-ll_ahu, wa'ammA al-tta_tA'ubu
fa'i-nnamA huwa min al-^s^say.tAni,  fa'i-_dA ta_tA'a ba 'a.hadukum
falyaruddahu mA asta.tA`a, fa'i-nna 'a.hadakum 'i_dA ta_tA'aba .da.hika
minhu al-^s^say.tAnu.\\
\end{arabtext}
\noindent
\textbf{Artinya}:
\par
\indent
"Sesungguhnya Allah menyukai bersin dan membenci menguap. Apabila salah
seorang dari kalian bersin dan memuji Allah (mengucapkan
\textit{Alhamdulill\^{a}h}), maka hendaklah setiap Muslim yang mendengarnya
berkata kepada orang yang bersin: '\textit{Yarhamukall\^{a}h} (artinya,
semoga Allah merahmatimu).' Adapun menguap itu datangnya dari syaitan. Maka
apabila salah seorang dari kalian menguap, hendaklah ia berusaha untuk
menahan semampunya. Sebab syaitan akan tertawa tatkala salah seorang dari
kalian menguap."\\\\
\par
\noindent
\textbf{Tingkatan Doa dan Sanad}: HR. Al-Bukhari (no. 6226). Lihat
\textit{Fathul B\^{a}ri} (X/611 no. 6226).\\
\textbf{Referensi}: Yazid bin Abdul Qadir Jawas. 2016. Kumpulan Do'a dari
Al-Quran dan As-Sunnah yang Shahih. Bogor: Pustaka Imam Asy-Syafi'i.
\end{document}