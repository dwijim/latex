\documentclass[a4paper,12pt]{article}
\usepackage{arabtex}
\usepackage[bahasa] {babel}
\usepackage[top=2cm,left=3cm,right=3cm,bottom=3cm]{geometry}
\title{\Large Berlindung dari Fitnah dan Berbagai Keburukan}
\author{\small Hanifah Atiya Budianto\\
		\small contact.us@latex-dailyprayers.com}
\begin{document}
\sffamily
\maketitle
\fullvocalize
\setcode{arabtex}
\begin{arabtext}
\noindent
al-ll_ahumma 'inni-y 'a`u-w_du bika mina al-`a^gzi, wAl-kasali, wAl-^gubni,
wAl-harami, wAl-bu_hli, wa'a`u-w_du bika min `a_dAbi al-qabri, wamin
fitnaTi al-ma.hyA wAl-mamAti.\\
\end{arabtext}
\noindent
\textbf{Artinya}:
\par
\indent
"Ya Allah, sungguh aku berlindung kepada-Mu dari kelemahan, kemalasan,
sifat pengecut, pikun, dan kekikiran. Aku juga berlindung kepada-Mu dari
adzab kubur serta fitnah kehidupan dan kematian."\\\\
\par
\noindent
\textbf{Tingkatan Doa dan Sanad}: \textbf{Shahih}: HR. Al-Bukhari (no.
2823, 6367), Muslim (no. 2706) dari Anas bin Malik r.a.\\
\textbf{Referensi}: Yazid bin Abdul Qadir Jawas. 2016. Kumpulan Do'a dari
Al-Quran dan As-Sunnah yang Shahih. Bogor: Pustaka Imam Asy-Syafi'i.
\end{document}