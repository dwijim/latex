\documentclass[a4paper,12pt]{article}
\usepackage{arabtex}
\usepackage[bahasa] {babel}
\usepackage[top=2cm,left=3cm,right=3cm,bottom=3cm]{geometry}
\usepackage{xcolor, framed}
\definecolor{shadecolor}{rgb}{0.8,0.8,0.8}
\title{\Large Doa Saat Mengalami Kesusahan, Kesedihan, dan Penawar
Kedukaan}
\author{\small Hanifah Atiya Budianto\\
		\small contact.us@latex-dailyprayers.com}
\begin{document}
\sffamily
\maketitle
\fullvocalize
\setcode{arabtex}
\begin{arabtext}
\noindent
lA 'il_aha 'illA al-ll_ahu al-`a.zi-ymu al-.hali-ymu, lA 'il_aha 'illA
al-ll_ahu rabbu al-`ar^si al-`a.zi-ymi, lA 'il_aha 'illA al-ll_ahu rabbu
al-ssamAwAti, warabbu al-'ar.di, warabbu al-`ar^si al-kari-ymi.\\
\end{arabtext}
\noindent
\textbf{Artinya}:
\par
\indent
"Tidak ada ilah yang berhak diibadahi dengan benar melainkan hanya Allah,
Rabb Yang Mahaagung lagi Maha Penyantun. Tidak ada ilah yang berhak
diibadahi dengan benar melainkan Allah, Pemilik Arsy yang agung. Tidak ada
ilah yang berhak diibadahi dengan benar melainkan hanya Allah, Rabb langit
dan Rabb bumi, Pemilik Arsy yang mulia."\\
\par
\noindent
\textbf{Tingkatan Doa dan Sanad}: \textbf{Shahih}: HR. Al-Bukhari (no.
6345, 6346, 7426, 7431), Muslim (no. 2730), at-Tirmidzi (no. 3435), Ibnu
Majah (no. 3883), dan Ahmad (I/228, 259, 268, 280) dari Ibnu Abbas r.a.\\
\textbf{Referensi}: Yazid bin Abdul Qadir Jawas. 2016. Kumpulan Do'a dari
Al-Quran dan As-Sunnah yang Shahih. Bogor: Pustaka Imam Asy-Syafi'i.
\end{document}