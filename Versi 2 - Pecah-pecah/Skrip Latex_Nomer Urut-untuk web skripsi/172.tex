\documentclass[a4paper,12pt]{article}
\usepackage{arabtex}
\usepackage[bahasa] {babel}
\usepackage[top=2cm,left=3cm,right=3cm,bottom=3cm]{geometry}
\title{\Large Doa Berlindung dari Kebinasaan dan Kehancuran}
\author{\small Hanifah Atiya Budianto\\
		\small contact.us@latex-dailyprayers.com}
\begin{document}
\sffamily
\maketitle
\fullvocalize
\setcode{arabtex}
\begin{arabtext}
\noindent
al-ll_ahumma 'inni-y 'a`u-w_du bika mina al-ttaraddi-y, wAl-hadmi,
wAl-.garaqi, wAl-.hari-yqi, wa'a`u-w_du bika 'an yata_habba.taniya
al-^s^sa-y.tAnu `inda al-ma-wti, wa'a`u-w_du bika 'an 'amu-wta fi-y
sabi-ylika mudbiraN, wa'a`u-w_du bika 'an 'amu-wta ladi-y.gaN.\\
\end{arabtext}
\noindent
\textbf{Artinya}:
\par
\indent
"Ya Allah, sesungguhnya aku berlindung kepada-Mu dari kebinasaan (jatuh
dari tempat yang tinggi), kehancuran (tertimpa sesuatu), tenggelam,
kebakaran dan aku berlindung kepada-Mu dari dikuasai syaitan pada saat
menjelang mati, dan aku berlindung kepada-Mu dari mati dalam keadaan
berpaling dari jalan-Mu, dan aku berlindung kepada-Mu dari mati dalam
keadaan tersengat."\\\\
\par
\noindent
\textbf{Tingkatan Doa dan Sanad}: \textbf{Shahih}: HR. An-Nasai (VIII/282),
Abu Dawud (no. 1552) dari Abu Yasar r.a dan \textit{Shah\^{i}h an-Nasai}
(III/1123, no. 5104). Lafazh ini milik an-Nasai.\\
\textbf{Referensi}: Yazid bin Abdul Qadir Jawas. 2016. Kumpulan Do'a dari
Al-Quran dan As-Sunnah yang Shahih. Bogor: Pustaka Imam Asy-Syafi'i.
\end{document}