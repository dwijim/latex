\documentclass[a4paper,12pt]{article}
\usepackage{arabtex}
\usepackage[bahasa] {babel}
\usepackage[top=2cm,left=3cm,right=3cm,bottom=3cm]{geometry}
\title{\Large Doa Mohon Ampunan dan Kasih Sayang}
\author{\small Hanifah Atiya Budianto\\
		\small contact.us@latex-dailyprayers.com}
\begin{document}
\sffamily
\maketitle
\fullvocalize
\setcode{arabtex}
\begin{arabtext}
\noindent
al-ll_ahumma 'inni-y .zalamtu nafsi-y .zulmaN ka_ti-yraN, walA ya.gfiru
al-_d_dunu-wba 'illA 'anta, fA.gfir li-y ma.gfiraTaN min `indika,
wAr.hamni-y, 'innaka 'anta al-.gafu-wru al-rra.hi-ymu.\\
\end{arabtext}
\noindent
\textbf{Artinya}:
\par
"Ya Allah, sesungguhnya aku telah menzhalimi diriku dengan kezhaliman yang
banyak, dan tidak ada yang dapat mengampuni dosa melainkan Engkau. Oleh
karena itu ampunilah aku dengan ampunan yang datang dari sisi-Mu, dan
rahmatilah aku, sesungguhnya Engkau adalah Yang Maha Pengampun lagi Maha
Penyayang."\\\\
\par
\noindent
\textbf{Tingkatan Doa dan Sanad}: \textbf{Shahih}: HR. Al-Bukhari (no.
834), Bab "ad-Du'\^{a}' qabla Sal\^{a}m" dan Muslim (no. 2705 [48]) dari
Abu Bakar ash-Shiddiq r.a. Doa ini dibaca setelah tasyahud akhir sebelum
salam.\\
\textbf{Referensi}: Yazid bin Abdul Qadir Jawas. 2016. Kumpulan Do'a dari
Al-Quran dan As-Sunnah yang Shahih. Bogor: Pustaka Imam Asy-Syafi'i.
\end{document}