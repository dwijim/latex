\documentclass[a4paper,12pt]{article}
\usepackage{arabtex}
\usepackage[bahasa] {babel}
\usepackage[top=2cm,left=3cm,right=3cm,bottom=3cm]{geometry}
\title{\Large Doa Istiftah}
\author{\small Hanifah Atiya Budianto\\
		\small contact.us@latex-dailyprayers.com}
\begin{document}
\sffamily
\maketitle
\fullvocalize
\setcode{arabtex}
- Setelah membaca doa istiftah, membaca ta'awwudz:\\
\begin{arabtext}
\noindent
'a`u-w_du bi-al-ll_ahi al-ssami-y`i al-`ali-ymi mina al-^s^sa-y.tAni
al-rra^gi-ymi min hamzihi wanaf_hihi wanaf_tihi.\\
\end{arabtext}
\noindent
\textbf{Artinya}:
\par
\indent
"Aku berlindung kepada Allah Yang Maha Mendengar lagi Maha Mengetahui dari
gangguan syaitan yang terkutuk, dari kegilaannya, kesombongannya, dan
syairnya yang tercela."{\scriptsize 1}\\
- Membaca surah Al-Fatihah.\\
- Mengucapkan "Aamiin" setelah \begin{arabtext} (walA al-.d.da-^Ali-yna)
\end{arabtext}
- Dalam shalat berjamaah, makmum tidak boleh mendahului imam.\\
- Membaca surah sesuai dengan apa yang dicontohkan oleh Rasulullah
shallallahu 'alaihi wa sallam.{\scriptsize 2}\\\\
\par
\noindent
\textbf{Tingkatan Doa dan Sanad}:
\begin{enumerate}
\item \textbf{Shahih}: HR. Abu Dawud (no. 775) dan at-Tirmidzi (no. 242).
Dengan dasar firman Allah dalam surah Fushshilat ayat 36, lihat
\textit{al-Kalimuth Tahyib} (no. 130), shahih. \textit{Shifatu Shal\^{a}tin
Nabi} (hlm. 95-96) dan \textit{Irw\^{a}-ul Ghal\^{i}l} (II/53-57, no. 342).
\item Lihat kitab \textit{Shifatu Shal\^{a}tin Nabi} karya Syaikh Muhammad
Nashiruddin al-Albani.
\end{enumerate}
\textbf{Referensi}: Yazid bin Abdul Qadir Jawas. 2016. Kumpulan Do'a dari
Al-Quran dan As-Sunnah yang Shahih. Bogor: Pustaka Imam Asy-Syafi'i.
\end{document}