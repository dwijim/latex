\documentclass[a4paper,12pt]{article}
\usepackage{arabtex}
\usepackage[bahasa] {babel}
\usepackage[top=2cm,left=3cm,right=3cm,bottom=3cm]{geometry}
\title{\Large Doa Minta Hujan}
\author{\small Hanifah Atiya Budianto\\
		\small contact.us@latex-dailyprayers.com}
\begin{document}
\sffamily
\maketitle
\fullvocalize
\setcode{arabtex}
\begin{arabtext}
\noindent
al-ll_ahumma asqinA .ga-y_taN mu.gi-y_taN mari-y'aN mari-y`aN, nA fi`aN
.ga-yra .dArriN, `A^gilaN .ga-yra ^A^giliN.\\
\end{arabtext}
\noindent
\textbf{Artinya}:
\par
\indent
"Ya Allah, berilah kami hujan yang merata, yang menyegarkan tubuh dan
menyuburkan tanaman, bermanfaat, tidak berbahaya. Kami mohon hujan dengan
segera, tidak ditunda-tunda."\\\\
\par
\noindent
\textbf{Tingkatan Doa dan Sanad}: \textbf{Shahih}: HR. Abu Dawud (no.
1169), dinyatakan shahih oleh al-Albani dalam \textit{Shah\^{i}h Abi Dawud}
(I/216). Dalam riwayat lain bahwa Nabi mengangkat kedua tangannya ketika
minta hujan, namun tidak melewati kepalanya (riwayat Abu Dawud, no. 1168)
sehingga terlihat kedua ketiak dan telapak tangannya ke arah bumi (Abu
Dawud, no. 1171). Lihat \textit{Shah\^{i}h al-Bukhari} (no. 1030, 1031) dan
\textit{Shahih Muslim} (no. 895, 896).\\
\textbf{Referensi}: Yazid bin Abdul Qadir Jawas. 2016. Kumpulan Do'a dari
Al-Quran dan As-Sunnah yang Shahih. Bogor: Pustaka Imam Asy-Syafi'i.
\end{document}