\documentclass[a4paper,12pt]{article}
\usepackage{arabtex}
\usepackage[bahasa] {babel}
\usepackage[top=2cm,left=3cm,right=3cm,bottom=3cm]{geometry}
\title{\Large Doa Supaya Dijadikan Hamba yang Bersyukur}
\author{\small Hanifah Atiya Budianto\\
		\small contact.us@latex-dailyprayers.com}
\begin{document}
\sffamily
\maketitle
\fullvocalize
\setcode{arabtex}
\begin{arabtext}
\noindent
rabbi 'awzi`n^I 'an 'a^skura ni`mataka allat^I 'an`amta `alaYYa
wa`alaY_a w_alidaYYa wa'an 'a`mala .s_ali.haN tar.d_ahu wa'a.sli.h liY fiY
_durriyyat^I 'inniY tubtu 'ilayka wa-'inniY mina al-muslimiyna.\\
\end{arabtext}
\noindent
\textbf{Artinya}:\\
\indent
"Ya Rabbku, berilah aku petunjuk agar aku dapat mensyukuri nikmat-Mu yang
telah Engkau limpahkan kepadaku dan kepada kedua orang tuaku dan agar aku
dapat berbuat kebajikan yang Engkau ridai; dan berilah aku kebaikan yang
akan mengalir sampai kepada anak cucuku. Sesungguhnya aku bertaubat kepada
Engkau dan sungguh, aku termasuk orang muslim." (QS. Al-Ahq\^{a}f [46]: 15).
\\\\
\noindent
\textbf{Referensi}: Yazid bin Abdul Qadir Jawas. 2016. Kumpulan Do'a dari
Al-Quran dan As-Sunnah yang Shahih. Bogor: Pustaka Imam Asy-Syafi'i.
\end{document}