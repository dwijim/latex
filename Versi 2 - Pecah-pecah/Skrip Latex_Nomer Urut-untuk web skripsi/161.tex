\documentclass[a4paper,12pt]{article}
\usepackage{arabtex}
\usepackage[bahasa] {babel}
\usepackage[top=2cm,left=3cm,right=3cm,bottom=3cm]{geometry}
\usepackage{xcolor, framed}
\definecolor{shadecolor}{rgb}{0.8,0.8,0.8}
\title{\Large Doa pada Shalat Jenazah}
\author{\small Hanifah Atiya Budianto\\
		\small contact.us@latex-dailyprayers.com}
\begin{document}
\sffamily
\maketitle
\fullvocalize
\setcode{arabtex}
\begin{arabtext}
\noindent
al-ll_ahumma `abduka wAbnu 'amatika 'i.htA^ga 'ilY ra.hmatika, wa-'anta
.ganiyyuN `an `a_dAbihi, 'in kAna mu.hsinaN fazid fi-y .hasanAtihi, wa-'in
kAna musi-y'aN fata^gAwaz `anhu.\\
\end{arabtext}
\noindent
\textbf{Artinya}:
\par
\indent
"Ya Allah, ini (adalah) hamba-Mu, anak hamba perempuan-Mu (Hawa),
membutuhkan rahmat-Mu, sedang Engkau tidak membutuhkan untuk menyiksanya.
Jika ia berbuat baik, tambahkanlah dalam amalan baiknya, dan jika dia orang
yang bersalah, maafkanlah kesalahannya [kemudian beliau berdoa dengan apa
yang Allah kehendaki]."\\\\
\par
\noindent
\textbf{Tingkatan Doa dan Sanad}: \textbf{Shahih}: HR. Ath-Thabrani dalam
\textit{al-Mu'jamul Kab\^{i}r} (XXII/249) tambahan dalam kurung miliknya,
dan Al-Hakim (I/359).Sanadnya shahih. Imam adz-Dzahabi menyetujuinya. Lihat
\textit{Ahk\^{a}mul Jan\^{a}-iz} (hlm. 159) karya Syaikh al-Albani.\\
\textbf{Referensi}: Yazid bin Abdul Qadir Jawas. 2016. Kumpulan Do'a dari
Al-Quran dan As-Sunnah yang Shahih. Bogor: Pustaka Imam Asy-Syafi'i.
\end{document}