\documentclass[a4paper,12pt]{article}
\usepackage{arabtex}
\usepackage[bahasa] {babel}
\usepackage[top=2cm,left=3cm,right=3cm,bottom=3cm]{geometry}
\title{\Large Membaca Shalawat{\scriptsize 1} Nabi setelah Tasyahud}
\author{\small Hanifah Atiya Budianto\\
		\small contact.us@latex-dailyprayers.com}
\begin{document}
\sffamily
\maketitle
\fullvocalize
\setcode{arabtex}
\begin{arabtext}
\noindent
al-ll_ahumma .salli `alY mu.hammadiN wa`alY ^Ali mu.hammadiN, kamA
.sallayta `alY 'ibrAhi-yma wa`alY ^Ali 'ibrAhi-yma, 'innaka .hami-yduN
ma^gi-yduN, al-ll_ahumma bArik `alY mu.hammadiN wa`alY ^Ali mu.hammadiN,
kamA bArakta `alY 'ibrAhi-yma wa`alY ^Ali 'ibrAhi-yma, 'innaka .hami-yduN
ma^gi-yduN.\\
\end{arabtext}
\noindent
\textbf{Artinya} :
\par
\indent
"Ya Allah, berikanlah shalawat kepada Nabi Muhammad beserta keluarga
Muhammad, sebagaimana Engkau telah memberikan shalawat kepada Ibrahim dan
keluarga Ibrahim. Sesungguhnya Engkau Maha Terpuji lagi Mahamulia.
Berikanlah berkah kepada Muhammad dan keluarga Muhammad sebagaimana Engkau
telah memberi berkah kepada Ibrahim beserta keluarga Ibrahim. Sesungguhnya
Engkau Maha Terpuji lagi Mahamulia."{\scriptsize 2}\\\\
\par
\noindent
\textbf{Tingkatan Doa dan Sanad}:
\begin{enumerate}
\item Tidak ada tambahan lafazh "sayyidinaa" dalam shalawat dan tidak ada
satu pun riwayat yang shahih dari Nabi Shallallahu ‘alaihi wa sallam, dan
lafazh ini pun tidak diucapkan oleh para Sahabat radhiyallahu 'anhum.
\item \textbf{Shahih}: HR. Al-Bukhari (no. 3370)/\textit{Fathul B\^{a}ri}
(VI/408), Muslim (no. 406), Abu Dawud (no. 976, 977, 978), at-Tirmidzi (no.
483), an-Nasai (III/47-48), Ahmad (IV/243-244), Ibnu Majah (no. 904), dan
selainnya dari Ka'ab bin Ujrah r.a.
\end{enumerate}
\textbf{Referensi}: Yazid bin Abdul Qadir Jawas. 2016. Kumpulan Do'a dari
Al-Quran dan As-Sunnah yang Shahih. Bogor: Pustaka Imam Asy-Syafi'i.
\end{document}