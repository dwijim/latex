\documentclass[a4paper,12pt]{article}
\usepackage{arabtex}
\usepackage[bahasa] {babel}
\usepackage[top=2cm,left=3cm,right=3cm,bottom=3cm]{geometry}
\title{\Large Doa Istiftah}
\author{\small Hanifah Atiya Budianto\\
		\small contact.us@latex-dailyprayers.com}
\begin{document}
\sffamily
\maketitle
\fullvocalize
\setcode{arabtex}
\begin{arabtext}
\noindent
sub.hAnaka al-ll_ahumma wabi.hamdika, watabAraka asmuka, wata`Al_aY
^gadduka, walA 'il_aha .gayruka.\\
\end{arabtext}
\noindent
\textbf{Artinya}:
\par
\indent
"Mahasuci Engkau ya Allah, aku memuji-Mu, Mahaberkah Nama-Mu. Mahatinggi
kekayaan dan kebesaran-Mu, tidak ada ilah yang berhak diibadahi dengan
benar selain Engkau."\\\\
\par
\noindent
\textbf{Tingkatan Doa dan Sanad}: \textbf{Shahih}: HR. Muslim (no. 399 [52])
dan ad-Daraquthni (I/628-629, no. 1127, 1132) dari Umar bin al-Khathab r.a.
secara \textit{mauquf}. Diriwayatkan juga oleh ad-Daraquthni (no. 1133) dan
ath-Thabrani dalam \textit{ad-Du'\^{a}'} (no. 506), dari Anas bin Malik
r.a. secara marfu'. Sanadnya shahih. Lihat kitab \textit{Silsilah
ash-Shah\^{i}hah} (no. 2996), \textit{Ashlu Shifatu Shal\^{a}tin Nabi}
(I/254), serta kitab \textit{Irw\^{a}-ul Ghal\^{i}l} (II/51-53).\\
\textbf{Referensi}: Yazid bin Abdul Qadir Jawas. 2016. Kumpulan Do'a dari
Al-Quran dan As-Sunnah yang Shahih. Bogor: Pustaka Imam Asy-Syafi'i.
\end{document}