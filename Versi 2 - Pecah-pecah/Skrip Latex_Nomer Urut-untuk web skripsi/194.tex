\documentclass[a4paper,12pt]{article}
\usepackage{arabtex}
\usepackage[bahasa] {babel}
\usepackage[top=2cm,left=3cm,right=3cm,bottom=3cm]{geometry}
\title{\Large Doa Memohon Surga dan Berlindung dari Neraka}
\author{\small Hanifah Atiya Budianto\\
		\small contact.us@latex-dailyprayers.com}
\begin{document}
\sffamily
\maketitle
\fullvocalize
\setcode{arabtex}
\begin{arabtext}
\noindent
al-ll_ahumma rabba ^gibrA'iyla, wami-ykA'iyla, warabba 'isrAfi-yla,
'a`u-w_du bika min .harri al-nnAri wamin `a_dAbi al-qabri.\\
\end{arabtext}
\noindent
\textbf{Artinya}:
\par
\indent
"Ya Allah, Rabb Malaikat Jibril, Mika-il dan Rabb Malaikat Israfril, aku
berlindung kepada-Mu dari panasnya api Neraka dan dari adzab kubur."\\\\
\par
\noindent
\textbf{Tingkatan Doa dan Sanad}: \textbf{Hasan}: HR. An-Nasai (VIII/278)
dari Aisyah r.a. Lihat \textit{Silsilah Ah\^{a}d\^{i}ts ash-Shah\^{i}hah}
(no. 1544).\\
\textbf{Referensi}: Yazid bin Abdul Qadir Jawas. 2016. Kumpulan Do'a dari
Al-Quran dan As-Sunnah yang Shahih. Bogor: Pustaka Imam Asy-Syafi'i.
\end{document}