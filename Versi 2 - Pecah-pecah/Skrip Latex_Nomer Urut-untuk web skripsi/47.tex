\documentclass[a4paper,12pt]{article}
\usepackage{arabtex}
\usepackage[bahasa] {babel}
\usepackage[top=2cm,left=3cm,right=3cm,bottom=3cm]{geometry}
\title{\Large Doa dan Dzikir sebelum Tidur}
\author{\small Hanifah Atiya Budianto\\
		\small contact.us@latex-dailyprayers.com}
\begin{document}
\sffamily
\maketitle
\fullvocalize
\setcode{arabtex}
\begin{arabtext}
\noindent
bi-asmika rabbi-y wa.da`tu ^ganbi-y, wabika 'arfa`uhu, 'in 'amsakta nafsi-y
fAr.hamhA, wa-'in 'arsaltahA fA.hfa.zhA bimA ta.hfa.zu bihi `ibAdaka
al-.s.sAli.hi-yna.\\
\end{arabtext}
\noindent
\textbf{Artinya}:\\
\indent
"Dengan nama-Mu, wahai Rabbku, aku meletakkan lambungku (tidur). Dengan
nama-Mu pula aku bangun. Apabila Engkau mencabut nyawaku, maka berikanlah
rahmat-Mu padanya. Apabila Engkau membiarkannya hidup maka peliharalah,
sebagaimana Engkau selalu memelihara hamba-hamba-Mu yang shalih."\\\\
Baca juga doa dan dzikir sebelum tidur ke 1-5\\
\par
\noindent
\textbf{Tingkatan Doa dan Sanad}: Rasulullah \textit{Shallallahu ‘alaihi wa
sallam} bersabda: "Apabila seseorang di antara kalian bangkit dari tempat
tidurnya kemudian ingin tidur lagi, hendaknya dia mengibaskan ujung kainnya
3x, dan menyebut nama Allah, karena dia tidak tahu apa yang ditinggalkannya
di atas tempat tidur setelah bangkit. Apabila dia ingin berbaring, maka
hendaklah membaca: '\textit{Bismika Rabb}i ...'"  \textbf{Shahih}: HR.
Al-Bukhari (no. 6320), Muslim (no. 2714), at-Tirmidzi (no. 3401), dan
an-Nasai dalam kitab \textit{'Amalul Yaum wal Lailah} (no. 796).\\
\textbf{Referensi}: Yazid bin Abdul Qadir Jawas. 2016. Kumpulan Do'a dari
Al-Quran dan As-Sunnah yang Shahih. Bogor: Pustaka Imam Asy-Syafi'i.
\end{document}