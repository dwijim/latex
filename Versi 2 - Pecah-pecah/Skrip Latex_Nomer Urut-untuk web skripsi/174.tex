\documentclass[a4paper,12pt]{article}
\usepackage{arabtex}
\usepackage[bahasa] {babel}
\usepackage[top=2cm,left=3cm,right=3cm,bottom=3cm]{geometry}
\title{\Large Berlindung dari Fitnah dan Berbagai Keburukan}
\author{\small Hanifah Atiya Budianto\\
		\small contact.us@latex-dailyprayers.com}
\begin{document}
\sffamily
\maketitle
\fullvocalize
\setcode{arabtex}
\begin{arabtext}
\noindent
al-ll_ahumma 'inni-y 'a`u-w_du bika mina al-`a^gzi, wAl-kasali, wAl-^gubni,
wAl-bu_hli, wAl-harami, wAl-qaswaTi, wAl-.gaflaTi, wAl-`aylaTi,
wAl-_d_dillaTi, wAl-maskanaTi, wa'a`u-w_du bika mina al-faqri, wAl-kufri,
wAl-fusu-wqi, wAl-^s^siqAqi, wAl-nnifAqi, wAl-ssum`aTi, wAl-rriyA'i,
wa'a`u-w_du bika mina al-.s.samami, wAl-bakami, wAl-^gunu-wni,
wAl-^gu_dAmi, wAl-bara.si, wasayyi -'i al-'asqAmi.\\
\end{arabtext}
\noindent
\textbf{Artinya}:
\par
\indent
"Ya Allah, aku berlindung kepada-Mu dari kelemahan, kemalasan, sifat yang
pengecut, kekikiran, pikun, kekerasan hati, lalai, berat tanggungan,
kehinaan, dan kerendahan. Dan aku berlindung kepada-Mu dari kemiskinan,
kekufuran, kefasikan, perpecahan, kemunafikan, \textit{sum'ah} (amalnya
ingin didengar orang), \textit{riya'} (amalnya ingin dilihat orang) serta
aku berlindung kepada-Mu dari tuli, bisu, gila, sakit lepra, belang, dan
dari keburukan berbagai jenis penyakit."\\\\
\par
\noindent
\textbf{Tingkatan Doa dan Sanad}: \textbf{Shahih}: HR. Al-Hakim (I/530) dan
Ibnu Hibban (no. 2446 - \textit{Maw\^{a}riduzh Zh\^{a}m-\^{a}n} dan no.
1019-\textit{at-Ta'liqatul His\^{a}n}) dari Anas bin Malik r.a. Lihat
\textit{Shah\^{i}hul J\^{a}mi'} (no. 1285) dan \textit{Irw\^{a}-ul
Ghal\^{i}l} (III/357). Dishahihkan al-Hakim dan disetujui adz-Dzahabi.\\
\textbf{Referensi}: Yazid bin Abdul Qadir Jawas. 2016. Kumpulan Do'a dari
Al-Quran dan As-Sunnah yang Shahih. Bogor: Pustaka Imam Asy-Syafi'i.
\end{document}