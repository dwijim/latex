\documentclass[a4paper,12pt]{article}
\usepackage{arabtex}
\usepackage[bahasa] {babel}
\usepackage[top=2cm,left=3cm,right=3cm,bottom=3cm]{geometry}
\title{\Large Doa ketika Mendengar Adzan}
\author{\small Hanifah Atiya Budianto\\
		\small contact.us@latex-dailyprayers.com}
\begin{document}
\sffamily
\maketitle
\fullvocalize
\setcode{arabtex}
\indent
Terdapat lima hal yang disunnahkan ketika adzan dikumandangkan:\\\\
5. Berdoa untuk diri sendiri dengan doa yang dikehendaki antara adzan
dan iqamah, sebab doa pada saat itu dikabulkan oleh Allah.
\begin{arabtext}
\noindent
al-ddu`A'u lA yuraddu bayna al-'a _dAni wAl-'iqAmaTi.\\
\end{arabtext}
\noindent
\textbf{Artinya}:
\par
\indent
"Tidak ditolak doa antara adzan dan iqamah."\\\\
Baca juga doa ketika mendengar adzan ke 1-5\\
\par
\noindent
\textbf{Tingkatan Doa dan Sanad}: \textbf{Shahih}: HR. Abu Dawud (no. 521),
at-Tirmidzi (no. 212, 3595), Ahmad (III/119, 155, 225), an-Nasai dalam
\textit{'Amalul Yaum wal Lailah} (no. 67, 68, 69), Ibnu Khuzaimah (no. 425,
426, 427). Lihat penjelasan Ibnu Qayyim tentang lima hal ini dalam
\textit{Shah\^{i}h al-Wabilish Shayyib} (hlm. 182-185), \textit{Z\^{a}dul
Ma'\^{a}d} (II/391-392).\\
\textbf{Referensi}: Yazid bin Abdul Qadir Jawas. 2016. Kumpulan Do'a dari
Al-Quran dan As-Sunnah yang Shahih. Bogor: Pustaka Imam Asy-Syafi'i.
\end{document}