\documentclass[a4paper,12pt]{article}
\usepackage{arabtex}
\usepackage[bahasa] {babel}
\usepackage[top=2cm,left=3cm,right=3cm,bottom=3cm]{geometry}
\title{\Large Doa Bersin dan Menguap}
\author{\small Hanifah Atiya Budianto\\
		\small contact.us@latex-dailyprayers.com}
\begin{document}
\sffamily
\maketitle
\fullvocalize
\setcode{arabtex}
\begin{arabtext}
\noindent
'i_dA `a.tasa 'a.hadukum falyaqul: al-.hamduli-ll_ahi, walyaqul lahu
'a_hu-whu 'aw.sA.hibuhu: yar.hamuka al-ll_ahu, fa'i-_dA qala lahu:
yar.hamuka al-ll_ahu falyaqul: yahdi-ykumu al-ll_ahu wayu.sli.hu bAlakum.\\
\end{arabtext}
\noindent
\textbf{Artinya}:
\par
\indent
"Apabila salah seorang di antara kalian bersin, hendaklah ia berkata:
\textit{Alhamdulill\^{a}h} 'Segala puji bagi Allah,' lantas saudara atau
temannya berkata: \textit{yarkhamukall\^{a}h} 'Semoga Allah memberikan
rahmat kepada-Mu.' Apabila teman atau saudaranya berkata demikian, bacalah:
\textit{yahdiykumull\^{a}h wayushlikhu balakum} 'Semoga Allah memberi
petunjuk kepadamu dan memperbaiki keadaanmu."\\\\
\par
\noindent
\textbf{Tingkatan Doa dan Sanad}: \textbf{Shahih}: HR. Al-Bukhari (no. 6224)
dari Sahabat Abu Hurairah r.a.\\
\textbf{Referensi}: Yazid bin Abdul Qadir Jawas. 2016. Kumpulan Do'a dari
Al-Quran dan As-Sunnah yang Shahih. Bogor: Pustaka Imam Asy-Syafi'i.
\end{document}