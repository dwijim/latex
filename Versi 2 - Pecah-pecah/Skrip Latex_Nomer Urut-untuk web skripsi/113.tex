\documentclass[a4paper,12pt]{article}
\usepackage{arabtex}
\usepackage[bahasa] {babel}
\usepackage[top=2cm,left=3cm,right=3cm,bottom=3cm]{geometry}
\title{\Large Bacaan Setelah Salam}
\author{\small Hanifah Atiya Budianto\\
		\small contact.us@latex-dailyprayers.com}
\begin{document}
\sffamily
\maketitle
\fullvocalize
\setcode{arabtex}
\begin{arabtext}
\noindent
sub.hAna al-ll_ahi\\
al-.hamdu li-ll_ahi\\
Aal-ll_ahu 'akbaru\\
\end{arabtext}
\noindent
\textbf{Artinya}:
\par
\noindent
"Mahasuci Allah" \textbf{[33x]}\\
"Segala puji bagi Allah" \textbf{[33x]}\\
"Allah Mahabesar" \textbf{[33x]}\\\\
Baca juga bacaan setelah salam ke 1-8\\
\par
\noindent
\textbf{Tingkatan Doa dan Sanad}: "Siapa yang membaca dzikir ini tiap
selesai shalat akan diampuni kesalahannya, meski seperti buih di lautan."
\textbf{Shahih}: HR. Muslim (no. 597), Ahmad (II/371, 483), Ibnu Khuzaimah
(no. 750), al-Baihaqi (II/187).\\
\textbf{Referensi}: Yazid bin Abdul Qadir Jawas. 2016. Kumpulan Do'a dari
Al-Quran dan As-Sunnah yang Shahih. Bogor: Pustaka Imam Asy-Syafi'i.
\end{document}