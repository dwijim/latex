\documentclass[a4paper,12pt]{article}
\usepackage{arabtex}
\usepackage[bahasa] {babel}
\usepackage[top=2cm,left=3cm,right=3cm,bottom=3cm]{geometry}
\title{\Large Bacaan bagi Orang yang Ragu dalam Beriman}
\author{\small Hanifah Atiya Budianto\\
		\small contact.us@latex-dailyprayers.com}
\begin{document}
\sffamily
\maketitle
\fullvocalize
\setcode{arabtex}
\noindent
\begin{enumerate}
\item Bagi siapa saja yang ragu-ragu dalam beriman, maka hendaklah ia
memohon perlindungan kepada Allah.{\scriptsize 1}
\item Berhenti dari keraguan.{\scriptsize 2}\\
\indent Hendaklah mengucapkan:\\
\begin{arabtext}
\noindent
^Amantu bi-al-ll_ahi warusulihi.\\
\end{arabtext}
\noindent
\textbf{Artinya}:
\par
\indent
"Aku beriman kepada Allah dan kepada (kebenaran) para Rasul (utusan)-Nya."
{\scriptsize 3}\\\\
Baca bacaan bagi orang yang ragu dalam beriman ke 2 juga.\\\\
\end{enumerate}
\par
\noindent
\textbf{Tingkatan Doa dan Sanad}:
\begin{enumerate}
\item \textbf{Shahih}: HR. Al-Bukhari/\textit{Fathul B\^{a}ri} (VI/336) dan
Muslim (I/120).
\item \textbf{Shahih}: HR. Al-Bukhari/\textit{Fathul B\^{a}ri} (VI/336) dan
Muslim (I/120), pada Bab "Bay\^{a}nil Waswasah fil \^{I}m\^{a}n wa m\^{a}
Yaq\^{u}luhu man Wajadaha".
\item \textbf{Shahih}: HR. Muslim (no. 134).
\end{enumerate}
\textbf{Referensi}: Yazid bin Abdul Qadir Jawas. 2016. Kumpulan Do'a dari
Al-Quran dan As-Sunnah yang Shahih. Bogor: Pustaka Imam Asy-Syafi'i.
\end{document}