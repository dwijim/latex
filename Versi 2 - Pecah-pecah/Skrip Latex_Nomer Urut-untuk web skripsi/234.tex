\documentclass[a4paper,12pt]{article}
\usepackage{arabtex}
\usepackage[bahasa] {babel}
\usepackage[top=2cm,left=3cm,right=3cm,bottom=3cm]{geometry}
\title{\Large Doa Berlindung Dari Hutang dan Agar Dapat Melunasinya}
\author{\small Hanifah Atiya Budianto\\
		\small contact.us@latex-dailyprayers.com}
\begin{document}
\sffamily
\maketitle
\fullvocalize
\setcode{arabtex}
\begin{arabtext}
\noindent
al-ll_ahumma 'inni-y 'a`u-w_du bika mina al-hammi wAl-.hazani, wAl-`a^gzi
wAl-kasali, wAl-bu_hli, wAl-^gubni, wa.dala`i al-dda-yni, wa.galabaTi
al-rri^gAli.\\
\end{arabtext}
\noindent
\textbf{Artinya}:
\par
\indent
"Ya Allah, sungguh aku  berlindung kepada-Mu dari kesusahan, kesedihan,
kelemahan, kemalasan, sifat kikir, sifat pengecut, lilitan utang, dan
dikuasai orang lain."\\\\
\par
\noindent
\textbf{Tingkatan Doa dan Sanad}: \textbf{Shahih}: HR. Al-Bukhari (no.
6363). Rasulullah SAW. sering memanjatkan doa ini. Lihat \textit{Fathul
B\^{a}ri} (XI/173).\\
\textbf{Referensi}: Yazid bin Abdul Qadir Jawas. 2016. Kumpulan Do'a dari
Al-Quran dan As-Sunnah yang Shahih. Bogor: Pustaka Imam Asy-Syafi'i.
\end{document}