\documentclass[a4paper,12pt]{article}
\usepackage{arabtex}
\usepackage[bahasa] {babel}
\usepackage[top=2cm,left=3cm,right=3cm,bottom=3cm]{geometry}
\title{\Large Doa di Akhir Shalat Witir}
\author{\small Hanifah Atiya Budianto\\
		\small contact.us@latex-dailyprayers.com}
\begin{document}
\sffamily
\maketitle
\fullvocalize
\setcode{arabtex}
\begin{arabtext}
\noindent
sub.hAna al-maliki al-qudduwsi, sub.hAna al-maliki al-qudduwsi, sub.hAna
al-maliki al-qudduwsi.\\
\end{arabtext}
\noindent
\textbf{Artinya}:
\par
\indent
"Mahasuci Allah Raja Yang Mahasuci, Mahasuci Allah Raja Yang Mahasuci,
Mahasuci Allah Raja Yang Mahasuci." \textbf{[Nabi mengangkat dan
memanjangkan suaranya pada ucapan yang ketiga]}.\\\\
Baca juga doa di akhir shalat witir ke 1-2\\
\par
\noindent
\textbf{Tingkatan Doa dan Sanad}: \textbf{Shahih}: Abu Dawud (no. 1430),
an-Nasai (III/245), dan Ahmad (V/123), Ibnu Hibban (no. 2441 -
\textit{at-Ta'liqatul His\^{a}n}), Ibnus Sunni (no. 706), serta al-Baghawi
dalam \textit{Syarhus Sunnah} (IV/98, no. 972). Lihat juga
\textit{Shah\^{i}h Kit\^{a}b al-Adzk\^{a}r} (I/255) dan
\textit{Z\^{a}dul Ma'\^{a}d} (I/337).\\
\textbf{Referensi}: Yazid bin Abdul Qadir Jawas. 2016. Kumpulan Do'a dari
Al-Quran dan As-Sunnah yang Shahih. Bogor: Pustaka Imam Asy-Syafi'i.
\end{document}