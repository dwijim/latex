\documentclass[a4paper,12pt]{article}
\usepackage{arabtex}
\usepackage[bahasa] {babel}
\usepackage[top=2cm,left=3cm,right=3cm,bottom=3cm]{geometry}
\title{\Large Bacaan ketika Berada di Atas Bukit Shafa dan Marwah}
\author{\small Hanifah Atiya Budianto\\
		\small contact.us@latex-dailyprayers.com}
\begin{document}
\sffamily
\maketitle
\fullvocalize
\setcode{arabtex}
\par
\indent
Dari Jabir r.a., ia berkata: "Ketika Nabi berada dekat dengan bukit Shafa,
beliau membaca:\\
\begin{arabtext}
\noindent
( 'inna al-.s.safA wa-ulmarwaTa min ^sa`a-^A'iri al-llahi ) ( 'abda'u bimA
bada'a al-ll_ahu bihi ).\\
\end{arabtext}
\noindent
\textbf{Artinya}:
\par
\indent
\textit{'Sesungguhnya Shafa dan Marwah adalah termasuk syi'ar agama Allah.}
Aku memulai sa'i dengan apa yang didahulukan Allah.' [\textbf{Dibaca 1x
ketika naik bukit Shafa}]\\\\
\par
\noindent
\textbf{Tingkatan Doa dan Sanad}: \textbf{Shahih}: HR. Muslim (no. 1218
[147]) dari Jabir bin Abdillah r.a., Bab "Hajjatun Nabi".\\\\
\textbf{Perhatian}: Mohon baca menu Bantuan, terdapat perbedaan penulisan 
doa pada Latex dan buku.\\
\textbf{Referensi}: Yazid bin Abdul Qadir Jawas. 2016. Kumpulan Do'a dari
Al-Quran dan As-Sunnah yang Shahih. Bogor: Pustaka Imam Asy-Syafi'i.
\end{document}