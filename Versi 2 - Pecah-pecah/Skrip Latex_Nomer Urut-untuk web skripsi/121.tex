\documentclass[a4paper,12pt]{article}
\usepackage{arabtex}
\usepackage[bahasa] {babel}
\usepackage[top=2cm,left=3cm,right=3cm,bottom=3cm]{geometry}
\title{\Large Doa Pengantin Pria kepada Istri}
\author{\small Hanifah Atiya Budianto\\
		\small contact.us@latex-dailyprayers.com}
\begin{document}
\sffamily
\maketitle
\fullvocalize
\setcode{arabtex}
\par
\indent
"Apabila seseorang di antara kalian menikah dengan wanita atau membeli
hamba sahaya, maka peganglah ubun-ubunnya, lalu bacalah Bismillah serta
doakanlah dengan ucapan doa berikut:\\
\begin{arabtext}
\noindent
al-ll_ahumma 'inni-y 'as'aluka _hayrahaa, wa_hayra maa ^gabaltahaa `alayhi,
wa'a`u-wbika min ^sarrihaa, wa^sarri maa ^gabaltahaa `alayhi.\\
\end{arabtext}
\noindent
\textbf{Artinya}:
\par
\indent
'Ya Allah, sesungguhnya aku mohon kepada-Mu kebaikannya dan kebaikan
tabi'atnya (wataknya). Dan aku mohon perlindungan kepada-Mu dari
keburukannya dan keburukan tabiatnya.'\\
\par
\indent
Apabila seseorang membeli unta, hendaklah dipegang puncak punuknya,
kemudian berkata seperti itu."\\\\
\par
\noindent
\textbf{Tingkatan Doa dan Sanad}: \textbf{Shahih}: HR. Abu Dawud (no.
2160), Ibnu Majah (no. 1918), al-Hakim (II/185) dan al-Baihaqi (VIII/148).
Lihat \textit{Shah\^{i}h Ibni Majah} (I/324) dan \textit{\^{A}d\^{a}buz
Zif\^{a}f fis Sunnah al-Muthahharah} (hlm. 92-93).\\
\textbf{Referensi}: Yazid bin Abdul Qadir Jawas. 2016. Kumpulan Do'a dari
Al-Quran dan As-Sunnah yang Shahih. Bogor: Pustaka Imam Asy-Syafi'i.
\end{document}