\documentclass[a4paper,12pt]{article}
\usepackage{arabtex}
\usepackage[bahasa] {babel}
\usepackage[top=2cm,left=3cm,right=3cm,bottom=3cm]{geometry}
\title{\Large Doa Sujud Tilawah}
\author{\small Hanifah Atiya Budianto\\
		\small contact.us@latex-dailyprayers.com}
\begin{document}
\sffamily
\maketitle
\fullvocalize
\setcode{arabtex}
\begin{arabtext}
\noindent
sa^gada wa^ghiya lilla_di-y _halaqahu wa^saqqa sam`ahu waba.sarahu,
bi.hawlihi waquwwatihi (fatabAraka al-llahu 'a.hsanu al-_h_aliqiyna).\\
\end{arabtext}
\noindent
\textbf{Artinya}:
\par
\indent
"Wajahku bersujud kepada Rabb yang menciptakannya, yang telah membelah
pendengarannya dan penglihatannya dengan daya dan kekuatan-Nya, maka
Mahasuci Allah Sebaik-baik Pencipta."\\\\
\par
\noindent
\textbf{Tingkatan Doa dan Sanad}: Nabi \textit{Shallallahu alaihi wa sallam}
mengucapkan dalam sujud al-Qur'an (sujud tilawah) pada waktu malam, yakni
beliau mengucapkan (berkali-kali): "\textit{Sajada wajh\^{i}...}".
\textbf{Shahih}: HR. Abu Dawud (no. 1414), At-Tirmidzi (no. 580), An-Nasai
(II/222), Ahmad (VI/30-31), dan al-Hakim (I/220) dari Aisyah r.a. Hadits ini
dishahihkan oleh Imam At-Tirmidzi, al-Hakim, an-Nawawi, adz-Dzahabi, Syaikh
al-Albani, dan dihasankan oleh al-Hafizh Ibnu Hajar al-Asqalani dalam
\textit{Nat\^{a}-ijul Afk\^{a}r} (II/116-118). Lihat \textit{Shah\^{i}h
at-Tirmidzi} (I/80, no. 474), \textit{Shah\^{i}h Sunan Abi Dawud}
(V/157-158, no. 1273), dan \textit{Shah\^{i}h al-Adzk\^{a}r} (no. 150/122).
Adapun tambahan di dalam kurung diriwayatkan oleh al-Hakim (I/220). Tambahan
ini dishahihkan oleh Al-Hakim, juga oleh adz-Dzahabi dan an-Nawawi.\\
\textbf{Referensi}: Yazid bin Abdul Qadir Jawas. 2016. Kumpulan Do'a dari
Al-Quran dan As-Sunnah yang Shahih. Bogor: Pustaka Imam Asy-Syafi'i.
\end{document}