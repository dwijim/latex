\documentclass[a4paper,12pt]{article}
\usepackage{arabtex}
\usepackage[bahasa] {babel}
\usepackage[top=2cm,left=3cm,right=3cm,bottom=3cm]{geometry}
\title{\Large Doa Mohon Karunia Allah saat Mendengar Kokok Ayam, dan
Berlindung kepada-Nya saat Mendengar Ringkikkan Keledai dan Lolongan
Anjing}
\author{\small Hanifah Atiya Budianto\\
		\small contact.us@latex-dailyprayers.com}
\begin{document}
\sffamily
\maketitle
\fullvocalize
\setcode{arabtex}
\begin{arabtext}
\noindent
'i_dA sami`tum nubA.ha al-kilAbi wanahi-yqa al-.hami-yri bi-al-llayli
fata`awwa_duW bi-al-ll_ahi mina (al-^s^say.tAni) fa-'innahunna yarayna mAlA
tarawna.\\
\end{arabtext}
\noindent
\textbf{Artinya}:
\par
\indent
"Jika kalian mendengar lolongan anjing dan ringkikan keledai pada malam
hari, berlindunglah kepada Allah (dari syaitan), karena ia melihat apa yang
tidak dapat kalian lihat."\\\\
\par
\noindent
\textbf{Tingkatan Doa dan Sanad}: \textbf{Shahih}: HR. Abu Dawud (no.
5103), Ahmad (III/306, 355-356), dan Ibnus Sunni (no.311) dalam kitab
\textit{'Amalul Yaum wal Lailah} dari Jabir bin Abdillah r.a. Lihat
\textit{Shah\^{i}h al-Adabul Mufrad} (no. 937).\\
\textbf{Referensi}: Yazid bin Abdul Qadir Jawas. 2016. Kumpulan Do'a dari
Al-Quran dan As-Sunnah yang Shahih. Bogor: Pustaka Imam Asy-Syafi'i.
\end{document}