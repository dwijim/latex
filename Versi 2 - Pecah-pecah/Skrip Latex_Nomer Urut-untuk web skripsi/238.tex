\documentclass[a4paper,12pt]{article}
\usepackage{arabtex}
\usepackage[bahasa] {babel}
\usepackage[top=2cm,left=3cm,right=3cm,bottom=3cm]{geometry}
\title{\Large Doa Menghadapi Musuh dan Orang yang Berkuasa}
\author{\small Hanifah Atiya Budianto\\
		\small contact.us@latex-dailyprayers.com}
\begin{document}
\sffamily
\maketitle
\fullvocalize
\setcode{arabtex}
\begin{arabtext}
\noindent
al-ll_ahumma 'anta `a.dudi-y, wa'anta na.si-yri-y, bika 'a.hu-wlu, wabika
'a.su-wlu, wabika 'uqAtilu.\\
\end{arabtext}
\noindent
\textbf{Artinya}:
\par
\indent
"Ya Allah, Engkau adalah Penolongku. Engkau adalah Pembelaku. Dan dengan
pertolongan-Mu aku bergerak, dengan bantuan-Mu aku menyergap, dengan
pertolongan-Mu pula aku berperang."\\\\
\par
\noindent
\textbf{Tingkatan Doa dan Sanad}: \textbf{Shahih}: HR. Abu Dawud (no. 2632)
dan at-Tirmidzi (no. 3584) dari Anas r.a. Lihat \textit{Shah\^{i}h
at-Tirmidzi} (III/183) dan \textit{al-Kalimuth Thayyib} (hlm. 120, no.
126).\\
\textbf{Referensi}: Yazid bin Abdul Qadir Jawas. 2016. Kumpulan Do'a dari
Al-Quran dan As-Sunnah yang Shahih. Bogor: Pustaka Imam Asy-Syafi'i.
\end{document}