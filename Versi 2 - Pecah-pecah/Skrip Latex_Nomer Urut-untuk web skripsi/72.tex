\documentclass[a4paper,12pt]{article}
\usepackage{arabtex}
\usepackage[bahasa] {babel}
\usepackage[top=2cm,left=3cm,right=3cm,bottom=3cm]{geometry}
\title{\Large Doa ketika Mendengar Adzan}
\author{\small Hanifah Atiya Budianto\\
		\small contact.us@latex-dailyprayers.com}
\begin{document}
\sffamily
\maketitle
\fullvocalize
\setcode{arabtex}
\indent
Terdapat lima hal yang disunnahkan ketika adzan dikumandangkan:\\\\
1. Menjawab adzan seperti apa yang diucapkan muadzin, kecuali pada
lafazh: "hayya ‘alas shalah" dan lafazh "hayya ‘alal falah", maka kita
mengucapkan:
\begin{arabtext}
\noindent
lA .hawla walA quwwaTa 'illA bi-al-ll_ahi.\\
\end{arabtext}
\noindent
\textbf{Artinya}:
\par
\indent
"Tidak ada daya dan kekuatan kecuali dengan pertolongan Allah."\\\\
Baca juga doa ketika mendengar adzan ke 1-5\\
\par
\noindent
\textbf{Tingkatan Doa dan Sanad}: "Barang siapa menjawab adzan dengan ikhlas
dari hatinya, ia akan masuk Surga. "Lihat \textit{Syarah Muslim} (IV/85-86
no. 385). Dan apabila seorang muadzin mengucapkan: \textit{"ash shalatu
khairum minannaum"}, maka hendaklah dijawab seperti itu juga.\\
\textbf{Referensi}: Yazid bin Abdul Qadir Jawas. 2016. Kumpulan Do'a dari
Al-Quran dan As-Sunnah yang Shahih. Bogor: Pustaka Imam Asy-Syafi'i.
\end{document}