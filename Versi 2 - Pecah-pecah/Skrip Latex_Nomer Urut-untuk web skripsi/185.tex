\documentclass[a4paper,12pt]{article}
\usepackage{arabtex}
\usepackage[bahasa] {babel}
\usepackage[top=2cm,left=3cm,right=3cm,bottom=3cm]{geometry}
\title{\Large Doa Mendapatkan Kebaikan Dunia dan Akhirat}
\author{\small Hanifah Atiya Budianto\\
		\small contact.us@latex-dailyprayers.com}
\begin{document}
\sffamily
\maketitle
\fullvocalize
\setcode{arabtex}
\begin{arabtext}
\noindent
al-ll_ahumma 'inni-y 'as'aluka al-^gannaTa wa'a`u-w_du bika mina al-nnAri.
\end{arabtext}
\noindent
\textbf{Artinya}:
\par
\indent
"Ya Allah, aku memohon kepada-Mu agar dimasukkan ke dalam Surga dan aku
berlindung kepada-Mu dari siksa Neraka."\\\\
\par
\noindent
\textbf{Tingkatan Doa dan Sanad}: \textbf{Shahih}: HR. Abu Dawud (no. 792),
Ibnu Majah (no. 910), dan Ibnu Khuzaimah (no. 725). Dishahihkan oleh Imam
Ibnu Khuzaimah, Imam an-Nawawi, dan Syaikh al-Albani.\\
\textbf{Referensi}: Yazid bin Abdul Qadir Jawas. 2016. Kumpulan Do'a dari
Al-Quran dan As-Sunnah yang Shahih. Bogor: Pustaka Imam Asy-Syafi'i.
\end{document}