\documentclass[a4paper,12pt]{article}
\usepackage{arabtex}
\usepackage[bahasa] {babel}
\usepackage[top=2cm,left=3cm,right=3cm,bottom=3cm]{geometry}
\title{\Large Doa Penghilang Kegelisahan dan Rasa Takut serta Menolak
Gangguan Syaitan ketika Tidur}
\author{\small Hanifah Atiya Budianto\\
		\small contact.us@latex-dailyprayers.com}
\begin{document}
\sffamily
\maketitle
\fullvocalize
\setcode{arabtex}
\begin{arabtext}
\noindent
'a`u-w_du bikalimAti al-ll_ahi al-ttAmmAti min .ga.dabihi, wa`iqAbihi,
wa^sarri `ibAdihi, wamin hamazAti al-^s^sayA .ti-yni, wa'an ya.h.duru-wni.
\\
\end{arabtext}
\noindent
\textbf{Artinya}:
\par
\indent
"Aku berlindung dengan perantara kalimat-kalimat Allah yang sempurna dari
murka dan siksa-Nya, serta dari kejahatan hamba-hamba-Nya, dan dari godaan
syaitan-syaitan, juga dari kedatangan mereka kepadaku."\\\\
\par
\noindent
\textbf{Tingkatan Doa dan Sanad}: \textbf{Shahih}: HR. Abu Dawud (no.
3893), at-Tirmidzi (no. 3528) Ibnu Sunni dalam \textit{'Amalul Yaum wal
Lailah} (no. 748), dan lainnya. Lihat \textit{Silsilah Ah\^{a}d\^{i}ts
ash-Shah\^{i}hah} (no. 264).\\
\textbf{Referensi}: Yazid bin Abdul Qadir Jawas. 2016. Kumpulan Do'a dari
Al-Quran dan As-Sunnah yang Shahih. Bogor: Pustaka Imam Asy-Syafi'i.
\end{document}