\documentclass[a4paper,12pt]{article}
\usepackage{arabtex}
\usepackage[bahasa] {babel}
\usepackage[top=2cm,left=3cm,right=3cm,bottom=3cm]{geometry}
\title{\Large Doa dan Dzikir sebelum Tidur}
\author{\small Hanifah Atiya Budianto\\
		\small contact.us@latex-dailyprayers.com}
\begin{document}
\sffamily
\maketitle
\fullvocalize
\setcode{arabtex}
\begin{arabtext}
\noindent
al-ll_ahumma _halaqta nafsi-y wa'anta tawaffAhA, laka mamAtuhA wama.hyAhA,
'in 'a.hyaytahA fA.hfa.zhA, wa-'in 'amattahA fA.gfirlahA. al-ll_ahumma
'inni-y 'as'aluka al-`AfiyaTa.\\
\end{arabtext}
\noindent
\textbf{Artinya}:\\
\indent
"Ya Allah, sesungguhnya Engkau telah menciptakan diriku, dan Engkaulah
yang akan mematikannya. Mati dan hidupnya hanyalah milik-Mu. Jika Engkau
menghidupkan jiwaku ini, maka peliharalah ia. Dan jika Engkau mematikannya,
maka ampunilah ia. Ya Allah, sesungguhnya aku mohon keselamatan kepada-Mu."
\\\\
Baca juga doa dan dzikir sebelum tidur ke 1-5\\
\par
\noindent
\textbf{Tingkatan Doa dan Sanad}: \textbf{Shahih}: HR. Muslim (no. 2712
[60]) dan Ahmad (II/79). Juga Ibnus Sunni dalam \textit{'Amalul Yaum wal
Lailah} (no. 721).\\
\textbf{Referensi}: Yazid bin Abdul Qadir Jawas. 2016. Kumpulan Do'a dari
Al-Quran dan As-Sunnah yang Shahih. Bogor: Pustaka Imam Asy-Syafi'i.
\end{document}