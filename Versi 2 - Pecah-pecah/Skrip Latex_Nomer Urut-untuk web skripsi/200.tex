\documentclass[a4paper,12pt]{article}
\usepackage{arabtex}
\usepackage[bahasa] {babel}
\usepackage[top=2cm,left=3cm,right=3cm,bottom=3cm]{geometry}
\title{\Large Doa Mohon Ampunan dan Kasih Sayang}
\author{\small Hanifah Atiya Budianto\\
		\small contact.us@latex-dailyprayers.com}
\begin{document}
\sffamily
\maketitle
\fullvocalize
\setcode{arabtex}
\begin{arabtext}
\noindent
al-ll_ahumma .tahhir ni-y mina al-_d_dunu-wbi wAl-_ha.tAyA, al-ll_ahumma
naqqini-y minhA, kamA yunaqqY al-_t_tawbu al-'abya.du mina al-ddanasi,
al-ll_ahumma .tahhir ni-y bi-al-_t_tal^gi, wAl-baradi, wAl-mA'i al-bAridi.
\end{arabtext}
\noindent
\textbf{Artinya}:
\par
"Ya Allah, sucikanlah aku dari berbagai dosa dan kesalahan. Ya Allah,
bersihkan diriku darinya sebagaimana dibersihkannya kain putih dari
kotoran. Ya Allah, sucikanlah diriku dengan salju, embun, dan air yang
dingin."\\\\
\par
\noindent
\textbf{Tingkatan Doa dan Sanad}: \textbf{Shahih}: HR. Muslim (no. 476
[204]), an-Nasai (I/198, 199) dan at-Tirmidzi (no. 3547) dari Abdullah bin
Abi Aufa. Lafazh ini milik an-Nasai. Lihat \textit{Shah\^{i}h an-Nasai}
(I/86).\\
\textbf{Referensi}: Yazid bin Abdul Qadir Jawas. 2016. Kumpulan Do'a dari
Al-Quran dan As-Sunnah yang Shahih. Bogor: Pustaka Imam Asy-Syafi'i.
\end{document}