\documentclass[a4paper,12pt]{article}
\usepackage{arabtex}
\usepackage[bahasa] {babel}
\usepackage[top=2cm,left=3cm,right=3cm,bottom=3cm]{geometry}
\title{\Large Membaca Shalawat{\scriptsize 1} Nabi setelah Tasyahud}
\author{\small Hanifah Atiya Budianto\\
		\small contact.us@latex-dailyprayers.com}
\begin{document}
\sffamily
\maketitle
\fullvocalize
\setcode{arabtex}
\begin{arabtext}
\noindent
al-ll_ahumma .salli `alaY mu.hammadiN wa`alaY 'azwA^gihi wa_durriyyatihi,
kamA .salla-yta `alaY ^Ali 'ibrAhi-yma, wabArik `alaY mu.hammadiN wa`alaY
'azwA^gihi wa_durriyyatihi, kamA bArakta `alaY ^Ali 'ibrAhi-yma, 'innaka
.hamiyduN ma^gi-yduN.\\
\end{arabtext}
\noindent
\textbf{Artinya}:
\par
\indent
"Ya Allah, berikanlah shalawat kepada Muhammad, istri-istri dan
keturunannya, sebagaimana Engkau telah memberikan shalawat kepada keluarga
Nabi Ibrahim. Berikanlah berkah kepada Muhammad, istri-istri dan
keturunannya, sebagaimana Engkau telah memberikan berkah kepada keluarga
Ibrahim. Sesungguhnya Engkau Maha Terpuji lagi Mahamulia."{\scriptsize 2}
\\\\
\par
\noindent
\textbf{Tingkatan Doa dan Sanad}:
\begin{enumerate}
\item Tidak ada tambahan lafazh "sayyidinaa" dalam shalawat dan tidak ada
satu pun riwayat yang shahih dari Nabi Shallallahu ‘alaihi wa sallam, dan
lafazh ini pun tidak diucapkan oleh para Sahabat radhiyallahu 'anhum.
\item \textbf{Shahih}: HR. Malik dalam \textit{al-Muwaththa'} (I/152, no.
66), al-Bukhari (no. 3369)/\textit{Fathul Bari} (VI/407), Muslim (no. 407
[69]), Abu Dawud (no. 979), dan lainnya. Lafazh tersebut diriwayatkan oleh
Muslim dari Abu Humaid as-Sa'idi r.a.
\end{enumerate}
\textbf{Referensi}: Yazid bin Abdul Qadir Jawas. 2016. Kumpulan Do'a dari
Al-Quran dan As-Sunnah yang Shahih. Bogor: Pustaka Imam Asy-Syafi'i.
\end{document}