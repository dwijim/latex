\documentclass[a4paper,12pt]{article}
\usepackage{arabtex}
\usepackage[bahasa] {babel}
\usepackage[top=2cm,left=3cm,right=3cm,bottom=3cm]{geometry}
\title{\Large Doa Sesudah Makan}
\author{\small Hanifah Atiya Budianto\\
		\small contact.us@latex-dailyprayers.com}
\begin{document}
\sffamily
\maketitle
\fullvocalize
\setcode{arabtex}
\begin{arabtext}
\noindent
al-.hamdu lill_ahi .hamdaN ka_ti-yraN .tayyibaN mubArakaN fi-yhi, .ga-yra
makfiyyiN walA muwadda-`iN, walA musta.gnaN_A `anhu rabbanA.\\
\end{arabtext}
\noindent
\textbf{Artinya}:
\par
\indent
"Segala puji bagi Allah (aku memuji-Nya) dengan pujian yang banyak, yang
baik dan penuh berkah, yang senantiasa dibutuhkan, diperlukan dan tidak
bisa ditinggalkan (pengharapan kepada-Nya) wahai Rabb kami."\\\\
\par
\noindent
\textbf{Tingkatan Doa dan Sanad}: \textbf{Shahih}: HR. Al-Bukhari (no.
5458), Abu Dawud (no. 3849), Ahmad (V/252, 256), at-Tirmidzi (no. 3456),
Ibnus Sunni dalam \textit{'Amalul Yaum wal Lailah} (no. 468, 484),
al-Baghawi dalam \textit{Syarhus Sunnah} (no. 2828) dari Abu Umamah
al-Bahili r.a.\\
\textbf{Referensi}: Yazid bin Abdul Qadir Jawas. 2016. Kumpulan Do'a dari
Al-Quran dan As-Sunnah yang Shahih. Bogor: Pustaka Imam Asy-Syafi'i.
\end{document}