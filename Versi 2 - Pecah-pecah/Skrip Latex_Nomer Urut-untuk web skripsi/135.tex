\documentclass[a4paper,12pt]{article}
\usepackage{arabtex}
\usepackage[bahasa] {babel}
\usepackage[top=2cm,left=3cm,right=3cm,bottom=3cm]{geometry}
\title{\Large Doa Naik Kendaraan}
\author{\small Hanifah Atiya Budianto\\
		\small contact.us@latex-dailyprayers.com}
\begin{document}
\sffamily
\maketitle
\fullvocalize
\setcode{arabtex}
\begin{arabtext}
\noindent
bismi al-ll_ahi, al-.hamduli-ll_ahi (sub.h_ana alla_diY sa_h_hara lanA
h_a_dA wamA kunnA lahu muqrini-yna wa-'inn^A 'il_aY rabbinA lamunqalibuwna)
al-.hamduli-ll_ahi, al-.hamduli-ll_ahi, al-.hamduli-ll_ahi, al-ll_ahu
'akbaru, al-ll_ahu 'akbaru, al-ll_ahu 'akbaru, sub.hAnaka 'inni-y .zalamtu
nafsi-y fA.gfirli-y, fa'i-nnahu lA ya.gfiru al-_d_dunu-wba 'illA 'anta.\\
\end{arabtext}
\noindent
\textbf{Artinya}:
\par
\indent
"Dengan nama Allah, segala puji bagi Allah, \textit{Mahasuci Rabb yang
menundukkan kendaraan ini untuk kami, padahal kami sebelumnya tidak mampu
menguasainya. Dan sesungguhnya kami akan kembali kepada Rabb kami (di hari
Kiamat)}. Segala puji bagi Allah (3x), Allah Mahabesar (3x), Mahasuci
Eangkau. Ya Allah, sesungguhnya aku menganiaya diriku, maka ampunilah aku.
Sesungguhnya tidak ada yang dapat mengampuni dosa-dosa kecuali Engkau."\\\\
\par
\noindent
\textbf{Tingkatan Doa dan Sanad}: \textbf{Shahih}: HR. Abu Dawud (no.
2602), at-Tirmidzi (no. 3446), Shahih Abi Dawud (II/493 no. 2267), dan
Shahih at-Tirmidzi (III/156, no. 2742).\\
\textbf{Referensi}: Yazid bin Abdul Qadir Jawas. 2016. Kumpulan Do'a dari
Al-Quran dan As-Sunnah yang Shahih. Bogor: Pustaka Imam Asy-Syafi'i.
\end{document}