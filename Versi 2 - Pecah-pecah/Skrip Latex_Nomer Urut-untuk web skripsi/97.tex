\documentclass[a4paper,12pt]{article}
\usepackage{arabtex}
\usepackage[bahasa] {babel}
\usepackage[top=2cm,left=3cm,right=3cm,bottom=3cm]{geometry}
\title{\Large Doa Sujud Tilawah}
\author{\small Hanifah Atiya Budianto\\
		\small contact.us@latex-dailyprayers.com}
\begin{document}
\sffamily
\maketitle
\fullvocalize
\setcode{arabtex}
\begin{arabtext}
\noindent
al-ll_ahumma aktub li-y bihA `indaka 'a^graN, wa.da` `anni-y bihA wizraN,
wA^g`alhA li-y `indaka _du_hraN, wataqabbalhA minni-y kamA taqabbaltahA min
`abdika dAwuda.\\
\end{arabtext}
\noindent
\textbf{Artinya}:
\par
\indent
"Ya Allah, tulislah untukku dengan sujudku pahala di sisi-Mu dan ampuni
dosaku dengannya, serta jadikanlah ia simpanan untukku di sisi-Mu, dan juga
terimalah sujudku sebagaimana Engkau menerimanya dari hamba-Mu, Dawud."\\\\
\par
\noindent
\textbf{Tingkatan Doa dan Sanad}: \textbf{Hasan}: HR. At-Tirmidi (no. 579
dan no. 3424), \textit{Shah\^{i}h at-Tirmidzi} (I/180 no. 473), dan al-Hakim
(I/220). At-Tirmidzi mengatakan hasan. Menurut al-Hakim, hadits tersebut
shahih. Dan adz-Dzahabi sependapat dengannya.\\
\textbf{Referensi}: Yazid bin Abdul Qadir Jawas. 2016. Kumpulan Do'a dari
Al-Quran dan As-Sunnah yang Shahih. Bogor: Pustaka Imam Asy-Syafi'i.
\end{document}