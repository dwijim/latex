\documentclass[a4paper,12pt]{article}
\usepackage{arabtex}
\usepackage[bahasa] {babel}
\usepackage[top=2cm,left=3cm,right=3cm,bottom=3cm]{geometry}
\title{\Large Doa Mohon Ampun dan Rahmat Allah}
\author{\small Hanifah Atiya Budianto\\
		\small contact.us@latex-dailyprayers.com}
\begin{document}
\sffamily
\maketitle
\fullvocalize
\setcode{arabtex}
\begin{arabtext}
\noindent
rabban^A 'innanA sami`nA munAdiyaN yunAdiY lil-'iym_ani 'an -'a-aminuW
birabbikum fa'_a-mannA rabbanA fa-a.gfir lanA _dunuwbanA wakaffir `annA
sayyi'AtinA watawaffanA ma`a al-'abrAri  $\odot$ rabbanA wa -'a-atinA mA
wa`adttanA `al_aY rusulika walA tu_hzinA yawma al-qiy_amaTi 'innaka lA
tu_hlifu almiy`Ada
\end{arabtext}
\noindent
\textbf{Artinya}:\\
\indent
"Ya Rabb kami, sesungguhnya kami mendengar orang yang menyeru kepada iman,
(yaitu) 'Berimanlah kamu kepada Rabbmu,' maka kami pun beriman. Ya Rabb
kami, ampunilah dosa-dosa kami dan hapuskanlah kesalahan-kesalahan kami,
dan matikanlah kami beserta orang-orang yang berbakti. Ya Rabb kami,
berilah kami apa yang telah Engkau janjikan kepada kami melalui
Rasul-Rasul-Mu. Dan janganlah Engkau hinakan kami pada hari Kiamat.
Sungguh, Engkau tidak pernah mengingkari janji." (QS. Ali 'Imran [3]:
193-194).\\\\
\noindent
\textbf{Perhatian}: Mohon baca menu Bantuan, terdapat perbedaan penulisan 
doa pada Latex dan buku.\\
\textbf{Referensi}: Yazid bin Abdul Qadir Jawas. 2016. Kumpulan Do'a dari
Al-Quran dan As-Sunnah yang Shahih. Bogor: Pustaka Imam Asy-Syafi'i.
\end{document}