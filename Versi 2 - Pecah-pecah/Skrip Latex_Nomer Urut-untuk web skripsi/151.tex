\documentclass[a4paper,12pt]{article}
\usepackage{arabtex}
\usepackage[bahasa] {babel}
\usepackage[top=2cm,left=3cm,right=3cm,bottom=3cm]{geometry}
\title{\Large Doa Melihat Hilal (Awal Bulan Hijriyyah)}
\author{\small Hanifah Atiya Budianto\\
		\small contact.us@latex-dailyprayers.com}
\begin{document}
\sffamily
\maketitle
\fullvocalize
\setcode{arabtex}
\begin{arabtext}
\noindent
al-ll_ahumma 'ahillahu `ala-ynaa bilyumni wAl-'i-ymAni, wAl-ssalAmaTi
wAl-'i-slAmi, rabbiY warabbuka al-ll_ahu.\\
\end{arabtext}
\noindent
\textbf{Artinya}:
\par
\indent
"Ya Allah, tampakkan bulan itu kepada kami dengan membawa keberkahan dan
keimanan, keselamatan dan Islam. Rabbku dan Rabbmu (wahai bulan sabit)
adalah Allah."\\\\
\par
\noindent
\textbf{Tingkatan Doa dan Sanad}: \textbf{Shahih}: HR. At-Tirmidzi (no.
3451), Ahmad (I/162), dan al-Hakim (IV/285) dari Thalhah bin Ubaidillah
r.a. Diriwayatkan oleh ad-Darimi (II/3-4) dan Ibnu Hibban ( no. 885 -
\textit{at-Ta'l\^{i}q\^{a}tul His\^{a}n}) dari Ibnu Umar r.a. Lihat
\textit{Silsilah Ah\^{a}d\^{i}ts ash-Shah\^{i}hah} (no. 1816) dan
\textit{al-Ikhb\^{a}r bima La Yashihhu min Ah\^{a}ditsil Adzk\^{a}r}
(hlm. 282).\\
\textbf{Referensi}: Yazid bin Abdul Qadir Jawas. 2016. Kumpulan Do'a dari
Al-Quran dan As-Sunnah yang Shahih. Bogor: Pustaka Imam Asy-Syafi'i.
\end{document}