\documentclass[a4paper,12pt]{article}
\usepackage{arabtex}
\usepackage[bahasa] {babel}
\usepackage[top=2cm,left=3cm,right=3cm,bottom=3cm]{geometry}
\title{\Large Doa Istiftah}
\author{\small Hanifah Atiya Budianto\\
		\small contact.us@latex-dailyprayers.com}
\begin{document}
\sffamily
\maketitle
\fullvocalize
\setcode{arabtex}
\begin{arabtext}
\noindent
al-ll_ahumma laka al-.hamdu, 'anta qayyimu al-ssamAwAti wAl-'ar.di waman
fi-yhinna, walaka al-.hamdu laka mulku al-ssamAwAti wAl-'ar.di waman
fi-yhinna, walaka al-.hamdu, 'anta nu-wru al-ssamAwAti wAl-'ar.di, walaka
al-.hamdu, 'anta maliku al-ssamAwAti wAl-'ar.di, walaka al-.hamdu 'anta
al-.haqqu, wawa`duka al-.haqqu, waliqA'uka .haqquN, waqawluka .haqquN,
wAl-^gannaTu .haqquN, wAl-nnAru .haqquN, wAl-nnabiyyu-wna .haqquN,
wamu.hammaduN .sallY al-ll_ahu `ala-yhi wasallama .haqquN, wAl-ssA`aTu
.haqquN, al-ll_ahumma laka 'aslamtu, wabika ^Amantu, wa`alayka tawakkaltu,
wa-'ilayka 'anabtu, wabika _hA.samtu, wa-'ilayka .hAkamtu, fA.gfirli-y mA
qaddamtu wamA-'a_h_hartu, wamA-'asrartu, wamA-'a`lantu, 'anta al-muqaddimu,
wa-'anta al-mu'a_h_hiru, lA 'il_aha 'illA 'anta.\\
\end{arabtext}
\noindent
\textbf{Artinya}:
\par
\indent
"Ya Allah, bagi-Mu segala puji, Engkaulah Pemelihara seluruh langit dan
bumi, serta segenap makhluk yang ada padanya. Bagi-Mu segala puji, bagi-Mu
kerajaan langit dan bumi, serta segenap makhluk yang ada padanya. Bagi-Mu
segala puji, Engkau adalah cahaya langit dan bumi. Bagi-Mu segala puji,
Engkaulah penguasa langit dan bumi. Bagi-Mu segala puji, Engkaulah Yang
Mahabenar, janji-Mu benar, pertemuan dengan-Mu adalah benar, firman-Mu
benar, adanya Surga itu benar, adanya Neraka adalah benar, diutusnya para
Nabi \textit{'alaihimussalatu wassalam} adalah benar, Nabi Muhammad
shallallahu ‘alaihi wa sallam adalah benar, dan adanya Kiamat adalah benar.
Ya Allah, hanya kepada-Mu aku berserah, hanya pada-Mu aku beriman, hanya
kepada-Mu aku bertawakal, hanya pada-Mulah aku bertaubat, hanya dengan
(pertolongan)-Mu aku berdebat dan hanya kepada-Mu aku berhukum (mohon
keputusan). Oleh karena itu, ampunilah dosa-dosaku yang telah lalu dan yang
akan datang, yang aku lakukan secara sembunyi-sembunyi atau
terang-terangan. Engkaulah Yang mendahulukan dan mengakhirkan. Tidak ada
ilah yang berhak diibadahi dengan benar kecuali Engkau."\\\\
\par
\noindent
\textbf{Tingkatan Doa dan Sanad}: \textbf{Shahih}: HR. Al-Bukhari (no. 1120,
6317, 7385, 7442, 7499). Muslim juga meriwayatkannya dengan ringkas (no. 769
[199]) dari Ibnu Abbas r.a. Doa istiftah ini dibaca ketika shalat malam
(Tahajud).\\
\textbf{Referensi}: Yazid bin Abdul Qadir Jawas. 2016. Kumpulan Do'a dari
Al-Quran dan As-Sunnah yang Shahih. Bogor: Pustaka Imam Asy-Syafi'i.
\end{document}