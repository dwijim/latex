\documentclass[a4paper,12pt]{article}
\usepackage{arabtex}
\usepackage[bahasa] {babel}
\usepackage[top=2cm,left=3cm,right=3cm,bottom=3cm]{geometry}
\title{\Large Doa Agar Diberi Ilmu Yang Bermanfaat dan Berlindung Dari Ilmu
Yang Tidak Bermanfaat}
\author{\small Hanifah Atiya Budianto\\
		\small contact.us@latex-dailyprayers.com}
\begin{document}
\sffamily
\maketitle
\fullvocalize
\setcode{arabtex}
\begin{arabtext}
\noindent
al-ll_ahumma 'inni-y 'as'aluka `ilmaN nAfi`aN, wa-rizqaN .tayyibaN,
wa-`amalaN mutaqabbalaN.\\
\end{arabtext}
\noindent
\textbf{Artinya}:
\par
\indent
"Ya Allah, sesungguhnya aku memohon kepada-Mu ilmu yang bermanfaat, rizki
yang baik, dan amal yang diterima."\\\\
\par
\noindent
\textbf{Tingkatan Doa dan Sanad}: \textbf{Shahih}: HR. Ibnu Majah (no. 925)
dan ath-Thabrani dalam \textit{al-Mu'jamus Shagh\^{i}r} (I/260). Lihat
\textit{Shah\^{i}h Ibni Majah} (I/152, no. 753).\\
\textbf{Referensi}: Yazid bin Abdul Qadir Jawas. 2016. Kumpulan Do'a dari
Al-Quran dan As-Sunnah yang Shahih. Bogor: Pustaka Imam Asy-Syafi'i.
\end{document}