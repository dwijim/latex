\documentclass[a4paper,12pt]{article}
\usepackage{arabtex}
\usepackage[bahasa] {babel}
\usepackage[top=2cm,left=3cm,right=3cm,bottom=3cm]{geometry}
\title{\Large Bacaan Setelah Salam}
\author{\small Hanifah Atiya Budianto\\
		\small contact.us@latex-dailyprayers.com}
\begin{document}
\sffamily
\maketitle
\fullvocalize
\setcode{arabtex}
\begin{arabtext}
\noindent
lA 'il_aha 'illA al-ll_ahu wa.hdahu lA ^sari-yka lahu, lahu al-mulku walahu
al-.hamdu wahuwa `alY kulli ^say'iN qadi-yruN, Aal-ll_ahumma lA mAni`a limA
'a`.ta-yta, walA mu`.tiya limA mana`ta, walA yanfa`u _dA al-^gaddi minka
al-^gaddu.\\
\end{arabtext}
\noindent
\textbf{Artinya}:
\par
\indent
"Tidak ada ilah yang berhak diibadahi dengan benar melainkan hanya Allah
Yang Maha Esa, tiada sekutu bagi-Nya. Bagi-Nya segala kerajaan dan bagi-Nya
pula segala pujian. Dia Mahakuasa atas segala sesuatu. Ya Allah, tidak ada
yang mencegah apa yang Engkau beri dan tidak ada yang memberi apa yang
Engkau cegah. Tidak berguna kekayaan dan kemuliaan bagi pemiliknya dari
(siksa)-Mu."\\\\
Baca juga bacaan setelah salam ke 1-8\\
\par
\noindent
\textbf{Tingkatan Doa dan Sanad}: \textbf{Shahih}: HR. Al-Bukhari (no. 844),
Muslim (no. 593), Ahmad (IV/245, 247, 250, 254, 255), Abu Dawud (no. 1505),
an-Nasai (III/70, 71), ad-Darimi (I/311), dan Ibnu Khuzaimah (no. 742) dari
al-Mughirah bin Syu'bah r.a.\\
\textbf{Referensi}: Yazid bin Abdul Qadir Jawas. 2016. Kumpulan Do'a dari
Al-Quran dan As-Sunnah yang Shahih. Bogor: Pustaka Imam Asy-Syafi'i.
\end{document}