\documentclass[a4paper,12pt]{article}
\usepackage{arabtex}
\usepackage[bahasa] {babel}
\usepackage[top=2cm,left=3cm,right=3cm,bottom=3cm]{geometry}
\title{\Large Doa Keluar Rumah}
\author{\small Hanifah Atiya Budianto\\
		\small contact.us@latex-dailyprayers.com}
\begin{document}
\sffamily
\maketitle
\fullvocalize
\setcode{arabtex}
\begin{arabtext}
\noindent
al-ll_ahumma 'innI 'a`uw_dubika 'an 'a.dilla, 'aw 'u.dalla, 'aw 'azilla,
'aw 'uzalla, 'aw 'a.zlima, 'aw 'u.zlama, 'aw 'a^ghala, 'aw yu^ghala
`alayya.\\
\end{arabtext}
\noindent
\textbf{Artinya}:
\par
\indent
"Ya Allah, sesungguhnya aku berlindung kepada-Mu, janganlah sampai aku
sesat atau disesatkan (syaitan atau orang jahat), tergelincir atau
digelincirkan orang lain, menganiaya atau dianiaya orang lain, dan berbuat
bodoh atau dibodohi orang lain."\\\\
\par
\noindent
\textbf{Tingkatan Doa dan Sanad}: \textbf{Shahih}: HR. Abu Dawud (no. 5094,
at-Tirmidzi (no. 3427), an-Nasai (VII/268), Ibnu Majah (no. 3884) dari Ummu
Salamah r.a. Lihat kitab \textit{Hid\^{a}yatur Ruw\^{a}t} (III/12, no.
2376). Sanad hadits ini shahih.\\
\textbf{Referensi}: Yazid bin Abdul Qadir Jawas. 2016. Kumpulan Do'a dari
Al-Quran dan As-Sunnah yang Shahih. Bogor: Pustaka Imam Asy-Syafi'i.
\end{document}