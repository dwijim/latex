\documentclass[a4paper,12pt]{article}
\usepackage{arabtex}
\usepackage[bahasa] {babel}
\usepackage[top=2cm,left=3cm,right=3cm,bottom=3cm]{geometry}
\title{\Large Doa Agar Diberi Ilmu Yang Bermanfaat dan Berlindung Dari Ilmu
Yang Tidak Bermanfaat}
\author{\small Hanifah Atiya Budianto\\
		\small contact.us@latex-dailyprayers.com}
\begin{document}
\sffamily
\maketitle
\fullvocalize
\setcode{arabtex}
\begin{arabtext}
\noindent
al-ll_ahumma 'inni-y 'a`u-w_dubika min qalbiN lA ya_h^sa`u, wa-min du`A'iN
lA yusma`u, wa-min nafsiN lA ta^sba`u, wa-min `ilmiN lA yanfa`u, 'a`u-w_du
bika min h_a'ulA'i al-'arba`i.\\
\end{arabtext}
\noindent
\textbf{Artinya}:
\par
\indent
"Ya Allah, sesungguhnya aku berlindung kepada-Mu dari hati yang tidak
khusyu', doa yang tidak didengar, nafsu yang tidak pernah puas, dan dari
ilmu yang tidak bermanfaat. Aku berlindung kepada-Mu dari keempat hal
tersebut."\\\\
\par
\noindent
\textbf{Tingkatan Doa dan Sanad}: \textbf{Shahih}: HR. At-Tirmidzi (no.
3482), an-Nasai (VIII/254-255) dari Abdullah bin Amr, Abu Dawud (no. 1548),
dan selainnya dari Abu Hurairah r.a. Lihat \textit{Shah\^{i}h
al-J\^{a}mi'ish Shagh\^{i}r} (no. 1297), \textit{Shah\^{i}h an-Nasai}
(III/1113, no. 5053), dan \textit{Shah\^{i}h Sunan Abi Dawud} (no. 1384)
terbitan Gharras.\\
\textbf{Referensi}: Yazid bin Abdul Qadir Jawas. 2016. Kumpulan Do'a dari
Al-Quran dan As-Sunnah yang Shahih. Bogor: Pustaka Imam Asy-Syafi'i.
\end{document}