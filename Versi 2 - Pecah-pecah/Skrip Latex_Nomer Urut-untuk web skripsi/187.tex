\documentclass[a4paper,12pt]{article}
\usepackage{arabtex}
\usepackage[bahasa] {babel}
\usepackage[top=2cm,left=3cm,right=3cm,bottom=3cm]{geometry}
\title{\Large Doa Mohon Diperbaiki Urusan Dunia dan Akhirat}
\author{\small Hanifah Atiya Budianto\\
		\small contact.us@latex-dailyprayers.com}
\begin{document}
\sffamily
\maketitle
\fullvocalize
\setcode{arabtex}
\begin{arabtext}
\noindent
al-ll_ahumma 'a.sli.h li-y di-yni-y alla_di-y huwa `i.smaTu 'amri-y,
wa'a.sli.h li-y dunyAya allati-y fi-yhA ma`A ^si-y, wa'a.sli.h li-y
^A_hirati-y Aallati-y fi-yhA ma`Adi-y, wA^g`ali al-.hayATa ziyAdaTaN li-y
fi-y kulli _ha-yriN, wA^g`ali al-mawta rA.haTaN li-y min kulli ^sarriN.\\
\end{arabtext}
\noindent
\textbf{Artinya}:
\par
\indent
"Ya Allah, perbaikilah agamaku bagiku yang ia merupakan benteng pelindung
bagi urusanku. Dan perbaikilah duniaku bagiku, yang ia menjadi tempat
hidupku. Serta perbaikilah akhiratku yang ia menjadi tempat kembaliku.
Jadikanlah kehidupan ini sebagai tambahan bagiku dalam setiap kebaikan,
serta jadikanlah kematian sebagai kebebasan bagiku dari segala kejahatan."
\\\\
\par
\noindent
\textbf{Tingkatan Doa dan Sanad}: \textbf{Shahih}: HR. Muslim (no. 2720)
dari Abu Hurairah.\\
\textbf{Referensi}: Yazid bin Abdul Qadir Jawas. 2016. Kumpulan Do'a dari
Al-Quran dan As-Sunnah yang Shahih. Bogor: Pustaka Imam Asy-Syafi'i.
\end{document}