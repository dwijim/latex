\documentclass[a4paper,12pt]{article}
\usepackage{arabtex}
\usepackage[bahasa] {babel}
\usepackage[top=2cm,left=3cm,right=3cm,bottom=3cm]{geometry}
\title{\Large Doa Mohon Ampun dan Rahmat Allah}
\author{\small Hanifah Atiya Budianto\\
	  \small contact.us@latex-dailyprayers.com}
\begin{document}
\sffamily
\maketitle
\fullvocalize
\setcode{arabtex}
\begin{arabtext}
\noindent
rabbi 'inn^I 'a`uw_du bika 'an 'as'alaka mA laysa liY bihi `ilmuN
wa-'illA ta.gfir liY watar.hamn^I 'akun mmina al-_h_asiriyna
\end{arabtext}
\noindent
\textbf{Artinya}:\\
\indent
"Ya Rabbku, sesungguhnya aku berlindung kepada-Mu untuk memohon kepada-Mu
sesuatu yang aku tidak mengetahui (hakikatnya). Kalau Engkau tidak
mengampuniku, dan (tidak) menaruh belas kasihan kepadaku, niscaya aku
termasuk orang yang rugi." (QS. Hud [11]: 47).\\\\
\noindent
\textbf{Perhatian}: Mohon baca menu Bantuan, terdapat perbedaan penulisan
doa pada Latex dan buku.\\
\textbf{Referensi}: Yazid bin Abdul Qadir Jawas. 2016. Kumpulan Do'a dari
Al-Quran dan As-Sunnah yang Shahih. Bogor: Pustaka Imam Asy-Syafi'i.
\end{document}