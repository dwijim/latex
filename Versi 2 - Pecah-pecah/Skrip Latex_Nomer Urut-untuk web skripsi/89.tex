\documentclass[a4paper,12pt]{article}
\usepackage{arabtex}
\usepackage[bahasa] {babel}
\usepackage[top=2cm,left=3cm,right=3cm,bottom=3cm]{geometry}
\title{\Large Doa Bangkit dari Ruku'}
\author{\small Hanifah Atiya Budianto\\
		\small contact.us@latex-dailyprayers.com}
\begin{document}
\sffamily
\maketitle
\fullvocalize
\setcode{arabtex}
\begin{arabtext}
\noindent
rabbanA laka al-.hamdu mil'a al-ssamAwAti wamil'a al-'ar.di wamil'a mA
^si'ta min ^say'iN ba`du. 'ahla al-_t_tanA'i wAl-ma^gdi, 'a.haqqu mAqAla
al-`abdu, wakullunA laka `abduN. al-ll_ahumma lA mAni`a limA 'a`.tayta,
walA mu`.tiya limA mana`ta, walA yanfa`u _dA al-^gaddi minka al-^gaddu.\\
\end{arabtext}
\noindent
\textbf{Artinya}:
\par
\indent
"Wahai Rabb kami, bagi-Mu segala pujian (kami memuji-Mu dengan) pujian
sepenuh langit dan sepenuh bumi, sepenuh apa yang Engkau kehendaki setelah
itu. Wahai Rabb yang layak dipuji dan diagungkan, Yang paling benar
dikatakan oleh seorang hamba dan kami seluruhnya adalah hamba-Mu. Ya Allah,
tidak ada yang akan dapat menghalangi apa yang Engkau berikan dan tidak ada
yang dapat memberi apa yang Engkau halangi, tidak bermanfaat kekayaan dan
kemuliaan bagi pemilik keduanya dari adzab-Mu."\\\\
\par
\noindent
\textbf{Tingkatan Doa dan Sanad}: \textbf{Shahih}: HR. Muslim (no. 477
[205]), Abu Awanah (II/176), Abu Dawud (no. 847) dari Abu Sa'id al-Khudri
r.a.\\
\textbf{Referensi}: Yazid bin Abdul Qadir Jawas. 2016. Kumpulan Do'a dari
Al-Quran dan As-Sunnah yang Shahih. Bogor: Pustaka Imam Asy-Syafi'i.
\end{document}