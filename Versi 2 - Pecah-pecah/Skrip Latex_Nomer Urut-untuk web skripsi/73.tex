\documentclass[a4paper,12pt]{article}
\usepackage{arabtex}
\usepackage[bahasa] {babel}
\usepackage[top=2cm,left=3cm,right=3cm,bottom=3cm]{geometry}
\title{\Large Doa ketika Mendengar Adzan}
\author{\small Hanifah Atiya Budianto\\
		\small contact.us@latex-dailyprayers.com}
\begin{document}
\sffamily
\maketitle
\fullvocalize
\setcode{arabtex}
\indent
Terdapat lima hal yang disunnahkan ketika adzan dikumandangkan:\\\\
2. Setelah muadzin selesai adzan, maka kita mengucapkan:{\scriptsize 1}
\begin{arabtext}
\noindent
(wa'anA) 'a^shadu 'an lA 'i-l_aha 'illA al-ll_ahu wa.hdahu lA^sari-yka lahu
wa ('a^shadu) 'anna mu.hammadaN `abduhu warasu-wluhu, ra.di-ytu
bi-al-ll_ahi rabbaN, wabimu.hammadiN rasu-wlaN wabi-al-'i-slA-mi di-ynaN.\\
\end{arabtext}
\noindent
\textbf{Artinya}:
\par
\indent
"Dan aku bersaksi bahwa tidak ada ilah yang berhak diibadahi dengan benar
melainkan Allah Yang Esa, tidak ada sekutu bagi-Nya, dan aku pun bersaksi
bahwa Muhammad adalah hamba-Nya dan Rasul-Nya, aku ridha Allah sebagai
Rabb, Muhammad sebagai Rasul dan Islam sebagai agama(ku)."{\scriptsize 2}
\\\\
Baca juga doa ketika mendengar adzan ke 1-5\\
\par
\noindent
\textbf{Tingkatan Doa dan Sanad}:
\begin{enumerate}
\item Ada yang berpendapat bahwa dzikir ini dibaca setelah muadzin membaca
syahadat. Lihat kitab \textit{ats-Tsamar al-Mustath\^{a}b f\^{i} Fiqhis
Sunnah wal Kit\^{a}b} (hlm. I/172-185) karya Syaikh al-Albani,
\textit{Maus\^{u}'ah al-Fiqhiyyah al-Muyassarah f\^{i} Fiqhil Kit\^{a}b was
Sunnah al-Muthahhara}h (hlm. 371) karya Husain al-Audah al-Awayisyah,
\textit{Shah\^{i}h al-W\^{a}bilish Shayyib} (hlm. 184), dan
\textit{Tash-h\^{i}hud Du'\^{a}'} (hlm. 370-372).
\item \textbf{Shahih}: HR. Muslim (no. 386), at-Tirmidzi (no. 210), Abu
Dawud (no. 525), an-Nasai (II/26), Ibnu Majah (no. 721), Ahmad (I/181),
Ibnu Khuzaimah (no. 421) dan yang lainnya dari Sa'ad bin Abi Waqqash r.a.
\end{enumerate}
\textbf{Referensi}: Yazid bin Abdul Qadir Jawas. 2016. Kumpulan Do'a dari
Al-Quran dan As-Sunnah yang Shahih. Bogor: Pustaka Imam Asy-Syafi'i.
\end{document}