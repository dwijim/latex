\documentclass[a4paper,12pt]{article}
\usepackage{arabtex}
\usepackage[bahasa] {babel}
\usepackage[top=2cm,left=3cm,right=3cm,bottom=3cm]{geometry}
\title{\Large Doa Istiftah}
\author{\small Hanifah Atiya Budianto\\
		\small contact.us@latex-dailyprayers.com}
\begin{document}
\sffamily
\maketitle
\fullvocalize
\setcode{arabtex}
\begin{arabtext}
\noindent
al-ll_ahumma rabba ^gibrA'iyla, wami-ykA'iyla, wa-'isrAfi-yla fA.tira
al-ssamAwAti wAl-'ar.di, `Alima al-.gaybi wAl-^s^sahAdaTi, 'anta ta.hkumu
bayna `ibAdika fi-ymA kAnuW fi-yhi ya_htalifu-wna. ihdini-y limA a_htulifa
fi-yhi mina al-.haqqi bi-'i_dnika 'innaka tahdi-y man ta^sA'u 'ilY
.sirA.tiN mustaqi-ymiN.\\
\end{arabtext}
\noindent
\textbf{Artinya}:
\par
\indent
"Ya Allah, Rabb Malaikat Jibril, Mika-il dan Israfil. Pencipta seluruh
langit dan bumi. Yang Maha Mengetahui semua yang ghaib dan yang nyata.
Engkau yang memutuskan hukum di antara hamba-hamba-Mu tentang apa-apa yang
mereka perselisihkan. Dengan izin-Mu tunjukkanlah aku kepada kebenaran
(yaitu, tetapkan aku di atas kebenaran) dari apa yang mereka perselisihkan.
Sesungguhnya Engkau memberi petunjuk kepada siapa yang Engkau kehendaki ke
jalan yang lurus."\\\\
\par
\noindent
\textbf{Tingkatan Doa dan Sanad}: \textbf{Shahih}: HR. Muslim (no. 770
[200]), Abu Dawud (no. 767), Ibnu Majah (no. 1357). Nabi membaca doa
istiftah ini ketika shalat malam.\\
\textbf{Referensi}: Yazid bin Abdul Qadir Jawas. 2016. Kumpulan Do'a dari
Al-Quran dan As-Sunnah yang Shahih. Bogor: Pustaka Imam Asy-Syafi'i.
\end{document}