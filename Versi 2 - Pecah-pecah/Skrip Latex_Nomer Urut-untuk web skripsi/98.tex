\documentclass[a4paper,12pt]{article}
\usepackage{arabtex}
\usepackage[bahasa] {babel}
\usepackage[top=2cm,left=3cm,right=3cm,bottom=3cm]{geometry}
\title{\Large Doa Tasyahud}
\author{\small Hanifah Atiya Budianto\\
		\small contact.us@latex-dailyprayers.com}
\begin{document}
\sffamily
\maketitle
\fullvocalize
\setcode{arabtex}
\begin{arabtext}
\noindent
al-tta.hiyyAtu al-mubArakAtu al-.s.salawAtu al-.t.tayyibAtu li-ll_ahi,
al-ssalAmu `alayka 'ayyuhA alnnabiyyu wara.hmaTu al-ll_ahi wabarakAtuhu,
al-ssalAmu `alaynA wa`alY `ibAdi al-ll_ahi al-.s.sA li.hi-yna, 'a^shadu
'an lA 'il_aha 'illA al-ll_ahu, wa'a^shadu 'anna mu.hammadaN rasuwlu
al-ll_ahi.\\
\end{arabtext}
\noindent
\textbf{Artinya}:
\par
\indent
"Semua kesejahteraan, kerajaan, dan kekekalan segala yang diberkahi; semua
doa untuk mengagungkan Allah; dan seluruh perkataan yang baik dan amal
shalih hanyalah milik Allah. Semoga kesejahteraan, rahmat, dan karunia
Allah tercurah untukmu wahai Nabi wahai Nabi. Semoga kesejahteraan
diberikan kepada kami dan hamba-hamba Allah yang shalih. Aku bersaksi bahwa
tidak ada ilah yang berhak diibadahi dengan benar selain Allah dan aku
bersaksi bahwasanya Muhammad adalah utusan Allah."\\\\
\par
\noindent
\textbf{Tingkatan Doa dan Sanad}: \textbf{Shahih}: HR. Muslim (no. 403
[60]), Abu Awanah (II/228), dari Abdullah bin Abbas; bahwasanya ia berkata:
"Rasulullah SAW. mengajarkan kepada kami tasyahud sebagaimana mengajarkan
surah dari al-Qur-an."\\
\textbf{Referensi}: Yazid bin Abdul Qadir Jawas. 2016. Kumpulan Do'a dari
Al-Quran dan As-Sunnah yang Shahih. Bogor: Pustaka Imam Asy-Syafi'i.
\end{document}