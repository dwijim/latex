\documentclass[a4paper,12pt]{article}
\usepackage{arabtex}
\usepackage[bahasa] {babel}
\usepackage[top=2cm,left=3cm,right=3cm,bottom=3cm]{geometry}
\title{\Large Doa Apabila Angin Bertiup Kencang}
\author{\small Hanifah Atiya Budianto\\
		\small contact.us@latex-dailyprayers.com}
\begin{document}
\sffamily
\maketitle
\fullvocalize
\setcode{arabtex}
\begin{arabtext}
\noindent
al-ll_ahumma 'inni-y 'as'aluka _ha-yrahA, wa_ha-yra mA fi-yhA, wa_ha-yra
mA 'ursilat bihi, wa'a`u-w_du bika min ^sarrihA, wa^sarri mA fi-yhA,
wa^sarri mA 'ursilat bihi.\\
\end{arabtext}
\noindent
\textbf{Artinya}:
\par
\indent
"Ya Allah, sungguh kepada-Mu aku memohon kebaikan angin ini, kebaikan apa-
apa yang ada padanya dan kebaikan tujuan angin ini dihembuskan. Aku
berlindung kepada-Mu dari kejelekan angin ini, kejelekan apa-apa yang ada
padanya dan kejelekan tujuan angin ini dihembuskan."\\\\
\par
\noindent
\textbf{Tingkatan Doa dan Sanad}: \textbf{Shahih}: HR. Muslim (no. 899
[15]) dan at-Tirmidzi (no. 3449) dari Aisyah r.a.\\
\textbf{Referensi}: Yazid bin Abdul Qadir Jawas. 2016. Kumpulan Do'a dari
Al-Quran dan As-Sunnah yang Shahih. Bogor: Pustaka Imam Asy-Syafi'i.
\end{document}