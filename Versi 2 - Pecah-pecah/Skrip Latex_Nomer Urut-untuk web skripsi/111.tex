\documentclass[a4paper,12pt]{article}
\usepackage{arabtex}
\usepackage[bahasa] {babel}
\usepackage[top=2cm,left=3cm,right=3cm,bottom=3cm]{geometry}
\title{\Large Bacaan Setelah Salam}
\author{\small Hanifah Atiya Budianto\\
		\small contact.us@latex-dailyprayers.com}
\begin{document}
\sffamily
\maketitle
\fullvocalize
\setcode{arabtex}
\begin{arabtext}
\noindent
lA 'il_aha 'illA al-ll_ahu wa.hdahu lA ^sari-yka lahu, lahu al-mulku,
walahu al-.hamdu, yu.hyi-y wayumi-ytu, wahuwa `alY kulli ^sa-y'iN
qadi-yruN.\\
\end{arabtext}
\noindent
\textbf{Artinya}:
\par
\indent
"Tidak ada ilah yang berhak diibadahi dengan benar melainkan hanya Allah
Yang Maha Esa, tiada sekutu bagi-Nya, bagi-Nya kerajaan, dan bagi-Nya
segala pujian. Dialah yang menghidupkan (orang yang sudah mati atau memberi
ruh janin yang akan dilahirkan) dan yang mematikan. Dialah Yang Mahakuasa
atas segala sesuatu." \textbf{[Dibaca 10x setiap setelah shalat Maghrib dan
Shubuh]}\\\\
Baca juga bacaan setelah salam ke 1-8\\
\par
\noindent
\textbf{Tingkatan Doa dan Sanad}: \textbf{Shahih}: HR. Ahmad (IV/227) dan
at-Tirmidzi (no. 3474). At-Tirmidzi berkata: "\textit{Hasan gharib shahih.}"
Diriwayatkan juga oleh Ahmad (V/420). Lihat \textit{Shah\^{i}h Targh\^{i}b
wat Tarh\^{i}b} (I/322-323, no. 474, 475, 477), \textit{Z\^{a}dul Ma'\^{a}d}
(I/300-301) dan \textit{Silsilah Ah\^{a}d\^{i}ts ash-Shah\^{i}hah} (no. 113,
114, 2563).\\
\textbf{Referensi}: Yazid bin Abdul Qadir Jawas. 2016. Kumpulan Do'a dari
Al-Quran dan As-Sunnah yang Shahih. Bogor: Pustaka Imam Asy-Syafi'i.
\end{document}