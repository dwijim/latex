\documentclass[a4paper,12pt]{article}
\usepackage{arabtex}
\usepackage[bahasa] {babel}
\usepackage[top=2cm,left=3cm,right=3cm,bottom=3cm]{geometry}
\title{\Large Bacaan bagi Orang yang Ragu dalam Beriman}
\author{\small Hanifah Atiya Budianto\\
		\small contact.us@latex-dailyprayers.com}
\begin{document}
\sffamily
\maketitle
\fullvocalize
\setcode{arabtex}
\noindent
Selain itu, hendaklah dia membaca firman-Nya:
\begin{arabtext}
\noindent
huwa al-'awwalu wa-al-'a_hiru wAl-.z.z_ahiru wa-al-bA.tinu wahuwa bikulli
^saY'iN `aliymuN.\\
\end{arabtext}
\noindent
\textbf{Artinya}:
\par
\indent
\textit{"Dialah yang awal (Allah telah ada sebelum segala sesuatu ada),
yang akhir (disaat segala sesuatu telah hancur, Allah masih tetap kekal),
yang zhahir (Dialah yang nyata, sebab banyak bukti yang menyatakan adanya
Allah), yang bathin (tidak ada sesuatu yang bisa menghalangi-Nya. Allah
lebih dekat kepada hamba-Nya daripada mereka kepada dirinya). Dialah Yang
Maha Mengetahui atas segala sesuatu."} (QS. Al-Had\^{i}d [57]: 3).\\\\
Baca bacaan bagi orang yang ragu dalam beriman ke 1 juga.\\\\
\par
\noindent
\textbf{Tingkatan Doa dan Sanad}: \textbf{Atsar Hasan}: Diriwayatkan oleh
Abu Dawud (no. 5110), pada Bab "F\^{i} Raddil Waswasah" dari perkataan Ibnu
Abbas r.a. Lihat \textit{Shah\^{i}h Abi Dawud} (III/962).\\
\textbf{Referensi}: Yazid bin Abdul Qadir Jawas. 2016. Kumpulan Do'a dari
Al-Quran dan As-Sunnah yang Shahih. Bogor: Pustaka Imam Asy-Syafi'i.
\end{document}