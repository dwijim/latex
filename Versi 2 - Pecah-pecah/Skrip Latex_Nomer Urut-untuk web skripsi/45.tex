\documentclass[a4paper,12pt]{article}
\usepackage{arabtex}
\usepackage[bahasa] {babel}
\usepackage[top=2cm,left=3cm,right=3cm,bottom=3cm]{geometry}
\title{\Large Doa dan Dzikir sebelum Tidur}
\author{\small Hanifah Atiya Budianto\\
		\small contact.us@latex-dailyprayers.com}
\begin{document}
\sffamily
\maketitle
\fullvocalize
\setcode{arabtex}
\noindent
- Membaca 2 ayat terakhir dari surah al-Baqarah:\\
\begin{arabtext}
\noindent
-'a-amana al-rrasuwlu bima-^A 'unzila 'ila-yhi min rrabbihi,
wa-al-mu'-minuwna kulluN -'a-amana bi-al-llahi wamal_a-^A'ikatihi,
wakutubihi, warusulihi, lA nufarriqu ba-yna 'a.hadiN mmin rrusulihi,
waqAluW sami`nA wa'a.ta`nA, .gufrAnaka rabbanA wa-'ila-yka al-ma.siyru
(285). lA yukallifu al-llahu nafsaN 'illA wus`ahA lahA mA kasabat
wa`ala-yhA mA aktasabat, rabbanA lA tu'A _hi_dn^A 'in nnasi-yn^A
'aw'a_h.ta'nA, rabbanA walA ta.hmil `ala-yn^A 'i.sraN kamA .hamaltahu,
`ala alla_diyna min qablinA, rabbanA walA tu.hammilnA mA lA .tAqaTalanA
bihi, wa-a`fu `annA wa-a.gfirlanA wa-ar.hamn^A, 'anta ma-wl_anA
fa-an.surnA `alY al-qa-wmi al-k_afiriyna. (286)\\
\end{arabtext}
\noindent
\textbf{Artinya}:\\
\indent
\textit{"Rasul (Muhammad) telah beriman kepada apa (Al-Qur-an) yang
diturunkan kepadanya dari Rabbnya, demikian pula orang-orang yang beriman.
Semua beriman kepada Allah, Malaikat-Malaikat-Nya, Kitab-Kitab-Nya dan
Rasul-Rasul-Nya. (Mereka berkata): 'Kami tidak membeda-bedakan seorang
pun dari Rasul-Rasul-Nya,' dan mereka berkata: 'Kami dengar dan kami taat.'
(Mereka berdoa): 'Ampunilah kami ya Rabb kami dan kepada Engkaulah tempat
kami kembali.'  Allah tidak membebani seseorang melainkan sesuai dengan
kesanggupannya. Ia mendapat pahala (dari kebajikan) yang diusahakan dan
mendapat siksa (dari kejahatan) yang dikerjakannya. (Mereka berdoa): 'Ya
Rabb kami, janganlah Engkau hukum kami jika kami lupa atau kami melakukan
kesalahan. Ya Rabb kami, janganlah Engkau bebankan kepada kami beban yang
berat sebagaimana Engkau bebankan kepada orang-orang sebelum kami. Ya Rabb
kami, janganlah Engkau pikulkan kepada kami apa yang tidak sanggup kami
memikulnya. Maafkanlah kami; ampunilah kami; dan rahmatilah kami. Engkaulah
Pelindung kami, maka tolonglah kami menghadapi orang-orang kafir.'"}
(QS. Al-Baqarah [2]: 285-286).\\\\
Baca juga doa dan dzikir sebelum tidur ke 1-5\\
\par
\noindent
\textbf{Tingkatan Doa dan Sanad}: "Siapa membaca dua ayat tersebut pada
malam hari, maka keduanya telah mencukupinya." \textbf{Shahih}: HR.
Al-Bukhari (no. 5051) dan Muslim (no. 807, 808). \textit{Fathul B\^{a}ri}
(IX/94).\\\\
\textbf{Perhatian}: Mohon baca menu Bantuan, terdapat perbedaan penulisan 
doa pada Latex dan buku.\\
\textbf{Referensi}: Yazid bin Abdul Qadir Jawas. 2016. Kumpulan Do'a dari
Al-Quran dan As-Sunnah yang Shahih. Bogor: Pustaka Imam Asy-Syafi'i.
\end{document}