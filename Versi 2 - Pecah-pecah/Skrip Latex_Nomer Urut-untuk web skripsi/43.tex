\documentclass[a4paper,12pt]{article}
\usepackage{arabtex}
\usepackage[bahasa] {babel}
\usepackage[top=2cm,left=3cm,right=3cm,bottom=3cm]{geometry}
\title{\Large Doa dan Dzikir sebelum Tidur}
\author{\small Hanifah Atiya Budianto\\
		\small contact.us@latex-dailyprayers.com}
\begin{document}
\sffamily
\maketitle
\fullvocalize
\setcode{arabtex}
\begin{arabtext}
\noindent
^gama`a kaffayhi _tumma nafa_ta fi-yhimA faqara'a fi-yhimA: (qul huwa
al-llahu 'a.haduN) (qul 'a`uw_du birabbi al-falaqi) (qul 'a`uw_du birabbi
al-nnAsi) _tumma yamsa.hu bihimA mA asta.tA`a min ^gasadihi yabda'u bihimA
`alY ra'sihi wawa^ghihi wamA 'aqbala min ^gasadihi.
\end{arabtext}
\noindent
\textbf{Artinya}:\\
\indent
"Rasulullah \textit{Shallallahu ‘alaihi wa sallam} merapatkan dua telapak
tangan lantas ditiup dan dibacakan:
\textit{Qul huwall\^{a}hu ahad} (surah Al-Ikhl\^{a}s), \textit{Qul
a'\^{u}dzu bi Rabbil falaq} (surah Al-Falaq), dan \textit{Qul a'\^{u}dzu bi
Rabbin n\^{a}s} (surah An-N\^{a}s), kemudian mengusap tubuh yang dapat
dijangkau dengan dua telapak tangan yang dimulai dari kepala, wajah, hingga
tubuh bagian depan sebanyak 3x."\\\\
Baca juga doa dan dzikir sebelum tidur ke 1-5\\
\par
\noindent
\textbf{Tingkatan Doa dan Sanad} : \textbf{Shahih}: HR. Al-Bukhari (no.
5017), Abu Dawud (no. 5056), an-Nasai dalam \textit{'Amalul Yaum wal Lailah}
(no. 793), at-Tirmidzi (no. 3402), dan Ahmad (VI/116). Lihat
\textit{Silsilah Ah\^{a}d\^{i}ts ash-Shah\^{i}hah} (no. 3104).\\
\textbf{Referensi}: Yazid bin Abdul Qadir Jawas. 2016. Kumpulan Do'a dari
Al-Quran dan As-Sunnah yang Shahih. Bogor: Pustaka Imam Asy-Syafi'i.
\end{document}