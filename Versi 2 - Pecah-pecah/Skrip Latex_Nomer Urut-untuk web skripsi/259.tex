\documentclass[a4paper,12pt]{article}
\usepackage{arabtex}
\usepackage[bahasa] {babel}
\usepackage[top=2cm,left=3cm,right=3cm,bottom=3cm]{geometry}
\title{\Large Doa pada Hari Arafah}
\author{\small Hanifah Atiya Budianto\\
		\small contact.us@latex-dailyprayers.com}
\begin{document}
\sffamily
\maketitle
\fullvocalize
\setcode{arabtex}
\par
\indent
Rasulullah SAW. bersabda: "Doa terbaik (yang mustajab) adalah pada hari
Arafah, dan sebaik-baik apa yang aku dan para Nabi baca adalah:\\
\begin{arabtext}
\noindent
lA 'il_aha 'illA al-ll_ahu wa.hdahu lA ^sari-yka lahu, lahu al-mulku,
walahu al-.hamdu wahuwa `alY kulli ^saY'iN qadi-yruN.\\
\end{arabtext}
\noindent
\textbf{Artinya}:
\par
\indent
'Tidak ada ilah yang berhak diibadahi dengan benar melainkan Allah Yang
Maha Esa, tidak ada sekutu bagi-Nya. Bagi-Nya kerajaan dan pujian. Dialah
Yang Mahakuasa atas segala sesuatu'".\\\\
\par
\noindent
\textbf{Tingkatan Doa dan Sanad}: \textbf{Hasan}: HR. At-Tirmidzi (no.
3585); \textit{Shah\^{i}h at-Tirmidzi} (III/184). Lihat \textit{Silsilah
Ah\^{a}d\^{i}hts ash-Shah\^{i}hah} (IV/6, no. 1503).\\
\textbf{Referensi}: Yazid bin Abdul Qadir Jawas. 2016. Kumpulan Do'a dari
Al-Quran dan As-Sunnah yang Shahih. Bogor: Pustaka Imam Asy-Syafi'i.
\end{document}