\documentclass[a4paper,12pt]{article}
\usepackage{arabtex}
\usepackage[bahasa] {babel}
\usepackage[top=2cm,left=3cm,right=3cm,bottom=3cm]{geometry}
\title{\Large Doa Berlindung dari Teman dan Tetangga yang Jahat}
\author{\small Hanifah Atiya Budianto\\
		\small contact.us@latex-dailyprayers.com}
\begin{document}
\sffamily
\maketitle
\fullvocalize
\setcode{arabtex}
\begin{arabtext}
\noindent
al-ll_ahumma 'inni-y 'a`u-w_du bika min yawmi al-ssu-w'i, wamin la-ylaTi
al-ssu-w'i, wamin sA`aTi al-ssu-w'i, wamin .sA.hibi al-ssu-w'i,
wamin ^gAri al-ssu-w'i fi-y dAri al-muqAmaTi.\\
\end{arabtext}
\noindent
\textbf{Artinya}:
\par
\indent
"Ya Allah, sesungguhnya aku berlindung kepada-Mu dari hari yang buruk,
malam yang buruk, saat yang buruk, teman yang jahat, dan tetangga yang
jahat di tempat tinggal tetapku."\\\\
\par
\noindent
\textbf{Tingkatan Doa dan Sanad}: \textbf{Hasan}: HR. Ath-Thabrani, dalam
\textit{al-Mu'jamul Kab\^{i}r} (XVII/294, no. 810). Imam al-Haitsami
berkata dalam kitabnya \textit{Majma'uz Zaw\^{a}-id} (X/144):
"\textit{Rijal} (perawi) hadits ini shahih." Lihat juga \textit{Silsilah
Ah\^{a}d\^{i}ts ash-Shah\^{i}hah} (no. 1443).\\
\textbf{Referensi}: Yazid bin Abdul Qadir Jawas. 2016. Kumpulan Do'a dari
Al-Quran dan As-Sunnah yang Shahih. Bogor: Pustaka Imam Asy-Syafi'i.
\end{document}