\documentclass[a4paper,12pt]{article}
\usepackage{arabtex}
\usepackage[bahasa] {babel}
\usepackage[top=2cm,left=3cm,right=3cm,bottom=3cm]{geometry}
\title{\Large Doa Masuk WC}
\author{\small Hanifah Atiya Budianto\\
		\small contact.us@latex-dailyprayers.com}
\begin{document}
\sffamily
\maketitle
\fullvocalize
\setcode{arabtex}
\begin{arabtext}
\noindent
( bismi al-ll_ahi ) al-ll_ahumma 'inni-y 'a`u-w_du bika mina al-_hubu_ti
wAl-_habA'i_ti.\\
\end{arabtext}
\noindent
\textbf{Artinya}:
\par
\indent
"Dengan nama Allah. Ya Allah, sungguh aku berlindung kepada-Mu dari godaan
syaitan yang laki-laki dan syaitan yang perempuan."\\\\
\par
\noindent
\textbf{Tingkatan Doa dan Sanad}: \textbf{Shahih}: HR. Al-Bukhari (no. 142)
dan Muslim (no. 375), juga at-Tirmidzi (no. 606), Ibnu Majah (297, 298).
Adapun tambahan \textit{Bismill\^{a}h} pada awal hadits, lihat
\textit{Fathul B\^{a}ri} (I/244). Dishahihkan oleh Syaikh al-Albani dalam
\textit{Irw\^{a}-ul Ghal\^{i}l} (no.50). \\
\textbf{Referensi}: Yazid bin Abdul Qadir Jawas. 2016. Kumpulan Do'a dari
Al-Quran dan As-Sunnah yang Shahih. Bogor: Pustaka Imam Asy-Syafi'i.
\end{document}