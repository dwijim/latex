\documentclass[a4paper,12pt]{article}
\usepackage{arabtex}
\usepackage[bahasa] {babel}
\usepackage[top=2cm,left=3cm,right=3cm,bottom=3cm]{geometry}
\title{\Large Doa Kaffaratul Majelis}
\author{\small Hanifah Atiya Budianto\\
		\small contact.us@latex-dailyprayers.com}
\begin{document}
\sffamily
\maketitle
\fullvocalize
\setcode{arabtex}
\begin{arabtext}
\noindent
al-ll_ahumma aqsim lanA min _ha^syatika mA ta.hu-wlu bihi baynanA wabayna
ma`A .si-yka, wamin .tA`atika mA tuballi.gunA bihi ^gannataka, wamina
al-yaqi-yni mA tuhawwinu bihi `alaynA ma.sA'iba al-ddunyA, al-ll_ahumma
matti`nA bi-'asmA`inA, wa'ab.sArinA, waquwwatinA mA 'a.hyaytanA, wA^g`alhu
al-wAri_ta minnA, wA^g`al _ta'ranA `alY man .zalamanA, wAn.surnA `alY man
`AdAnA, walA ta^g`al mu.si-ybatanA fi-y di-yninA, walA ta^g`ali al-ddunyA
'akbara hamminA, walA mabla.ga `ilminA, walA tusalli.t `alaynA man lA
yar.hamunA.\\
\end{arabtext}
\noindent
\textbf{Artinya}:
\par
\indent
"Ya Allah, anugerahkanlah untuk kami rasa takut kepada-Mu, yang menghalangi
antara kami dengan perbuatan maksiat kepada-Mu, dan (anugerahkanlah kepada
kami) ketaatan kepada-Mu yang akan menyampaikan kami ke Surga-Mu, dan
(anugerahkanlah pula) keyakinan yang dapat menyebabkan ringannya bagi kami
segala musibah di dunia ini. Ya Allah, anugerahkanlah kenikmatan kepada
kami melalui pendengaran kami, pengelihatan kami dan dalam kekuatan kami
selama kami masih hidup, serta jadikanlah ia sebagai warisan dari kami. Dan
jadikan ia balasan kami atas orang-orang yang menganiaya kami, dan
tolonglah kami terhadap orang yang memusuhi kami, serta janganlah Engkau
jadikan musibah ada dalam urusan agama kami, dan janganlah Engkau jadikan
dunia ini sebagai cita-cita terbesar dan puncak dari ilmu kami, dan jangan
Engkau jadikan orang-orang yang tidak mengasihi kami berkuasa atas
kami."\\\\
\par
\noindent
\textbf{Tingkatan Doa dan Sanad}: \textbf{Shahih}: HR. At-Tirmidzi (no.
3502) al-Hakim (I/528) dan Ibnus Sunni dalam \textit{Amalul Yaum wal
Lailah} (no. 446) dan an-Nasai dalam \textit{Amalul Yaum wal Lailah} (no.
4040, dari Abdullah bin Umar r.a. Hadits ini dishahihkan oleh al-Hakim dan
disepakati oleh adz-Dzahabi. Abdullah bin Umar r.a. berkata: "Rasulullah
Shallallahu ‘alaihi wa sallam seringkali mengucapkan doa ini bagi
Sahabat-Sahabat beliau sebelum bangkit dari majelis." Lihat
\textit{Shah\^{i}h at-Tirmidzi} (III/168, no. 2783) dan \textit{Shah\^{i}hul
J\^{a}mi'} (no. 1268), \textit{Shah\^{i}h al-Kalimith Thayyib} (no. 226).\\
\textbf{Keterangan}: \textit{Kaff\^{a}ratul majelis} artinya penghapus
dosa akibat apa saja yang terjadi di majelis. Dibaca setelah selesai dari
majelis dzikir, majelis ilmu, dan yang semisalnya. \\
\textbf{Referensi}: Yazid bin Abdul Qadir Jawas. 2016. Kumpulan Do'a dari
Al-Quran dan As-Sunnah yang Shahih. Bogor: Pustaka Imam Asy-Syafi'i.
\end{document}