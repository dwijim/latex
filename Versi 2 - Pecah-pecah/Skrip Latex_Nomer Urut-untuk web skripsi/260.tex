\documentclass[a4paper,12pt]{article}
\usepackage{arabtex}
\usepackage[bahasa] {babel}
\usepackage[top=2cm,left=3cm,right=3cm,bottom=3cm]{geometry}
\title{\Large Bacaan di Masy'aril Haram}
\author{\small Hanifah Atiya Budianto\\
		\small contact.us@latex-dailyprayers.com}
\begin{document}
\sffamily
\maketitle
\fullvocalize
\setcode{arabtex}
\begin{arabtext}
\noindent
... rakiba .sallY al-ll_ahu `alayhi wasallama al-qa.swA'a .hattY 'atY
al-ma^s`ara al-.harAma fAstaqbala al-qiblaTa fada`Ahu wakabbarahu
wahallalahu wawa.h.hadahu falam yazal wAqifaN .hattY 'asfara ^giddaN
fadafa`a qabla 'an ta.tlu`a al-^s^samsu ....\\
\end{arabtext}
\noindent
\textbf{Artinya}:
\par
\indent
"Nabi SAW. naik unta beliau yang bernama al-Qashwa hingga di Masy'aril
Haram, lalu beliau menghadap Kiblat, berdoa, membaca takbir
(\textit{All\^{a}hu Akbar}) dan tahlil (\textit{L\^{a} Il\^{a}ha
Illall\^{a}h}) serta kalimat tauhid. Beliau pun terus berdoa hingga fajar
menyingsing. Lantas beliau berangkat (ke Mina) sebelum Matahari terbit."
\\\\
\par
\noindent
\textbf{Tingkatan Doa dan Sanad}: \textbf{Shahih}: HR. Muslim (no. 1218).\\
\textbf{Referensi}: Yazid bin Abdul Qadir Jawas. 2016. Kumpulan Do'a dari
Al-Quran dan As-Sunnah yang Shahih. Bogor: Pustaka Imam Asy-Syafi'i.
\end{document}