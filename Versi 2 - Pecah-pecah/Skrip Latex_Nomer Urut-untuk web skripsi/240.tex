\documentclass[a4paper,12pt]{article}
\usepackage{arabtex}
\usepackage[bahasa] {babel}
\usepackage[top=2cm,left=3cm,right=3cm,bottom=3cm]{geometry}
\title{\Large Doa Menghadapi Musuh dan Orang yang Berkuasa}
\author{\small Hanifah Atiya Budianto\\
		\small contact.us@latex-dailyprayers.com}
\begin{document}
\sffamily
\maketitle
\fullvocalize
\setcode{arabtex}
\begin{arabtext}
\noindent
( .hasbunA al-llahu wani`ma al-waki-ylu )\\
\end{arabtext}
\noindent
\textbf{Artinya}:
\par
\indent
\textit{"Cukuplah Allah menjadi Penolong kami. Dan Dia adalah sebaik-baik
Pelindung."} (QS. Ali 'Imran [3]: 173).\\\\
\par
\noindent
\textbf{Tingkatan Doa dan Sanad}: Kalimat ini diucapkan oleh Nabi Ibrahim
a.s. ketika dilemparkan ke dalam api, dan juga diucapkan Nabi Muhammad SAW.
ketika orang-orang berkata: "\textit{Sesungguhnya manusia telah
mengumpulkan pasukan untuk menyerangmu}." (QS. Ali 'Imran [3]: 173).
\textbf{Shahih}: HR. Al-Bukhari (no. 4563, 4564).\\
\textbf{Referensi}: Yazid bin Abdul Qadir Jawas. 2016. Kumpulan Do'a dari
Al-Quran dan As-Sunnah yang Shahih. Bogor: Pustaka Imam Asy-Syafi'i.
\end{document}