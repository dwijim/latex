\documentclass[a4paper,12pt]{article}
\usepackage{arabtex}
\usepackage[bahasa] {babel}
\usepackage[top=2cm,left=3cm,right=3cm,bottom=3cm]{geometry}
\title{\Large Doa Agar Diberi Ilmu Yang Bermanfaat dan Berlindung Dari Ilmu
Yang Tidak Bermanfaat}
\author{\small Hanifah Atiya Budianto\\
		\small contact.us@latex-dailyprayers.com}
\begin{document}
\sffamily
\maketitle
\fullvocalize
\setcode{arabtex}
\begin{arabtext}
\noindent
al-ll_ahumma 'inni-y 'as'aluka `ilmaN nAfi`aN, wa-'a`u-wbika min `ilmiN lA
yanfa`u.\\
\end{arabtext}
\noindent
\textbf{Artinya}:
\par
\indent
"Ya Allah, sesunguhnya aku memohon kepada-Mu ilmu yang bermanfaat, dan aku
berlindung kepada-Mu dari ilmu yang tidak bermanfaat."\\\\
\par
\noindent
\textbf{Tingkatan Doa dan Sanad}: \textbf{Hasan Shahih}: HR. Ibnu Majah
(no. 3843), an-Nasai dalam \textit{Sunanul Kubra} (no. 7818), Ibnu Hibban
(no. 82 - \textit{at-Ta'l\^{i}q\^{a}tul His\^{a}n}), Ibnu Abi Syaibah dalam
\textit{al-Mushannaf} (no. 27127, 29610). Dan ini adalah lafazh an-Nasai
dan Ibnu Hibban.\\
\textbf{Referensi}: Yazid bin Abdul Qadir Jawas. 2016. Kumpulan Do'a dari
Al-Quran dan As-Sunnah yang Shahih. Bogor: Pustaka Imam Asy-Syafi'i.
\end{document}