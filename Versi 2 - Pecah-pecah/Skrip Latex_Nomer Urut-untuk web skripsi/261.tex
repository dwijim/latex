\documentclass[a4paper,12pt]{article}
\usepackage{arabtex}
\usepackage[bahasa] {babel}
\usepackage[top=2cm,left=3cm,right=3cm,bottom=3cm]{geometry}
\title{\Large Bertakbir Setiap Melempar Jumrah}
\author{\small Hanifah Atiya Budianto\\
		\small contact.us@latex-dailyprayers.com}
\begin{document}
\sffamily
\maketitle
\fullvocalize
\setcode{arabtex}
\begin{arabtext}
\noindent
'inna rasu-wla al-ll_ahi .sallaY al-ll_ahu `ala-yhi wasallama kAna 'i_dA
ramY al-^gamraTa ... bisab`i .ha.sayAtiN, yukabbiru kullamA ramY
bi.ha.sATiN, _tumma taqaddama 'amAmahA fawaqafa mustaqbila al-qiblaTi,
rAfi`aN yadayhi yad`uw, wakAna yu.ti-ylu al-wuqu-wfa. _tumma ya'tiY
al-^gamraTa al-_t_tAniyaTa fayarmi-yhA bisab`i .ha.sayAtiN yukabbiru
kullamA ramY bi.ha.sATiN... fayaqifu mustaqbila al-qiblaTi rAfi`aN yadayhi
yad`uw. _tumma ya'tiy al-^gamraTa allatiy `inda al-`aqabaTi fayarmi-yhA
bisab`i .ha.sayAtiN, yukabbiru `inda kulli .ha.sATiN, _tumma yan.sarifu
walA yaqifu `indahA.\\
\end{arabtext}
\noindent
\textbf{Artinya}:
\par
\indent
"Sesungguhnya Rasulullah Shallallahu ‘alaihi wa sallam. melempar Jumratul
Ula (jumrah pertama di dekat Masjid Khaif) dengan tujuh batu kerikil dan
bertakbir setiap kali melemparnya. Kemudian beliau maju dan berdiri lama
menghadap kiblat, lantas berdoa sambil mengangkat kedua tangan. Selanjutnya
beliau melakukan hal yang sama pada Jumratus Tsaniyah (jumrah kedua), lalu
berdoa. Kemudian itu beliau melempar Jumratul Aqabah (jumrah ketiga) dengan
tujuh batu kerikil, bertakbir setiap kali melempar, lalu beliau langsung
pergi dari situ dan tidak diam padanya (yakni tidak berdoa)."\\\\
\par
\noindent
\textbf{Tingkatan Doa dan Sanad} : \textbf{Shahih}: HR. Al-Bukhari (no.
1753). Bab "Ad-Du'\^{a}' 'indal Jamrataini" - \textit{Fathul B\^{a}ri}
(III/584)-dan Muslim (no. 1218).\\
\textbf{Referensi}: Yazid bin Abdul Qadir Jawas. 2016. Kumpulan Do'a dari
Al-Quran dan As-Sunnah yang Shahih. Bogor: Pustaka Imam Asy-Syafi'i.
\end{document}