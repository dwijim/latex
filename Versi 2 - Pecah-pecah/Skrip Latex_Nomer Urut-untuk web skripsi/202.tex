\documentclass[a4paper,12pt]{article}
\usepackage{arabtex}
\usepackage[bahasa] {babel}
\usepackage[top=2cm,left=3cm,right=3cm,bottom=3cm]{geometry}
\title{\Large Doa Berlindung dari Teman dan Tetangga yang Jahat}
\author{\small Hanifah Atiya Budianto\\
		\small contact.us@latex-dailyprayers.com}
\begin{document}
\sffamily
\maketitle
\fullvocalize
\setcode{arabtex}
\begin{arabtext}
\noindent
al-ll_ahumma 'inni-y 'a`u-w_du bika min ^gAri al-ssu-w'i fi-y dAri
al-muqAmaTi, fa-'inna ^gAra al-bAdiyaTi yata.hawwalu.\\
\end{arabtext}
\noindent
\textbf{Artinya}:
\par
\indent
"Ya Allah, sesungguhnya aku berlindung kepada-Mu dari tetangga yang jahat
di tempat tinggal tetapku, karena sungguh tetangga orang-orang Badui (desa)
itu berpindah-pindah."\\\\
\par
\noindent
\textbf{Tingkatan Doa dan Sanad}: \textbf{Hasan}: HR. Al-Hakim (1/532) -
lalu dishahihkannya dan disepakati oleh azd-Dzahabi, An-Nasa-i (VIII/274),
dan al-Bukhari dalam \textit{al-Adabul Mufrad} (no. 117). Lihat
\textit{Shah\^{i}hul J\^{a}mi}' (no. 1290).\\
\textbf{Referensi}: Yazid bin Abdul Qadir Jawas. 2016. Kumpulan Do'a dari
Al-Quran dan As-Sunnah yang Shahih. Bogor: Pustaka Imam Asy-Syafi'i.
\end{document}