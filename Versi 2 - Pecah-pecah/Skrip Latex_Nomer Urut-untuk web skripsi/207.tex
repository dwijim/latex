\documentclass[a4paper,12pt]{article}
\usepackage{arabtex}
\usepackage[bahasa] {babel}
\usepackage[top=2cm,left=3cm,right=3cm,bottom=3cm]{geometry}
\title{\Large Doa Menghadapi Kesulitan}
\author{\small Hanifah Atiya Budianto\\
		\small contact.us@latex-dailyprayers.com}
\begin{document}
\sffamily
\maketitle
\fullvocalize
\setcode{arabtex}
\begin{arabtext}
\noindent
lA-'il_aha 'illA 'anta, sub.hAnaka, 'inni-y kuntu mina al-.z.zAlimi-yna.\\
\end{arabtext}
\noindent
\textbf{Artinya}:
\par
\indent
"Tidak ada ilah yang berhak diibadahi dengan benar melainkan hanya Engkau.
Mahasuci Engkau, sesungguhnya aku termasuk orang-orang yang zhalim".\\\\
\par
\noindent
\textbf{Tingkatan Doa dan Sanad}: \textbf{Shahih}: HR. At-Tirmidzi (no.
3505) dan al-Hakim (I/505) dan lainnya, dishahihkan oleh al-Hakim dan
disepakati oleh adz-Dzahabi. Lihat \textit{Shah\^{i}h al-J\^{a}mi-us
Shagh\^{i}r} (no. 3383) dengan lafazh (yang artinya): "Doa Dzun Nun (Nabi
Yunus), ketika dia berdoa di dalam perut ikan paus adalah: 'Tidak ada ilah
yang berhak diibadahi dengan benar melainkan hanya Engkau. Mahasuci Engkau,
sesungguhnya aku termasuk orang-orang yang zhalim.' Sesungguhnya tidak ada
seorang Muslim pun yang memanjatkan doa dengan kalimat tersebut dalam suatu
hal apa pun, melainkan Allah akan mengabulkan untuknya".\\
\textbf{Referensi}: Yazid bin Abdul Qadir Jawas. 2016. Kumpulan Do'a dari
Al-Quran dan As-Sunnah yang Shahih. Bogor: Pustaka Imam Asy-Syafi'i.
\end{document}