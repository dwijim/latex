\documentclass[a4paper,12pt]{article}
\usepackage{arabtex}
\usepackage[bahasa] {babel}
\usepackage[top=2cm,left=3cm,right=3cm,bottom=3cm]{geometry}
\title{\Large Doa Shalat Istikharah}
\author{\small Hanifah Atiya Budianto\\
		\small contact.us@latex-dailyprayers.com}
\begin{document}
\sffamily
\maketitle
\fullvocalize
\setcode{arabtex}
\par
\indent
Jabir bin Abdillah menuturkan: "Rasulullah shallallahu ‘alaihi wa sallam
mengajari kami shalat Istikharah untuk memutuskan segala sesuatu sebagaimana
mengajari  surah Al-Qur-an." Beliau pun bersabda: "Apabila seseorang di
antara kalian mempunyai satu rencana untuk mengerjakan sesuatu, hendaknya
ia melakukan shalat sunnah (Istikharah) dua rakaat, kemudian bacalah doa
ini:\\
\begin{arabtext}
\noindent
al-ll_ahumma 'inniy 'asta_hi-yruka bi`ilmika, wa-'astaqdiruka biqudratika,
wa-'as'aluka min fa.dlika al`a.ziymi, fa-'innaka taqdiru walA 'aqdiru,
wata`lamu walA 'a`lamu, wa'anta `allAmu al.guyu-wbi. al-ll_ahumma 'in kunta
ta`lamu 'anna h_a_dA al-'amra (wayusammiy .hA^gatahu) _hayruN li-y fi-y
di-yni-y, wama`A^si-y, wa`AqibaTi 'amri-y ('aw qala : `A^gili 'amri-y
wa-^A^gilihi) fAqdurhu li-y wayassirhu li-y, _tumma bArik li-y fi-yhi,
wa-'in kunta ta`lamu 'anna h_a_dA al-'amra ^sarruN li-y fi-y di-yni-y,
wama`A^si-y, wa-`AqibaTi 'amri-y ('awqAla : `A^gili 'amri-y wa-^A^gilihi)
fA.srifhu `anni-y wA.srifni-y `anhu, wAqdurliya al-_hayra .hay_tu kAna,
_tumma 'ar.dini-y bihi.\\
\end{arabtext}
\noindent
\textbf{Artinya}:
\par
\indent
'Ya Allah, sesungguhnya aku meminta pilihan yang tepat kepada-Mu dengan
ilmu-Mu, dan aku memohon kekuatan kepada-Mu (untuk mengatasi masalahku)
dengan kemahakuasaan-Mu. Aku mohon kepada-Mu sesuatu dari anugerah-Mu Yang
Mahaagung, sesungguhnya Engkau Mahakuasa, sedangkan aku tidak kuasa,
Engkau mengetahui, sedangkan aku tidak mengetahui dan Engkaulah yang Maha
Mengetahui perihal yang ghaib. Ya Allah, apabila Engkau mengetahui bahwa
urusan ini (hendaknya menyebutkan masalahnya) lebih baik dalam agamaku,
kehidupanku, dan akibatnya terhadap diriku, baik di dunia atau di akhirat,
maka takdirkanlah ia untukku, dan mudahkan jalannya, kemudian berilah
keberkahan. Akan tetapi apabila Engkau mengetahui bahwa urusan ini membawa
keburukan bagiku dalam agamaku, kehidupanku, dan akibatnya terhadap diriku,
baik di dunia atau di akhirat, maka singkirkan urusan tersebut, dan jauhkan
aku darinya, serta takdirkanlah bagiku kebaikan di mana saja kebaikan
berada, kemudian jadikanlah aku ridha dalam menerimanya."\\
\par
\indent
Tidak akan menyesal orang yang beristikharah kepada al-Khaliq (Allah Azza
wa Jalla) dan bermusyawarah dengan orang-orang Mukmin serta berhati-hati
menangani persoalannya. Allah SWT. berfirman:\\
\begin{arabtext}
\noindent
wa^sAwirhum fiY al'amri fa-'i_dA `azamta fatawakkal `alaY al-llahi ....
(109)\\
\end{arabtext}
\noindent
\textbf{Artinya}:
\par
\indent
\textit{"Dan bermusyawarahlah dengan mereka (para Sahabat) dalam urusan itu
(peperangan, perekonomian, politik, dan lain-lain). Bila kamu telah
membulatkan tekad, bertakwakallah kepada Allah ...." (QS. Ali 'Imran
[3]: 159)}\\\\
\par
\noindent
\textbf{Tingkatan Doa dan Sanad}: \textbf{Shahih}: HR. Al-Bukhari (no.
1162, 6382, 7390), Abu Dawud (no. 1538), at-Tirmidzi (no. 480), an-Nasai
(VI/80), dan Ibnu Majah (no. 1383).\\
\textbf{Referensi}: Yazid bin Abdul Qadir Jawas. 2016. Kumpulan Do'a dari
Al-Quran dan As-Sunnah yang Shahih. Bogor: Pustaka Imam Asy-Syafi'i.
\end{document}