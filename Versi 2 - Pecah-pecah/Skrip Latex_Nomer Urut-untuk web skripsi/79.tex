\documentclass[a4paper,12pt]{article}
\usepackage{arabtex}
\usepackage[bahasa] {babel}
\usepackage[top=2cm,left=3cm,right=3cm,bottom=3cm]{geometry}
\title{\Large Doa Istiftah}
\author{\small Hanifah Atiya Budianto\\
		\small contact.us@latex-dailyprayers.com}
\begin{document}
\sffamily
\maketitle
\fullvocalize
\setcode{arabtex}
\begin{arabtext}
\noindent
wa^g^gahtu wa^ghiya lilla_di-y fa.tara al-ssamAwAti wAl-'ar.da .hani-yfaN
wamA 'anA mina al-mu^sriki-yna, 'inna .salAti-y, wanusuki-y, wama.hyAya,
wamamAti-y li-ll_ahi rabbi al-`Alami-yna, lA^sari-yka lahu wabi_d_alika
'umirtu wa'anA mina al-muslimi-yna. al-ll_ahumma 'anta almaliku lA 'il_aha
'illA 'anta. 'anta rabbi-y wa'anA `abduka, .zalamtu nafsi-y wA`taraftu
bi_danbi-y fA.gfirli-y _dunu-wbi-y ^gami-y`aN 'innahu lA ya.gfiru
al-_d_dunu-wba 'illA 'anta. wAhdini-y li-'a.hsani al-'a_hlAqi lA yahdi-y
li-'a.hsanihA 'illA 'anta, wA.srif `anni-y sayyi'ahA, lA ya.srifu `anni-y
sayyi'ahA 'illA 'anta, labbayka wasa`dayka, wAl-_hayru kulluhu fi-y
yadayka, wAl-^s^sarru laysa 'ilayka, 'anAbika wa-'ilayka, tabArakta
wata`Alayta, 'asta.gfiruka wa'atu-wbu 'ilayka.\\
\end{arabtext}
\noindent
\textbf{Artinya}:
\par
\indent
"Aku menghadapkan wajahku kepada Rabb Pencipta langit dan bumi, dalam
keadaan lurus dan aku tidak termasuk orang-orang yang musyrik. Sesungguhnya
shalatku, ibadahku, hidupku serta matiku adalah untuk Allah. Rabb alam
semesta, tidak ada sekutu bagi-Nya. Demikianlah aku diperintah dan bahwa
aku termasuk orang Muslim. Ya Allah, Engkau adalah Raja, tidak ada ilah
yang berhak diibadahi dengan benar kecuali Engkau, Engkau Rabbku sedangkan
aku ini adalah hamba-Mu. Aku menganiaya diriku, aku mengakui dosa-dosaku
(yang pernah aku lakukan). Oleh karena itu, ampunilah seluruh dosaku,
sesungguhnya tidak ada yang dapat mengampuni dosa-dosa, kecuali Engkau.
Tunjukkan aku pada akhlak yang baik (mulia), tidak ada yang dapat
menunjukkan kepada akhlak yang mulia kecuali Engkau. Hindarkan aku dari
akhlak yang buruk, tidak ada yang dapat menjauhkanku darinya kecuali
Engkau. Aku penuhi panggilan-Mu, aku mohon pertolongan-Mu, seluruh kebaikan
berada di kedua tangan-Mu, kejelekan tidak dinisbatkan kepada-Mu. Aku hidup
dengan pertolongan dan rahmat-Mu, dan kepada-Mu (aku kembali). Mahasuci
Engkau dan Mahatinggi. Aku memohon ampunan dan bertaubat kepada-Mu."\\\\
\par
\noindent
\textbf{Tingkatan Doa dan Sanad}: \textbf{Shahih}: HR. Muslim (no. 771
[201]), Abu Dawud (no. 760), an-Nasai (II/130), Ahmad (I/94-95, 102), dan
selainnya. Doa ini dibaca saat shalat wajib dan saat shalat sunnah
(\textit{Shifatu Shal\^{a}tin Nabi karya Syaikh al-Albani}).\\
\textbf{Referensi}: Yazid bin Abdul Qadir Jawas. 2016. Kumpulan Do'a dari
Al-Quran dan As-Sunnah yang Shahih. Bogor: Pustaka Imam Asy-Syafi'i.
\end{document}