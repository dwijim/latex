\documentclass[a4paper,12pt]{article}
\usepackage{arabtex}
\usepackage[bahasa] {babel}
\usepackage[top=2cm,left=3cm,right=3cm,bottom=3cm]{geometry}
\title{\Large Memohon Akhlak yang Baik}
\author{\small Hanifah Atiya Budianto\\
		\small contact.us@latex-dailyprayers.com}
\begin{document}
\sffamily
\maketitle
\fullvocalize
\setcode{arabtex}
\begin{arabtext}
\noindent
al-ll_ahumma 'a.hsanta _halqi-y fa'a.hsin _huluqi-y.\\
\end{arabtext}
\noindent
\textbf{Artinya}:
\par
\indent
"Ya Allah, sebagaimana Engkau  telah menciptakanku dengan baik, maka
perbaiki pula akhlakku."\\\\
\par
\noindent
\textbf{Tingkatan Doa dan Sanad}: \textbf{Shahih}: HR. Ahmad (VI/68, 155;
I/403) dan dishahihkan oleh al-Albani dalam \textit{Irw\^{a}-ul Ghal\^{i}l}
(I/155, dibawah pembahasan hadits no. 74). Hadits doa ini bersifat
\textit{mutlaq}, tidak terikat (tidak harus diucapkan) di depan cermin.
Lihat kitab \textit{al-Kalmuth Thayyib} (hlm. 171).\\
\textbf{Referensi}: Yazid bin Abdul Qadir Jawas. 2016. Kumpulan Do'a dari
Al-Quran dan As-Sunnah yang Shahih. Bogor: Pustaka Imam Asy-Syafi'i.
\end{document}