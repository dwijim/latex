\documentclass[a4paper,12pt]{article}
\usepackage{arabtex}
\usepackage[bahasa] {babel}
\usepackage[top=2cm,left=3cm,right=3cm,bottom=3cm]{geometry}
\title{\Large Doa agar Diberi Kekuatan Iman dan Berbagai Kebaikan}
\author{\small Hanifah Atiya Budianto\\
		\small contact.us@latex-dailyprayers.com}
\begin{document}
\sffamily
\maketitle
\fullvocalize
\setcode{arabtex}
\begin{arabtext}
\noindent
al-ll_ahumma 'inni-y 'as'aluka mina al-_ha-yri kullihi, `A^gilihi
wa-^A^gilihi, mA `alimtu minhu wamA lam 'a`lam, wa'a`u-w_du bika mina
al-^s^sarri kullihi, `A^gilihi wa-^A^gilihi, mA `alimtu minhu wamA lam
'a`lam. al-ll_ahuma 'inni-y 'as'aluka min _ha-yri mA sa'alaka `abduka
wanabiyyuka, wa'a`u-w_du bika min ^sarri mA `A_da bihi `abduka
wanabiyyuka. al-ll_ahumma 'inni-y 'as'aluka al-^gannaTa, wamA qarraba
'ila-yhA, min qawliN 'aw `amaliN, wa'a`u-w_du bika mina al-nnAri, wamA
qarraba 'ila-yhA, min qawliN 'a-w `amaliN, wa'as'aluka 'an ta^g`ala kulla
qa.dA'iN qa.da-ytahu li-y _hayraN.\\
\end{arabtext}
\noindent
\textbf{Artinya}:
\par
\indent
"Ya Allah, sesungguhnya aku  memohon kepada-Mu seluruh kebaikan, baik yang
sekarang maupun yang akan datang, yang aku ketahui maupun yang tidak aku
ketahui. Dan aku memohon perlindungan kepada-Mu dari seluruh kejahatan,
baik yang sekarang maupun yang akan datang, baik yang kuketahui maupun yang
tidak kuketahui. Ya Allah, sesungguhnya aku memohon kebaikan yang diminta
oleh hamba dan Nabi-Mu, dan aku pun berlindung kepada-Mu dari kejahatan
yang hamba dan Nabi-Mu berlindung kepada-Mu darinya. Ya Allah, sesungguhnya
aku memohon kepada-Mu Surga dan apa-apa yang dapat mendekatkan kepadanya
baik berupa ucapan maupun perbuatan. Dan aku juga berlindung kepada-Mu dari
Neraka dan apa-apa yang dapat mendekatkan kepadanya, baik berupa ucapan
maupun perbuatan. Dan aku memohon kepada-Mu supaya  Engkau menjadikan
seluruh ketetapan yang telah Engkau tetapkan bagiku merupakan suatu
kebaikan."\\\\
\par
\noindent
\textbf{Tingkatan Doa dan Sanad}: \textbf{Shahih}: HR. Ibnu Majah (no.
3846), Ibnu Hibban (no. 2413 - \textit{al-Maw\^{a}rid}), Ahmad (VI/134),
al-Hakim (I/521-522), dan lafazh hadits tersebut adalah milik Ibnu Majah.
Lihat kitab \textit{Shah\^{i}h Ibni Majah} (II/327, no. 3102) dan
\textit{Silsilah Ah\^{a}d\^{i}ts ash-Shah\^{i}hah} (no. 1542).\\
\textbf{Referensi}: Yazid bin Abdul Qadir Jawas. 2016. Kumpulan Do'a dari
Al-Quran dan As-Sunnah yang Shahih. Bogor: Pustaka Imam Asy-Syafi'i.
\end{document}