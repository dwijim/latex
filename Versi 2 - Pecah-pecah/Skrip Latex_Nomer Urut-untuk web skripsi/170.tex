\documentclass[a4paper,12pt]{article}
\usepackage{arabtex}
\usepackage[bahasa] {babel}
\usepackage[top=2cm,left=3cm,right=3cm,bottom=3cm]{geometry}
\title{\Large Doa Berlindung terhadap Berbagai Kesusahan, Kesengsaraan,
dan Hilangnya Kenikmatan}
\author{\small Hanifah Atiya Budianto\\
		\small contact.us@latex-dailyprayers.com}
\begin{document}
\sffamily
\maketitle
\fullvocalize
\setcode{arabtex}
\begin{arabtext}
\noindent
al-ll_ahumma 'inni-y 'a`u-w_du bika mina al-^gu-w`i, fa'i-nnahu bi'sa
al-.d.da^gi-y`u, wa'a`u-w_du bika mina al-_hiyAnaTi, fa'i-nnahA bi'sati
al-bi.tAnaTu.\\
\end{arabtext}
\noindent
\textbf{Artinya}:
\par
\indent
"Ya Allah, sesungguhnya aku berlindung kepada-Mu dari kelaparan, karena ia
adalah seburuk-buruk teman berbaring. Aku juga berlindung kepada-Mu dari
pengkhianatan, karena ia merupakan seburuk-buruk kawan."\\\\
\par
\noindent
\textbf{Tingkatan Doa dan Sanad}: \textbf{Shahih}: HR. Abu Dawud (no.
1547), an-Nasai (VIII/263), serta Ibnu Majah (no. 3354). Lihat
\textit{Shah\^{i}h an-Nasai} (III/1112, no. 5051).\\
\textbf{Referensi}: Yazid bin Abdul Qadir Jawas. 2016. Kumpulan Do'a dari
Al-Quran dan As-Sunnah yang Shahih. Bogor: Pustaka Imam Asy-Syafi'i.
\end{document}