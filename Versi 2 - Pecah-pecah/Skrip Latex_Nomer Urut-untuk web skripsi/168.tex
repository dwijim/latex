\documentclass[a4paper,12pt]{article}
\usepackage{arabtex}
\usepackage[bahasa] {babel}
\usepackage[top=2cm,left=3cm,right=3cm,bottom=3cm]{geometry}
\title{\Large Doa Berlindung terhadap Berbagai Kesusahan, Kesengsaraan,
dan Hilangnya Kenikmatan}
\author{\small Hanifah Atiya Budianto\\
		\small contact.us@latex-dailyprayers.com}
\begin{document}
\sffamily
\maketitle
\fullvocalize
\setcode{arabtex}
\begin{arabtext}
\noindent
al-ll_ahumma 'inni-y 'a`u-w_du bika min zawAli ni`matika, wata.hawwuli
`Afiyatika, wafu^gA'aTi niqmatika, wa^gami-y`i sa_ha.tika.\\
\end{arabtext}
\noindent
\textbf{Artinya}:
\par
\indent
"Ya Allah, sesungguhnya aku berlindung kepada-Mu dari hilangnya nikmat-Mu,
berubahnya afiat (kesejahteraan) dari-Mu, dari hukuman-Mu yang datang
tiba-tiba, dan dari seluruh kemarahan-Mu."\\\\
\par
\noindent
\textbf{Tingkatan Doa dan Sanad}: \textbf{Shahih}: HR. Muslim (no. 2739
[96]) dan Abu Dawud (no. 1545) dari Abdullah bin Umar r.a.\\
\textbf{Referensi}: Yazid bin Abdul Qadir Jawas. 2016. Kumpulan Do'a dari
Al-Quran dan As-Sunnah yang Shahih. Bogor: Pustaka Imam Asy-Syafi'i.
\end{document}