\documentclass[a4paper,12pt]{article}
\usepackage{arabtex}
\usepackage[bahasa] {babel}
\usepackage[top=2cm,left=3cm,right=3cm,bottom=3cm]{geometry}
\title{\Large Membaca Shalawat{\scriptsize 1} Nabi setelah Tasyahud}
\author{\small Hanifah Atiya Budianto\\
		\small contact.us@latex-dailyprayers.com}
\begin{document}
\sffamily
\maketitle
\fullvocalize
\setcode{arabtex}
\begin{arabtext}
\noindent
al-ll_ahumma .salli `alaY mu.hammadiN wa`alaY ^Ali mu.hammadiN, kamA
.salla-yta `alaY ^Ali 'ibrAhi-yma, wabArik `alaY mu.hammadiN wa`alaY ^Ali
mu.hammadiN, kamA bArakta `alaY ^Ali 'ibrAhi-yma fiy al-`Alami-yna, 'innaka
.hami-yduN ma^gi-yduN.\\
\end{arabtext}
\noindent
\textbf{Artinya}:
\par
\indent
"Ya Allah, berikanlah shalawat kepada Muhammad dan keluarga Muhammad
sebagaimana Engkau telah memberi shalawat kepada keluarga Ibrahim. Dan
berkahilah Muhammad dan keluarga Muhammad sebagaimana Engkau telah
memberkahi keluarga Ibrahim atas sekalian alam, sesungguhnya Engkau Maha
Terpuji (lagi) Mahamulia."{\scriptsize 2}\\\\
\par
\noindent
\textbf{Tingkatan Doa dan Sanad}:
\begin{enumerate}
\item Tidak ada tambahan lafazh "sayyidinaa" dalam shalawat dan tidak ada
satu pun riwayat yang shahih dari Nabi Shallallahu ‘alaihi wa sallam, dan
lafazh ini pun tidak diucapkan oleh para Sahabat radhiyallahu 'anhum.
\item \textbf{Shahih}: HR. Malik dalam \textit{al-Muwaththa'} (I/152, no.
67), Muslim (no. 405 [65]), Abu Dawud (no. 980), Ahmad (IV/118, V/273-274),
at-Tirmidzi (no. 3220), an-Nasai (III/45), \textit{'Amalul Yaum wal Lailah}
(no. 48), dan selainnya dari Abu Mas'ud al-Anshari r.a.
\end{enumerate}
\textbf{Referensi}: Yazid bin Abdul Qadir Jawas. 2016. Kumpulan Do'a dari
Al-Quran dan As-Sunnah yang Shahih. Bogor: Pustaka Imam Asy-Syafi'i.
\end{document}