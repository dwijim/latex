\documentclass[a4paper,12pt]{article}
\usepackage{arabtex}
\usepackage[bahasa] {babel}
\usepackage[top=2cm,left=3cm,right=3cm,bottom=3cm]{geometry}
\title{\Large Doa setelah Tasyahud Akhir sebelum Salam}
\author{\small Hanifah Atiya Budianto\\
		\small contact.us@latex-dailyprayers.com}
\begin{document}
\sffamily
\maketitle
\fullvocalize
\setcode{arabtex}
\begin{arabtext}
\noindent
al-ll_ahumma 'inni-y 'a`u-w_du bika min `a_dA bi al-qabri, wa'a`u-w_du bika
min fitnnaTi al-masiy.hi al-dda^g^gAli, wa'a`u-w_du bika min fitnaTi
al-ma.hyA wAl-mamAti. al-ll_ahumma 'inni-y 'a`u-w_du bika mina al-ma'_tami
wAl-ma.grami.\\
\end{arabtext}
\noindent
\textbf{Artinya}:
\par
\indent
"Ya Allah, sesungguhnya aku berlindung kepada-Mu dari siksa kubur. Aku
berlindung kepada-Mu dari fitnah al-Masih ad-Dajjal. Aku juga berlindung
kepada-Mu dari fitnah kehidupan dan fitnah sesudah mati. Ya Allah,
sesungguhnya aku berlindung kepada-Mu dari perbuatan dosa dan dari utang."
\\\\
\par
\noindent
\textbf{Tingkatan Doa dan Sanad}: \textbf{Shahih}: HR. Al-Bukhari (no. 832)
dan Muslim (no. 589 [129]), dan an-Nasai (III/56-57) dari Aisyah r.a.\\
\textbf{Referensi}: Yazid bin Abdul Qadir Jawas. 2016. Kumpulan Do'a dari
Al-Quran dan As-Sunnah yang Shahih. Bogor: Pustaka Imam Asy-Syafi'i.
\end{document}