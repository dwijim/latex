\documentclass[a4paper,12pt]{article}
\usepackage{arabtex}
\usepackage[bahasa] {babel}
\usepackage[top=2cm,left=3cm,right=3cm,bottom=3cm]{geometry}
\title{\Large Doa Ketika Berbuka Puasa Di Rumah Orang Lain}
\author{\small Hanifah Atiya Budianto\\
		\small contact.us@latex-dailyprayers.com}
\begin{document}
\sffamily
\maketitle
\fullvocalize
\setcode{arabtex}
\begin{arabtext}
\noindent
'af.tara `indakumu al-.s.sA'imu-wna, wa'akala .ta`Amakumu al-'abrAru,
wa.sallat `ala-ykumu al-malA'ikaTu.\\
\end{arabtext}
\noindent
\textbf{Artinya}:
\par
\indent
"Orang-orang yang berpuasa telah berbuka di tempat kalian, dan orang-orang
yang baik telah makan makananmu, dan para Malaikat mendoakan (rahmat /
kebaikan) untuk kalian."\\\\
\par
\noindent
\textbf{Tingkatan Doa dan Sanad}: \textbf{Shahih}: HR. Abu Dawud (no.
3854), an-Nasa'I dalam Kitab \textit{'Amalul Yaum wal Lailah} (no. 298,
299), Ibnu Sunni dalam \textit{'Amalul Yaum wal Lailah} (no. 482), Ahmad
(III/118, 138). Doa ini boleh juga dibaca setelah makan di rumah orang
lain. Lihat \textit{\^{A}d\^{a}buz Zif\^{a}f} (hlm. 170-171).\\
\textbf{Referensi}: Yazid bin Abdul Qadir Jawas. 2016. Kumpulan Do'a dari
Al-Quran dan As-Sunnah yang Shahih. Bogor: Pustaka Imam Asy-Syafi'i.
\end{document}