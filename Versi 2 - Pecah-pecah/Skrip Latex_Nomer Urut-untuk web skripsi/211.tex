\documentclass[a4paper,12pt]{article}
\usepackage{arabtex}
\usepackage[bahasa] {babel}
\usepackage[top=2cm,left=3cm,right=3cm,bottom=3cm]{geometry}
\usepackage{xcolor, framed}
\definecolor{shadecolor}{rgb}{0.8,0.8,0.8}
\title{\Large Doa Saat Mengalami Kesusahan, Kesedihan, dan Penawar
Kedukaan}
\author{\small Hanifah Atiya Budianto\\
		\small contact.us@latex-dailyprayers.com}
\begin{document}
\sffamily
\maketitle
\fullvocalize
\setcode{arabtex}
\begin{arabtext}
\noindent
al-ll_ahumma 'inni-y `abduka, wAbnu `abdika, wAbnu 'amatika, nA.siyati-y
biyadika, mA.diN fiyya .hukmuka, `adluN fiyya qa.dA'uka. 'as'aluka bikulli
asmiN huwa laka, samma-yta bihi nafsaka, 'a-w 'anzaltahu fi-y kitAbika,
'a-w `allamtahu 'a.hadaN min _halqika, 'awi asta'_tarta bihi fi-y `ilmi
al-.ga-ybi `indaka, 'an ta^g`ala al-qur-^Ana rabi-y`a qalbi-y, wanu-wra
.sadri-y, wa^galA'a .huzni-y, wa_dahAba hammi-y.\\
\end{arabtext}
\noindent
\textbf{Artinya}:
\par
\indent
"Ya Allah, sesungguhnya aku adalah hamba-Mu, anak hamba-Mu (Adam), dan anak
hamba perempuan-Mu (Hawa), ubun-ubunku berada di tangan-Mu, hukum-Mu
berlaku terhadap diriku dan ketetapan-Mu adil pada diriku. Aku memohon
kepada-Mu dengan segala Nama yang menjadi milik-Mu, yang Engkau namai
diri-Mu dengannya, atau yang Engkau turunkan di dalam Kitab-Mu, atau yang
Engkau ajarkan kepada seorang dari makhluk-Mu, atau yang Engkau rahasiakan
di dalam ilmu ghaib di sisi-Mu, maka dengannya aku memohon supaya Engkau
menjadikan al-Qur-an penyejuk bagi hatiku, cahaya bagi dadaku, pelipur bagi
kesedihanku, dan penghilang kesusahanku."
\begin{shaded*}
\noindent
Melainkan Allah akan menghilangkan kesedihannya dan kesusahannya (orang
yang mengucapkan doa ini) serta menggantikan semuanya itu dengan
kegembiraan.
\end{shaded*}
\par
\noindent
\\ \textbf{Tingkatan Doa dan Sanad}: \textbf{Shahih}: HR. Ahmad (I/391,
452), Ibnu Hibban (\textit{at-Ta'l\^{i}q\^{a}tul His\^{a}n} [no. 968]),
al-Hakim (I/509), dan ath-Thabrani dalam \textit{al-Mu'jamul Kab\^{i}r}
(X/169-170, no. 352) dari Abdullah bin Mas'ud r.a. Dihasankan al-Hafizh
dalam \textit{Takhr\^{i}j al-Adzk\^{a}r}. Dishahihkan Syaikh al-Albani.
Lihat \textit{al-Kalimuth Thayyib} (hlm. 119, no. 124) dan \textit{Silsilah
Ah\^{a}d\^{i}ts ash-Shah\^{i}hah} (no. 199).\\
\textbf{Referensi}: Yazid bin Abdul Qadir Jawas. 2016. Kumpulan Do'a dari
Al-Quran dan As-Sunnah yang Shahih. Bogor: Pustaka Imam Asy-Syafi'i.
\end{document}