\documentclass[a4paper,12pt]{article}
\usepackage{arabtex}
\usepackage[bahasa] {babel}
\usepackage[top=2cm,left=3cm,right=3cm,bottom=3cm]{geometry}
\title{\Large Doa Diberi Kebahagiaan dan Terhindar dari Kesengsaraan}
\author{\small Hanifah Atiya Budianto\\
		\small contact.us@latex-dailyprayers.com}
\begin{document}
\sffamily
\maketitle
\fullvocalize
\setcode{arabtex}
\begin{arabtext}
\noindent
al-ll_ahumma laka al-.hamdu kulluhu, al-ll_ahumma lA qAbi.da limA basa.tta,
walA bAsi.ta limA qaba.dta, walA hAdiya liman 'a.dlalta, walA mu.dilla
liman hada-yta, walA mu`.tiya limA mana`ta, walA mAni`a limA 'a`.tayta,
walA muqarriba limA bA`adta, walA mubA`ida limA qarrabta, al-ll_ahumma
absu.t `ala-ynA min barakAtika, wara.hmatika, wafa.dlika, warizqika,
al-ll_ahumma 'inni-y 'as'aluka al-nna`i-yma al-muqi-yma, alla_di-y lA
ya.hu-wlu walA yazu-wlu, al-ll_ahumma 'inni-y 'as'aluka al-nna`i-yma ya-wma
al-`a-ylaTi, wAl-'amna yawma al-_hawfi, al-ll_ahumma 'inni-y `A'i_duN bika
min ^sarri mA 'a`.ta-ytanA, wa^sarri mA mana`tanA, al-ll_ahumma .habbib
'ila-ynA al-'i-ymAna, wazayyinhu fi-y qulu-wbinA, wakarrih 'ila-ynA
al-kufra, wAl-fusu-wqa, wAl-`i.syAna, wA^g`alnA mina al-rrA^sidi-yna,
al-ll_ahumma tawaffanA muslimi-yna, wa-'a.hyinA muslimi-yna, wa-'al.hiqnA
bi-al-.s.sAli.hi-yna, .ga-yra _hazAyA walA maftu-wni-yna, al-ll_ahumma
qAtili al-kafaraTa alla_di-yna yuka_d_dibu-wna rusulaka, waya.suddu-wna
`an sabi-ylika, wA^g`al `alayhim ri^gzaka wa-`a_dAbaka, al-ll_ahumma qAtili
al-kafaraTa alla_di-yna 'uwtuW al-kitAba, 'il_aha al.haqqi (^Ami-yn).\\
\end{arabtext}
\noindent
\textbf{Artinya}:
\par
\indent
"Ya Allah, segala puji hanya bagi-Mu. Ya Allah, tidak ada yang dapat
menahan apa yang telah Engkau lapangkan dan tidak ada yang dapat
melapangkan apa yang Engkau tahan, tidak ada yang dapat memberikan petunjuk
kepada orang yang telah Engkau sesatkan, dan tidak ada yang dapat
menyesatkan orang yang telah Engkau beri petunjuk, tidak ada yang dapat
memberi apa yang telah Engkau cegah, dan tidak ada yang dapat mencegah apa
yang Engkau berikan, tidak ada yang dapat mendekatkan apa yang telah Engkau
jauhkan, dan tidak ada pula yang dapat menjauhkan apa yang telah Engkau
dekatkan. Ya Allah, lapangkanlah keberkahan, juga rahmat, karunia, beserta
rizki-Mu kepada kami. Ya Allah, sesungguhnya aku memohon kepada-Mu
kenikmatan yang abadi yang tidak akan berubah dan tidak pula lenyap. Ya
Allah, sesungguhnya aku memohon kenikmatan pada hari kesengsaraan, dan
keamanan pada hari ketakutan. Ya Allah, sungguh aku berlindung kepada-Mu
dari kejelekan apa yang Engkau berikan kepada kami dan kejelekan apa yang
Engkau cegah dari sisi kami. Ya Allah, jadikan kami cinta terhadap
keimanan. Hiasilah ia dalam hati kami dan tanamkanlah kebencian kepada kami
terhadap kekufuran, kefasikan, dan kemaksiatan, serta jadikanlah kami
termasuk orang-orang yang mengikuti jalan yang lurus. Ya Allah, wafatkan
dan hidupkanlah kami dalam keadaan Muslim, dan pertemukan kami dengan
orang-orang yang shalih dalam keadaan tidak terhina dan tidak pula
terfitnah. Ya Allah, perangilah orang-orang kafir yang mendustakan
Rasul-Rasul-Mu dan menghadang jalan-Mu, timpakan kepada mereka siksaan
serta adzab. Ya Allah, perangilah orang-orang kafir yang telah diberi
al-Kitab, wahai Ilah Yang Mahabenar (kabulkanlah, ya Allah)."\\\\
\par
\noindent
\textbf{Tingkatan Doa dan Sanad}: \textbf{Shahih}: HR. Ahmad dengan
lafazhnya (III/424), al-Hakim (I/507)-yang dalam kurung miliknya
(III/23-24)-al-Bukhari dalam \textit{al-Adabul Mufrad} (no. 699).
Dishahihkan oleh Syaikh al-Albani dalam \textit{Takhr\^{i}j Fiqhis
S\^{i}rah} (hlm. 284) dan \textit{Shah\^{i}h al-Adabil Mufrad} (no. 541).\\
\textbf{Referensi}: Yazid bin Abdul Qadir Jawas. 2016. Kumpulan Do'a dari
Al-Quran dan As-Sunnah yang Shahih. Bogor: Pustaka Imam Asy-Syafi'i.
\end{document}