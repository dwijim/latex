\documentclass[a4paper,12pt]{article}
\usepackage{arabtex}
\usepackage[bahasa] {babel}
\usepackage[top=2cm,left=3cm,right=3cm,bottom=3cm]{geometry}
\title{\Large Doa Qunut Witir}
\author{\small Hanifah Atiya Budianto\\
		\small contact.us@latex-dailyprayers.com}
\begin{document}
\sffamily
\maketitle
\fullvocalize
\setcode{arabtex}
\begin{arabtext}
\noindent
al-ll_ahumma ahdini-y fi-yman hadayta, wa`Afini-y fi-yman `Afa-yta,
watawallaniy fi-yman tawalla-yta, wabArik li-y fi-ymA 'a`.tayta, waqini-y
^sarramA qa.dayta, fa-'innaka taq.di-y walA yuq.dY `alayka, wa-'innahu lA
ya_dillu man wAlayta, (walA ya`izzu man `Adayta), tabArakta rabbanA
wata`Alayta.\\
\end{arabtext}
\noindent
\textbf{Artinya}:
\par
\indent
"Ya Allah, berikanlah aku petunjuk sebagaimana orang yang telah Engkau beri
petunjuk, berilah aku perlindungan (dari penyakit dan apa yang tidak
disukai) sebagaimana orang yang telah Engkau lindungi, tolonglah aku
sebagaimana orang-orang yang Engkau tolong. Berikanlah berkah terhadap
apa-apa yang telah Engkau berikan kepadaku, jauhkanlah aku dari kejelekan
apa yang Engkau telah takdirkan, sesungguhnya Engkaulah yang menjatuhkan
hukum, dan tidak ada orang yang memberikan hukuman kepada-Mu. Dan
sesungguhnya orang yang Engkau bela tidak akan terhina, dan tidak akan
mulia orang yang Engkau musuhi. Mahasuci Engkau, wahai Rabb kami
Mahatinggi."\\\\
\par
\noindent
\textbf{Tingkatan Doa dan Sanad}: \textbf{Shahih}: HR. Abu Dawud (no.
1425), at-Tirmidzi (no. 464), Ibnu Majah (no. 1178), an-Nasai (III/248),
Ahmad (I/199; 200), al-Baihaqi (II/209, 497-498). Sedang doa yang terdapat
di dalam kurung menurut riwayat al-Baihaqi. Hadits ini diriwayatkan dari
al-Hasan bin Ali: "Nabi SAW. mengajarkanku beberapa kalimat yang dapat aku
baca dalam shalat Witir ...." Lihat \textit{Shah\^{i}h at-Tirmidzi}
(I/144), \textit{Shah\^{i}h Ibni Majah} (I/194), \textit{Irw\^{a}-ul
Ghal\^{i}l} (II/172), dan \textit{Shah\^{i}h Kit\^{a}b al-Adzk\^{a}r}
(I/176-177, no. 155/125). Sanadnya shahih.\\
\textbf{Referensi}: Yazid bin Abdul Qadir Jawas. 2016. Kumpulan Do'a dari
Al-Quran dan As-Sunnah yang Shahih. Bogor: Pustaka Imam Asy-Syafi'i.
\end{document}