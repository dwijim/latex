\documentclass[a4paper,12pt]{article}
\usepackage{arabtex}
\usepackage[bahasa] {babel}
\usepackage[top=2cm,left=3cm,right=3cm,bottom=3cm]{geometry}
\title{\Large Berlindung dari Segala Penyakit}
\author{\small Hanifah Atiya Budianto\\
		\small contact.us@latex-dailyprayers.com}
\begin{document}
\sffamily
\maketitle
\fullvocalize
\setcode{arabtex}
\begin{arabtext}
\noindent
al-ll_ahumma 'inni-y 'a`u-w_du bika mina al-bara.si, wAl-^gunu-wni,
wAl-^gu_dAmi, wamin sayyi -'i al-'asqAmi.\\
\end{arabtext}
\noindent
\textbf{Artinya}:
\par
\indent
"Ya Allah, sungguh aku berlindung kepada-Mu dari penyakit belang, gila,
lepra, dan dari keburukan segala macam penyakit."\\\\
\par
\noindent
\textbf{Tingkatan Doa dan Sanad}: \textbf{Shahih}: HR. Abu Dawud (no.
1554), an-Nasai (VIII / 270), Ahmad (III/192), Ibnu Hibban (no. 1013 -
\textit{at-Ta'liqatul His\^{a}n}) dan lainnya, dari Anas bin Malik r.a.
Lihat \textit{Shah\^{i}h al-J\^{a}mi-us Shagh\^{i}r} (no. 1281).\\
\textbf{Referensi}: Yazid bin Abdul Qadir Jawas. 2016. Kumpulan Do'a dari
Al-Quran dan As-Sunnah yang Shahih. Bogor: Pustaka Imam Asy-Syafi'i.
\end{document}