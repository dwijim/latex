\documentclass[a4paper,12pt]{article}
\usepackage{arabtex}
\usepackage[bahasa] {babel}
\usepackage[top=2cm,left=3cm,right=3cm,bottom=3cm]{geometry}
\title{\Large Bacaan Setelah Salam}
\author{\small Hanifah Atiya Budianto\\
		\small contact.us@latex-dailyprayers.com}
\begin{document}
\sffamily
\maketitle
\fullvocalize
\setcode{arabtex}
\par
\indent
Lalu untuk melengkapinya menjadi seratus dengan membaca:
\begin{arabtext}
\noindent
lA 'il_aha 'illA al-ll_ahu wa.hdahu lA ^sari-yka lahu, lahu al-mulku walahu
al-.hamdu, wahuwa `alY kulli ^say'iN qadi-yruN.\\
\end{arabtext}
\noindent
\textbf{Artinya}:
\par
\indent
"Tidak ada ilah yang berhak diibadahi dengan benar melainkan hanya Allah
Yang Maha Esa, tiada sekutu bagi-Nya, bagi-Nya kerajaan, bagi-Nya segala
puji. Dan Dia Mahakuasa atas segala sesuatu."{\scriptsize 1}\\\\
\indent
Kemudian membaca surah Al-Ikhlash, Al-Falaq, dan An-Nas pada setiap selesai
shalat fardhu.{\scriptsize 2}\\
\indent
Kemudian membaca ayat Kursi setiap selesai shalat (fardhu).{\scriptsize 3}
\\\\
Baca juga bacaan setelah salam ke 1-8\\
\par
\noindent
\textbf{Tingkatan Doa dan Sanad}:
\begin{enumerate}
\item "Siapa yang membaca dzikir ini tiap selesai shalat akan diampuni
kesalahannya, meski seperti buih di lautan." \textbf{Shahih}: HR. Muslim
(no. 597), Ahmad (II/371, 483), Ibnu Khuzaimah (no. 750), al-Baihaqi
(II/187).
\item \textbf{Shahih}: HR. Abu Dawud (no. 1523), an-Nasai (III/68), Ibnu
Khuzaimah (no. 755), dan Hakim (I/253) dari Uqbah bin Amir r.a. Lihat
\textit{Shah\^{i}h at-Tirmidzi} (III/8, no. 2324) dan \textit{Fathul
B\^{a}ri} (IX/62). Tiga surah ini dinamakan \textit{al-Mu'awwadz\^{a}t}.
\item "Siapa yang membacanya setiap selesai shalat maka tidak ada yang
menghalanginya masuk Surga selain belum datangnya kematian."
\textbf{Shahih}: HR. An-Nasai dalam \textit{'Amalul Yaum wal Lailah} (no.
100) dan Ibnus Sunni (no. 124) dari Abu Umamah r.a. Dishahihkan Syaikh
al-Albani dalam \textit{Shah\^{i}hul J\^{a}mi'} (no. 6464) dan
\textit{ash-Shah\^{i}hah} (no. 972).
\end{enumerate}
\textbf{Referensi}: Yazid bin Abdul Qadir Jawas. 2016. Kumpulan Do'a dari
Al-Quran dan As-Sunnah yang Shahih. Bogor: Pustaka Imam Asy-Syafi'i.
\end{document}