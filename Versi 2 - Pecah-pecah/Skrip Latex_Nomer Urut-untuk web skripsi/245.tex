\documentclass[a4paper,12pt]{article}
\usepackage{arabtex}
\usepackage[bahasa] {babel}
\usepackage[top=2cm,left=3cm,right=3cm,bottom=3cm]{geometry}
\title{\Large Doa Penghilang Gangguan Syaitan tatkala Shalat atau Membaca
Al-Qur'an}
\author{\small Hanifah Atiya Budianto\\
		\small contact.us@latex-dailyprayers.com}
\begin{document}
\sffamily
\maketitle
\fullvocalize
\setcode{arabtex}
\begin{arabtext}
\noindent
'a`u-w_du bi-al-ll_ahi mina al-^s^sa-y.tAni al-rra^gi-ymi.\\
\end{arabtext}
\noindent
\textbf{Artinya}:
\par
\indent
"Aku berlindung kepada Allah dari godaan syaitan yang terkutuk."\\
Lalu meludahlah ke kirimu tiga kali.\\\\
\par
\noindent
\textbf{Tingkatan Doa dan Sanad}: \textbf{Shahih}: HR. Muslim (no. 2203).\\
\textbf{Referensi}: Yazid bin Abdul Qadir Jawas. 2016. Kumpulan Do'a dari
Al-Quran dan As-Sunnah yang Shahih. Bogor: Pustaka Imam Asy-Syafi'i.
\end{document}