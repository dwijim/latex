\documentclass[a4paper,12pt]{article}
\usepackage{arabtex}
\usepackage[bahasa] {babel}
\usepackage[top=2cm,left=3cm,right=3cm,bottom=3cm]{geometry}
\title{\Large Doa untuk Keselamatan}
\author{\small Hanifah Atiya Budianto\\
		\small contact.us@latex-dailyprayers.com}
\begin{document}
\sffamily
\maketitle
\fullvocalize
\setcode{arabtex}
\begin{arabtext}
\noindent
al-ll_ahumma a.gfirli-y, wAhdini-y, wArzuqni-y, wa`Afini-y, 'a`u-w_du
bi-al-ll_ahi min .di-yqi al-maqAmi ya-wma al-qiyAmaTi.\\
\end{arabtext}
\noindent
\textbf{Artinya}:
\par
\indent
"Ya Allah , ampunilah aku, berikanlah petunjuk kepadaku, karuniakanlah
rizki kepadaku, berikan keselamatan bagiku. Aku berlindung kepada Allah
SWT. dari kesempitan tempat berdiri kelak pada hari Kiamat."\\\\
\par
\noindent
\textbf{Tingkatan Doa dan Sanad}: \textbf{Hasan Shahih}: HR. Abu Dawud (no.
766), an-Nasai (III/209), Ibnu Majah (no. 1356), dan yang lainnya. Lihat
\textit{Shah\^{i}h Sunan Abi Dawud} (III/352-353, no. 742).\\
\textbf{Referensi}: Yazid bin Abdul Qadir Jawas. 2016. Kumpulan Do'a dari
Al-Quran dan As-Sunnah yang Shahih. Bogor: Pustaka Imam Asy-Syafi'i.
\end{document}