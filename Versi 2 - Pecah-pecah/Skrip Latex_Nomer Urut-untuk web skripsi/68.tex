\documentclass[a4paper,12pt]{article}
\usepackage{arabtex}
\usepackage[bahasa] {babel}
\usepackage[top=2cm,left=3cm,right=3cm,bottom=3cm]{geometry}
\title{\Large Doa Pergi ke Masjid}
\author{\small Hanifah Atiya Budianto\\
		\small contact.us@latex-dailyprayers.com}
\begin{document}
\sffamily
\maketitle
\fullvocalize
\setcode{arabtex}
\begin{arabtext}
\noindent
al-ll_ahumma a^g`al fi-y qalbi-y nu-wraN , wafi-y lisAni-y nu-wraN,
wA^g`al fi-y sam`iy nuwraN wA^g`al fi-y ba.sari-y nu-wraN, wA^g`al min
_halfi-y nu-wraN, wamin 'amAmi-y nu-wraN wA^g`al min fawqi-y nu-wraN, wamin
ta.hti-y nu-wraN, al-ll_ahumma 'a`.tini-y nu-wraN.\\
\end{arabtext}
\noindent
\textbf{Artinya}:
\par
\indent
"Ya Allah, jadikanlah cahaya pada hatiku, cahaya pada lidahku, cahaya pada
pendengaranku, dan cahaya pada pengelihatanku, cahaya dari belakangku,
cahaya dari hadapanku, cahaya dari atasku, serta cahaya dari bawahku. Ya
Allah, berikanlah padaku cahaya."\\\\
\par
\noindent
\textbf{Tingkatan Doa dan Sanad}: \textbf{Shahih}: HR. Muslim (no. 763
[191])-\textit{Syarah Muslim} (V/51)-Lafazh ini miliknya-diriwayatkan oleh
Imam al-Bukhari (no. 6316). Dan al-Hafizh Ibnu Hajar al-Asqalani
menyebutkan doa ini dalam \textit{Fathul B\^{a}ri} (XI/116) dengan banyak
tambahan di dalamnya. Untuk lebih jelasnya, lihat kitab tersebut.\\
\textbf{Referensi}: Yazid bin Abdul Qadir Jawas. 2016. Kumpulan Do'a dari
Al-Quran dan As-Sunnah yang Shahih. Bogor: Pustaka Imam Asy-Syafi'i.
\end{document}