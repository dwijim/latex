\documentclass[a4paper,12pt]{article}
\usepackage{arabtex}
\usepackage[bahasa] {babel}
\usepackage[top=2cm,left=3cm,right=3cm,bottom=3cm]{geometry}
\title{\Large Doa Berlindung terhadap Berbagai Kesusahan, Kesengsaraan,
dan Hilangnya Kenikmatan}
\author{\small Hanifah Atiya Budianto\\
		\small contact.us@latex-dailyprayers.com}
\begin{document}
\sffamily
\maketitle
\fullvocalize
\setcode{arabtex}
\begin{arabtext}
\noindent
al-ll_ahumma 'inni-y 'a`u-w_du bika mina al-faqri, wAl-qillaTi,
wAl-_d_dillaTi, wa'a`uw_du bika min 'an 'a.zlima 'uw 'u.zlama.\\
\end{arabtext}
\noindent
\textbf{Artinya}:
\par
\indent
"Ya Allah, sesungguhnya aku berlindung kepada-Mu dari kefakiran,
kekurangan, serta kehinaan, dan aku pun berlindung kepada-Mu dari
menzhalimi ataupun dizhalimi orang lain."\\\\
\par
\noindent
\textbf{Tingkatan Doa dan Sanad}: \textbf{Shahih}: HR. An-Nasai (VIII/261),
Abu Dawud (no. 1544) dari Abu Hurairah r.a.\\
\textbf{Referensi}: Yazid bin Abdul Qadir Jawas. 2016. Kumpulan Do'a dari
Al-Quran dan As-Sunnah yang Shahih. Bogor: Pustaka Imam Asy-Syafi'i.
\end{document}