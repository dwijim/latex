\documentclass[a4paper,12pt]{article}
\usepackage{arabtex}
\usepackage[bahasa] {babel}
\usepackage[top=2cm,left=3cm,right=3cm,bottom=3cm]{geometry}
\title{\Large Doa Sesudah Makan}
\author{\small Hanifah Atiya Budianto\\
		\small contact.us@latex-dailyprayers.com}
\begin{document}
\sffamily
\maketitle
\fullvocalize
\setcode{arabtex}
\begin{arabtext}
\noindent
al-.hamdu li-ll_ahi alla_di-y 'a.t`amaniy h_a_dA warazaqani-yhi min
.ga-yri .hawliN minni-y walA quwwaTiN.\\
\end{arabtext}
\noindent
\textbf{Artinya}:
\par
\indent
"Segala puji bagi Allah yang telah memberi makanan ini kepadaku dan yang
telah memberi rizki kepadaku tanpa daya dan kekuatan dariku."\\\\
\par
\noindent
\textbf{Tingkatan Doa dan Sanad}: \textbf{Shahih}: HR. Abu Dawud (no.
4023), at-Tirmidzi (no. 3458), Ibnu Majah (no. 3285), Ibnus Sunni (no.
467), Ahmad (III/439) dan al-Hakim (I/507; IV/192), Lihat
\textit{Irw\^{a}-ul Ghal\^{i}l} (no. 1989).\\
\textbf{Referensi}: Yazid bin Abdul Qadir Jawas. 2016. Kumpulan Do'a dari
Al-Quran dan As-Sunnah yang Shahih. Bogor: Pustaka Imam Asy-Syafi'i.
\end{document}