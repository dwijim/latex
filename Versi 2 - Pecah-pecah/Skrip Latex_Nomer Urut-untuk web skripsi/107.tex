\documentclass[a4paper,12pt]{article}
\usepackage{arabtex}
\usepackage[bahasa] {babel}
\usepackage[top=2cm,left=3cm,right=3cm,bottom=3cm]{geometry}
\title{\Large Doa setelah Tasyahud Akhir sebelum Salam}
\author{\small Hanifah Atiya Budianto\\
		\small contact.us@latex-dailyprayers.com}
\begin{document}
\sffamily
\maketitle
\fullvocalize
\setcode{arabtex}
\begin{arabtext}
\noindent
al-ll_ahumma 'inni-y 'as'aluka bi-'anna laka al-.hamda lA 'il_aha 'anta
wa.hdaka lA ^sari-yka laka, al-mannAnu, yA badi-y`a al-ssamAwAti wAl-'ar.di
yA _dAl^galAli wAl-'ikrAmi, yA.hayyu yA qayyu-wmu 'inni-y 'as'aluka
(al-^gannaTa wa'a`u-w_du bika mina al-nnAri).\\
\end{arabtext}
\noindent
\textbf{Artinya}:
\par
\indent
"Ya Allah, sesungguhnya aku memohon kepada-Mu. Sesungguhnya bagi-Mu segala
pujian, tidak ada ilah yang berhak diibadahi dengan benar kecuali Engkau
Yang Maha Esa, tiada sekutu bagi-Mu, Mahapemberi nikmat, Pencipta langit
dan bumi tanpa contoh sebelumnya. Wahai Rabb Yang memiliki keagungan dan
kemuliaan, wahai Rabb Yang Mahahidup, Yang berdiri sendiri (mengurusi
makhluk-Nya) sesungguhnya aku mohon kepada-Mu agar dimasukkan [ke Surga dan
aku berlindung kepada-Mu dari siksa Neraka]."\\\\
\par
\noindent
\textbf{Tingkatan Doa dan Sanad}: Sabda Rasulullah Shallallahu ‘alaihi wa
sallam: "Sesungguhnya dia meminta kepada Allah dengan nama-Nya yang teragung
(\textit{ismullabi a'zham}). Apabila ia minta kepada Allah maka akan
dipenuhi dan apabila ia berdoa maka akan dikabulkan." \textbf{Shahih}: HR.
Abu Dawud (no. 1495), an-Nasai (III/52), Ibnu Majah (no. 3858) Ahmad
(III/158, 245) dan Ibnu Mandah dalam \textit{Kitabut Tauhid} (no. 355). Dan
tambahan dalam kurung miliknya dari Anas bin Malik r.a.\\
\textbf{Referensi}: Yazid bin Abdul Qadir Jawas. 2016. Kumpulan Do'a dari
Al-Quran dan As-Sunnah yang Shahih. Bogor: Pustaka Imam Asy-Syafi'i.
\end{document}