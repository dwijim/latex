\documentclass[a4paper,12pt]{article}
\usepackage{arabtex}
\usepackage[bahasa] {babel}
\usepackage[top=2cm,left=3cm,right=3cm,bottom=3cm]{geometry}
\title{\Large Doa setelah Tasyahud Akhir sebelum Salam}
\author{\small Hanifah Atiya Budianto\\
		\small contact.us@latex-dailyprayers.com}
\begin{document}
\sffamily
\maketitle
\fullvocalize
\setcode{arabtex}
\begin{arabtext}
\noindent
al-ll_ahumma 'inni-y .zalamtu nafsi-y .zulmaN ka_ti-yraN, walA ya.gfiru
al-ddunu-wba 'illA 'anta, fA.gfir li-y ma.gfiraTaN min `indika, wAr.hamni-y
'innaka 'anta al-.gafu-wru al-rra.hi-ymu.\\
\end{arabtext}
\noindent
\textbf{Artinya}:
\par
\indent
"Ya Allah, sesungguhnya aku banyak menganiaya diriku, dan tidak ada yang
dapat mengampuni dosa-dosa kecuali Engkau. Oleh karena itulah, ampunilah
dosa-dosaku dengan ampunan dari sisi Engkau, dan berilah rahmat kepadaku.
Sesungguhnya Engkau Maha Pengampun lagi Maha Penyayang."\\\\
\par
\noindent
\textbf{Tingkatan Doa dan Sanad}: \textbf{Shahih}: HR. Al-Bukhari (no. 834,
6326, 7387, 7388), dan Muslim (no. 2705 [48]) dari Sahabat Abu Bakar
ash-Shiddiq r.a.\\
\textbf{Referensi}: Yazid bin Abdul Qadir Jawas. 2016. Kumpulan Do'a dari
Al-Quran dan As-Sunnah yang Shahih. Bogor: Pustaka Imam Asy-Syafi'i.
\end{document}