\documentclass[a4paper,12pt]{article}
\usepackage{arabtex}
\usepackage[bahasa] {babel}
\usepackage[top=2cm,left=3cm,right=3cm,bottom=3cm]{geometry}
\title{\Large Doa Berlindung Dari Hutang dan Agar Dapat Melunasinya}
\author{\small Hanifah Atiya Budianto\\
		\small contact.us@latex-dailyprayers.com}
\begin{document}
\sffamily
\maketitle
\fullvocalize
\setcode{arabtex}
\begin{arabtext}
\noindent
al-ll_ahumma akfini-y bi.halAlika `an .harAmika, wa'a.gnini-y bifa.dlika
`amman siwAka.\\
\end{arabtext}
\noindent
\textbf{Artinya}:
\par
\indent
"Ya Allah, cukupilah aku dengan rizki-Mu yang halal (hingga terhindar) dari
yang haram. Cukupilah aku dengan karunia-Mu (hingga aku tidak minta) kepada
selain-Mu."\\\\
\par
\noindent
\textbf{Tingkatan Doa dan Sanad}: \textbf{Hasan}: HR. At-Tirmidzi (no.
3563), Ahamd (I/153) dan al-Hakim (I/538) dari Ali bin Abi Thalib r.a.\\
\textbf{Referensi}: Yazid bin Abdul Qadir Jawas. 2016. Kumpulan Do'a dari
Al-Quran dan As-Sunnah yang Shahih. Bogor: Pustaka Imam Asy-Syafi'i.
\end{document}