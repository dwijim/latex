\documentclass[a4paper,12pt]{article}
\usepackage{arabtex}
\usepackage[bahasa] {babel}
\usepackage[top=2cm,left=3cm,right=3cm,bottom=3cm]{geometry}
\title{\Large Doa Apabila Singgah di Suatu Tempat dalam Safar atau Selainnya}
\author{\small Hanifah Atiya Budianto\\
		\small contact.us@latex-dailyprayers.com}
\begin{document}
\sffamily
\maketitle
\fullvocalize
\setcode{arabtex}
\begin{arabtext}
\noindent
'a`u-w_du bikalimAti al-ll_ahi al-ttAmmAti min ^sarri mA _halaqa.\\
\end{arabtext}
\noindent
\textbf{Artinya}:
\par
\indent
"Aku berlindung dengan kalimat-kalimat Allah yang sempurna, dari kejahatan
apa yang diciptakan-Nya."\\\\
\par
\noindent
\textbf{Tingkatan Doa dan Sanad}: \textbf{Shahih}: HR. Muslim (no. 2708
[53]), at-Tirmidzi (no. 3437), Ibnu Majah (no. 3547), Ahmad (VI/377) dan
lainnya. Nabi SAW. bersabda: "Barang siapa yang menempati (atau singgah) di
suatu tempat kemudian mengucapkan (doa di atas), maka tidak ada sesuatu pun
yang bisa membahayakannya, sampai dia meninggalkan tempat tersebut."\\
\textbf{Referensi}: Yazid bin Abdul Qadir Jawas. 2016. Kumpulan Do'a dari
Al-Quran dan As-Sunnah yang Shahih. Bogor: Pustaka Imam Asy-Syafi'i.
\end{document}