\documentclass[a4paper,12pt]{article}
\usepackage{arabtex}
\usepackage[bahasa] {babel}
\usepackage[top=2cm,left=3cm,right=3cm,bottom=3cm]{geometry}
\title{\Large Doa di Akhir Shalat Witir}
\author{\small Hanifah Atiya Budianto\\
		\small contact.us@latex-dailyprayers.com}
\begin{document}
\sffamily
\maketitle
\fullvocalize
\setcode{arabtex}
\begin{arabtext}
\noindent
al-ll_ahumma 'inni-y 'a`u-w_du biri.dAka min sa_ha.tika, wabimu`AfAtika min
`uqu-wbatika, wa'a`u-w_du bika minka, lA 'u.h.si-y _tanA'aN `alayka, 'anta
kamA 'a_tnayta `alY nafsika.\\
\end{arabtext}
\noindent
\textbf{Artinya}:
\par
\indent
"Ya Allah, sesungguhnya aku berlindung dengan keridhaan-Mu dari
kemurkaan-Mu, dengan keselamatan-Mu dari hukuman-Mu, dan berlindung
kepada-Mu dari siksaan-Mu. Aku tidaklah mampu menghitung pujian dan
sanjungan kepada-Mu, Engkau adalah sebagaimana Engkau menyanjung/memuji
diri-Mu sendiri."\\\\
Baca juga doa di akhir shalat witir ke 1-2\\
\par
\noindent
\textbf{Tingkatan Doa dan Sanad}: \textbf{Shahih}: HR. Abu Dawud (no.
1427), at-Tirmidzi (no. 3566), Ibnu Majah (no. 1179), an-Nasai (III/249),
Ahmad (I/98, I/96, 118, 150). Lihat \textit{Shah\^{i}h at-Tirmidzi}
(III/180), \textit{Shah\^{i}h Ibni Majah} (I/194), \textit{Irw\^{a}-ul
Ghal\^{i}l} (II/175), dan \textit{Shah\^{i}h Kit\^{a}b al-Adzk\^{a}r}
(I/255-256, no. 246/184).\\
\textbf{Referensi}: Yazid bin Abdul Qadir Jawas. 2016. Kumpulan Do'a dari
Al-Quran dan As-Sunnah yang Shahih. Bogor: Pustaka Imam Asy-Syafi'i.
\end{document}