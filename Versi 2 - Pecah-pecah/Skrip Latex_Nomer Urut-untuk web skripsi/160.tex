\documentclass[a4paper,12pt]{article}
\usepackage{arabtex}
\usepackage[bahasa] {babel}
\usepackage[top=2cm,left=3cm,right=3cm,bottom=3cm]{geometry}
\usepackage{xcolor, framed}
\definecolor{shadecolor}{rgb}{0.8,0.8,0.8}
\title{\Large Doa pada Shalat Jenazah}
\author{\small Hanifah Atiya Budianto\\
		\small contact.us@latex-dailyprayers.com}
\begin{document}
\sffamily
\maketitle
\fullvocalize
\setcode{arabtex}
\begin{arabtext}
\noindent
al-ll_ahumma a.gfir li.hayyinA wamayyitinA, wa^sAhidinA wa.gA'ibinA,
wa.sa.gi-yrinA wakabi-yrinA, wa_dakarinA wa-'un_tAnA. al-ll_ahumma man
'a.hya-ytahu minnA fa'a.hyihi `alY al-'islAmi, waman tawaffa-ytahu minnA
fatawaffahu `alY al-'iymAni, al-ll_ahumma lA ta.hrimnA 'a^grahu walA
tu.dillanA ba`dahu.\\
\end{arabtext}
\noindent
\textbf{Artinya}:
\par
\indent
"Ya Allah, ampuni orang yang masih hidup di antara kami dan yang sudah
mati, yang hadir dan yang tidak hadir, yang masih kecil maupun dewasa,
laki-laki maupun perempuan. Ya Allah, orang yang Engkau hidupkan di antara
kami, hidupkanlah dengan memegang ajaran Islam, dan yang Engkau wafatkan di
antara kami, maka wafatkanlah dalam keadaan beriman. Ya Allah, jangan
halangi kami untuk memperoleh pahalanya dan jangan sesatkan kami
sepeninggalnya."\\\\
\par
\noindent
\textbf{Tingkatan Doa dan Sanad}: \textbf{Shahih}: HR. Abu Dawud (no.
3201), at-Tirmidzi (no. 1024), Ibnu Majah (no. 1498) dan Ahmad (II/368) dan
lainnya. Lihat \textit{Ahk\^{a}mul Jan\^{a}-iz} (hlm. 157-158).\\
\textbf{Referensi}: Yazid bin Abdul Qadir Jawas. 2016. Kumpulan Do'a dari
Al-Quran dan As-Sunnah yang Shahih. Bogor: Pustaka Imam Asy-Syafi'i.
\end{document}