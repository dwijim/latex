\documentclass[a4paper,12pt]{article}
\usepackage{arabtex}
\usepackage[bahasa] {babel}
\usepackage[top=2cm,left=3cm,right=3cm,bottom=3cm]{geometry}
\title{\Large Doa Setelah Wudhu}
\author{\small Hanifah Atiya Budianto\\
		\small contact.us@latex-dailyprayers.com}
\begin{document}
\sffamily
\maketitle
\fullvocalize
\setcode{arabtex}
\begin{arabtext}
\noindent
'a^shadu 'an lA 'il_aha 'illA al-ll_ahu wa.hdahu lA ^sari-yka lahu
wa'a^shadu 'anna mu.hammadaN `abduhu warasu-wluhu.\\
\end{arabtext}
\noindent
\textbf{Artinya}:
\par
\indent
"Aku bersaksi bahwa tidak ada ilah yang berhak diibadahi dengan benar
kecuali hanya Allah, Yang Maha Esa, tiada sekutu bagi-Nya. Dan aku bersaksi
bahwa Muhammad adalah hamba dan Rasul-Nya."\\\\
\par
\noindent
\textbf{Tingkatan Doa dan Sanad}: \textbf{Shahih}: HR. Muslim (I/209-210,
no. 234).\\
\textbf{Referensi}: Yazid bin Abdul Qadir Jawas. 2016. Kumpulan Do'a dari
Al-Quran dan As-Sunnah yang Shahih. Bogor: Pustaka Imam Asy-Syafi'i.
\end{document}