\documentclass[a4paper,12pt]{article}
\usepackage{arabtex}
\usepackage[bahasa] {babel}
\usepackage[top=2cm,left=3cm,right=3cm,bottom=3cm]{geometry}
\title{\Large Doa Malam Lailatul Qadar}
\author{\small Hanifah Atiya Budianto\\
		\small contact.us@latex-dailyprayers.com}
\begin{document}
\sffamily
\maketitle
\fullvocalize
\setcode{arabtex}
\begin{arabtext}
\noindent
al-ll_ahumma 'innaka `afuwwuN, tu.hibbu al-`afwa, fA`fu `anni-y.\\
\end{arabtext}
\noindent
\textbf{Artinya}:
\par
\indent
"Ya Allah, sesungguhnya Engkau Maha Pemaaf, Engkau menyukai pemaafan.
Karena itu, berilah maaf kepadaku."\\\\
\par
\noindent
\textbf{Tingkatan Doa dan Sanad}: \textbf{Shahih}: HR. At-Tirmidzi (no.
3513), Ibnu Majah (no. 3850), Ahmad (VI/171), al-Hakim (I/530), an-Nasai
dalam \textit{'Amalul Yaum wal Laila}h (no. 878). Lihat \textit{Shah\^{i}h
at-Tirmidzi} (III/170, no. 2789) dan \textit{Silsilah Ah\^{a}d\^{i}ts
ash-Shah\^{i}hah} (no. 3337).\\
\textbf{Referensi}: Yazid bin Abdul Qadir Jawas. 2016. Kumpulan Do'a dari
Al-Quran dan As-Sunnah yang Shahih. Bogor: Pustaka Imam Asy-Syafi'i.
\end{document}