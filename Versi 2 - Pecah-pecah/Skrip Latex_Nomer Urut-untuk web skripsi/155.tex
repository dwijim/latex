\documentclass[a4paper,12pt]{article}
\usepackage{arabtex}
\usepackage[bahasa] {babel}
\usepackage[top=2cm,left=3cm,right=3cm,bottom=3cm]{geometry}
\title{\Large Doa Apabila Melihat Orang yang Mengalami Cobaan}
\author{\small Hanifah Atiya Budianto\\
		\small contact.us@latex-dailyprayers.com}
\begin{document}
\sffamily
\maketitle
\fullvocalize
\setcode{arabtex}
\begin{arabtext}
\noindent
al-.hamdu li-ll_ahi a-lla_di-y `AfAni-y mimmA abtalAka bihi wafa.d.dalani-y
`alY ka_ti-yriN mimman _halaqa taf.di-ylaN.\\
\end{arabtext}
\noindent
\textbf{Artinya}:
\par
\indent
"Segala puji bagi Allah yang telah menyelamatkan aku dari musibah yang
Allah timpakan kepadamu. Dan Allah telah memberi kemuliaan kepadaku
melebihi orang banyak."\\\\
\par
\noindent
\textbf{Tingkatan Doa dan Sanad}: \textbf{Shahih}: HR. At-Tirmidzi (no.
3431), Ibnu Majah (no. 3892) dan lihat \textit{Silsilah Ah\^{a}d\^{i}ts
ash-Shah\^{i}hah} (no. 602). \\
\textbf{Referensi}: Yazid bin Abdul Qadir Jawas. 2016. Kumpulan Do'a dari
Al-Quran dan As-Sunnah yang Shahih. Bogor: Pustaka Imam Asy-Syafi'i.
\end{document}