\documentclass[a4paper,12pt]{article}
\usepackage{arabtex}
\usepackage[bahasa] {babel}
\usepackage[top=2cm,left=3cm,right=3cm,bottom=3cm]{geometry}
\title{\Large Doa Masuk Masjid}
\author{\small Hanifah Atiya Budianto\\
		\small contact.us@latex-dailyprayers.com}
\begin{document}
\sffamily
\maketitle
\fullvocalize
\setcode{arabtex}
\begin{arabtext}
\noindent
bismi al-ll_ahi wAl-.s.salATu wAl-ssalAmu `al_aY rasu-wli al-ll_ahi,
al-ll_ahumma afta.h li-y 'abwAba ra.hmatika.\\
\end{arabtext}
\noindent
\textbf{Artinya}:
\par
\indent
"Dengan nama Allah, semoga shalawat dan salam tercurah kepada Rasulullah.
{\scriptsize 1} Ya Allah, bukakanlah untukku pintu-pintu rahmat-Mu."
{\scriptsize 2}\\\\
\par
\noindent
\textbf{Tingkatan Doa dan Sanad}:
\begin{enumerate}
\item \textbf{Shahih}: HR. Abu Dawud (no. 465), lihat \textit{Shah\^{i}h
al-J\^{a}mi'ish Shagh\^{i}r} (no. 514); Ibnus Sunni dalam \textit{'Amalul
Yaum wal Lailah} (no. 88). Dinyatakan hasan oleh Syaikh al-Albani dalam
\textit{al-Kalimuth Thayyib} (hlm. 92, no. 64, catatan kaki no. 52).
\item \textbf{Shahih}: HR. Muslim (no. 713).
\end{enumerate}
\textbf{Referensi}: Yazid bin Abdul Qadir Jawas. 2016. Kumpulan Do'a dari
Al-Quran dan As-Sunnah yang Shahih. Bogor: Pustaka Imam Asy-Syafi'i.
\end{document}