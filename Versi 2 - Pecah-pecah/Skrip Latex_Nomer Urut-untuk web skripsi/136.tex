\documentclass[a4paper,12pt]{article}
\usepackage{arabtex}
\usepackage[bahasa] {babel}
\usepackage[top=2cm,left=3cm,right=3cm,bottom=3cm]{geometry}
\title{\Large Doa Safar}
\author{\small Hanifah Atiya Budianto\\
		\small contact.us@latex-dailyprayers.com}
\begin{document}
\sffamily
\maketitle
\fullvocalize
\setcode{arabtex}
\begin{arabtext}
\noindent
al-ll_ahu 'akbaru, al-ll_ahu 'akbaru, al-ll_ahu 'akbaru, (sub.h_ana
alla_diY sa_h_hara lanA h_a_dA wamA kunnA lahu muqri ni-yna wa-'inna-^A
'il_aY rabbinA lamunqalibuwna) al-ll_ahumma 'innA nas'aluka fi-y safarinA
h_a_dA al-birra wAl-ttaqwY, wamina al-`amali mAtar.dY, al-ll_ahumma
hawwin `ala-ynA safaranA h_a_dA wA.twi `annA bu`dahu, al-ll_ahumma 'anta
al-.s.sA.hibu fi-y al-ssafari wAl-_hali-yfaTu fi-y al-'ahli, al-ll_ahumma
'inni-y 'a`u-w_dubika min wa-`_tA'i al-ssafari waka-^AbaTi al-man.zari
wasu-w'i al-munqalabi fi-y almAli wAl-'ahli.\\
\end{arabtext}
\noindent
\textbf{Artinya}:
\par
\indent
"Allah Mahabesar (3x). \textit{Mahasuci Rabb yang menundukkan kendaraan ini
untuk kami, sedang sebelumnya kami tidak mampu menguasainya. Dan
sesungguhnya kami akan kembali kepada Rabb kami (di hari Kiamat)}. Ya
Allah, sesungguhnya kami memohon kepada-Mu kebaikan dan takwa dalam
perjalanan ini, kami mohon perbuatan yang Engkau ridhai. Ya Allah,
mudahkanlah perjalanan ini untuk kami, dan dekatkan jaraknya. Ya Allah,
Engkaulah Pendampingku dalam bepergian dan yang mengurusi keluarga(ku). Ya
Allah, sesungguhnya aku berlindung kepada-Mu dari kesulitan dalam
bepergian, pemandangan yang menyedihkan dan kepulangan yang buruk dalam
harta dan keluarga."\\\\
\par
\noindent
\textbf{Tingkatan Doa dan Sanad}: \textbf{Shahih}: HR. Muslim (no. 1342)
dari Ibnu Umar r.a.\\
\textbf{Referensi}: Yazid bin Abdul Qadir Jawas. 2016. Kumpulan Do'a dari
Al-Quran dan As-Sunnah yang Shahih. Bogor: Pustaka Imam Asy-Syafi'i.
\end{document}