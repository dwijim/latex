\documentclass[a4paper,12pt]{article}
\usepackage{arabtex}
\usepackage[bahasa] {babel}
\usepackage[top=2cm,left=3cm,right=3cm,bottom=3cm]{geometry}
\title{\Large Doa agar Diberi Kekuatan Iman dan Berbagai Kebaikan}
\author{\small Hanifah Atiya Budianto\\
		\small contact.us@latex-dailyprayers.com}
\begin{document}
\sffamily
\maketitle
\fullvocalize
\setcode{arabtex}
\begin{arabtext}
\noindent
al-ll_ahumma 'inni-y 'as'aluka min fa.dlika wara.hmatika, fa-'innahu lA
yamlikuhA 'illA 'anta.\\
\end{arabtext}
\noindent
\textbf{Artinya}:
\par
\indent
"Ya Allah, sungguh aku memohon kepada-Mu karunia dan rahmat-Mu, karena
tidak ada yang memilikinya kecuali hanya Engkau."\\\\
\par
\noindent
\textbf{Tingkatan Doa dan Sanad}: \textbf{Shahih}: HR. Abu Nu'aim dalam
kitab \textit{Hilyatul Auliya'} dan ath-Thabrani dalam \textit{al-Mu'jamul
Kab\^{i}r} (X/178, no. 10379). Lihat \textit{Majma'uz Zaw\^{a}-id} (X/159),
\textit{Shah\^{i}hul J\^{a}mi'} (no. 1278) serta \textit{Silsilah
Ah\^{a}d\^{i}ts ash-Shah\^{i}hah} (no. 1543).\\
\textbf{Referensi}: Yazid bin Abdul Qadir Jawas. 2016. Kumpulan Do'a dari
Al-Quran dan As-Sunnah yang Shahih. Bogor: Pustaka Imam Asy-Syafi'i.
\end{document}