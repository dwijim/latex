\documentclass[a4paper,12pt]{article}
\usepackage{arabtex}
\usepackage[bahasa] {babel}
\usepackage[top=2cm,left=3cm,right=3cm,bottom=3cm]{geometry}
\title{\Large Bacaan Setelah Salam}
\author{\small Hanifah Atiya Budianto\\
		\small contact.us@latex-dailyprayers.com}
\begin{document}
\sffamily
\maketitle
\fullvocalize
\setcode{arabtex}
\begin{arabtext}
\noindent
lA 'il_aha 'illA al-ll_ahu wa.hdahu lA ^sari-yka lahu, lahu al-mulku walahu
al-.hamdu wahuwa `alY kulli ^say'iN qadi-yruN, lA .ha-wla walA quwwaTa
'illA bi-al-ll_ahi, lA 'il_aha 'illA al-ll_ahu, walA na`budu 'illA 'iyyAhu,
lahu al-nni`maTu, walahu al-fa.dlu, walahu al-_t_tanA'u al-.hasanu, lA
'il_aha 'illA al-ll_ahu mu_hli.si-yna lahu al-ddi-yna wala-w kariha
al-kAfiru-wna.\\
\end{arabtext}
\noindent
\textbf{Artinya}:
\par
\indent
"Tidak ada ilah yang berhak diibadahi dengan benar melainkan hanya Allah
Yang Maha Esa, tiada sekutu bagi-Nya. Bagi-Nya segala kerajaan dan pujian.
Dia Maha berkuasa atas segala sesuatu. Tidak ada daya dan kekuatan kecuali
(dengan pertolongan) Allah semata. Tidak ada ilah yang berhak diibadahi
dengan benar melainkan hanya Allah. Kami tidak beribadah kecuali
kepada-Nya. Bagi-Nya nikmat, anugerah, dan pujian yang baik. Tidak ada ilah
yang berhak diibadahi dengan benar melainkan hanya Allah, dengan memurnikan
ibadah hanya kepada-Nya, meskipun orang-orang kafir tidak menyukainya." \\\\
Baca juga bacaan setelah salam ke 1-8\\
\par
\noindent
\textbf{Tingkatan Doa dan Sanad}: \textbf{Shahih}: HR. Muslim (no. 594), Abu
Dawud (no. 1506, 1507), Ahmad (IV/4,5), an-Nasai (III/70), Ibnu Khuzaimah
(no. 740, 741) dari Abdullah bin az-Zubair r.a.\\
\textbf{Referensi}: Yazid bin Abdul Qadir Jawas. 2016. Kumpulan Do'a dari
Al-Quran dan As-Sunnah yang Shahih. Bogor: Pustaka Imam Asy-Syafi'i.
\end{document}