\documentclass[a4paper,12pt]{article}
\usepackage{arabtex}
\usepackage[bahasa] {babel}
\usepackage[top=2cm,left=3cm,right=3cm,bottom=3cm]{geometry}
\title{\Large Doa Agar Diberi Ketetapan Hati}
\author{\small Hanifah Atiya Budianto\\
		\small contact.us@latex-dailyprayers.com}
\begin{document}
\sffamily
\maketitle
\fullvocalize
\setcode{arabtex}
\begin{arabtext}
\noindent
yA muqaliba al-qulu-wbi, _tabbit qalbi-y `alY di-ynika.\\
\end{arabtext}
\noindent
\textbf{Artinya}:
\par
\indent
"Wahai Yang membolak-balikkan hati, teguhkanlah hatiku pada agamu-Mu."\\\\
\par
\noindent
\textbf{Tingkatan Doa dan Sanad}: \textbf{Shahih}: HR. At-Tirmidzi (no.
3522), Ahmad (VI/302, 315) dari Ummu Salamah r.a., dan al-Hakim (I/525)
dari an-Nawas bin Sam'an. Dishahihkan dan disepakati oleh adz-Dzahabi.
Lihat juga \textit{Shah\^{i}h at-Tirmidzi} (III/171), no. 2792). Ummu
Salamah berkata: "Doa itu adalah doa Nabi Shallallahu alaihi wa sallam
yang paling sering dibaca."\\
\textbf{Referensi}: Yazid bin Abdul Qadir Jawas. 2016. Kumpulan Do'a dari
Al-Quran dan As-Sunnah yang Shahih. Bogor: Pustaka Imam Asy-Syafi'i.
\end{document}