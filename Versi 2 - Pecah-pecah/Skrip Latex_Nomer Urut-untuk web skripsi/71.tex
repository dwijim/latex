\documentclass[a4paper,12pt]{article}
\usepackage{arabtex}
\usepackage[bahasa] {babel}
\usepackage[top=2cm,left=3cm,right=3cm,bottom=3cm]{geometry}
\title{\Large Doa Keluar Masjid}
\author{\small Hanifah Atiya Budianto\\
		\small contact.us@latex-dailyprayers.com}
\begin{document}
\sffamily
\maketitle
\fullvocalize
\setcode{arabtex}
\begin{arabtext}
\noindent
bismi al-ll_ahi waal-.s.salATu wAl-ssalAmu `al_aY rasu-wli  al-ll_ahi,
Aal-ll_ahumma 'inniy 'as'aluka min fa.dlika, Aal-ll_ahumma a`.simniy mina
al-^s^say.taani al-rra^giymi.\\
\end{arabtext}
\noindent
\textbf{Artinya}:
\par
\indent
"Dengan nama Allah, semoga shalawat dan salam selalu terlimpahkan kepada
Rasulullah. Ya Allah, sesungguhnya aku memohon kepada-Mu karunia-Mu, Ya
Allah, lindungilah aku dari godaan syaitan yang terkutuk."\\\\
\par
\noindent
\textbf{Tingkatan Doa dan Sanad}: \textbf{Shahih}: HR. Muslim (no. 713),
dan Ibnus Sunni dalam \textit{'Amalul Yaum wal Lailah} (no. 88). Adapun
tambahan: (\textit{Allaahumma'shimnii minasy-syaithoonir-rojiim}) adalah
dari Ibnu Majah (no. 773). \textit{Shah\^{i}h Ibnu Majah} (no. 627).\\
\textbf{Referensi}: Yazid bin Abdul Qadir Jawas. 2016. Kumpulan Do'a dari
Al-Quran dan As-Sunnah yang Shahih. Bogor: Pustaka Imam Asy-Syafi'i.
\end{document}