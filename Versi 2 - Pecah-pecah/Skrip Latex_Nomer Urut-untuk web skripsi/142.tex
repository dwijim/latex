\documentclass[a4paper,12pt]{article}
\usepackage{arabtex}
\usepackage[bahasa] {babel}
\usepackage[top=2cm,left=3cm,right=3cm,bottom=3cm]{geometry}
\title{\Large Doa Masuk Pasar}
\author{\small Hanifah Atiya Budianto\\
		\small contact.us@latex-dailyprayers.com}
\begin{document}
\sffamily
\maketitle
\fullvocalize
\setcode{arabtex}
\begin{arabtext}
\noindent
lA 'il_aha 'illA al-ll_ahu wa.hdahu lA ^sari-yka lahu, lahu al-mulku,
walahu al-.hamdu,yu.hyi-y wayumi-ytu, wahuwa .hayyuN lA yamu-wtu, biyadihi
al-_ha-yru, wahuwa `alY kulli ^say'iN qadi-yruN.\\
\end{arabtext}
\noindent
\textbf{Artinya}:
\par
\indent
"Tidak ada Ilah yang berhak diibadahi dengan benar melainkan hanya Allah,
Yang Maha Esa, tiada sekutu bagi-Nya. Bagi-Nya kerajaan, bagi-Nya segala
puji. Dialah Yang Menghidupkan dan Yang Mematikan. Dialah Yang Hidup, tidak
akan mati. Di tangan-Nya kebaikan, Dialah Yang Mahakuasa atas segala
sesuatu."\\\\
\par
\noindent
\textbf{Tingkatan Doa dan Sanad}: \textbf{Hasan}: HR. At-Tirmidzi (no. 3428
dan 3429), Ibnu Majah (no. 2235), al-Hakim (I/538). Lihat takhrij hadits
ini dalam \textit{Shah\^{i}h al-W\^{a}hbilish Shayyib} (hlm. 250-255).\\
\textbf{Referensi}: Yazid bin Abdul Qadir Jawas. 2016. Kumpulan Do'a dari
Al-Quran dan As-Sunnah yang Shahih. Bogor: Pustaka Imam Asy-Syafi'i.
\end{document}