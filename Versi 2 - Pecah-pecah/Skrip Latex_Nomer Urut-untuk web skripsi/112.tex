\documentclass[a4paper,12pt]{article}
\usepackage{arabtex}
\usepackage[bahasa] {babel}
\usepackage[top=2cm,left=3cm,right=3cm,bottom=3cm]{geometry}
\title{\Large Bacaan Setelah Salam}
\author{\small Hanifah Atiya Budianto\\
		\small contact.us@latex-dailyprayers.com}
\begin{document}
\sffamily
\maketitle
\fullvocalize
\setcode{arabtex}
\begin{arabtext}
\noindent
Aal-ll_ahumma 'a`inni-y `alY _dikrika, wa^sukrika, wa.husni `ibAdatika.\\
\end{arabtext}
\noindent
\textbf{Artinya}:
\par
\indent
"Ya Allah, tolong aku agar selalu berdzikir kepada-Mu, bersyukur kepada-Mu,
serta beribadah dengan baik kepada-Mu."\\\\
Baca juga bacaan setelah salam ke 1-8\\
\par
\noindent
\textbf{Tingkatan Doa dan Sanad}: \textbf{Shahih}: HR. Abu Dawud (no. 1522),
an-Nasai (III/53), Ahmad (V/245) dan al-Hakim (I/273 dan III/273). Hadits
ini dishahihkan oleh al-Hakim dan disepakati adz-Dzahabi, yang mana
kedudukan hadits itu seperti yang dikatakan keduanya, bahwa Nabi
\textit{shallallahu alaihi wa sallam} pernah memberikan wasiat kepada
Mu'adz agar dia mengucapkannya di setiap akhir shalat.\\
\textbf{Referensi}: Yazid bin Abdul Qadir Jawas. 2016. Kumpulan Do'a dari
Al-Quran dan As-Sunnah yang Shahih. Bogor: Pustaka Imam Asy-Syafi'i.
\end{document}