\documentclass[a4paper,12pt]{article}
\usepackage{arabtex}
\usepackage[bahasa] {babel}
\usepackage[top=2cm,left=3cm,right=3cm,bottom=3cm]{geometry}
\title{\Large Berlindung dari Berbuat Buruk}
\author{\small Hanifah Atiya Budianto\\
		\small contact.us@latex-dailyprayers.com}
\begin{document}
\sffamily
\maketitle
\fullvocalize
\setcode{arabtex}
\begin{arabtext}
\noindent
al-ll_ahumma ^gannibni-y munkarAti al-'a_hlAqi, wAl-'ahwA'i, wAl-'a`mAli,
wAl-'adwA'i.\\
\end{arabtext}
\noindent
\textbf{Artinya}:
\par
\indent
"Ya Allah, jauhkanlah aku dari berbagai kemunkaran akhlak, hawa nafsu, amal
perbuatan, dan, segala macam penyakit."\\\\
\par
\noindent
\textbf{Tingkatan Doa dan Sanad}: \textbf{Shahih}: HR. Ibnu Hibban (no.
956-\textit{at-Ta'liqatul His\^{a}n}) al-Hakim (I/532), dan dia berkata:
"Hadits ini shahih sesuai syarat Muslim." Dishahihkannya dan disepakati
oleh adz-Dzahabi. Lihat \textit{Shah\^{i}h al-Adzk\^{a}r} (1187/938).\\
\textbf{Referensi}: Yazid bin Abdul Qadir Jawas. 2016. Kumpulan Do'a dari
Al-Quran dan As-Sunnah yang Shahih. Bogor: Pustaka Imam Asy-Syafi'i.
\end{document}