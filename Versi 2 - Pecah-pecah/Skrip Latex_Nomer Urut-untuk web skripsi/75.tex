\documentclass[a4paper,12pt]{article}
\usepackage{arabtex}
\usepackage[bahasa] {babel}
\usepackage[top=2cm,left=3cm,right=3cm,bottom=3cm]{geometry}
\title{\Large Doa ketika Mendengar Adzan}
\author{\small Hanifah Atiya Budianto\\
		\small contact.us@latex-dailyprayers.com}
\begin{document}
\sffamily
\maketitle
\fullvocalize
\setcode{arabtex}
\indent
Terdapat lima hal yang disunnahkan ketika adzan dikumandangkan:\\\\
4. Membaca doa setelah adzan:
\begin{arabtext}
\noindent
al-ll_ahumma rabba h_a_dihi al-dda`waTi al-ttAmmaTi, wAl-.s.salATi
al-qA'imaTi, ^Ati mu.hammadaN ni al-wasi-ylaTa wAl-fa.di-ylaTa, wAb`a_thu
maqAmaN ma.hmu-wdA-ni alla_di-y wa`adtahu.\\
\end{arabtext}
\noindent
\textbf{Artinya}:
\par
\indent
"Ya Allah, Rabb Pemilik panggilan yang sempurna (adzan) ini dan shalat
(wajib) yang didirikan. Berikanlah \textit{al-wasilah} (derajat di Surga),
dan keutamaan kepada Muhammad \textit{Shallallahu ‘alaihi wa sallam}. Dan
bangkitkanlah beliau sehingga bisa menempati maqam terpuji yang telah
Engkau janjikan."\\\\
Baca juga doa ketika mendengar adzan ke 1-5\\
\par
\noindent
\textbf{Tingkatan Doa dan Sanad}: \textbf{Shahih}: HR. Al-Bukhari (no.
614)--Lihat \textit{Fathul B\^{a}ri} (II/94)--Abu Dawud (no. 529),
at-Tirmidzi (no. 211), an-Nasai (II/26-27), Ibnu Majah (no. 722). Adapun
tambahan \textit{"innaka la tukhliful-mi’ad(u)"} adalah \textbf{lemah}, jadi ia tidak
boleh diamalkan. Lihat kitab \textit{Irw\^{a}-ul Ghal\^{i}l} (I/260, 261).
Tidak ada juga tambahan: \textit{"waddarajata ar-rafi'ata" "ya
arhamarrohimina"} karena tidak ada asalnya.\\
\textbf{Referensi}: Yazid bin Abdul Qadir Jawas. 2016. Kumpulan Do'a dari
Al-Quran dan As-Sunnah yang Shahih. Bogor: Pustaka Imam Asy-Syafi'i.
\end{document}