\documentclass[a4paper,12pt]{article}
\usepackage{arabtex}
\usepackage[bahasa] {babel}
\usepackage[top=2cm,left=3cm,right=3cm,bottom=3cm]{geometry}
\title{\Large Doa ketika Mendengar Adzan}
\author{\small Hanifah Atiya Budianto\\
		\small contact.us@latex-dailyprayers.com}
\begin{document}
\sffamily
\maketitle
\fullvocalize
\setcode{arabtex}
\indent
Terdapat lima hal yang disunnahkan ketika adzan dikumandangkan:\\\\
3. Membaca shalawat kepada Nabi Muhammad \textit{Shallallahu ‘alaihi wa
sallam}.
\begin{arabtext}
\noindent
al-ll_ahumma .salli `alY mu.hammadiN wa`alY ^Ali mu.hammadiN, kamA .sallayta
`alY 'ibrAhi-yma wa`alY ^Ali 'ibrAhi-yma, 'innaka .hami-yduN ma^gi-yduN,
al-ll_ahumma bArik `alY mu.hammadiN wa`alY ^Ali mu.hammadiN, kamA bArakta
`alY 'ibrAhi-yma wa`alY ^Ali 'ibrAhi-yma, 'innaka .hami-yduN ma^gi-yduN.\\
\end{arabtext}
\noindent
\textbf{Artinya}:
\par
\indent
"Ya Allah, berikanlah shalawat kepada Nabi Muhammad beserta keluarga
Muhammad, sebagaimana Engkau telah memberikan shalawat kepada Ibrahim dan
keluarga Ibrahim. Sesungguhnya Engkau Maha Terpuji lagi Mahamulia.
Berikanlah berkah kepada Muhammad dan keluarga Muhammad sebagaimana Engkau
telah memberi berkah kepada Ibrahim beserta keluarga Ibrahim. Sesungguhnya
Engkau Maha Terpuji lagi Mahamulia."\\\\
Baca juga doa ketika mendengar adzan ke 1-5\\
\par
\noindent
\textbf{Tingkatan Doa dan Sanad}: \textbf{Shahih}: HR. Muslim (no. 384),
an-Nasai (II/25-26), Abu Dawud (no. 523), Ibnu Khuzaimah (no. 418), Ahmad
(II/168), Al-Baihaqi (I/409-410)dari Abdullah bin Amr bin al-Ash r.a.\\
\textbf{Referensi}: Yazid bin Abdul Qadir Jawas. 2016. Kumpulan Do'a dari
Al-Quran dan As-Sunnah yang Shahih. Bogor: Pustaka Imam Asy-Syafi'i.
\end{document}