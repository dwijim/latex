\documentclass[a4paper,12pt]{article}
\usepackage{arabtex}
\usepackage[bahasa] {babel}
\usepackage[top=2cm,left=3cm,right=3cm,bottom=3cm]{geometry}
\title{\Large Doa Tasyahud}
\author{\small Hanifah Atiya Budianto\\
		\small contact.us@latex-dailyprayers.com}
\begin{document}
\sffamily
\maketitle
\fullvocalize
\setcode{arabtex}
\begin{arabtext}
\noindent
al-tta.hiyyAtu li-ll_ahi, wAl-.s.salawAtu wAl-.t.tayyibAtu, al-ssalAmu
`alayka 'ayyuhA alnnabiyyu wara.hmaTu al-ll_ahu wabarakAtuhu, al-ssalAmu
`alaynA wa`alY `ibAdi al-ll_ahi al-.s.sAli.hiyna, 'a^shadu 'an lA 'il_aha
'illA al-ll_ahu, wa'a^shadu 'anna mu.hammadaN `abduhu warasu-wluhu.\\
\end{arabtext}
\noindent
\textbf{Artinya}:
\par
\indent
"Semua kesejahteraan, kerajaan, dan kekekalan; semua doa untuk mengagungkan
Allah; dan seluruh perkataan yang baik dan amal shalih hanyalah milik Allah
tercurah kepadamu Nabi. Semoga keselamatan dicurahkan kepada kami semua dan
hamba-hamba Allah yang shalih. Aku bersaksi bahwa tidak ada ilah yang
berhak diibadahi dengan benar selain Allah semata, tidak ada sekutu
bagi-Nya dan aku bersaksi bahwa Muhammad adalah hamba dan Rasul-Nya."\\\\
\par
\noindent
\textbf{Sanad}: HR. Al-Bukhari (no. 831, 835, 1202) dan
Muslim (no. 402 [55]).\\
\textbf{Referensi}: Yazid bin Abdul Qadir Jawas. 2016. Kumpulan Do'a dari
Al-Quran dan As-Sunnah yang Shahih. Bogor: Pustaka Imam Asy-Syafi'i.
\end{document}