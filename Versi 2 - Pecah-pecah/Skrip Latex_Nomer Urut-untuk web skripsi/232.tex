\documentclass[a4paper,12pt]{article}
\usepackage{arabtex}
\usepackage[bahasa] {babel}
\usepackage[top=2cm,left=3cm,right=3cm,bottom=3cm]{geometry}
\title{\Large Doa agar Menjadi Orang yang Banyak Berdzikir, Bersyukur, dan
Taat}
\author{\small Hanifah Atiya Budianto\\
		\small contact.us@latex-dailyprayers.com}
\begin{document}
\sffamily
\maketitle
\fullvocalize
\setcode{arabtex}
\begin{arabtext}
\noindent
al-l_ahumma 'a`inni-y `alY _dikrika, wa^sukrika, wa.husni `ibAdatika.\\
\end{arabtext}
\noindent
\textbf{Artinya}:
\par
\indent
"Ya Allah, tolonglah aku untuk dapat berdzikir kepada-Mu, dapat bersyukur
kepada-Mu, dan dapat beribadah dengan baik kepada-Mu."\\\\
\par
\noindent
\textbf{Tingkatan Doa dan Sanad}: \textbf{Shahih}: HR. Abu Dawud (no. 1522),
Ahmad (V/244-245, 247), an-Nasai (III/53), dan al-Hakim (I/273 dan III/273)
dan dishahihkannya, juga disepakati oleh adz-Dzahabi. Nabi SAW. pernah
berwasiat kepada Mu'adz r.a. agar dia mengucapkan dzikir tersebut pada
setiap akhir shalatnya atau sesudah salam dari shalat wajib.\\
\textbf{Referensi}: Yazid bin Abdul Qadir Jawas. 2016. Kumpulan Do'a dari
Al-Quran dan As-Sunnah yang Shahih. Bogor: Pustaka Imam Asy-Syafi'i.
\end{document}