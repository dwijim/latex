\documentclass[a4paper,12pt]{article}
\usepackage{arabtex}
\usepackage[bahasa] {babel}
\usepackage[top=2cm,left=3cm,right=3cm,bottom=3cm]{geometry}
\title{\Large Doa Ruku'}
\author{\small Hanifah Atiya Budianto\\
		\small contact.us@latex-dailyprayers.com}
\begin{document}
\sffamily
\maketitle
\fullvocalize
\setcode{arabtex}
\begin{arabtext}
\noindent
sub.hAna rabbiya al-`a.zi-ymi.\\
\end{arabtext}
\noindent
\textbf{Artinya}:
\par
\indent
"Mahasuci Rabbku Yang Mahaagung." \textbf{[Dibaca 3x]}.\\\\
\par
\noindent
\textbf{Tingkatan Doa dan Sanad}: \textbf{Shahih}: HR. Ahmad (V/382, 394),
Abu Dawud (no. 871), an-Nasai (II/190), at-Tirmidzi (no. 262) dan Ibnu Majah
(no. 888), lafazh ini miliknya. Lihat \textit{Irw\^{a}-ul Ghal\^{i}l} (no.
333 dan 334).\\
\textbf{Referensi}: Yazid bin Abdul Qadir Jawas. 2016. Kumpulan Do'a dari
Al-Quran dan As-Sunnah yang Shahih. Bogor: Pustaka Imam Asy-Syafi'i.
\end{document}