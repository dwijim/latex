\documentclass[a4paper,12pt]{article}
\usepackage{arabtex}
\usepackage[bahasa] {babel}
\usepackage[top=2cm,left=3cm,right=3cm,bottom=3cm]{geometry}
\title{\Large Doa Diberi Kebahagiaan dan Terhindar dari Kesengsaraan}
\author{\small Hanifah Atiya Budianto\\
		\small contact.us@latex-dailyprayers.com}
\begin{document}
\sffamily
\maketitle
\fullvocalize
\setcode{arabtex}
\begin{arabtext}
\noindent
al-ll_ahumma bi`ilmika al-.ga-yba, waqudratika `alY al-_halqi, 'a.hyini-y
mA `alimta al-.hayATa _ha-yraN li-y, watawaffani-y 'i_dA `alimta al-wafATa
_ha-yraN li-y, al-ll_ahumma wa-'as'aluka _ha^syataka fiy al-.ga-ybi
wAl-^s^sahAdaTi, wa-'as'aluka kalimaTa al-.haqqi fiy al-rri.dA
wAl-.ga.dabi, wa-'as'aluka al-qa.sda fiy al-faqri wAl-.ginY, wa-'as'aluka
na`i-ymaN lA yanfadu, wa-'as'aluka qurraTa `a-yniN lA tanqa.ti`u,
wa-'as'aluka al-rri.dA ba`da al-qa.dA'i, wa'as'aluka barda al-`ay^si ba`da
al-ma-wti, wa-'as'aluka la_d_daTa al-nna.zari 'ilY wa^ghika, wAl-^s^sa-wqa
'ilY liqA'ika, fi-y .ga-yri .darrA'a mu.dirraTiN, walA fitnaTiN
mu.dillaTiN, al-ll_ahumma zayyinnA bizi-ynaTi al-'iymAni, wA^g`alnA hudATaN
muhtadi-yna.\\
\end{arabtext}
\noindent
\textbf{Artinya}:
\par
\indent
"Ya Allah, dengan pengetahuan-Mu terhadap yang ghaib dan kekuasaan-Mu atas
semua makhluk, hidupkanlah aku jika Engkau mengetahui kehidupan itu lebih
baik bagiku, dan matikanlah aku jika Engkau mengetahui kematian itu lebih
baik bagiku. Ya Allah, dan aku mohon rasa takut kepada-Mu baik dalam
keadaan sembunyi maupun ketika terang-terangan. Dan aku pun memohon
kepada-Mu perkataan yang benar baik dalam keadaan senang maupun dalam
keadaan marah. Aku mohon kepada-Mu kesederhanaan baik saat dalam keadaan
fakir maupun saat dalam keadaan kaya. Aku memohon kepada-Mu nikmat yang
tidak pernah habis. Dan aku memohon kepada-Mu penyejuk hati yang tidak
pernah putus. Aku mohon kepada-Mu kerelaan menerima segala hal setelah
ditetapkan. Aku memohon kepada-Mu ketenteraman hidup setelah kematian. Dan
aku memohon pula kepada-Mu kenikmatan memandang wajah-Mu, juga kerinduan
untuk bertemu dengan-Mu, bukan ketika dalam keadaan kesusahan yang
membinasakan dan cobaan yang menyesatkan. Ya Allah, hiasilah kami dengan
hiasan iman dan jadikan kami termasuk orang-orang yang memberi petunjuk dan
diberi petunjuk."\\\\
\par
\noindent
\textbf{Tingkatan Doa dan Sanad}: \textbf{Shahih}: HR. An-Nasai
(III/54-55), Ahmad (IV/264), dan al-Hakim (I/524) dan lainnya dari Ammar
bin Yasir r.a. Sanadnya \textit{jayyid}. Lihat \textit{Shah\^{i}h
al-J\^{a}mi-us Shagh\^{i}r} (no. 1301). Lafazh doa ini boleh juga dibaca
setelah tasyahud sebelum salam. Lihat \textit{Shah\^{i}h al-Kalimith
Thayyib} (no. 106) Pasal 16, dan \textit{Shifatu Shal\^{a}tin Nabi} (hlm.
184) karya Syaikh Muhammad Nashiruddin al-Albani.\\
\textbf{Referensi}: Yazid bin Abdul Qadir Jawas. 2016. Kumpulan Do'a dari
Al-Quran dan As-Sunnah yang Shahih. Bogor: Pustaka Imam Asy-Syafi'i.
\end{document}