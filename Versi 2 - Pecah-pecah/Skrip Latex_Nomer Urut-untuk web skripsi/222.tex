\documentclass[a4paper,12pt]{article}
\usepackage{arabtex}
\usepackage[bahasa] {babel}
\usepackage[top=2cm,left=3cm,right=3cm,bottom=3cm]{geometry}
\title{\Large Doa Agar Diberi Ilmu Yang Bermanfaat dan Berlindung Dari Ilmu
Yang Tidak Bermanfaat}
\author{\small Hanifah Atiya Budianto\\
		\small contact.us@latex-dailyprayers.com}
\begin{document}
\sffamily
\maketitle
\fullvocalize
\setcode{arabtex}
\begin{arabtext}
\noindent
al-ll_ahumma anfa`ni-y nimA `allamtani-y, wa-`allimni-y mA yanfa`uni-y,
wazidni-y `ilmaN.\\
\end{arabtext}
\noindent
\textbf{Artinya}:
\par
\indent
"Ya Allah, berilah manfaat bagiku atas apa yang Engkau ajarkan kepadaku,
dan ajarkanlah kepadaku apa-apa yang bermanfaat bagiku, serta tambahkanlah
ilmu kepadaku."\\\\
\par
\noindent
\textbf{Tingkatan Doa dan Sanad}: \textbf{Shahih}: HR. At-Tirmidzi (no.
3599), Ibnu Majah (no. 251, 3833). Lihat \textit{Shah\^{i}h at-Tirmidzi}
(III/185, no. 2845) dan \textit{Shah\^{i}h Ibni Majah} (I/47, no. 203) dari
Abu Hurairah r.a. Lihat juga \textit{Silsilah Ah\^{a}d\^{i}ts
ash-Shah\^{i}hah} (no. 3151).\\
\textbf{Referensi}: Yazid bin Abdul Qadir Jawas. 2016. Kumpulan Do'a dari
Al-Quran dan As-Sunnah yang Shahih. Bogor: Pustaka Imam Asy-Syafi'i.
\end{document}