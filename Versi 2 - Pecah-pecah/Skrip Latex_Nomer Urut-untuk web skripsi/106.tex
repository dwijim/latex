\documentclass[a4paper,12pt]{article}
\usepackage{arabtex}
\usepackage[bahasa] {babel}
\usepackage[top=2cm,left=3cm,right=3cm,bottom=3cm]{geometry}
\title{\Large Doa setelah Tasyahud Akhir sebelum Salam}
\author{\small Hanifah Atiya Budianto\\
		\small contact.us@latex-dailyprayers.com}
\begin{document}
\sffamily
\maketitle
\fullvocalize
\setcode{arabtex}
\begin{arabtext}
\noindent
al-ll_ahumma 'inni-y 'as'aluka yA Aal-ll_ahu bi-'annaka al-wA.hidu
al-'a.hadu al-.s.samadu alla_di-y lam yalid walam yu-wlad walam yakun lahu
kufuwaN 'a.haduN, 'an ta.gfirali-y _dunu-wbi-y 'innaka 'anta al-.gafu-wru
al-rra.hi-ymu.\\
\end{arabtext}
\noindent
\textbf{Artinya}:
\par
\indent
"Ya Allah, sesungguhnya aku memohon kepada-Mu. Ya Allah, dengan bersaksi
Engkau adalah Rabb Yang Maha Esa, Mahatunggal yang tidak membutuhkan
sesuatu, tapi segala sesuatu yang butuh kepada-Mu, tidak beranak dan tidak
diperanakan (tidak mempunyai ibu maupun bapak), tidak seorang pun yang
menyamai-Mu, aku memohon agar Engkau mengampuni dosa-dosaku. Sesungguhnya
Engkau Maha Pengampun lagi Maha Penyayang."\\\\
\par
\noindent
\textbf{Tingkatan Doa dan Sanad}: \textbf{Shahih}: HR. An-Nasai (III/52) -
lafazh ini ialah miliknya-dan Ahmad (IV/338) dari Mihjan bin al-Adru r.a.
Dinyatakan shahih oleh Syaikh al-Albani dalam \textit{Shah\^{i}h an-Nasai}
(I/279, no. 1234).\\
\textbf{Referensi}: Yazid bin Abdul Qadir Jawas. 2016. Kumpulan Do'a dari
Al-Quran dan As-Sunnah yang Shahih. Bogor: Pustaka Imam Asy-Syafi'i.
\end{document}