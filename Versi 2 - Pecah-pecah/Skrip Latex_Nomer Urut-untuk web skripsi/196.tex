\documentclass[a4paper,12pt]{article}
\usepackage{arabtex}
\usepackage[bahasa] {babel}
\usepackage[top=2cm,left=3cm,right=3cm,bottom=3cm]{geometry}
\title{\Large Doa Mohon Ampunan dan Kasih Sayang}
\author{\small Hanifah Atiya Budianto\\
		\small contact.us@latex-dailyprayers.com}
\begin{document}
\sffamily
\maketitle
\fullvocalize
\setcode{arabtex}
\begin{arabtext}
\noindent
rabbi a.gfirli-y, watub `alayya, 'innaka 'anta al-ttawwAbu al-.gafu-wru.\\
\end{arabtext}
\noindent
\textbf{Artinya}:
\par
\indent
"Ya Rabbku, ampunilah aku, terimalah taubatku, sesungguhnya Engkau adalah
Yang Maha Penerima taubat lagi Yang Maha Pengampun."\\\\
\par
\noindent
\textbf{Tingkatan Doa dan Sanad}: Abdullah bin Umar berkata: "Aku
menghitung kalimat yang diucapkan Rasulullah: '\textit{Rabbighfirl\^{i}
watub 'alayya innaka antat taww\^{a}bul ghaf\^{u}r}' dalam satu majelis
sebanyak seratus kali." \textbf{Hasan Shahih}: HR. Abu Dawud (no. 1516),
at-Tirmidzi (no. 3434), Ibnu Majah (no. 3814). Lafazhnya milik at-Tirmidzi,
dan dia menyatakan: "Hadits \textit{hasan shahih gharib}." Lihat
\textit{Shah\^{i}h al-J\^{a}mi-us Shaghir} (no. 3486) dan \textit{Silsilah
Ah\^{a}d\^{i}ts ash-Shah\^{i}hah} (no. 556).\\
\textbf{Referensi}: Yazid bin Abdul Qadir Jawas. 2016. Kumpulan Do'a dari
Al-Quran dan As-Sunnah yang Shahih. Bogor: Pustaka Imam Asy-Syafi'i.
\end{document}