\documentclass[a4paper,12pt]{article}
\usepackage{arabtex}
\usepackage[bahasa] {babel}
\usepackage[top=2cm,left=3cm,right=3cm,bottom=3cm]{geometry}
\title{\Large Doa Mohon Ampunan dan Kasih Sayang}
\author{\small Hanifah Atiya Budianto\\
		\small contact.us@latex-dailyprayers.com}
\begin{document}
\sffamily
\maketitle
\fullvocalize
\setcode{arabtex}
\begin{arabtext}
\noindent
al-ll_ahumma 'inni-y 'as'aluka yA Aal-ll_ahu, bi-'annaka al-wA.hidu
al-'a.hadu al-.s.samadu, alla_di-y lam yalid walam yu-wlad, walam yakun
lahu kufu-waN 'a.haduN, 'an ta.gfira li-y _dunu-wbi-y, 'innaka 'anta
al-.gafu-wru al-rra.hi-ymu.\\
\end{arabtext}
\noindent
\textbf{Artinya}:
\par
"Ya Allah, sesungguhnya aku memohon kepada-Mu ya Allah, karena Engkau
adalah satu-satunya Yang Maha Esa, yang bergantung kepada-Mu seluruh
makhluk, yang tidak beranak dan tidak pula diperanakkan, serta tidak ada
seorang pun yang sebanding dengan-Nya, agar Engkau memberikan ampunan
kepadaku atas dosa-dosaku, sesungguhnya Engkau Maha Pengampun lagi Maha
Penyayang."\\\\
\par
\noindent
\textbf{Tingkatan Doa dan Sanad}: \textbf{Shahih}: HR. An-Nasai dengan
lafazhnya (III/52), Ahmad (IV/338). Lihat \textit{Shah\^{i}h an-Nasai}
(I/279). Pada akhir riwayat, Nabi SAW. bersabda: "Allah telah mengampuni
dosanya." - Beliau mengucapkannya tiga kali.\\
\textbf{Referensi}: Yazid bin Abdul Qadir Jawas. 2016. Kumpulan Do'a dari
Al-Quran dan As-Sunnah yang Shahih. Bogor: Pustaka Imam Asy-Syafi'i.
\end{document}