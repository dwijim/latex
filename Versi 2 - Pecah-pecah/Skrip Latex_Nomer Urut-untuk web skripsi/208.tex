\documentclass[a4paper,12pt]{article}
\usepackage{arabtex}
\usepackage[bahasa] {babel}
\usepackage[top=2cm,left=3cm,right=3cm,bottom=3cm]{geometry}
\title{\Large Doa Menghadapi Kesulitan}
\author{\small Hanifah Atiya Budianto\\
		\small contact.us@latex-dailyprayers.com}
\begin{document}
\sffamily
\maketitle
\fullvocalize
\setcode{arabtex}
\begin{arabtext}
\noindent
al-ll_ahumma ra.hmataka 'ar^gu-w, falA takilni-y 'ilY nafsi-y .tarfaTa
`a-yniN, wa'a.sli.h li-y ^sa'ni-y kullahu, lA 'il_aha 'illA 'anta.\\
\end{arabtext}
\noindent
\textbf{Artinya}:
\par
\indent
"Ya Allah, rahmat-Mu yang selalu aku harapkan, maka janganlah Engkau
serahkan urusanku kepada diriku meski hanya sekejap mata, dan perbaikilah
urusanku semuanya, tidak ada ilah yang berhak diibadahi dengan benar selain
Engkau".\\\\
\par
\noindent
\textbf{Tingkatan Doa dan Sanad}: \textbf{Hasan}: HR. Abu Dawud (no. 5090)
dan Ahmad (V/42). Dihasankan oleh Syaikh al-Albani dan selainnya. Lihat
kitab \textit{Shah\^{i}h al-Adabil Mufrad} (no. 539) dan \textit{Shah\^{i}h
al-Adzk\^{a}r} (351/251).\\
\textbf{Referensi}: Yazid bin Abdul Qadir Jawas. 2016. Kumpulan Do'a dari
Al-Quran dan As-Sunnah yang Shahih. Bogor: Pustaka Imam Asy-Syafi'i.
\end{document}