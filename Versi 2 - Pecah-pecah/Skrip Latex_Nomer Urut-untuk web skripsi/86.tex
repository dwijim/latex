\documentclass[a4paper,12pt]{article}
\usepackage{arabtex}
\usepackage[bahasa] {babel}
\usepackage[top=2cm,left=3cm,right=3cm,bottom=3cm]{geometry}
\title{\Large Doa Ruku'}
\author{\small Hanifah Atiya Budianto\\
		\small contact.us@latex-dailyprayers.com}
\begin{document}
\sffamily
\maketitle
\fullvocalize
\setcode{arabtex}
\begin{arabtext}
\noindent
subbu-w.huN quddu-wsuN, rabbu al-malA'ikaTi wAl-rru-w.hi.\\
\end{arabtext}
\noindent
\textbf{Artinya}:
\par
\indent
"Engkau Rabb Yang Mahasuci (dari kekurangan dan yang tidak layak bagi
kebesaran-Mu), Rabb seluruh Malaikat dan Jibril."\\\\
\par
\noindent
\textbf{Tingkatan Doa dan Sanad}: \textbf{Shahih}: HR. Muslim (no. 487), Abu
Dawud (no. 872), an-Nasai (II/191), dan Ahmad (VI/35).\\
\textbf{Referensi}: Yazid bin Abdul Qadir Jawas. 2016. Kumpulan Do'a dari
Al-Quran dan As-Sunnah yang Shahih. Bogor: Pustaka Imam Asy-Syafi'i.
\end{document}