\documentclass[a4paper,12pt]{article}
\usepackage{arabtex}
\usepackage[bahasa] {babel}
\usepackage[top=2cm,left=3cm,right=3cm,bottom=3cm]{geometry}
\usepackage{xcolor, framed}
\definecolor{shadecolor}{rgb}{0.8,0.8,0.8}
\title{\Large Doa Saat Mengalami Kesusahan, Kesedihan, dan Penawar
Kedukaan}
\author{\small Hanifah Atiya Budianto\\
		\small contact.us@latex-dailyprayers.com}
\begin{document}
\sffamily
\maketitle
\fullvocalize
\setcode{arabtex}
\begin{arabtext}
\noindent
al-ll_ahu, al-ll_ahu rabbi-y, lA 'u^sriku bihi ^say'aN.\\
\end{arabtext}
\noindent
\textbf{Artinya}:
\par
\indent
"Allah, Allah  adalah Rabbku, aku tidak menyekutukan-Nya dengan sesuatu
apapun."\\\\
\par
\noindent
\textbf{Tingkatan Doa dan Sanad}: \textbf{Shahih}: HR. Abu Dawud (no.
1525), Ibnu Majah (no. 3882), dan lihat \textit{Silsilah Ah\^{a}d\^{i}ts
ash-Shah\^{i}hah} (no. 2755).\\
\textbf{Referensi}: Yazid bin Abdul Qadir Jawas. 2016. Kumpulan Do'a dari
Al-Quran dan As-Sunnah yang Shahih. Bogor: Pustaka Imam Asy-Syafi'i.
\end{document}