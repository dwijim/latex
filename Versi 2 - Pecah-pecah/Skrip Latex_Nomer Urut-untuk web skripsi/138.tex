\documentclass[a4paper,12pt]{article}
\usepackage{arabtex}
\usepackage[bahasa] {babel}
\usepackage[top=2cm,left=3cm,right=3cm,bottom=3cm]{geometry}
\title{\Large Disunnahkan bagi Musafir agar Bertakbir pada Jalan Mendaki
dan Bertasbih ketika Menurun}
\author{\small Hanifah Atiya Budianto\\
		\small contact.us@latex-dailyprayers.com}
\begin{document}
\sffamily
\maketitle
\fullvocalize
\setcode{arabtex}
\begin{arabtext}
\noindent
`an ^gAbiribni `abdi al-ll_ahi ra.di-y al-l_ahu `anhumA: kunnA 'i_dA
.sa`idnA kabbarnA, wa 'i_dA nazalnA sabba.hnA.\\
\end{arabtext}
\noindent
\textbf{Artinya}:
\par
\indent
Dari Jabir bin Abdillah radhiyallahu 'anhuma, ia berkata: "Kami membaca
takbir apabila berjalan naik, dan kami membaca tasbih apabila berjalan
menurun."\\\\
\par
\noindent
\textbf{Tingkatan Doa dan Sanad}: \textbf{Shahih}: HR. Al-Bukhari (no.
2993)/\textit{Fathul B\^{a}ri} (VI/135).\\
\textbf{Referensi}: Yazid bin Abdul Qadir Jawas. 2016. Kumpulan Do'a dari
Al-Quran dan As-Sunnah yang Shahih. Bogor: Pustaka Imam Asy-Syafi'i.
\end{document}