\documentclass[a4paper,12pt]{article}
\usepackage{arabtex}
\usepackage[bahasa] {babel}
\usepackage[top=2cm,left=3cm,right=3cm,bottom=3cm]{geometry}
\title{\Large Berlindung dari Sifat yang Jelek dan Mohon Dibersihkan Hati}
\author{\small Hanifah Atiya Budianto\\
		\small contact.us@latex-dailyprayers.com}
\begin{document}
\sffamily
\maketitle
\fullvocalize
\setcode{arabtex}
\begin{arabtext}
\noindent
al-ll_ahumma 'inni-y 'a`u-w_du bika mina al-`a-^gzi, wAl-kasali,
wAl-^gubni, wAl-bu_hli, wAl-harami, wa`a_dAbi al-qabri, al-ll_ahumma
^Ati nafsi-y taqwAhA wazakkihA 'anta _ha-yru zakkAhA, 'anta waliyyuhA
wama-wlAhA, al-ll_ahumma 'inni-y 'a`u-w_du bika min `ilmiN lA yanfa`u,
wamin qalbiN lA ya_h^sa`u, wamin nafsiN lA ta^sba`u, waman da`waTiN lA
yusta^gAbu lahA.\\
\end{arabtext}
\noindent
\textbf{Artinya}:
\par
\indent
"Ya Allah, sesungguhnya aku memohon perlindungan kepadamu dari kelemahan,
kemalasan, sifat pengecut, kekikiran, pikun, dan dari adzab kubur. Ya
Allah, berikanlah ketakwaan pada diriku dan sucikanlah ia, karena Engkaulah
sebaik-baik Rabb yang mensucikannya, Engkau Pelindung dan Pemeliharanya. Ya
Allah, sesungguhnya aku berlindung kepada-Mu dari ilmu yang tidak
bermanfaat, hati yang tidak khusyu', nafsu yang tidak pernah puas, dan doa
yang tidak dikabulkan."\\\\
\par
\noindent
\textbf{Tingkatan Doa dan Sanad}: \textbf{Shahih}: HR. Muslim (no. 2722)
dan an-Nasai (VIII/269) dari Zaid bin al-Arqam r.a.\\
\textbf{Referensi}: Yazid bin Abdul Qadir Jawas. 2016. Kumpulan Do'a dari
Al-Quran dan As-Sunnah yang Shahih. Bogor: Pustaka Imam Asy-Syafi'i.
\end{document}