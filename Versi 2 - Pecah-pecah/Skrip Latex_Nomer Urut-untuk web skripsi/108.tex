\documentclass[a4paper,12pt]{article}
\usepackage{arabtex}
\usepackage[bahasa] {babel}
\usepackage[top=2cm,left=3cm,right=3cm,bottom=3cm]{geometry}
\title{\Large Bacaan Setelah Salam}
\author{\small Hanifah Atiya Budianto\\
		\small contact.us@latex-dailyprayers.com}
\begin{document}
\sffamily
\maketitle
\fullvocalize
\setcode{arabtex}
\begin{arabtext}
\noindent
'asta.gfiru al-ll_aha.\\
Aal-ll_ahumma 'anta al-ssalAmu, waminka al-ssalAmu, tabArakta yA_dA
al-^galAli wAl-'ikrAmi.\\
\end{arabtext}
\noindent
\textbf{Artinya}:
\par
\indent
"Aku memohon ampun kepada Allah \textbf{[3x]}.\\
Ya Allah, Engkau Mahasejahtera, dan dari-Mu kesejahteraan, Mahasuci Engkau,
wahai Rabb Pemilik keagungan dan kemuliaan.\\\\
Baca juga bacaan setelah salam ke 1-8\\
\par
\noindent
\textbf{Tingkatan Doa dan Sanad}: \textbf{Shahih}: HR. Muslim (no. 591
[135]), Ahmad (V/275, 279), Abu Dawud (no. 1513), Ibnu Khuzaimah (no. 737),
an-Nasai (III/68), ad-Darimi (I/311), dan Ibnu Majah (no. 928) dari Tsauban.
\\
\textbf{Referensi}: Yazid bin Abdul Qadir Jawas. 2016. Kumpulan Do'a dari
Al-Quran dan As-Sunnah yang Shahih. Bogor: Pustaka Imam Asy-Syafi'i.
\end{document}