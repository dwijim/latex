 \documentclass[a4paper,12pt]{article}
\usepackage{arabtex} 
\usepackage[bahasa] {babel}
\usepackage[top=1cm,left=1cm,right=1cm,bottom=1cm]{geometry}
\begin{document}
\sffamily 
\fullvocalize
\setcode{arabtex}
1------------------ Doa Mohon Ampun dan Rahmat Allah
\begin{arabtext}
\noindent
1) rabbi 'inn^I 'a`uw_du bika 'an 'as'alaka mA laysa liY bihi `ilmuN wa-'illA ta.gfir liY watar.hamn^I 'akun mmina al-_h_asiriyna\\
2) rabban^A -'a-amannA fa-a.gfir lanA wa-ar.hamnA wa'anta _hayru al-rr_a.himiyna.\\
3) rabbi a.gfir wa-ar.ham wa'anta _hayru al-rr_a.himiyna.\\
4) rabban^A 'innan^A -'a-amannA fa-a.gfir lanA _dunuwbanA waqinA `a_dAba al-nnAri.\\
5) rabbi 'inniY .zalamtu nafsiY fa-a.gfir liY.\\
6) rabban^A 'innanA sami`nA munAdiyaN yunAdiY lil-'iym_ani 'an -'a-aminuW birabbikum fa'aamannA rabbanA fa-a.gfir lanA _dunuwbanA wakaffir `annA sayyi'AtinA watawaffanA ma`a al-'abrAri  $\odot$ rabbanA wa -'a-atinA mA wa`adttanA `al_aY rusulika walA tu_hzinA yawma al-qiy_amaTi 'innaka lA tu_hlifu almiy`Ada.\\
7) rabbanA lA tu'A_hi_dn^A 'in nnasiyn^A 'aw 'a_h.ta'nA, rabbanA walA ta.hmil `alayn^A 'i.sraN kamA .hamaltah_u `alaY alla_dina min qablinA, rabbanA walA tu.hammilnA mA lA .tAqaTa lanA bih_i wa-a`fu `annA wa-a.gfir lanA wa-ar.hamna-^A, 'anta mawl_anA fa-an.surnA `alaY al-qawmi alk_afiriyna.\\
8) rabbanA a.gfir lanA _dunuwbanA wa-'isrAfanA f^I 'amrinA wa_tabbit 'aqdAmanA wa-an.surnA `alaY alqawmi al-k_afiriyna.\\
9) rabbanA .zalamn^A 'anfusanA wa-'in llam ta.gfir lanA watar.hamnA lanakuw nanna mina al-_h_asiriyna.\\
\end{arabtext}
\\
11------------------ Doa Supaya Dijadikan Hamba yang Bersyukur
\begin{arabtext}
\noindent
1) rabbi 'awzi`n^I 'an 'a^skura ni`mataka allat^I 'an`amta `alaYYa wa`alaY_a w_alidaYYa wa'an 'a`mala .s_ali.haN tar.d_ahu wa'ad_hilniY bira.hmatika fiY `ibAdika al-.s.s_ali.hiyna.\\
2) rabbi 'awzi`n^I 'an 'a^skura ni`mataka allat^I 'an`amta `alaYYa wa`alaY_a w_alidaYYa wa'an 'a`mala .s_ali.haN tar.d_ahu wa'a.sli.h liY fiY _durriyyat^I 'inniY tubtu 'ilayka wa-'inniY mina al-muslimiyna.
\end{arabtext}
\\
12------------------ Doa supaya Dilapangkan Hati serta Dimudahkan Urusan
\begin{arabtext}
\noindent
1) lla-^A 'il_aha 'illa-^A 'anta sub.h_anaka 'inniY kuntu mina al-.z.z_alimiyna.\\
2) rabbi a^sra.h liY .sadriY  $\odot$ wayassir l^I 'amriY  $\odot$ wa-a.hlul `uqdaTaN mmin  $\odot$ llisAniY yafqahu-W qawliY  $\odot$.\\
3) rabbana-^A -'a-atinA min lladunka ra.hmaTaN wahayyi' lanA min 'amrinA ra^sadaN
\end{arabtext}
\\
21------------------ Doa Agar Dibimbing Ke Jalan Yang Lurus
\begin{arabtext}
\noindent
`asa_A_a rabb^I 'an yahdiyaniY sawa'A'a al-ssabiyli.
\end{arabtext}
\\
22------------------ Doa dan Dzikir sebelum Tidur
\begin{arabtext}
\noindent
1) ^gama`a kaffayhi _tumma nafa_ta fi-yhimA faqara'a fi-yhimA: (qul huwa al-llahu 'a.haduN) (qul 'a`uw_du birabbi al-falaqi) (qul 'a`uw_du birabbi al-nnAsi) _tumma yamsa.hu bihimA mA asta.tA`a min ^gasadihi yabda'u bihimA `alY ra'sihi wawa^ghihi wamA 'aqbala min ^gasadihi.\\
2) al-llahu l^A 'il_aha 'ilA huwa al-.haYYu al-qayyu-wmu, lA ta'_hu_duhu sinaTuN walA na-wmuN, llahu mA fiY al-ssam_aw_ati wamA fiY al-'ar.di, man _dA alla_diY ya^sfa`u `indahu 'illA bi-'i-_dnihi, ya`lamu mA ba-yna 'aydi-yhim wamA _halfahum walA yu.hi-y.tuwna bi^saY'iN mmin `ilmihi, 'illA bimA ^sa-^A'a, wasi`a kursiyyuhu al-ssam_aw_ati wAl-'ar.da, walA ya'uduhu .hif.zuhumA, wahuwa al-`aliYYu al-`a.zi-ymu.\\
3) -'a-amana al-rrasuwlu bima-^A 'unzila 'ila-yhi min rrabbihi, wa-al-mu'-minuwna kulluN -'a-amana bi-al-llahi wamal^A'ikatihi, wakutubihi, warusulihi, lA nufarriqu ba-yna 'a.hadiN mmin rrusulihi, waqAluW sami`nA wa'a.ta`nA, .gufrAnaka rabbanA wa-'ila-yka al-ma.siyru (285). lA yukallifu al-llahu nafsaN 'illA wus`ahA lahA mA kasabat wa`ala-yhA mA aktasabat, rabbanA lA tu'A _hi_dn^A 'in nnasi-yn^A 'aw'a_h.ta'nA, rabbanA walA ta.hmil `ala-yn^A 'i.sraN kamA .hamaltahu, `ala alla_diyna min qablinA, rabbanA walA tu.hammilnA mA lA .tAqaTalanA bihi, wa-a`fu `annA wa-a.gfirlanA wa-ar.hamn^A, 'anta ma-wl_anA fa-an.surnA `alY al-qa-wmi al-k_afiriyna. (286)\\
4) al-ll_ahumma 'aslamtu tafsi-y 'ila-yka, wawa^g^gahtu wa^ghi-y 'ila-yka, wafawwa.dtu  'amri-y 'ila-yka, wa'al^ga'tu .zahri-y 'ila-yka, ra.gbaTaN warahbaTaN 'ila-yka, lA mal^ga'a walA man^gA minka 'illA 'ila-yka ^Amantu bikitAbika alla_diY 'anzalta wabinabiyyika alla_diY 'arsalta.\\
5) bi-asmika rabbi-y wa.da`tu ^ganbi-y, wabika 'arfa`uhu, 'in 'amsakta nafsi-y fAr.hamhA, wa-'in 'arsaltahA fA.hfa.zhA bimA ta.hfa.zu bihi `ibAdaka al-.s.sAli.hi-yna.\\
6) al-ll_ahumma _halaqta nafsi-y wa'anta tawaffAhA, laka mamAtuhA wama.hyAhA, 'in 'a.hyaytahA fA.hfa.zhA, wa-'in 'amattahA fA.gfirlahA. al-ll_ahumma 'inni-y 'as'aluka al-`AfiyaTa.\\
7) al-ll_ahumma qini-y `a_dAbaka ya-wma tab`a_tu `ibAdaka.\\
8) bi-asmika al-ll_ahumma 'amu-wtu wa'a.hyA.
\end{arabtext}
\\
50------------------ Bacaan Setelah Salam
\begin{arabtext}
\noindent
1) 'asta.gfiru al-ll_aha.  Aal-ll_ahumma 'anta al-ssalAmu, waminka al-ssalAmu, tabArakta yA_dA al-^galAli wAl-'ikrAmi.\\
2) lA 'il_aha 'illA al-ll_ahu wa.hdahu lA ^sari-yka lahu, lahu al-mulku walahu al-.hamdu wahuwa `alY kulli ^say'iN qadi-yruN, Aal-ll_ahumma lA mAni`i limA 'a`.ta-yta, walA mu`.tiya limA mana`ta, walA yanfa`u _dA al-^gaddi minka al-^gaddu.\\
3) lA 'il_aha 'illA al-ll_ahu wa.hdahu lA ^sari-yka lahu, lahu al-mulku walahu al-.hamdu wahuwa `alY kulli ^say'iN qadi-yruN, lA .ha-wla walA quwwaTa 'illA bi-al-ll_ahi, lA 'il_aha 'illA al-ll_ahu, walA na`budu 'illA 'iyyAhu, lahu al-nni`maTu, walahu al-fa.dlu, walahu al-_t_tanA'u al-.hasanu, lA 'il_aha 'illA al-ll_ahu mu_hli.si-yna lahu al-ddi-yna wala-w kariha al-kAfiru-wna.\\
4) lA 'il_aha 'illA al-ll_ahu wa.hdahu lA ^sari-yka lahu, lahu al-mulku, walahu al-.hamdu, yu.hyi-y wayumi-ytu, wahuwa `alY kulli ^sa-y'iN qadi-yruN.\\
5) Aal-ll_ahumma 'a`inni-y `alY _dikrika, wa^sukrika, wa.husni `ibAdatika.\\
6) sub.hAna al-ll_ahi. al-.hamdu li-ll_ahi. Aal-ll_ahu 'akbaru\\
7) lA 'il_aha 'illA al-ll_ahu wa.hdahu lA ^sari-yka lahu, lahu al-mulku walahu al-.hamdu, wahuwa `alY kulli ^say'iN qadi-yruN.\\
8) Aal-ll_ahumma 'inni-y 'as'aluka `ilmaN nAfi`aN, warizqaN .tayyibaN, wa`amalaN wataqabbalaN.
\end{arabtext}
\\
151------------------ Bacaan ketika Berada di Atas Bukit Shafa dan Marwah\\
\begin{arabtext}
\noindent
1) ( 'inna al-.s.safA wa-ulmarwaTa min ^sa`a-^A'iri al-llahi ) ( 'abda'u bimA 
bada'a al-ll_ahu bihi ).\\
2) lA 'il_aha 'illA al-ll_ahu wa.hdahu lA ^sari-yka lahu, lahu al-mulku, 
walahu al-.hamdu, wahuwa `alaY kulli ^saY'iN qadi-yruN. lA 'il_aha 'illA 
al-ll_ahu wa.hdahu lA ^sari-yka lahu, 'an^gaza wa`dahu, wana.sara `abdahu, 
wahazama al-'a.hzAba wa.hdahu.\\
\end{arabtext}
\end{document}