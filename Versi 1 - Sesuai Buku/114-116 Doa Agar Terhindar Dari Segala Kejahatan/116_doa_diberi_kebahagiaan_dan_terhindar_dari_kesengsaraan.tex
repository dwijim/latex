\documentclass[a4paper,12pt]{article}
\usepackage{arabtex} 
\usepackage[bahasa] {babel}
\usepackage{calligra}
\usepackage[top=2cm,left=3cm,right=3cm,bottom=3cm]{geometry}
\title{\Large Doa Diberi Kebahagiaan dan Terhindar dari Kesengsaraan}
\author{\calligra Hanifah Atiya Budianto}
\begin{document}
\sffamily
\maketitle 
\fullvocalize
\setcode{arabtex}
\begin{arabtext}
\noindent
al-ll_ahumma laka al-.hamdu kulluhu, al-ll_ahumma lA qAbi.da limA basa.tta,
walA bAsi.ta limA qaba.dta, walA hAdiya liman 'a.dlalta, walA mu.dilla 
liman hada-yta, walA mu`.tiya limA mana`ta, walA mAni`a limA 'a`.tayta, 
walA muqarriba limA bA`adta, walA mubA`ida limA qarrabta, al-ll_ahumma 
absu.t `ala-ynA min barakAtika, wara.hmatika, wafa.dlika, warizqika, 
al-ll_ahumma 'inni-y 'as'aluka al-nna`i-yma al-muqi-yma, alla_di-y lA 
ya.hu-wlu walA yazu-wlu, al-ll_ahumma 'inni-y 'as'aluka al-nna`i-yma ya-wma
al-`a-ylaTi, wAl-'amna yawma al-_hawfi, al-ll_ahumma 'inni-y `A'i_duN bika 
min ^sarri mA 'a`.ta-ytanA, wa^sarri mA mana`tanA, al-ll_ahumma .habbib 
'ila-ynA al-'i-ymAna, wazayyinhu fi-y qulu-wbinA, wakarrih 'ila-ynA 
al-kufra, wAl-fusu-wqa, wAl-`i.syAna, wA^g`alnA mina al-rrA^sidi-yna, 
al-ll_ahumma tawaffanA muslimi-yna, wa-'a.hyinA muslimi-yna, wa-'al.hiqnA
bi-al-.s.sAli.hi-yna, .ga-yra _hazAyA walA maftu-wni-yna, al-ll_ahumma 
qAtili al-kafaraTa alla_di-yna yuka_d_dibu-wna rusulaka, waya.suddu-wna 
`an sabi-ylika, wA^g`al `alayhim ri^gzaka wa-`a_dAbaka, al-ll_ahumma qAtili
al-kafaraTa alla_di-yna 'uwtuW al-kitAba, 'il_aha al.haqqi (^Ami-yn).\\
\end{arabtext}
\noindent
\textbf{Artinya}:
\par
\indent
"Ya Allah, segala puji hanya bagi-Mu. Ya Allah, tidak ada yang dapat 
menahan apa yang telah Engkau lapangkan dan tidak ada yang dapat 
melapangkan apa yang Engkau tahan, tidak ada yang dapat memberikan petunjuk
kepada orang yang telah Engkau sesatkan, dan tidak ada yang dapat 
menyesatkan orang yang telah Engkau beri petunjuk, tidak ada yang dapat 
memberi apa yang telah Engkau cegah, dan tidak ada yang dapat mencegah apa 
yang Engkau berikan, tidak ada yang dapat mendekatkan apa yang telah Engkau
jauhkan, dan tidak ada pula yang dapat menjauhkan apa yang telah Engkau 
dekatkan. Ya Allah, lapangkanlah keberkahan, juga rahmat, karunia, beserta 
rizki-Mu kepada kami. Ya Allah, sesungguhnya aku memohon kepada-Mu 
kenikmatan yang abadi yang tidak akan berubah dan tidak pula lenyap. Ya 
Allah, sesungguhnya aku memohon kenikmatan pada hari kesengsaraan, dan 
keamanan pada hari ketakutan. Ya Allah, sungguh aku berlindung kepada-Mu 
dari kejelekan apa yang Engkau berikan kepada kami dan kejelekan apa yang 
Engkau cegah dari sisi kami. Ya Allah, jadikan kami cinta terhadap 
keimanan. Hiasilah ia dalam hati kami dan tanamkanlah kebencian kepada kami
terhadap kekufuran, kefasikan, dan kemaksiatan, serta jadikanlah kami 
termasuk orang-orang yang mengikuti jalan yang lurus. Ya Allah, wafatkan 
dan hidupkanlah kami dalam keadaan Muslim, dan pertemukan kami dengan 
orang-orang yang shalih dalam keadaan tidak terhina dan tidak pula 
terfitnah. Ya Allah, perangilah orang-orang kafir yang mendustakan 
Rasul-Rasul-Mu dan menghadang jalan-Mu, timpakan kepada mereka siksaan 
serta adzab. Ya Allah, perangilah orang-orang kafir yang telah diberi 
al-Kitab, wahai Ilah Yang Mahabenar (kabulkanlah, ya Allah)."{\scriptsize 
1}\\
\begin{arabtext}
\noindent
al-ll_ahumma bi`ilmika al-.ga-yba, waqudratika `alY al-_halqi, 'a.hyini-y 
mA `alimta al-.hayATa _ha-yraN li-y, watawaffani-y 'i_dA `alimta al-wafATa 
_ha-yraN li-y, al-ll_ahumma wa-'as'aluka _ha^syataka fiy al-.ga-ybi 
wAl-^s^sahAdaTi, wa-'as'aluka kalimaTa al-.haqqi fiy al-rri.dA 
wAl-.ga.dabi, wa-'as'aluka al-qa.sda fiy al-faqri wAl-.ginY, wa-'as'aluka 
na`i-ymaN lA yanfadu, wa-'as'aluka qurraTa `a-yniN lA tanqa.ti`u, 
wa-'as'aluka al-rri.dA ba`da al-qa.dA'i, wa'as'aluka barda al-`ay^si ba`da 
al-ma-wti, wa-'as'aluka la_d_daTa al-nna.zari 'ilY wa^ghika, wAl-^s^sa-wqa 
'ilY liqA'ika, fi-y .ga-yri .darrA'a mu.dirraTiN, walA fitnaTiN 
mu.dillaTiN, al-ll_ahumma zayyinnA bizi-ynaTi al-'iymAni, wA^g`alnA hudATaN
muhtadi-yna.\\
\end{arabtext}
\noindent
\textbf{Artinya}:
\par
\indent
"Ya Allah, dengan pengetahuan-Mu terhadap yang ghaib dan kekuasaan-Mu atas 
semua makhluk, hidupkanlah aku jika Engkau mengetahui kehidupan itu lebih 
baik bagiku, dan matikanlah aku jika Engkau mengetahui kematian itu lebih 
baik bagiku. Ya Allah, dan aku mohon rasa takut kepada-Mu baik dalam 
keadaan sembunyi maupun ketika terang-terangan. Dan aku pun memohon 
kepada-Mu perkataan yang benar baik dalam keadaan senang maupun dalam 
keadaan marah. Aku mohon kepada-Mu kesederhanaan baik saat dalam keadaan 
fakir maupun saat dalam keadaan kaya. Aku memohon kepada-Mu nikmat yang 
tidak pernah habis. Dan aku memohon kepada-Mu penyejuk hati yang tidak 
pernah putus. Aku mohon kepada-Mu kerelaan menerima segala hal setelah 
ditetapkan. Aku memohon kepada-Mu ketenteraman hidup setelah kematian. Dan 
aku memohon pula kepada-Mu kenikmatan memandang wajah-Mu, juga kerinduan 
untuk bertemu dengan-Mu, bukan ketika dalam keadaan kesusahan yang 
membinasakan dan cobaan yang menyesatkan. Ya Allah, hiasilah kami dengan 
hiasan iman dan jadikan kami termasuk orang-orang yang memberi petunjuk dan
diberi petunjuk."{\scriptsize 2}\\
\begin{arabtext}
\noindent
al-ll_ahumma a.hfa.zni-y bi-al-'islAmi qA'imaN, wA.hfa.zni-y bi-al-'islAmi 
qA`idaN, wA.hfa.zni-y bi-al-'islAmi rAqidaN, walA tu^smit bi-y `aduwwaN 
walA .hAsidaN. al-ll_ahumma 'inni-y 'as'aluka min kulli _ha-yriN 
_hazA'inuhu biyadika, wa-'a-`u-w_du bika min kulli ^sarriN _hazA'inuhu 
biyadika.\\
\end{arabtext}
\noindent
\textbf{Artinya}:
\par
\indent
"Ya Allah, peliharalah aku dengan Islam ini ketika sedang berdiri, 
peliharalah aku dengan Islam ini saat sedang duduk, dan peliharalah aku 
dengan Islam ini dalam keadaan tidur. Dan janganlah Engkau jadikan musuh 
dan orang yang dengki gembira karena kedukaanku. Ya Allah, sungguh aku 
memohon segala kebaikan yang tiap perbendaharaannya ada di tangan-Mu, dan 
aku berlindung kepada-Mu dari segala kejahatan yang tiap perbendaharaannya 
ada di tangan-Mu."{\scriptsize 3}\\\\
\par
\noindent
\textbf{Tingkatan Doa dan Sanad}: 
\begin{enumerate}
\item \textbf{Shahih}: HR. Ahmad dengan lafazhnya (III/424), al-Hakim 
(I/507)-yang dalam kurung miliknya (III/23-24)-al-Bukhari dalam 
\textit{al-Adabul Mufrad} (no. 699). Dishahihkan oleh Syaikh al-Albani 
dalam \textit{Takhr\^{i}j Fiqhis S\^{i}rah} (hlm. 284) dan 
\textit{Shah\^{i}h al-Adabil Mufrad} (no. 541).
\item \textbf{Shahih}: HR. An-Nasai (III/54-55), Ahmad (IV/264), dan 
al-Hakim (I/524) dan lainnya dari Ammar bin Yasir r.a. Sanadnya 
\textit{jayyid}. Lihat \textit{Shah\^{i}h al-J\^{a}mi-us Shagh\^{i}r} (no. 
1301). Lafazh doa ini boleh juga dibaca setelah tasyahud sebelum salam. 
Lihat \textit{Shah\^{i}h al-Kalimith Thayyib} (no. 106) Pasal 16, dan 
\textit{Shifatu Shal\^{a}tin Nabi} (hlm. 184) karya Syaikh Muhammad 
Nashiruddin al-Albani.
\item \textbf{Hasan}: HR. Al-Hakim (I/525), dan dishahihkannya lalu 
disepakati oleh adz-Dzahabi. Lihat \textit{Shah\^{i}hul J\^{a}mi'} (no. 
1260), serta 
\textit{Silsilah Ah\^{a}d\^{i}ts ash-Shah\^{i}hah} (IV/54, no. 1540). 
Sanadnya hasan.
\end{enumerate}
\textbf{Referensi}: Yazid bin Abdul Qadir Jawas. 2016. Kumpulan Do'a dari
Al-Quran dan As-Sunnah yang Shahih. Bogor: Pustaka Imam Asy-Syafi'i.
\index{bahagia}	
\index{sengsara}
\footnote{Hanifah Atiya Budianto 1417051063 - Jurusan Ilmu Komputer,
Universitas Lampung}
\end{document}