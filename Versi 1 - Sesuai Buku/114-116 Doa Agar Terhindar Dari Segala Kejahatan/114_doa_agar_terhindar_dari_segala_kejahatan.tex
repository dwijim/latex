\documentclass[a4paper,12pt]{article}
\usepackage{arabtex} 
\usepackage[bahasa] {babel}
\usepackage{calligra}
\usepackage[top=2cm,left=3cm,right=3cm,bottom=3cm]{geometry}
\title{\Large Doa Agar Terhindar dari Segala Kejahatan}
\author{\calligra Hanifah Atiya Budianto}
\begin{document}
\sffamily
\maketitle 
\fullvocalize
\setcode{arabtex}
\begin{arabtext}
\noindent
al-ll_ahumma rabba al-ssamAwAti (al-ssab`i), warabba al-'ar.di, warabba 
al-`ar^si al-`a.zi-ymi, rabbanA warabba kulli ^sa-y'iN, fAliqa al-.habbi 
wAl-nnawY, wamunzila al-ttawrATi wAl-'in^gi-yli wAl-furqAti, 'a`u-w_du 
bika min ^sarri kulli ^sa-y'iN 'anta ^A_hi_duN binA.siyatihi, al-ll_ahumma 
'anta al-'awwalu fala-ysa qablaka ^sa-y'uN, wa-'anta al-^A_hiru fala-ysa 
ba`daka ^sa-y'uN, wa'anta al-.z.zAhiru fala-ysa fa-wqaka ^sa-y'uN, wa'anta 
al-bA.tinu fala-ysa du-wnaka ^sa-y'uN, 'iq.di `annA al-dda-yna, wa'a.gninA 
mina al-faqri.\\
\end{arabtext}
\noindent
\textbf{Artinya}:
\par
\indent
"Ya Allah, Rabb langit [yang tujuh] dan Rabb bumi, Rabb Arsy yang agung, 
Rabb kami dan Rabb segala sesuatu, Pembelah biji serta benih, Rabb yang 
menurunkan Taurat, Injil, dan \textit{al-Furqan} (al-Qur’an), aku 
berlindung kepada-Mu dari kejahatan segala yang ubun-ubunnya Engkau pegang.
Ya Allah, Engkau yang paling pertama, tidak ada sesuatu pun sebelum-Mu, 
Engkau adalah yang paling akhir, tidak ada sesuatu pun setelah-Mu. Engkau 
adalah yang zhahir, tidak ada sesuatu pun yang mengungguli-Mu, dan Engkau 
adalah yang bathin, tidak ada sesuatu pun yang tersembunyi dari-Mu, 
lunasilah hutang kami dan cukupkanlah kami dari kefakiran (kemiskinan)."
\\\\
\par
\noindent
\textbf{Tingkatan Doa dan Sanad}: \textbf{Shahih}: HR. Muslim (no. 2713) 
dari Abu Hurairah r.a. Doa ini dibaca juga ketika hendak tidur.\\
\textbf{Referensi}: Yazid bin Abdul Qadir Jawas. 2016. Kumpulan Do'a dari
Al-Quran dan As-Sunnah yang Shahih. Bogor: Pustaka Imam Asy-Syafi'i.
\index{jahat}
\footnote{Hanifah Atiya Budianto 1417051063 - Jurusan Ilmu Komputer,
Universitas Lampung}
\end{document}