\documentclass[a4paper,12pt]{article}
\usepackage{arabtex} 
\usepackage[bahasa] {babel}
\usepackage{calligra}
\usepackage[top=2cm,left=3cm,right=3cm,bottom=3cm]{geometry}
\title{\Large Doa Sesudah Makan}
\author{\calligra Hanifah Atiya Budianto}
\begin{document}
\sffamily
\maketitle 
\fullvocalize
\setcode{arabtex}
\begin{arabtext}
\noindent
al-.hamdu li-ll_ahi alla_di-y 'a.t`amaniy h_a_dA warazaqani-yhi min 
.ga-yri .hawliN minni-y walA quwwaTiN.\\
\end{arabtext}
\noindent
\textbf{Artinya}:
\par
\indent
"Segala puji bagi Allah yang telah memberi makanan ini kepadaku dan yang 
telah memberi rizki kepadaku tanpa daya dan kekuatan dariku."
{\scriptsize 1}\\
\begin{arabtext}
\noindent
al-.hamdu lill_ahi .hamdaN ka_ti-yraN .tayyibaN mubArakaN fi-yhi, .ga-yra
makfiyyiN walA muwadda-`iN, walA musta.gnaN_A `anhu rabbanA.\\
\end{arabtext}
\noindent
\textbf{Artinya}:
\par
\indent
"Segala puji bagi Allah (aku memuji-Nya) dengan pujian yang banyak, yang 
baik dan penuh berkah, yang senantiasa dibutuhkan, diperlukan dan tidak 
bisa ditinggalkan (pengharapan kepada-Nya) wahai Rabb kami."{\scriptsize 2}
\\\\
\par
\noindent
\textbf{Tingkatan Doa dan Sanad}:
\begin{enumerate}
\item \textbf{Shahih}: HR. Abu Dawud (no. 4023), at-Tirmidzi (no. 3458),
Ibnu Majah (no. 3285), Ibnus Sunni (no. 467), Ahmad (III/439) dan al-Hakim 
(I/507; IV/192), Lihat \textit{Irw\^{a}-ul Ghal\^{i}l} (no. 1989).
\item \textbf{Shahih}: HR. Al-Bukhari (no. 5458), Abu Dawud (no. 3849), 
Ahmad (V/252, 256), at-Tirmidzi (no. 3456), Ibnus Sunni dalam 
\textit{'Amalul Yaum wal Lailah} (no. 468, 484), al-Baghawi dalam 
\textit{Syarhus Sunnah} (no. 2828) dari Abu Umamah al-Bahili r.a.
\end{enumerate}
\textbf{Referensi}: Yazid bin Abdul Qadir Jawas. 2016. Kumpulan Do'a dari
Al-Quran dan As-Sunnah yang Shahih. Bogor: Pustaka Imam Asy-Syafi'i.
\index{sesudah}	
\index{makan}
\footnote{Hanifah Atiya Budianto 1417051063 - Jurusan Ilmu Komputer,
Universitas Lampung}
\end{document}