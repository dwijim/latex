\documentclass[a4paper,12pt]{article}
\usepackage{arabtex} 
\usepackage[bahasa] {babel}
\usepackage{calligra}
\usepackage[top=2cm,left=3cm,right=3cm,bottom=3cm]{geometry}
\title{\Large Doa Sebelum Makan}
\author{\calligra Hanifah Atiya Budianto}
\begin{document}
\sffamily
\maketitle 
\fullvocalize
\setcode{arabtex}
\par
\indent
Apabila seseorang di antara kamu makan makanan, hendaklah membaca:\\
\begin{arabtext}
\noindent
bismi al-ll_ahi.\\
\end{arabtext}
\noindent
\textbf{Artinya}:
\par
\indent
"Dengan nama Alah (aku makan)."\
\par
\indent
Adapun apabila lupa membaca pada permulaannya, hendaklah membaca:\\
\begin{arabtext}
\noindent
bismi al-ll_ahi fi-y 'awwalihi wa'A_hirihi.\\
\end{arabtext}
\noindent
\textbf{Artinya}:
\par
\indent
"Dengan nama Allah (aku makan) di awal dan di akhirnya."{\scriptsize 1}\\
\par
\indent
Atau membaca:
\begin{arabtext}
\noindent
bismi al-ll_ahi 'awwalahu wa'A_hirahu.\\
\end{arabtext}
\noindent
\textbf{Artinya}:
\par
\indent
"Dengan nama Allah (aku makan), awal dan akhirnya."\\\\
\par
\noindent
\textbf{Tingkatan Doa dan Sanad}: \textbf{Shahih}: HR. Abu Dawud (no. 
3767), At-Tirmidzi (no. 1858), dan Shahih At-Tirmidzi (II/67).\\
\textbf{Referensi}: Yazid bin Abdul Qadir Jawas. 2016. Kumpulan Do'a dari
Al-Quran dan As-Sunnah yang Shahih. Bogor: Pustaka Imam Asy-Syafi'i.
\index{sebelum}
\index{makan}
\footnote{Hanifah Atiya Budianto 1417051063 - Jurusan Ilmu Komputer,
Universitas Lampung}
\end{document}