\documentclass[a4paper,12pt]{article}
\usepackage{arabtex} 
\usepackage[bahasa] {babel}
\usepackage{calligra}
\usepackage[top=2cm,left=3cm,right=3cm,bottom=3cm]{geometry}
\title{\Large Doa agar Menjadi Orang yang Banyak Berdzikir, Bersyukur, dan 
Taat}
\author{\calligra Hanifah Atiya Budianto}
\begin{document}
\sffamily
\maketitle 
\fullvocalize
\setcode{arabtex}
\begin{arabtext}
\noindent
rabbi 'a`inni-y walAtu`in `alayya, wAn.surni-y walA tan.sur `alayya, wAmkur 
li-y walA tamkur `alayya, wAhdini-y wayassiri al-hudY 'ilayya, wAn.surni-y 
`alY man ba.gY `alayya, rabbi a^g`alni-y laka ^sakkAraN, laka _dakkAraN, 
laka rahhAbaN, laka mi.twA`aN, 'ila-yka mu_hbitaN, laka 'awwAhaN muni-ybaN, 
rabbi taqabbal tawbati-y, wA.gsil .hawbati-y, wa'a^gib da`wati-y, wa_tabbit 
.hu^g^gati-y, wAhdi qalbi-y, wasaddid lisAni-y, wAslul sa_hiymaTa qalbi-y.
\\
\end{arabtext}
\noindent
\textbf{Artinya}:
\par
\indent
"Rabbku, tolonglah aku dan jangan Engkau tolong (orang yang hendak 
mencelakakan) atas diriku. Dan belalah aku dan jangan Engkau bela (orang 
yang akan mencelakakan) atas diriku. Perdayakanlah untukku dan jangan 
sampai aku diperdayai orang lain. Berilah aku petunjuk dan mudahkan ia 
untukku. Dan tolonglah aku atas orang yang menzhalimiku. Rabbku, jadikanlah
aku orang yang senantiasa bersyukur kepada-Mu, selalu berdzikir kepada-Mu, 
selalu takut kepada-Mu, selalu taat kepada-Mu, khusyu' patuh, banyak 
berdoa, serta bertaubat kepada-Mu. Rabbku, terimalah taubatku, bersihkan 
dosa-dosaku, perkenankanlah doaku, tetapkanlah hujjahku, beri petunjuk 
kepada hatiku, luruskanlah lidahku, dan hilangkanlah kejelekkan dalam 
hatiku."{\scriptsize 1}\\
\begin{arabtext}
\noindent
al-l_ahumma 'a`inni-y `alY _dikrika, wa^sukrika, wa.husni `ibAdatika.\\
\end{arabtext}
\noindent
\textbf{Artinya}:
\par
\indent
"Ya Allah, tolonglah aku untuk dapat berdzikir kepada-Mu, dapat bersyukur 
kepada-Mu, dan dapat beribadah dengan baik kepada-Mu."{\scriptsize 2}\\\\
\par
\noindent
\textbf{Tingkatan Doa dan Sanad}:
\begin{enumerate}
\item \textbf{Shahih}: HR. Ahmad (I/227)-lafazh ini miliknya-Abu Dawud (no. 
1510), at-Tirmidzi (no. 3551), Ibnu Majah (no. 3830), Ibnu Hibban (no. 993 
- \textit{at-Ta'liqatul His\^{a}n}), al-Hakim (I/519-520), dan yang 
lainnya. Dishahihkan oleh al-Hakim dan disepakati oleh adz-Dzahabi. Lihat 
\textit{Shah\^{i}h at-Tirmidzi} (III/178, no. 2816).
\item \textbf{Shahih}: HR. Abu Dawud (no. 1522), Ahmad (V/244-245, 247), 
an-Nasai (III/53), dan al-Hakim (I/273 dan III/273) dan dishahihkannya, 
juga disepakati oleh adz-Dzahabi. Nabi SAW. pernah berwasiat kepada Mu'adz 
r.a. agar dia mengucapkan dzikir tersebut pada setiap akhir shalatnya atau 
sesudah salam dari shalat wajib.
\end{enumerate}
\textbf{Referensi}: Yazid bin Abdul Qadir Jawas. 2016. Kumpulan Do'a dari
Al-Quran dan As-Sunnah yang Shahih. Bogor: Pustaka Imam Asy-Syafi'i.
\index{orang}
\index{banyak}
\index{berdzikir}
\index{bersyukur}
\index{taat}
\footnote{Hanifah Atiya Budianto 1417051063 - Jurusan Ilmu Komputer,
Universitas Lampung}
\end{document}