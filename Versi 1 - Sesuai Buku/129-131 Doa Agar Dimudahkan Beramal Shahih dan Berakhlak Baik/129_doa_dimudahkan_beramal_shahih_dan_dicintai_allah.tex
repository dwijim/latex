\documentclass[a4paper,12pt]{article}
\usepackage{arabtex} 
\usepackage[bahasa] {babel}
\usepackage{calligra}
\usepackage[top=2cm,left=3cm,right=3cm,bottom=3cm]{geometry}
\title{\Large Doa Dimudahkan Beramal Shahih dan Dicintai Allah}
\author{\calligra Hanifah Atiya Budianto}
\begin{document}
\sffamily
\maketitle 
\fullvocalize
\setcode{arabtex}
\begin{arabtext}
\noindent
al-ll_ahumma 'inni-y 'as'aluka fi`la al-_ha-yrAti, watarka al-munkarAti, 
wa.hubba almasAki-yni, wa'anta.gfirali-y, watar.hamani-y, wa-'i_dA 'aradta 
fitnaTa qawmiN, fatawaffani-y .ga-yra maftu-wniN, wa'as'aluka .hubbaka, 
wa.hubba man yu.hibbuka, wa.hubba `amaliN yuqarribuni-y 'ilY .hubbika.\\
\end{arabtext}
\noindent
\textbf{Artinya}:
\par
\indent
"Ya Allah, sungguh aku memohon kepada-Mu supaya dapat melakukan berbagai 
perbuatan baik, meninggalkan semua perbuatan mungkar, mencintai orang-orang
miskin, dan agar Engkau mengampuni dan menyayangiku. Dan jika Engkau hendak
menimpakan suatu \textit{fitnah} (malapetaka) kepada suatu kaum, maka 
wafatkanlah aku dalam keadaan tidak terkena fitnah tersebut. Dan aku 
memohon akan rasa cinta kepada-Mu dan rasa cinta kepada tiap orang yang 
mencintai-Mu, juga rasa cinta kepada amal perbuatan yang dengannya dapat 
mendekatkan diriku ini sehingga lebih mencintai-Mu."\\\\
\par
\noindent
\textbf{Tingkatan Doa dan Sanad}: \textbf{Hasan Shahih}: HR. Ahmad dengan 
lafazhnya (V/243), at-Tirmidzi (no. 3235), dan al-Hakim (I/521).\\
\textbf{Referensi}: Yazid bin Abdul Qadir Jawas. 2016. Kumpulan Do'a dari
Al-Quran dan As-Sunnah yang Shahih. Bogor: Pustaka Imam Asy-Syafi'i.
\index{dimudahkan}	
\index{beramal}
\index{shahih}	
\index{dicintai}
\index{allah}	
\footnote{Hanifah Atiya Budianto 1417051063 - Jurusan Ilmu Komputer,
Universitas Lampung}
\end{document}