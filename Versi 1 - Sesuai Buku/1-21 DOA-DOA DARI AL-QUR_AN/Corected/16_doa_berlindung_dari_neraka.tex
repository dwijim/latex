\documentclass[a4paper,12pt]{article}
\usepackage{arabtex} 
\usepackage[bahasa] {babel}
\usepackage{calligra}
\usepackage[top=2cm,left=3cm,right=3cm,bottom=3cm]{geometry}
\title{\Large Doa Berlindung dari Neraka}
\author{\calligra Hanifah Atiya Budianto}
\begin{document}
\sffamily
\maketitle 
\fullvocalize
\setcode{arabtex}
\begin{arabtext}
\noindent
rabbanaa a.srif `annaa `a_daaba ^gahannama 'inna `a_daabahaa kaana 
.garaa-maN $\odot$ 'innahaa sa'A'at mustaqarraN wamuqaamaN.\\
\end{arabtext}
\noindent
\textbf{Artinya}:\\
\indent
"'Ya Rabb kami, jauhkanlah azab Jahanam dari kami, karena sesungguhnya 
azabnya itu membuat kebinasaan yang kekal,' sungguh, Jahanam itu 
seburuk-buruk tempat menetap dan tempat kediaman." (QS. Al-Furqan [25]: 
65-66).\\\\
\par
\noindent
\textbf{Referensi}: Yazid bin Abdul Qadir Jawas. 2016. Kumpulan Do'a dari
Al-Quran dan As-Sunnah yang Shahih. Bogor: Pustaka Imam Asy-Syafi'i.
\index{lindung}	
\index{neraka}
\footnote{Hanifah Atiya Budianto 1417051063 - Jurusan Ilmu Komputer,
Universitas Lampung}
\end{document}