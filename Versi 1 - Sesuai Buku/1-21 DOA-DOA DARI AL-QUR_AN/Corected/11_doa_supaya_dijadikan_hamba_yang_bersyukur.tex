\documentclass[a4paper,12pt]{article}
\usepackage{arabtex} 
\usepackage[bahasa] {babel}
\usepackage{calligra}
\usepackage[top=2cm,left=3cm,right=3cm,bottom=3cm]{geometry}
\title{\Large Doa Supaya Dijadikan Hamba yang Bersyukur}
\author{\calligra Hanifah Atiya Budianto}
\begin{document}
\sffamily
\maketitle 
\fullvocalize
\setcode{arabtex}
\begin{arabtext}
\noindent
rabbi 'awzi`n^I 'an 'a^skura ni`mataka allat^I 'an`amta `alaYYa 
wa`alaY_a w_alidaYYa wa'an 'a`mala .s_ali.haN tar.d_ahu wa'ad_hilniY 
bira.hmatika fiY `ibAdika al-.s.s_ali.hiyna.\\
\end{arabtext}
\noindent
\textbf{Artinya}:\\
\indent
"Ya Rabbku, anugerahkanlah aku ilham untuk tetap mensyukuri nikmat-Mu yang 
telah Engkau anugerahkan kepadaku dan kepada kedua orang tuaku dan agar aku 
mengerjakan kebajikan yang Engkau ridhai; dan masukkanlah aku dengan 
rahmat-Mu ke dalam golongan hamba-hamba-Mu yang shalih." (QS. An-Naml 
[27]: 19).\\\\
\begin{arabtext}
\noindent
rabbi 'awzi`n^I 'an 'a^skura ni`mataka allat^I 'an`amta `alaYYa 
wa`alaY_a w_alidaYYa wa'an 'a`mala .s_ali.haN tar.d_ahu wa'a.sli.h liY fiY 
_durriyyat^I 'inniY tubtu 'ilayka wa-'inniY mina al-muslimiyna.\\
\end{arabtext}
\noindent
\textbf{Artinya}:\\
\indent
"Ya Rabbku, berilah aku petunjuk agar aku dapat mensyukuri nikmat-Mu yang 
telah Engkau limpahkan kepadaku dan kepada kedua orang tuaku dan agar aku 
dapat berbuat kebajikan yang Engkau ridai; dan berilah aku kebaikan yang 
akan mengalir sampai kepada anak cucuku. Sesungguhnya aku bertaubat kepada 
Engkau dan sungguh, aku termasuk orang muslim." (QS. Al-Ahq\^{a}f [46]: 15).
\\\\
\noindent
\textbf{Referensi}: Yazid bin Abdul Qadir Jawas. 2016. Kumpulan Do'a dari
Al-Quran dan As-Sunnah yang Shahih. Bogor: Pustaka Imam Asy-Syafi'i.
\index{jadi}	
\index{hamba}
\index{syukur}
\footnote{Hanifah Atiya Budianto 1417051063 - Jurusan Ilmu Komputer,
Universitas Lampung}
\end{document}