\documentclass[a4paper,12pt]{article}
\usepackage{arabtex} 
\usepackage[bahasa] {babel}
\usepackage{calligra}
\usepackage[top=2cm,left=3cm,right=3cm,bottom=3cm]{geometry}
\title{\Large Doa agar Termasuk Golongan Orang yang Beriman}
\author{\calligra Hanifah Atiya Budianto}
\begin{document}
\sffamily
\maketitle 
\fullvocalize
\setcode{arabtex}
\begin{arabtext}
\noindent
rabbi hab liY .hukmaN wa'al.hiqniY bi-al-.s.s_ali.hiyna $\odot$ wa-a^g`al 
lliY lisAna .sidqiN fiY  al-'a_hiriyna $\odot$ wa-a^g`alniY min wara_taTi 
^gannaTi al-nna`iymi $\odot$ walA tu_hziniY yawma yub`a_tuwna $\odot$
\end{arabtext}
\noindent
\textbf{Artinya}:\\
\indent
"Ya Rabbku, berikanlah kepadaku ilmu dan masukkanlah aku ke dalam golongan 
orang-orang yang shalih, dan jadikanlah aku buah tutur yang baik bagi 
orang-orang (yang datang) kemudian, dan jadikanlah aku termasuk orang yang 
mewarisi Surga yang penuh kenikmatan ..., dan janganlah Engkau hinakan aku 
pada hari mereka dibangkitkan." (QS. Asy-Syu'ar\^{a}' [26]: 83-85 dan 87).\\
\begin{arabtext}
\noindent
rabbana-^A -'a-amannA fa-uktubnA ma`a al-^s^s_ahidiyna.\\
\end{arabtext}
\noindent
\textbf{Artinya}:\\
\indent
"Ya Rabb, kami telah beriman, maka catatlah kami bersama orang-orang yang 
menjadi saksi (atas kebenaran al-Qur-an dan kenabian Muhammad)." 
(QS. Al-M\^{a}-idah [5]: 83).\\\\
\par
\noindent
\textbf{Referensi}: Yazid bin Abdul Qadir Jawas. 2016. Kumpulan Do'a dari
Al-Quran dan As-Sunnah yang Shahih. Bogor: Pustaka Imam Asy-Syafi'i.
\index{golongan}	
\index{orang}
\index{beriman}
\footnote{Hanifah Atiya Budianto 1417051063 - Jurusan Ilmu Komputer,
Universitas Lampung}
\end{document}