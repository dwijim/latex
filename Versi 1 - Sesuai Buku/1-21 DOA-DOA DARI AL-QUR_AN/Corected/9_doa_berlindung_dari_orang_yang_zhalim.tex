\documentclass[a4paper,12pt]{article}
\usepackage{arabtex} 
\usepackage[bahasa] {babel}
\usepackage{calligra}
\usepackage[top=2cm,left=3cm,right=3cm,bottom=3cm]{geometry}
\title{\Large Doa Berlindung Dari Orang Yang Zhalim}
\author{\calligra Hanifah Atiya Budianto}
\begin{document}
\sffamily
\maketitle 
\fullvocalize
\setcode{arabtex}
\begin{arabtext}
\noindent
rabbi na^g^giniY mina al-qawmi al-.z.z_alimiyna.\\
\end{arabtext}
\noindent
\textbf{Artinya}:\\
\indent
"Ya Rabbku, selamatkanlah aku dari orang-orang yang zhalim itu." (QS. 
Al-Qashash [28]: 21).\\
\begin{arabtext}
\noindent
rabbanA lA ta^g`alnA ma`a al-qawmi al-.z.z_alimiyna.\\
\end{arabtext}
\noindent
\textbf{Artinya}:\\
\indent
"Ya Rabb kami, janganlah Engkau tenpatkan kami bersama orang-orang yang 
zhalim itu." (QS. Al-A'r\^{a}f [7]: 47).\\
\begin{arabtext}
\noindent
rabbi an.surniY `alaY al-qawmi al-mufsidiyna.\\
\end{arabtext}
\noindent
\textbf{Artinya}:\\
\indent
"Ya Rabbku, tolonglah aku (dengan menimpakan azab) atas golongan yang 
berbuat kerusakan itu." (QS. Al-'Ankab\^{u}t [29]: 30).\\\\
\par
\noindent
\textbf{Referensi}: Yazid bin Abdul Qadir Jawas. 2016. Kumpulan Do'a dari
Al-Quran dan As-Sunnah yang Shahih. Bogor: Pustaka Imam Asy-Syafi'i.
\index{berlindung}	
\index{orang}
\index{zhalim}
\footnote{Hanifah Atiya Budianto 1417051063 - Jurusan Ilmu Komputer,
Universitas Lampung}
\end{document}