\documentclass[a4paper,12pt]{article}
\usepackage{arabtex} 
\usepackage[bahasa] {babel}
\usepackage{calligra}
\usepackage[top=2cm,left=3cm,right=3cm,bottom=3cm]{geometry}
\title{\Large Doa supaya Dilapangkan Hati serta Dimudahkan Urusan}
\author{\calligra Hanifah Atiya Budianto}
\begin{document}
\sffamily
\maketitle 
\fullvocalize
\setcode{arabtex}
\begin{arabtext}
\noindent
lla-^A 'il_aha 'illa-^A 'anta sub.h_anaka 'inniY kuntu mina al-.z.z_alimiyna
\end{arabtext}
\noindent
\textbf{Artinya}:\\
\indent
"Tidak ada ilah selain Engkau, Mahasuci Engkau. Sungguh, aku termasuk 
orang-orang yang zhalim." (QS. Al-Anbiy\^{a}' [21]: 87).\\
\begin{arabtext}
\noindent
rabbi a^sra.h liY .sadriY  $\odot$ wayassir l^I 'amriY  $\odot$ wa-a.hlul 
`uqdaTaN mmin  $\odot$ llisAniY yafqahu-W qawliY  $\odot$ \\
\end{arabtext}
\noindent
\textbf{Artinya}:\\
\indent
"Ya Rabbku, lapangkanlah dadaku, dan mudahkanlah untukku urusanku, dan
lepaskanlah kekakuan dari lidahku, agar mereka mengerti perkataanku." 
(QS. Thaha [20]: 25-28).\\
\begin{arabtext}
\noindent
rabbana-^A -'a-atinA min lladunka ra.hmaTaN wahayyi' lanA min 'amrinA 
ra^sadaN
\end{arabtext}
\noindent
\textbf{Artinya}:\\
\indent
"Ya Rabb kami. Berikanlah rahmat kepada kami dari sisi-Mu dan 
sempurnakanlah petunjuk yang lurus bagi kami dalam urusan kami." (QS. 
Al-Kahfi [18]: 10)\\\\
\noindent
\textbf{Referensi}: Yazid bin Abdul Qadir Jawas. 2016. Kumpulan Do'a dari
Al-Quran dan As-Sunnah yang Shahih. Bogor: Pustaka Imam Asy-Syafi'i.
\index{lapang}	
\index{lapang}
\index{mudah}	
\index{urusan}
\footnote{Hanifah Atiya Budianto 1417051063 - Jurusan Ilmu Komputer,
Universitas Lampung}
\end{document}