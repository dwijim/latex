\documentclass[a4paper,12pt]{article}
\usepackage{arabtex} 
\usepackage[bahasa] {babel}
\usepackage{calligra}
\usepackage[top=2cm,left=3cm,right=3cm,bottom=3cm]{geometry}
\title{\Large Doa Mohon Diberi Keturunan yang Shalih}
\author{\calligra Hanifah Atiya Budianto}
\begin{document}
\sffamily
\maketitle 
\fullvocalize
\setcode{arabtex}
\begin{arabtext}
\noindent
rabbi lA ta_darniY fardaN wa'anta _hayru al-w_ari_tiyna.\\
\end{arabtext}
\noindent
\textbf{Artinya}:\\
\indent
"Ya Rabbku, janganlah Engkau biarkan aku hidup seorang diri (tanpa 
keturunan) dan Engkaulah ahli waris yang terbaik." (QS. Al-Anbiy\^{a}' 
[21]: 89).\\
\begin{arabtext}
\noindent
rabbi hab liY mina al-.s.s_ali.hiyna.\\
\end{arabtext}
\noindent
\textbf{Artinya}:\\
\indent
"Ya Rabbku, anugerahkanlah kepadaku (seorang anak) yang termasuk orang yang
shalih." (QS. Ash-Sh\^{a}ff\^{a}t [37]: 100).\\
\begin{arabtext}
\noindent
rabbi hab liY min lladunka _durriyyaTaN .tayyibaTaN 'innaka samiy`u 
al-ddu`a-^A'i.\\
\end{arabtext}
\noindent
\textbf{Artinya}:\\
\indent
"Ya Rabbku, berilah aku keturunan yang baik dari sisi-Mu. Sesungguhnya 
Engkau Maha Mendengar doa." (QS. Ali 'Imran [3]: 38).\\
\begin{arabtext}
\noindent
rabbanA hab lanA min 'azw_a^ginA wa_durriyy_atinA qurraTa 'a`yuniN 
wa-u^g`alnA lilmuttaqi-yna 'imAmaN.\\
\end{arabtext}
\noindent
\textbf{Artinya}:\\
\indent
"Ya Rabb kami, anugerahkanlah kepada kami pasangan kami dan keturunan kami 
sebagai penyenang hati (kami), dan jadikanlah kami pemimpin bagi 
orang-orang yang bertakwa." (QS. Al-Furq\^{a}n [25]: 74).\\\\
\par
\noindent
\textbf{Referensi}: Yazid bin Abdul Qadir Jawas. 2016. Kumpulan Do'a dari
Al-Quran dan As-Sunnah yang Shahih. Bogor: Pustaka Imam Asy-Syafi'i.
\index{mohon}	
\index{beri}
\index{keturunan}	
\index{shahih}
\footnote{Hanifah Atiya Budianto 1417051063 - Jurusan Ilmu Komputer,
Universitas Lampung}
\end{document}