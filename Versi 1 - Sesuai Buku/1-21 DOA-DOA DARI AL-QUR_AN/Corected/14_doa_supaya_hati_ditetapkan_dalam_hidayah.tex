\documentclass[a4paper,12pt]{article}
\usepackage{arabtex} 
\usepackage[bahasa] {babel}
\usepackage{calligra}
\usepackage[top=2cm,left=3cm,right=3cm,bottom=3cm]{geometry}
\title{\Large Doa Supaya Hati Ditetapkan dalam Hidayah}
\author{\calligra Hanifah Atiya Budianto}
\begin{document}
\sffamily
\maketitle 
\fullvocalize
\setcode{arabtex}
\begin{arabtext}
\noindent
rabbanaa lA tuzi.g quluwbanaa ba`da 'i_d hadaytanaa wahab lanaa min 
lladunka ra.hmaTaN 'innaka 'anta al-wahhaabu.\\
\end{arabtext}
\noindent
\textbf{Artinya}:\\
\indent
"Ya Rabb kami, janganlah Engkau condongkan hati kami kepada kesesatan 
setelah Engkau berikan petunjuk kepada kami, dan karuniakanlah kepada kami 
rahmat dari sisi-Mu, dan sesungguhnya Engkau Maha Pemberi." (QS. Ali 
'Imran [3]: 8).\\\\
\par
\noindent
\textbf{Referensi}: Yazid bin Abdul Qadir Jawas. 2016. Kumpulan Do'a dari
Al-Quran dan As-Sunnah yang Shahih. Bogor: Pustaka Imam Asy-Syafi'i.
\index{hati}	
\index{tetap}
\index{hidayah}	
\footnote{Hanifah Atiya Budianto 1417051063 - Jurusan Ilmu Komputer,
Universitas Lampung}
\end{document}