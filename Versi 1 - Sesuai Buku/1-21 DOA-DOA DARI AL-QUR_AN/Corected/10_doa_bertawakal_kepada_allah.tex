\documentclass[a4paper,12pt]{article}
\usepackage{arabtex} 
\usepackage[bahasa] {babel}
\usepackage{calligra}
\usepackage[top=2cm,left=3cm,right=3cm,bottom=3cm]{geometry}
\title{\Large Doa Bertawakal Kepada Allah}
\author{\calligra Hanifah Atiya Budianto}
\begin{document}
\sffamily
\maketitle 
\fullvocalize
\setcode{arabtex}
\begin{arabtext}
\noindent
rabbanA `alayka tawakkalnA wa'i-layka 'anabnA wa'i-layka al-ma.siyru.\\
\end{arabtext}
\noindent
\textbf{Artinya}:\\
\indent
"Ya Rabb kami, hanya kepada Engkau kami bertawakal dan hanya kepada Engkau 
kami bertobat dan hanya kepada Engkaulah kami kembali." (QS. Al-Mumtahanah 
[60]: 4).\\
\begin{arabtext}
\noindent
.hasbiYa al-llahu la-^A 'il_aha 'illA huwa `alayhi tawakkaltu wahuwa rabbu
al-`ar^si al-`a.ziymi.\\
\end{arabtext}
\noindent
\textbf{Artinya}:\\
\indent
"Cukuplah Allah bagiku; tidak ada ilah selain Dia. Hanya kepada-Nya aku
bertawakal, dan Dia adalah Rabb yang memiliki 'Arsy (singgasana) yang 
agung." (QS. At-Taubah [9]: 129).\\\\
\par
\noindent
\textbf{Referensi}: Yazid bin Abdul Qadir Jawas. 2016. Kumpulan Do'a dari
Al-Quran dan As-Sunnah yang Shahih. Bogor: Pustaka Imam Asy-Syafi'i.
\index{tawakal}	
\index{allah}
\footnote{Hanifah Atiya Budianto 1417051063 - Jurusan Ilmu Komputer,
Universitas Lampung}
\end{document}