\documentclass[a4paper,12pt]{article}
\usepackage{arabtex} 
\usepackage[bahasa] {babel}
\usepackage{calligra}
\usepackage[top=2cm,left=3cm,right=3cm,bottom=3cm]{geometry}
\title{\Large Doa Mohon Ampun dan Rahmat Allah}
\author{\calligra Hanifah Atiya Budianto}
\begin{document}
\sffamily
\maketitle 
\fullvocalize
\setcode{arabtex}
\begin{arabtext}
\noindent
rabbi 'inniY^A 'a`uw_du bika 'an 'as'alaka mA laysa liY bihi `ilmuN 
wa'illA ta_hfir liY watar.hamniY^A 'akun mmina al-_h_asiriyna
\end{arabtext}
\noindent
\textbf{Artinya}:\\
\indent
"Ya Rabbku, sesunguhnya aku berlindung kepada-Mu untuk memohon kepada-Mu
sesuatu yang aku tidak mengetahui (hakikatnya). Kalau Engkau tidak
mengampuniku, dan (tidak) menaruh belas kasihan kepadaku, niscaya aku
termasuk orang yang rugi." (QS. Hud [11]: 47). \\
\begin{arabtext}
\noindent
rabban^A -'a-amannA fa-a.gfir lanA wa-ar.hamnA wa'anta _hayru 
al-rr_a.himiyna.\\
\end{arabtext}
\noindent
\textbf{Artinya}:\\
\indent
"Ya Rabb kami, kami telah beriman, maka ampunilah kami dan berilah kami
rahmat, Engkau adalah pemberi rahmat yang terbaik." (QS. Al-Mu'min\^{u}m 
[23]:109). \\
\begin{arabtext}
\noindent
rabbi a.gfir wa-ar.ham wa'anta _hayru al-rr_a.himiyna.\\
\end{arabtext}
\noindent
\textbf{Artinya}:\\
\indent
"Ya Rabbku, berilah ampunan dan (berilah) rahmat, Engkaulah pemberi rahmat
yang terbaik." (QS. Al-Mu'min\^{u}m [23]: 118).\\
\begin{arabtext}
\noindent
rabban^A 'innan^A -'a-amannA fa-a.gfir lanA _dunuwbanA waqinA `a_dAba 
al-nnAri.\\
\end{arabtext}
\noindent
\textbf{Artinya}:\\
\indent
"Ya Rabb kami, kami benar-benar beriman, maka ampunilah dosa-dosa kami dan
lindungilah kami dari azab Neraka." (QS. Ali 'Imran [3]: 16).\\
\begin{arabtext}
\noindent
rabbi 'inniY .zalamtu nafsiY fa-a.gfir liY.\\
\end{arabtext}
\noindent
\textbf{Artinya}:\\
\indent
"Ya Rabbku, sesungguhnya aku telah menzalimi diri sendiri, maka ampunilah
aku." (QS. Al-Qashash [28]: 16).\\
\begin{arabtext}
\noindent
rabban^A 'innanA sami`nA munAdiyaN yunAdiY lil-'iym_ani 'an -'a-aminuW
birabbikum fa'aamannA, rabbanA fa-a.gfir lanA _dunuwbanA wakaffir `annA 
sayyi'AtinA watawaffanA ma`a al-'abrAri  $\odot$ rabbanA wa -'a-atinA mA 
wa`adttanA `al_aY rusulika walA tu_hzinA yawma al-qiy_amaTi, 'innaka lA 
tu_hlifu almiy`Ada
\end{arabtext}
\noindent
\textbf{Artinya}:\\
\indent
"Ya Rabb kami, sesungguhnya kami mendengar orang yang menyeru kepada iman,
(yaitu) 'Berimanlah kamu kepada Rabbmu,' maka kami pun beriman. Ya Rabb
kami, ampunilah dosa-dosa kami dan hapuskanlah kesalahan-kesalahan kami, 
dan matikanlah kami beserta orang-orang yang berbakti. Ya Rabb kami, 
berilah kami apa yang telah Engkau janjikan kepada kami melalui 
Rasul-Rasul-Mu. Dan janganlah Engkau hinakan kami pada hari Kiamat. Sungguh,
 Engkau tidak pernah mengingkari janji." (QS. Ali 'Imran [3]: 193-194).\\
\begin{arabtext}
\noindent
rabbanA lA tu'A_hi_dn^A 'in nnasiyn^A 'aw 'a_h.ta'nA, rabbanA walA ta.hmil
`alayn^A 'i.sraN kamA .hamaltah_u `alaY alla_dina min qablinA, rabbanA walA
tu.hammilnA mA lA .tAqaTa lanA bih_i wa-a`fu `annA wa-a.gfir lanA 
wa-ar.hamna-^A, 'anta mawl_anA fa-an.surnA `alaY al-qawmi alk_afiriyna.\\
\end{arabtext}
\noindent
\textbf{Artinya}:\\
\indent
"Ya Rabb kami, janganlah Engkau hukum kami jika kami lupa atau kami 
melakukan kesalahan. Ya Rabb kami, janganlah Engkau bebani kami dengan 
beban yang berat sebagaimana Engkau bebankan kepada orang-orang sebelum 
kami. Ya Rabb kami, janganlah Engkau pikulkan kepada kami apa yang tidak 
sanggup kami memikulnya. Maafkanlah kami, ampunilah kami, dan rahmatilah 
kami. Engkaulah pelindung kami, maka tolonglah kami menghadapi orang-orang
kafir." (QS. Al-Baqarah [2]: 286).\\
\begin{arabtext}
\noindent
rabbanA a.gfir lanA _dunuwbanA wa-'isrAfanA f^I 'amrinA wa_tabbit 
'aqdAmanA wa-an.surnA `alaY alqawmi al-k_afiriyna.\\
\end{arabtext}
\noindent \textbf{Artinya}:\\
\indent
"Ya Rabb kami, ampunilah dosa-dosa kami, dan tindakan-tindakan kami yang
berlebihan (dalam) urusan kami dan tetapkanlah pendirian kami, dan 
tolonglah kami terhadap orang-orang kafir." (QS. Ali 'Imran [3]: 147).\\
\begin{arabtext}
\noindent
rabbanA .zalamn^A 'anfusanA wa-'in llam ta.gfir lanA watar.hamnA lanakuw 
nanna mina al-_h_asiriyna.\\
\end{arabtext}
\noindent
\textbf{Artinya}: \\
\indent "Ya Rabb kami, kami telah menzalimi diri kami sendiri. Jika Engkau 
tidak mengampuni kami dan memberi rahmat kepada kami, niscaya kami termasuk
orang-orang yang rugi." (QS. Al-A'r\^{a}f [7]: 23).\\\\
\noindent
\textbf{Referensi}: Yazid bin Abdul Qadir Jawas. 2016. Kumpulan Do'a dari 
Al-Quran dan As-Sunnah yang Shahih. Bogor: Pustaka Imam Asy-Syafi'i.
\index{mohon}
\index{ampunan}	
\index{rahmat}
\index{allah}	
\footnote{Hanifah Atiya Budianto 1417051063 - Jurusan Ilmu Komputer,
Universitas Lampung}
\end{document}