\documentclass[a4paper,12pt]{article}
\usepackage{arabtex} 
\usepackage[bahasa] {babel}
\usepackage{calligra}
\usepackage[top=2cm,left=3cm,right=3cm,bottom=3cm]{geometry}
\title{\Large Doa Sebelum Jima' (Bersetubuh)}
\author{\calligra Hanifah Atiya Budianto}
\begin{document}
\sffamily
\maketitle 
\fullvocalize
\setcode{arabtex}
\begin{arabtext}
\noindent
bismi al-ll_ahi, al-ll_ahumma ^gannibnaa al-^s^say.taana wa^gannibi 
al-^s^say.taana maa razaqtanaa.\\
\end{arabtext}
\noindent
\textbf{Artinya}:
\par
\indent
"Dengan nama Allah, ya Allah, jauhkanlah kami dari syaitan dan jauhkanlah 
syaitan agar tidak mengganggu apa (anak) yang Engkau rizkikan kepada kami."
\\\\
\par
\noindent
\textbf{Tingkatan Doa dan Sanad}: \textbf{Shahih}: HR. Al-Bukhari (no. 141,
3271, 5165, 6388) dan Muslim (no. 1434) dari Ibnu Abbas. Sabda Nabi SAW.: 
"Apabila mereka ditakdirkan mendapatkan anak, maka anak itu tidak akan 
diganggu (dibahayakan) oleh syaitan selama-lamanya."\\
\textbf{Referensi}: Yazid bin Abdul Qadir Jawas. 2016. Kumpulan Do'a dari
Al-Quran dan As-Sunnah yang Shahih. Bogor: Pustaka Imam Asy-Syafi'i.
\index{sebelum}	
\index{jima}
\index{bersetubuh}
\footnote{Hanifah Atiya Budianto 1417051063 - Jurusan Ilmu Komputer,
Universitas Lampung}
\end{document}