\documentclass[a4paper,12pt]{article}
\usepackage{arabtex} 
\usepackage[bahasa] {babel}
\usepackage{calligra}
\usepackage[top=2cm,left=3cm,right=3cm,bottom=3cm]{geometry}
\title{\Large Doa Kepada Pengantin}
\author{\calligra Hanifah Atiya Budianto}
\begin{document}
\sffamily
\maketitle 
\fullvocalize
\setcode{arabtex}
\begin{arabtext}
\noindent
baaraka al-ll_ahu laka wabaaraka `alayka wa^gama`a baynakumaa fiy _hayriN.\\
\end{arabtext}
\noindent
\textbf{Artinya}:
\par
\indent
"Semoga Allah memberimu berkah serta memberkahi atas pernikahanmu, dan 
semoga Dia mengumpulkan kalian berdua dalam kebaikan."\\\\
\par
\noindent
\textbf{Tingkatan Doa dan Sanad}: \textbf{Shahih}: HR. Abu Dawud (no. 
2130), at-Tirmidzi (no. 1091), Ahmad (II/381), ad-Darimi (II/134), Ibnu 
Majah (no. 1905), dan al-Hakim (II/183). Sanadnya shahih. Lihat 
\textit{\^{A}d\^{a}buz Zif\^{a}f} (hlm. 175).\\
\textbf{Referensi}: Yazid bin Abdul Qadir Jawas. 2016. Kumpulan Do'a dari
Al-Quran dan As-Sunnah yang Shahih. Bogor: Pustaka Imam Asy-Syafi'i.
\index{pengantin}
\index{nikah}
\footnote{Hanifah Atiya Budianto 1417051063 - Jurusan Ilmu Komputer,
Universitas Lampung}
\end{document}