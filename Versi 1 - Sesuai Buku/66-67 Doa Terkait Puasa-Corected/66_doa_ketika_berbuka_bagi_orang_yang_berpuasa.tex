\documentclass[a4paper,12pt]{article}
\usepackage{arabtex} 
\usepackage[bahasa] {babel}
\usepackage{calligra}
\usepackage[top=2cm,left=3cm,right=3cm,bottom=3cm]{geometry}
\title{\Large Doa Ketika Berbuka bagi Orang yang Berpuasa}
\author{\calligra Hanifah Atiya Budianto}
\begin{document}
\sffamily
\maketitle 
\fullvocalize
\setcode{arabtex}
\begin{arabtext}
\noindent
_dahaba al-.z.zama'u, waabtallati al-`uru-wqu, wa_tabata al-'a^gru 'in 
^sA'a al-ll_ahu.\\
\end{arabtext}
\noindent
\textbf{Artinya}:
\par
\indent
"Telah hilang rasa haus, dan urat-urat telah basah serta pahala telah 
tetap, \textit{insya Allah}."\\\\
\par
\noindent
\textbf{Tingkatan Doa dan Sanad}: \textbf{Hasan}: HR. Abu Dawud (no. 2357),
ad-Daraquthni (II/401, no. 2247), al-Hakim (I/422). Lihat 
\textit{Irw\^{a}'ul Ghal\^{i}l} (IV/39, no. 920), dan \textit{Shah\^{i}h 
Abi Dawud} (III/449, no. 2066).\\
\textbf{Referensi}: Yazid bin Abdul Qadir Jawas. 2016. Kumpulan Do'a dari
Al-Quran dan As-Sunnah yang Shahih. Bogor: Pustaka Imam Asy-Syafi'i.
\index{buka}	
\index{orang}
\index{puasa}
\footnote{Hanifah Atiya Budianto 1417051063 - Jurusan Ilmu Komputer,
Universitas Lampung}
\end{document}