\documentclass[a4paper,12pt]{article}
\usepackage{arabtex} 
\usepackage[bahasa] {babel}
\usepackage{calligra}
\usepackage[top=2cm,left=3cm,right=3cm,bottom=3cm]{geometry}
\title{\Large Doa Kaffaratul Majelis}
\author{\calligra Hanifah Atiya Budianto}
\begin{document}
\sffamily
\maketitle 
\fullvocalize
\setcode{arabtex}
\begin{arabtext}
\noindent
al-ll_ahumma aqsim lanA min _ha^syatika mA ta.hu-wlu bihi baynanA wabayna 
ma`A .si-yka, wamin .tA`atika mA tuballi.gunA bihi ^gannataka, wamina 
al-yaqi-yni mA tuhawwinu bihi `alaynA ma.sA'iba al-ddunyA, al-ll_ahumma 
matti`nA bi-'asmA`inA, wa'ab.sArinA, waquwwatinA mA 'a.hyaytanA, wA^g`alhu 
al-wAri_ta minnA, wA^g`al _ta'ranA `alY man .zalamanA, wAn.surnA `alY man
`AdAnA, walA ta^g`al mu.si-ybatanA fi-y di-yninA, walA ta^g`ali al-ddunyA 
'akbara hamminA, walA mabla.ga `ilminA, walA tusalli.t `alaynA man lA 
yar.hamunA.\\
\end{arabtext}
\noindent
\textbf{Artinya}:
\par
\indent
"Ya Allah, anugerahkanlah untuk kami rasa takut kepada-Mu, yang menghalangi
antara kami dengan perbuatan maksiat kepada-Mu, dan (anugerahkanlah kepada
kami) ketaatan kepada-Mu yang akan menyampaikan kami ke Surga-Mu, dan 
(anugerahkanlah pula) keyakinan yang dapat menyebabkan ringannya bagi kami 
segala musibah di dunia ini. Ya Allah, anugerahkanlah kenikmatan kepada 
kami melalui pendengaran kami, pengelihatan kami dan dalam kekuatan kami 
selama kami masih hidup, serta jadikanlah ia sebagai warisan dari kami. Dan
jadikan ia balasan kami atas orang-orang yang menganiaya kami, dan 
tolonglah kami terhadap orang yang memusuhi kami, serta janganlah Engkau 
jadikan musibah ada dalam urusan agama kami, dan janganlah Engkau jadikan 
dunia ini sebagai cita-cita terbesar dan puncak dari ilmu kami, dan jangan 
Engkau jadikan orang-orang yang tidak mengasihi kami berkuasa atas 
kami."{\scriptsize 1}\\
\begin{arabtext}
\noindent
sub.hAnaka al-ll_ahumma wabi.hamdika, 'a^shadu 'an lA 'il_aha 'illA 'anta, 
'asta.gfiruka wa'atu-wbu 'ilayka.\\
\end{arabtext}
\noindent
\textbf{Artinya}:
\par
\indent
"Mahasuci Engkau, ya Allah, dan aku memuji-Mu. Aku bersaksi bahwa tidak ada
ilah yang berhak diibadahi dengan benar kecuali Engkau, serta aku meminta 
ampun dan bertaubat kepada-Mu."\\
\par
\indent
Rasulullah Shallallahu ‘alaihi wa sallam bersabda: "Barang siapa duduk 
dalam satu majelis, lalu ada kekeliruan dan banyak kesalahan, kemudian 
sebelum bangkit dari majelis itu ia mengucap: 
\textbf{'Subh\^{a}nakall\^{a}humma wabihamdika asyhadu all\^{a} 
Il\^{a}ha illa anta astaghfiruka wa at\^{u}bu ilaika'}, maka Allah 
akan menghapuskan kesalahannya yang terjadi di majelis tersebut." 
{\scriptsize 2}\\
\indent
Dari Aisyah r.a., dia berkata: "Setiap Rasulullah Shallallahu ‘alaihi wa 
sallam duduk di suatu tempat dan setiap melakukan shalat, beliau 
mengakhirinya dengan beberapa kalimat." Aisyah bertanya tentang beberapa 
kalimat tersebut." Beliau Shallallahu ‘alaihi wa sallam bersabda: "Ya, 
barang siapa yang berkata baik maka akan ditulis pada kebaikan itu (pahala 
bacaan kalimat ini), dan barang siapa yang berkata jelek maka kalimat 
inilah penghapusnya."\\
\indent
Kalimat yang dimaksudkan adalah: \\
\begin{arabtext}
\noindent
sub.hAnaka al-ll_ahumma wabi.hamdika, lA 'il_aha 'illA 'anta, 'asta.gfiruka
wa'atu-wbu 'ilayka.\\
\end{arabtext}
\noindent
\textbf{Artinya}:
\par
\indent
"Mahasuci Engkau ya Allah, aku memuji-Mu. Tidak ada ilah yang berhak 
diibadahi dengan benar selain Engkau, aku mohon ampun dan bertaubat 
kepada-Mu."{\scriptsize 3}\\
\begin{arabtext}
\noindent
al-ll_ahumma .salli wasallim `alY nabiyyinA mu.hammadiN wa`alY ^Alihi 
wa'a.s.hAbihi 'a^gma`i-yna, waman tabi`ahum bi-'i.hsAniN 'ilY yawmi 
al-ddi-yni.\\
\end{arabtext}
\noindent
\textbf{Artinya}:
\par
\indent
"Ya Allah, limpahkanlah shalawat dan salam kepada Nabi kami, Muhammad, 
serta kepada keluarga dan para Sahabat beliau secara keseluruhan, juga 
kepada orang-orang yang mengikuti mereka dengan baik sampai hari Kiamat 
kelak."\\\\
\par
\noindent
\textbf{Tingkatan Doa dan Sanad}:
\begin{enumerate}
\item \textbf{Shahih}: HR. At-Tirmidzi (no. 3502) al-Hakim (I/528) dan 
Ibnus Sunni dalam \textit{Amalul Yaum wal Lailah} (no. 446) dan an-Nasai 
dalam \textit{Amalul Yaum wal Lailah} (no. 4040, dari Abdullah bin Umar 
r.a. Hadits ini dishahihkan oleh al-Hakim dan disepakati oleh adz-Dzahabi. 
Abdullah bin Umar r.a. berkata: "Rasulullah Shallallahu ‘alaihi wa sallam 
seringkali mengucapkan doa ini bagi Sahabat-Sahabat beliau sebelum bangkit 
dari majelis." Lihat \textit{Shah\^{i}h at-Tirmidzi} (III/168, no. 2783) 
dan \textit{Shah\^{i}hul J\^{a}mi'} (no. 1268), \textit{Shah\^{i}h 
al-Kalimith Thayyib} (no. 226).
\item \textbf{Hasan shahih}: HR. At-Tirmidzi (no. 3433), an-Nasai dalam 
\textit{'Amalul Yaum wal Lailah} (no. 400), Ibnus Sunni dalam 
\textit{'Amalul Yaum wal Lailah} (no. 447), Ibnu Hibban (no. 
593-\textit{at-Ta'l\^{i}q\^{a}tul His\^{a}n}), dan al-Hakim (I/536-537) 
dari Abu Hurairah r.a. At-Tirmidzi berkata: "Hadits ini hasan shahih." 
Dishahihkan al-Hakim, dan disetujui adz-Dzahabi. Hadits ini ada 
\textit{syawahid} (penguat) juga dari Abu Barzah al-Aslami, Jubair bin 
Muth'im, dan Aisyah r.a.
\item \textbf{Shahih}: HR. An-Nasai (III/71-72) dan dalam \textit{'Amalul 
Yaum wal Lailah} (no. 403), serta Ahmad (VI/77). Lihat kitab \textit{Fathul
B\^{a}ri} (XIII/546), dan \textit{Silsilah Ah\^{a}d\^{i}ts 
ash-Shah\^{i}hah} (no. 3164).
\end{enumerate}
\textbf{Keterangan}: \textit{Kaff\^{a}ratul majelis} artinya penghapus 
dosa akibat apa saja yang terjadi di majelis. Dibaca setelah selesai dari 
majelis dzikir, majelis ilmu, dan yang semisalnya. \\
\textbf{Referensi}: Yazid bin Abdul Qadir Jawas. 2016. Kumpulan Do'a dari
Al-Quran dan As-Sunnah yang Shahih. Bogor: Pustaka Imam Asy-Syafi'i.
\index{kaffaratul}
\index{majelis}
\footnote{Hanifah Atiya Budianto 1417051063 - Jurusan Ilmu Komputer,
Universitas Lampung}
\end{document}