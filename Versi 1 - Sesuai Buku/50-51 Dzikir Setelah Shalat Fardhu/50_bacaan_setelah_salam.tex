\documentclass[a4paper,12pt]{article}
\usepackage{arabtex} 
\usepackage[bahasa] {babel}
\usepackage{calligra}
\usepackage[top=2cm,left=3cm,right=3cm,bottom=3cm]{geometry}
\title{\Large Bacaan Setelah Salam}
\author{\calligra Hanifah Atiya Budianto}
\begin{document}
\sffamily
\maketitle 
\fullvocalize
\setcode{arabtex}
\begin{arabtext}
\noindent
'asta.griru al-ll_aha.\\
Aal-ll_ahumma 'anta al-ssalAmu, waminka al-ssalAmu, tabArakta yA_dA 
al-^galAli wAl-'ikrAmi\\
\end{arabtext}
\noindent
\textbf{Artinya}:
\par
\indent
"Aku memohon ampun kepada Allah [3x]. Ya Allah, Engkau Mahasejahtera, dan 
dari-Mu kesejahteraan, Mahasuci Engkau, wahai Rabb Pemilik keagungan dan 
kemuliaan.{\scriptsize 1}\\
\begin{arabtext}
\noindent
lA 'il_aha 'illA al-ll_ahu wa.hdahu lA ^sari-yka lahu, lahu al-mulku walahu 
al-.hamdu wahuwa `alY kulli ^say'iN qadi-yruN, Aal-ll_ahumma lA mAni`i limA 
'a`.ta-yta, walA mu`.tiya limA mana`ta, walA yanfa`u _dA al-^gaddi minka 
al-^gaddu.\\
\end{arabtext}
\noindent
\textbf{Artinya}:
\par
\indent
"Tidak ada ilah yang berhak diibadahi dengan benar melainkan hanya Allah 
Yang Maha Esa, tiada sekutu bagi-Nya. Bagi-Nya segala kerajaan dan bagi-Nya 
pula segala pujian. Dia Mahakuasa atas segala sesuatu. Ya Allah, tidak ada 
yang mencegah apa yang Engkau beri dan tidak ada yang memberi apa yang 
Engkau cegah. Tidak berguna kekayaan dan kemuliaan bagi pemiliknya dari 
(siksa)-Mu."{\scriptsize 2}\\
\begin{arabtext}
\noindent
lA 'il_aha 'illA al-ll_ahu wa.hdahu lA ^sari-yka lahu, lahu al-mulku walahu 
al-.hamdu wahuwa `alY kulli ^say'iN qadi-yruN, lA .ha-wla walA quwwaTa 
'illA bi-al-ll_ahi, lA 'il_aha 'illA al-ll_ahu, walA na`budu 'illA 'iyyAhu, 
lahu al-nni`maTu, walahu al-fa.dlu, walahu al-_t_tanA'u al-.hasanu, lA 
'il_aha 'illA al-ll_ahu mu_hli.si-yna lahu al-ddi-yna wala-w kariha 
al-kAfiru-wna.\\
\end{arabtext}
\noindent
\textbf{Artinya}:
\par
\indent
"Tidak ada ilah yang berhak diibadahi dengan benar melainkan hanya Allah 
Yang Maha Esa, tiada sekutu bagi-Nya. Bagi-Nya segala kerajaan dan pujian. 
Dia Maha berkuasa atas segala sesuatu. Tidak ada daya dan kekuatan kecuali 
(dengan pertolongan) Allah semata. Tidak ada ilah yang berhak diibadahi 
dengan benar melainkan hanya Allah. Kami tidak beribadah kecuali 
kepada-Nya. Bagi-Nya nikmat, anugerah, dan pujian yang baik. Tidak ada ilah 
yang berhak diibadahi dengan benar melainkan hanya Allah, dengan memurnikan 
ibadah hanya kepada-Nya, meskipun orang-orang kafir tidak menyukainya." 
{\scriptsize 3}\\
\begin{arabtext}
\noindent
lA 'il_aha 'illA al-ll_ahu wa.hdahu lA ^sari-yka lahu, lahu al-mulku, 
walahu al-.hamdu, yu.hyi-y wayumi-ytu, wahuwa `alY kulli ^sa-y'iN 
qadi-yruN.\\
\end{arabtext}
\noindent
\textbf{Artinya}:
\par
\indent
"Tidak ada ilah yang berhak diibadahi dengan benar melainkan hanya Allah 
Yang Maha Esa, tiada sekutu bagi-Nya, bagi-Nya kerajaan, dan bagi-Nya 
segala pujian. Dialah yang menghidupkan (orang yang sudah mati atau memberi 
ruh janin yang akan dilahirkan) dan yang mematikan. Dialah Yang Mahakuasa 
atas segala sesuatu." \textbf{[Dibaca 10x setiap setelah shalat Maghrib dan 
Shubuh]} {\scriptsize 4}\\
\begin{arabtext}
\noindent
Aal-ll_ahumma 'a`inni-y `alY _dikrika, wa^sukrika, wa.husni `ibAdatika.\\
\end{arabtext}
\noindent
\textbf{Artinya}:
\par
\indent
"Ya Allah, tolong aku agar selalu berdzikir kepada-Mu, bersyukur kepada-Mu, 
serta beribadah dengan baik kepada-Mu."{\scriptsize 5}\\
\begin{arabtext}
\noindent
sub.hAna al-ll_ahi\\
al-.hamdu li-ll_ahi\\
Aal-ll_ahu 'akbaru\\
\end{arabtext}
\noindent
\textbf{Artinya}:
\par
\noindent
"Mahasuci Allah" [33x]\\
"Segala puji bagi Allah" [33x]\\
"Allah Mahabesar" [33x]\\
\par
\indent
Lalu untuk melengkapinya menjadi seratus dengan membaca:
\begin{arabtext}
\noindent
lA 'il_aha 'illA al-ll_ahu wa.hdahu lA ^sari-yka lahu, lahu al-mulku walahu 
al-.hamdu, wahuwa `alY kulli ^say'iN qadi-yruN.\\
\end{arabtext}
\noindent
\textbf{Artinya}:
\par
\indent
"Tidak ada ilah yang berhak diibadahi dengan benar melainkan hanya Allah 
Yang Maha Esa, tiada sekutu bagi-Nya, bagi-Nya kerajaan, bagi-Nya segala 
puji. Dan Dia Mahakuasa atas segala sesuatu."{\scriptsize 6}\\
\indent
Kemudian membaca surah Al-Ikhlash, Al-Falaq, dan An-Nas pada setiap selesai 
shalat fardhu.{\scriptsize 7}\\
\indent
Kemudian membaca ayat Kursi setiap selesai shalat (fardhu).{\scriptsize 8}
\\
\noindent
\textbf{Setelah shalat Shubuh membaca:}
\begin{arabtext}
\noindent
Aal-ll_ahumma 'inni-y 'as'aluka `ilmaN nAfi`aN, warizqaN .tayyibaN, 
wa`amalaN wataqabbalaN.\\
\end{arabtext}
\noindent
\textbf{Artinya}:
\par
\indent
"Ya Allah, sesungguhnya aku mohon kepada-Mu ilmu yang bermanfaat, rizki 
yang halal, dan amal yang diterima."{\scriptsize 9}\\\\
\par
\noindent
\textbf{Tingkatan Doa dan Sanad}:
\begin{enumerate}
\item \textbf{Shahih}: HR. Muslim (no. 591[135]), Ahmad (V/275, 279), Abu 
Dawud (no. 1513), Ibnu Khuzaimah (no. 737), an-Nasai (III/68), ad-Darimi 
(I/311), dan Ibnu Majah (no. 928) dari Tsauban.
\item \textbf{Shahih}: HR. Al-Bukhari (no. 844), Muslim (no. 593), Ahmad 
(IV/245, 247, 250, 254, 255), Abu Dawud (no. 1505), an-Nasai (III/70, 71), 
ad-Darimi (I/311), dan Ibnu Khuzaimah (no. 742) dari al-Mughirah bin Syu'bah 
r.a.
\item \textbf{Shahih}: HR. Muslim (no. 594), Abu Dawud (no. 1506, 1507), 
Ahmad (IV/4,5), an-Nasai (III/70), Ibnu Khuzaimah (no. 740, 741) dari 
Abdullah bin az-Zubair r.a.
\item \textbf{Shahih}: HR. Ahmad (IV/227) dan at-Tirmidzi (no. 3474). 
At-Tirmidzi berkata: "\textit{Hasan gharib shahih.}" Diriwayatkan juga oleh 
Ahmad (V/420). Lihat \textit{Shah\^{i}h Targh\^{i}b wat Tarh\^{i}b} 
(I/322-323, no. 474, 475, 477), \textit{Z\^{a}dul Ma'\^{a}d} (I/300-301) dan 
\textit{Silsilah Ah\^{a}d\^{i}ts ash-Shah\^{i}hah} (no. 113, 114, 2563).
\item \textbf{Shahih}: HR. Abu Dawud (no. 1522), an-Nasai (III/53), Ahmad 
(V/245) dan al-Hakim (I/273 dan III/273). Hadits ini dishahihkan oleh 
al-Hakim dan disepakati adz-Dzahabi, yang mana kedudukan hadits itu seperti 
yang dikatakan keduanya, bahwa Nabi shallallahu alaihi wa sallam pernah 
memberikan wasiat kepada Mu'adz agar dia mengucapkannya di setiap akhir 
shalat.
\item "Siapa yang membaca dzikir ini tiap selesai shalat akan diampuni 
kesalahannya, meski seperti buih di lautan." \textbf{Shahih}: HR. Muslim 
(no. 597), Ahmad (II/371, 483), Ibnu Khuzaimah (no. 750), al-Baihaqi 
(II/187).
\item \textbf{Shahih}: HR. Abu Dawud (no. 1523), an-Nasai (III/68), Ibnu 
Khuzaimah (no. 755), dan Hakim (I/253) dari Uqbah bin Amr r.a. Lihat 
\textit{Shah\^{i}h at-Tirmidzi} (III/8, no. 2324) dan \textit{Fathul 
B\^{a}ri} (IX/62). Tigas surah ini dinamakan \textit{al-Mu'awwadz\^{a}t}.
\item "Siapa yang membacanya setiap selesai shalat maka tidak ada yang 
menghalanginya masuk Surga selain belum datangnya kematian." 
\textbf{Shahih}: HR. An-Nasai dalam \textit{'Amalul Yaum wal Lailah} (no. 
100) dan Ibnus Sunni (no. 124) dari Abu Umamah r.a. Dishahihkan Syaikh 
al-Albani dalam \textit{Shah\^{i}hul J\^{a}mi'} (no. 6464) dan 
\textit{ash-Shah\^{i}hah} (no. 972).
\item \textbf{Shahih}: HR. Ibnu Majah (no. 925), \textit{Shah\^{i}h Ibni 
Majah} (I/152, no. 753), Ibnus Sunni dalam \textit{'Amalul Yaum wal Lailah} 
(no. 110), Ahmad (VI/322), dan lainnya. Lihat \textit{Shah\^{i}h Ibni Majah} 
(I/152) dan \textit{Majma'uz Zaw\^{a}-id} (X/111). Sanadnya shahih.
\end{enumerate}
\textbf{Referensi}: Yazid bin Abdul Qadir Jawas. 2016. Kumpulan Do'a dari
Al-Quran dan As-Sunnah yang Shahih. Bogor: Pustaka Imam Asy-Syafi'i.
\index{dzikir}	
\index{bacaan}
\index{setelah}	
\index{salam}
\index{shalat}	
\footnote{Hanifah Atiya Budianto 1417051063 - Jurusan Ilmu Komputer,
Universitas Lampung}
\end{document}