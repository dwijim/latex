\documentclass[a4paper,12pt]{article}
\usepackage{arabtex} 
\usepackage[bahasa] {babel}
\usepackage{calligra}
\usepackage[top=2cm,left=3cm,right=3cm,bottom=3cm]{geometry}
\title{\Large Seputar Kesalahan Umum Setelah Shalat}
\author{\calligra Hanifah Atiya Budianto}
\begin{document}
\sffamily
\maketitle 
\fullvocalize
\setcode{arabtex}
\par
\indent
Beberapa hal yang biasa dilakukan banyak orang setelah shalat fardhu (wajib) yang lima waktu tetapi tidak ada contoh dan dalil dari Rasulullah SAW. dan pada Sahabat ;ridhwanullah 'alaihim ajma'in.\\
\par
\indent
Di antara kesalahan dan bid'ah yang dimaksud ialah:
\begin{enumerate}
\item Mengusap muka setelah salam.1
\item Berdoa dan berdzikir secara berjamaah yang dipimpin oleh imam shalat.2
\item Berdzikir dengan bacaan yang tidak ada nash/dalilnya, baik lafazh maupun bilangannya, atau berdzikir dengan dasar hadits yang dha'if (lemah) atau hadits yang maudhu' (palsu).\\
Contoh bacaan yang tidak ada dalil:\\
\begin{itemize}
\item Sesudah salam membaca: "Alhamdulillah."
\item Membaca surah Al-Fatihah setelah salam.
\item Membaca beberapa ayat terakhir surah Al-Hasyr dan lainnya.
\end{itemize}
\item Menghitung dzikir dengan memakai biji tasbih atau yang serupa dengannya. Tidak ada satu pun hadits yang shahih mengenai menghitung dzikir dengan biji-bijian tasbih, bahkan sebagiannya maudhu' (palsu).3 Syaikh al-Albani r.a mengatakan: :Berdzikir dengan biji-bijian tasbih adalah bid'ah."4\\
\par
\indent
Syaikh Bakr Abu Zaid mengatakan bahwa berdzikir dengan menggunakan biji-biji tasbih menyerupai kaum Yahudi, Nasrani, dan Budha. Dan perbuatan ini adalah bid'ah dhalalah.\\
\par
\indent
Yang disunnahkan dalam berdzikir adalah menggunakan jari-jari tangan:\\
\begin{arabtext}
\noindent
........ \\ \\
\end{arabtext}
\par
\indent
Dari Abdullah bin Amr r.a, ia berkata: "Aku melihat Rasulullah SAW. menghitung bacaan tasbih dengan jari-jari tangan kanannya."5.\\
\par
\indent
Ketahuilah bahwa Rasulullah SAW. memerintahkan kepada para Sahabat wanita rda. agar menghitung Subhanallah, La ilaha illallah, dan mensucikan Allah dengan jari-jari. Karena sesungguhnya jari-jari tersebut akan ditanya dan diminta oleh Allah untuk berbicara kelak, pada hari Kiamat.6\\
\item Berdzikir dengan suara keras dan beramai-ramai (dengan cara koor atau berjamaah).\\
Allah SWT. memerintahkan kepada kita agar berdzikir dengan suara yang lembut tidak keras. Demikianlah sebagaimana tercantum dalam al-Qur-an:
\begin{arabtext}
\noindent
........ \\ \\
\end{arabtext}
\noindent
\textbf{Artinya} :
\par
\indent
"Berdoalah kepada Tuhanmu dengan berendah diri dan suara yang lembut. Sesungguhnya Allah tidak menyukai orang-orang yang melampaui batas."(QS. Al-A'raf[7]:55)\\
\begin{arabtext}
\noindent
........ \\ \\
\end{arabtext}
\noindent
\textbf{Artinya} :
\par
\indent
"Dan sebutlah (nama)Tuhanmu dalam hatimu dengan merendahkan diri dan rasa takut, dan dengan tidak mengeraskan suara, di waktu pagi dan petang, dan janganlah kamu termasuk orang-orang yang lalai."(QS. Al-A'raf[7]:205)7\\
\indent
Nabi SAW. melarang berdzikir dengan suara keras sebagaimana diriwayatkan oleh al-Bukhari, Muslim, dan lainnya.\\
\indent
Imam asy-Syafi'i menganjurkan agar imam atau makmum tidak mengeraskan bacaan dzikir.8\\
\item Imam dan makmum membiasakan/merutinkan berdoa setelah shalat fardhu berjamaah dan mengangkat tangan pada doa tersebut, padahal perbuatan ini tidak pernah dicontohkan Rasulullah SAW.
\item Saling berjabat tangan seusai shalat fardhu (bersalam-salaman). Tidak ada seorang pun dari Sahabat atau kalangan Salafush Shalih yang berjabat tangan (bersalam-salaman) kepada orang baik di sebelah kanan atau kiri, depan atau di belakangnya apabila mereka selesai melaksanakan shalat. Jika seandainya perbuatan itu baik, maka akan sampai kabar kepada kita, dan para ulama akan menukil serta menyampaikan riwayat shahih tentangnya kepada kita.9\\
\par
\indent
Para ulama mengatakan: "Perbuatan tersebut adalah bid'ah."10
\indent
Berjabat tangan dianjurkan, tetapi menetapkannya di setiap selesai shalat fardhu tidak ada contohnya, ataupun setelah shalat Shubuh dan Ashar, amak perbuatan ini adalah bid'ah.11\\
\end{enumerate}
\par
\noindent
\textbf{Tingkatan Doa dan Sanad}:
\begin{enumerate}
\item 
\item 
\item 
\item 
\item 
\item \textbf{Shahih}: 
\item \textbf{Hasan}: 
\item 
\item 
\item 
\item 
\item 
\item 
\end{enumerate}
\textbf{Referensi}: Yazid bin Abdul Qadir Jawas. 2016. Kumpulan Do'a dari
Al-Quran dan As-Sunnah yang Shahih. Bogor: Pustaka Imam Asy-Syafi'i.
\index{kesalahan}	
\index{setelah}	
\index{shalat}
\footnote{Hanifah Atiya Budianto 1417051063 - Jurusan Ilmu Komputer,
Universitas Lampung}
\end{document}