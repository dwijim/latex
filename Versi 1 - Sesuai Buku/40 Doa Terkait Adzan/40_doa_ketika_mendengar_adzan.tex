\documentclass[a4paper,12pt]{article}
\usepackage{arabtex} 
\usepackage[bahasa] {babel}
\usepackage{calligra}
\usepackage[top=2cm,left=3cm,right=3cm,bottom=3cm]{geometry}
\title{\Large Doa ketika Mendengar Adzan}
\author{\calligra Hanifah Atiya Budianto}
\begin{document}
\sffamily
\maketitle 
\fullvocalize
\setcode{arabtex}
\indent
Terdapat lima hal yang disunnahkan ketika adzan dikumandangkan:
\begin{enumerate}
\item Menjawab adzan seperti apa yang diucapkan muadzin, kecuali pada 
lafazh: \begin{arabtext} (.hayya `alaY al-.s.salATi) \end{arabtext} dan 
lafazh: \begin{arabtext} (.hayya `alaY al-falA.hi) \end{arabtext}, maka 
kita mengucapkan:
\begin{arabtext}
\noindent
lA .hawla walA quwwaTa 'illA bi-al-ll_ahi.\\
\end{arabtext}
\noindent
\textbf{Artinya}:
\par
\indent
"Tidak ada daya dan kekuatan kecuali dengan pertolongan Allah."
{\scriptsize 1}
\item Setelah muadzin selesai adzan, maka kita mengucapkan:{\scriptsize 2}
\begin{arabtext}
\noindent
(wa'anA) 'a^shadu 'an lA 'i-l_aha 'illA al-ll_ahu wa.hdahu lA^sari-yka lahu 
wa ('a^shadu) 'anna mu.hammadaN `abduhu warasu-wluhu, ra.di-ytu 
bi-al-ll_ahi rabbaN, wabimu.hammadiN rasu-wlaN wabi-al-'i-slA-mi di-ynaN.\\
\end{arabtext}
\noindent
\textbf{Artinya}:
\par
\indent
"Dan aku bersaksi bahwa tidak ada ilah yang berhak diibadahi dengan benar 
melainkan Allah Yang Esa, tidak ada sekutu bagi-Nya, dan aku pun bersaksi 
bahwa Muhammad adalah hamba-Nya dan Rasul-Nya, aku ridha Allah sebagai 
Rabb, Muhammad sebagai Rasul dan Islam sebagai agama(ku)."{\scriptsize 3}
\item Membaca shalawat kepada Nabi Muhammad \textit{Shallallahu ‘alaihi wa 
sallam}.{\scriptsize 4}
\item Membaca doa setelah adzan:
\begin{arabtext}
\noindent
al-ll_ahumma rabba h_a_dihi al-dda`waTi al-ttAmmaTi, wAl-.s.salATi 
al-qA'imaTi, ^Ati mu.hammadaN ni al-wasi-ylaTa wAl-fa.di-ylaTa, wAb`a_thu 
maqAmaN ma.hmu-wdA-ni alla_di-y wa`adtahu.\\
\end{arabtext}
\noindent
\textbf{Artinya}:
\par
\indent
"Ya Allah, Rabb Pemilik panggilan yang sempurna (adzan) ini dan shalat 
(wajib) yang didirikan. Berikanlah \textit{al-wasilah} (derajat di Surga), 
dan keutamaan kepada Muhammad \textit{Shallallahu ‘alaihi wa sallam}. Dan 
bangkitkanlah beliau sehingga bisa menempati maqam terpuji yang telah 
Engkau janjikan."{\scriptsize 5}
\item Berdoa untuk diri sendiri dengan doa yang dikehendaki antara adzan 
dan iqamah, sebab doa pada saat itu dikabulkan oleh Allah.
\begin{arabtext}
\noindent
al-ddu`A'u lA yuraddu bayna al-'a _dAni wAl-'iqAmaTi.\\
\end{arabtext}
\noindent
\textbf{Artinya}:
\par
\indent
"Tidak ditolak doa antara adzan dan iqamah."{\scriptsize 6}\\\\
\end{enumerate}
\par
\noindent
\textbf{Tingkatan Doa dan Sanad} :
\begin{enumerate}
\item "Barang siapa menjawab adzan dengan ikhlas dari hatinya, ia akan 
masuk Surga. "Lihat \textit{Syarah Muslim} (IV/85-86 no. 385). Dan apabila 
seorang muadzin mengucapkan: \begin{arabtext} (al-.s.salATu _ha-yruN mina 
al-nna-wmi) \end{arabtext}, maka hendaklah dijawab seperti itu juga.
\item Ada yang berpendapat bahwa dzikir ini dibaca setelah muadzin membaca 
syahadat. Lihat kitab \textit{ats-Tsamar al-Mustath\^{a}b f\^{i} Fiqhis 
Sunnah wal Kit\^{a}b} (hlm. I/172-185) karya Syaikh al-Albani, 
\textit{Maus\^{u}'ah al-Fiqhiyyah al-Muyassarah f\^{i} Fiqhil Kit\^{a}b was
Sunnah al-Muthahhara}h (hlm. 371) karya Husain al-Audah al-Awayisyah, 
\textit{Shah\^{i}h al-W\^{a}bilish Shayyib} (hlm. 184), dan 
\textit{Tash-h\^{i}hud Du'\^{a}'} (hlm. 370-372).
\item \textbf{Shahih}: HR. Muslim (no. 386), at-Tirmidzi (no. 210), Abu 
Dawud (no. 525), an-Nasai (II/26), Ibnu Majah (no. 721), Ahmad (I/181), 
Ibnu Khuzaimah (no. 421) dan yang lainnya dari Sa'ad bin Abi Waqqash r.a.
\item \textbf{Shahih}: HR. Muslim (no. 384), an-Nasai (II/25-26), Abu Dawud
(no. 523), Ibnu Khuzaimah (no. 418), Ahmad (II/168), Al-Baihaqi (I/409-410)
dari Abdullah bin Amr bin al-Ash r.a.
\item \textbf{Shahih}: HR. Al-Bukhari (no. 614)--Lihat 
\textit{Fathul B\^{a}ri} (II/94)--Abu Dawud (no. 529), at-Tirmidzi (no. 
211), an-Nasai (II/26-27), Ibnu Majah (no. 722). Adapun tambahan 
\begin{arabtext} ('innaka lA tu_hlifu al-mi-y`Adu) \end{arabtext} adalah 
\textbf{lemah}, jadi ia tidak boleh diamalkan. Lihat kitab 
\textit{Irw\^{a}-ul Ghal\^{i}l} (I/260, 261). Tidak ada juga tambahan: 
\begin{arabtext} (wAl-ddara^gaTa al-rrafi-y`aTa) (yA 'ar.hama 
al-rra.himi-yna) \end{arabtext} karena tidak ada asalnya.
\item \textbf{Shahih}: HR. Abu Dawud (no. 521), at-Tirmidzi (no. 212, 
3595), Ahmad (III/119, 155, 225), an-Nasai dalam \textit{'Amalul Yaum wal 
Lailah} (no. 67, 68, 69), Ibnu Khuzaimah (no. 425, 426, 427). Lihat penjelasan 
Ibnu Qayyim tentang lima hal ini dalam \textit{Shah\^{i}h al-Wabilish 
Shayyib} (hlm. 182-185), \textit{Z\^{a}dul Ma'\^{a}d} (II/391-392).
\end{enumerate}
\textbf{Referensi}: Yazid bin Abdul Qadir Jawas. 2016. Kumpulan Do'a dari
Al-Quran dan As-Sunnah yang Shahih. Bogor: Pustaka Imam Asy-Syafi'i.
\index{adzan}	
\index{dengar}
\footnote{Hanifah Atiya Budianto 1417051063 - Jurusan Ilmu Komputer,
Universitas Lampung}
\end{document}