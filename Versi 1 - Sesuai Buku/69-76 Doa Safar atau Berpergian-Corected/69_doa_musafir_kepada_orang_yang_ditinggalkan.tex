\documentclass[a4paper,12pt]{article}
\usepackage{arabtex} 
\usepackage[bahasa] {babel}
\usepackage{calligra}
\usepackage[top=2cm,left=3cm,right=3cm,bottom=3cm]{geometry}
\title{\Large Doa Musafir kepada Orang yang Ditinggalkan}
\author{\calligra Hanifah Atiya Budianto}
\begin{document}
\sffamily
\maketitle 
\fullvocalize
\setcode{arabtex}
\begin{arabtext}
\noindent
'astawdi `ukumu al-ll_aha a-lla_di-y lA ta.diy`u wadA'i`uhu.\\
\end{arabtext}
\noindent
\textbf{Artinya}:
\par
\indent
"Aku menitipkan kalian kepada Allah yang tidak akan hilang titipan-Nya."
\\\\
\par
\noindent
\textbf{Tingkatan Doa dan Sanad}: \textbf{Shahih}: HR. Ahmad (II/403) dan 
Ibnu Majah (no. 2825), An-Nasa'i dalam \textit{'Amalul Yaum wal Lailah} 
(no. 512), Ibnu Sunni dalam \textit{'Amalul Yaum wal Lailah} (no. 505), dan
ath-Thabrani dalam kitab \textit{ad-Du'\^{a}'} (no. 820)-lafazh ini milik 
Ibnu Sunni.\\
\textbf{Referensi}: Yazid bin Abdul Qadir Jawas. 2016. Kumpulan Do'a dari
Al-Quran dan As-Sunnah yang Shahih. Bogor: Pustaka Imam Asy-Syafi'i.
\index{musafir}	
\index{orang}
\index{ditinggalkan}
\footnote{Hanifah Atiya Budianto 1417051063 - Jurusan Ilmu Komputer,
Universitas Lampung}
\end{document}