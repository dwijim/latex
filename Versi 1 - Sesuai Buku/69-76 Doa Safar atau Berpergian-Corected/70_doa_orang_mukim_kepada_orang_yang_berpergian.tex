\documentclass[a4paper,12pt]{article}
\usepackage{arabtex} 
\usepackage[bahasa] {babel}
\usepackage{calligra}
\usepackage[top=2cm,left=3cm,right=3cm,bottom=3cm]{geometry}
\title{\Large Doa Orang Mukmin kepada Orang yang Bepergian}
\author{\calligra Hanifah Atiya Budianto}
\begin{document}
\sffamily
\maketitle 
\fullvocalize
\setcode{arabtex}
\begin{arabtext}
\noindent
'astawdi`u al-ll_aha di-ynaka wa'amAnataka wa_hawAti-yma `amalika.\\
\end{arabtext}
\noindent
\textbf{Artinya}:
\par
\indent
"Aku menitipkan agamamu, amanatmu, dan kesudahan amal perbuatanmu kepada 
Allah."\\\\
\par
\noindent
\textbf{Tingkatan Doa dan Sanad}: \textbf{Shahih}: HR. Ahmad (II/7), Abu 
Dawud (no. 2600), al-Hakim (I/442), dan at-Tirmidzi (no. 3443) dari Ibnu 
Umar r.a. Lihat \textit{Silsilah Ah\^{a}d\^{i}ts ash-Shah\^{i}hah} (no. 
14).\\
\textbf{Referensi}: Yazid bin Abdul Qadir Jawas. 2016. Kumpulan Do'a dari
Al-Quran dan As-Sunnah yang Shahih. Bogor: Pustaka Imam Asy-Syafi'i.
\index{mukmin}	
\index{orang}
\index{bepergian}
\footnote{Hanifah Atiya Budianto 1417051063 - Jurusan Ilmu Komputer,
Universitas Lampung}
\end{document}