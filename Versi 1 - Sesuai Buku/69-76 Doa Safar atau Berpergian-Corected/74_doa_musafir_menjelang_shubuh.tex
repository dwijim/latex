\documentclass[a4paper,12pt]{article}
\usepackage{arabtex} 
\usepackage[bahasa] {babel}
\usepackage{calligra}
\usepackage[top=2cm,left=3cm,right=3cm,bottom=3cm]{geometry}
\title{\Large Doa Musafir Menjelang Subuh}
\author{\calligra Hanifah Atiya Budianto}
\begin{document}
\sffamily
\maketitle 
\fullvocalize
\setcode{arabtex}
\begin{arabtext}
\noindent
samma`a sAmi`uN bi.hamdi al-ll_ahi, wa.husni balA'ihi `ala-ynaa. rabbanA 
.sA.hibnaa, wa'af.dil `ala-ynaa `A'i_daN bi-al-ll_ahi mina al-nnaari.\\
\end{arabtext}
\noindent
\textbf{Artinya}:
\par
\indent
"Semoga ada yang memperdengarkan/menyaksikan pujian kami kepada Allah (atas
nikmat) dan cobaan-Nya yang baik bagi kami. Wahai Rabb kami, dampingilah 
kami (periharalah kami) dan berikanlah karunia kepada kami dengan 
berlindung kepada Allah dari Api Neraka."\\\\
\par
\noindent
\textbf{Tingkatan Doa dan Sanad}: \textbf{Shahih}: HR. Muslim (no. 2718)
- \textit{Syarh an-Nawawi} (XVII/39)-dan Abu Dawud (no.  5086). Lihat 
\textit{Silsilah Ah\^{a}d\^{i}ts ash-Shah\^{i}hah} (no. 2638). \\
\textbf{Referensi}: Yazid bin Abdul Qadir Jawas. 2016. Kumpulan Do'a dari
Al-Quran dan As-Sunnah yang Shahih. Bogor: Pustaka Imam Asy-Syafi'i.
\index{musafir}	
\index{menjelang}
\index{subuh}
\footnote{Hanifah Atiya Budianto 1417051063 - Jurusan Ilmu Komputer,
Universitas Lampung}
\end{document}