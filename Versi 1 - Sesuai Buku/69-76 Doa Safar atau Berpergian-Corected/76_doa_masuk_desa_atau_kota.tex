\documentclass[a4paper,12pt]{article}
\usepackage{arabtex} 
\usepackage[bahasa] {babel}
\usepackage{calligra}
\usepackage[top=2cm,left=3cm,right=3cm,bottom=3cm]{geometry}
\title{\Large Doa Masuk Desa atau Kota}
\author{\calligra Hanifah Atiya Budianto}
\begin{document}
\sffamily
\maketitle 
\fullvocalize
\setcode{arabtex}
\begin{arabtext}
\noindent
al-ll_ahumma rabba al-ssamAwAti al-ssab`i wamA 'a.zlalna, warabba 
al-'a-r.di-yna al-ssab`i wamA 'aqlalna, warabba al-^s^sayA .ti-yni wamA 
'a.dlalna, warabba al-rriyA.hi wamA _dara-yna. fa'i-nnA nas'aluka _ha-yra 
h_a_dihi al-qaryaTi wa_ha-yra 'ahlihA, wa_ha-yra mA fi-yhA, wana`u-w_dubika
min ^sarrihA wa^sarri 'ahlihA wa^sarri mA fi-yhA.\\
\end{arabtext}
\noindent
\textbf{Artinya}:
\par
\indent
"Ya Allah, Rabb tujuh langit dan apa yang dinaunginya, Rabb tujuh bumi dan 
apa yang diatasnya, Rabb yang menguasai syaitan-syaitan dan apa yang mereka
sesatkan, Rabb yang menguasai angin dan apa yang dihembuskannya. Kami mohon
kepada-Mu kebaikan desa/kota ini, kebaikan penduduknya dan apa yang ada di 
dalamnya. Kami berlindung kepada-Mu dari keburukan desa/kota ini, keburukan
penduduknya dan apa yang ada di dalamnya."\\\\
\par
\noindent
\textbf{Tingkatan Doa dan Sanad}: \textbf{Shahih}: HR. An-Nasai dalam 
\textit{Sunanul Kubra} (no. 8775, 8776) dan \textit{'Amalul Yaum wal 
Lailah} (no. 547, 548). Ibnus Sunni dalam \textit{'Amalul Yaum wal Lailah} 
(524), Ibnu Khuzaimah (no. 2565), al-Hakim (II/100), dan lainnya dari 
Shuhaib bin Amr r.a. Al-Hakim menilai Hadits shahih. Imam adz-Dzahabi 
menyetujuinya. Lihat \textit{Shah\^{i}h al-Kalimith Thayyib} (no. 179), 
\textit{Silsilah Ah\^{a}d\^{i}ts ash-Shah\^{i}hah} (no. 2759), serta 
\textit{Shah\^{i}h al-Adzk\^{a}r} (no. 617/450).\\
\textbf{Referensi}: Yazid bin Abdul Qadir Jawas. 2016. Kumpulan Do'a dari
Al-Quran dan As-Sunnah yang Shahih. Bogor: Pustaka Imam Asy-Syafi'i.
\index{masuk}	
\index{desa}
\index{kota}
\footnote{Hanifah Atiya Budianto 1417051063 - Jurusan Ilmu Komputer,
Universitas Lampung}
\end{document}