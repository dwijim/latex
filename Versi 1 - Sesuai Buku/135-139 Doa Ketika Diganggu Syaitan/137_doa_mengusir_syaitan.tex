\documentclass[a4paper,12pt]{article}
\usepackage{arabtex} 
\usepackage[bahasa] {babel}
\usepackage{calligra}
\usepackage[top=2cm,left=3cm,right=3cm,bottom=3cm]{geometry}
\title{\Large Doa Mengusir Syaitan}
\author{\calligra Hanifah Atiya Budianto}
\begin{document}
\sffamily
\maketitle 
\fullvocalize
\setcode{arabtex}
\par
\noindent
\begin{enumerate}
\item Minta perlindungan Allah dari syaitan (dengan membaca: 
\textit{A'\^{u}dzu bill\^{a}hi minasy syaith\^{a}nir raj\^{i}m})
{\scriptsize 1} atau:
\begin{arabtext}
\noindent
'a`u-w_du bi-al-ll_ahi al-ssami-y`i al-`ali-ymi mina al-^s^sa-y.tAni 
al-rra^gi-ymi, min hamzihi wanaf_hihi wanaf_tihi.\\
\end{arabtext}
\noindent
\textbf{Artinya}:
\par
\indent
"Aku berlindung kepada Allah Yang Maha Mendengar lagi Maha Mengetahui dari 
gangguan syaitan yang terkutuk, dari kegilaannya, kesombongannya, dan 
syairnya yang tercela."{\scriptsize 2}
\item Ketika dikumandangkan adzan untuk shalat.{\scriptsize 3}
\item Membaca dzikir tertentu yang sudah diterangkan dalam hadits dan 
membaca al-Qur-an, misalnya: dua ayat terakhir dari surah Al-Baqarah, 
dzikir waktu pagi dan sore, dan ayat Kursi. {\scriptsize 4}\\\\
\end{enumerate}
\par
\noindent
\textbf{Tingkatan Doa dan Sanad}:
\begin{enumerate}
\item Dasarnya ayat-ayat al-Qur-an (Al-A'raf ayat 200, Al-Mu'min\^{u}m ayat
97-98, Fushshilat ayat 36) dan hadits Nabi yang shahih.
\item \textbf{Shahih}: HR. Abu Dawud (no. 775), at-Tirmidzi (no. 242), dan 
selainnya. Lihat \textit{al-Kalimuth Thayyib} (no. 130) dan 
\textit{Irw\^{a}-ul Ghal\^{i}l} (no. 341, 342).
\item \textbf{Shahih}: HR. Al-Bukhari (no. 608)/\textit{Fathul B\^{a}ri} 
(II/85), dan Muslim (no. 388, 389 [16-19]).
\item HR. Muslim (no. 780 [212]).
\end{enumerate}
\textbf{Referensi}: Yazid bin Abdul Qadir Jawas. 2016. Kumpulan Do'a dari
Al-Quran dan As-Sunnah yang Shahih. Bogor: Pustaka Imam Asy-Syafi'i.
\index{mengusir}
\index{syaitan}
\footnote{Hanifah Atiya Budianto 1417051063 - Jurusan Ilmu Komputer,
Universitas Lampung}
\end{document}