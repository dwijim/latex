\documentclass[a4paper,12pt]{article}
\usepackage{arabtex} 
\usepackage[bahasa] {babel}
\usepackage{calligra}
\usepackage[top=2cm,left=3cm,right=3cm,bottom=3cm]{geometry}
\title{\Large Doa Kaffarat Thiyarah}
\author{\calligra Hanifah Atiya Budianto}
\begin{document}
\sffamily
\maketitle 
\fullvocalize
\setcode{arabtex}
\par
\indent
Dari Abdullah bin Amr r.a., dia berkata bahwa Rasulullah bersabda: "Barang 
siapa mengurungkan niatnya karena thiyarah, maka ia telah berbuat syirik." 
Para Sahabat bertanya: "Lantas, apakah tebusannya?" Beliau menjawab: 
"Hendaklah ia mengucapkan:\\
\begin{arabtext}
\noindent
al-ll_ahumma lA _ha-yra 'illA _ha-yruka walA .ta-yra 'illA .ta-yruka walA 
'il_aha .ga-yruka.\\
\end{arabtext}
\noindent
\textbf{Artinya}:
\par
\indent
'Ya Allah, tidak ada kebaikan kecuali kebaikan dari Engkau, tidaklah burung
itu (yang dijadikan objek tathayyur) melainkan makhluk-Mu dan tiada ilah 
yang berhak diibadahi dengan benar kecuali Engkau.'"{\scriptsize 1}\\
\par
\indent
Tathayyur termasuk adat Jahiliyyah. Mereka biasa berpatokan pada burung. 
Apabila melihat burung itu terbang ke arah kanan, maka mereka gembira dan 
meneruskan niat. Apabila ia terbang ke arah kiri, mereka pun menganggap ia 
pembawa sial dan menangguhkan niat. Bahkan, mereka sengaja menerbangkan 
burung untuk meramal nasib.
\par
\indent
Syariat yang \textit{hanif} (lurus) ini telah melarang segala bentuk 
tathayyur. Sebab, \textit{thair} (burung) tidak memiliki keistimewaan apa 
pun hingga geraknya dijadikan petunjuk untung atau rugi. Di dalam banyak 
hadits, Rasulullah SAW. menegaskan: "Tidak ada thiyarah!"\\
\par
\indent
Penegasan tersebut juga dinukil dari sejumlah Sahabat r.a.\\
\par
\indent
Bukti lain yang menguatkan riwayat yang menafikan hal ini adalah larangan 
Rasulullah terhadap \textit{thiyarah} dan \textit{syu'm} (kesialan) secara 
umum serta pujian dari beliau terhadap orang-orang yang menjauhi keduanya. 
Dinukilkan bahwa beliau berabda:\\
\begin{arabtext}
\noindent
yad_hulu al-^gannaTa min 'ummati-y sab`u-wna 'alfaN bi.ga-yri .hisAbiN, 
humu alla_di-yna lA yastar qu-wna, walA yata.tayyaru-wna, wa`alY rabbihim 
yatawakkalu-wna.\\
\end{arabtext}
\noindent
\textbf{Artinya}:
\par
\indent
"Tujuh puluh ribu orang dari umatku akan masuk Jannah tanpa hisab. Mereka 
adalah orang-orang yang tidak meminta diruqyah, tidak bertathayyur dan 
hanya bertawakal kepada Allah semata."{\scriptsize 2}\\\\\\\\
\par
\noindent
\textbf{Tingkatan Doa dan Sanad}:
\begin{enumerate}
\item \textbf{Shahih}: HR. Ahmad (II/220). Dishahihkan Syaikh Ahmad Syakir 
dalam \textit{ta'liq Musnad Ahmad} (no. 7045), dan oleh Syaikh Nashiruddin 
al-Albani dalam \textit{Silsilah Ah\^{a}d\^{i}ts ash-Shah\^{i}hah} (no. 
1065).
\item \textbf{Shahih}: HR. Al-Bukhari (no. 6472) dari Sahabat Ibnu Abbas 
r.a. Diriwayatkan dengan lafazh panjang oleh al-Bukhari (no. 5705, 5752) 
dan Muslim (no. 220) juga dari Ibnu Abbas r.a.
\end{enumerate}
\textbf{Referensi}: Yazid bin Abdul Qadir Jawas. 2016. Kumpulan Do'a dari
Al-Quran dan As-Sunnah yang Shahih. Bogor: Pustaka Imam Asy-Syafi'i.
\index{kaffarat}	
\index{thiyarah}
\footnote{Hanifah Atiya Budianto 1417051063 - Jurusan Ilmu Komputer,
Universitas Lampung}
\end{document}