\documentclass[a4paper,12pt]{article}
\usepackage{arabtex} 
\usepackage[bahasa] {babel}
\usepackage{calligra}
\usepackage[top=2cm,left=3cm,right=3cm,bottom=3cm]{geometry}
\title{\Large Bacaan bagi Orang yang Ragu dalam Beriman}
\author{\calligra Hanifah Atiya Budianto}
\begin{document}
\sffamily
\maketitle 
\fullvocalize
\setcode{arabtex}
\noindent
\begin{enumerate}
\item Bagi siapa saja yang ragu-ragu dalam beriman, maka hendaklah ia 
memohon perlindungan kepada Allah.{\scriptsize 1}
\item Berhenti dari keraguan.{\scriptsize 2}\\
\indent Hendaklah mengucapkan:\\
\begin{arabtext}
\noindent
^Amantu bi-al-ll_ahi warusulihi.\\
\end{arabtext}
\noindent
\textbf{Artinya}:
\par
\indent
"Aku beriman kepada Allah dan kepada (kebenaran) para Rasul (utusan)-Nya."
{\scriptsize 3}\\\\
\indent
Selain itu, hendaklah dia membaca firman-Nya: 
\begin{arabtext}
\noindent
huwa al-'awwalu wa-al-'a_hiru wAl-.z.z_ahiru wa-al-bA.tinu wahuwa bikulli 
^saY'iN `aliymuN.\\
\end{arabtext}
\noindent
\textbf{Artinya}:
\par
\indent
\textit{"Dialah yang awal (Allah telah ada sebelum segala sesuatu ada), 
yang akhir (disaat segala sesuatu telah hancur, Allah masih tetap kekal), 
yang zhahir (Dialah yang nyata, sebab banyak bukti yang menyatakan adanya 
Allah), yang bathin (tidak ada sesuatu yang bisa menghalangi-Nya. Allah 
lebih dekat kepada hamba-Nya daripada mereka kepada dirinya). Dialah Yang 
Maha Mengetahui atas segala sesuatu."} (QS. Al-Had\^{i}d [57]: 3).
{\scriptsize 4}\\\\
\end{enumerate}
\par
\noindent
\textbf{Tingkatan Doa dan Sanad}:
\begin{enumerate}
\item \textbf{Shahih}: HR. Al-Bukhari/\textit{Fathul B\^{a}ri} (VI/336) dan
Muslim (I/120).
\item \textbf{Shahih}: HR. Al-Bukhari/\textit{Fathul B\^{a}ri} (VI/336) dan
Muslim (I/120), pada Bab "Bay\^{a}nil Waswasah fil \^{I}m\^{a}n wa m\^{a} 
Yaq\^{u}luhu man Wajadaha".
\item \textbf{Shahih}: HR. Muslim (no. 134).
\item \textbf{Atsar Hasan}: Diriwayatkan oleh Abu Dawud (no. 5110), pada 
Bab "F\^{i} Raddil Waswasah" dari perkataan Ibnu Abbas r.a. Lihat 
\textit{Shah\^{i}h Abi Dawud} (III/962).
\end{enumerate}
\textbf{Referensi}: Yazid bin Abdul Qadir Jawas. 2016. Kumpulan Do'a dari
Al-Quran dan As-Sunnah yang Shahih. Bogor: Pustaka Imam Asy-Syafi'i.
\index{orang}
\index{ragu}
\index{beriman}
\footnote{Hanifah Atiya Budianto 1417051063 - Jurusan Ilmu Komputer,
Universitas Lampung}
\end{document}