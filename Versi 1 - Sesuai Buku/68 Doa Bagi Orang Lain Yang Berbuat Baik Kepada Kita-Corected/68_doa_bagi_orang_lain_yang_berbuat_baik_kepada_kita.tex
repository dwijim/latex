\documentclass[a4paper,12pt]{article}
\usepackage{arabtex} 
\usepackage[bahasa] {babel}
\usepackage{calligra}
\usepackage[top=2cm,left=3cm,right=3cm,bottom=3cm]{geometry}
\title{\Large Doa Bagi Orang Lain yang Berbuat Baik kepada Kita}
\author{\calligra Hanifah Atiya Budianto}
\begin{document}
\sffamily
\maketitle 
\fullvocalize
\setcode{arabtex}
\begin{arabtext}
\noindent
^gazAka al-ll_ahu _ha-yraN.\\
\end{arabtext}
\noindent
\textbf{Artinya}:
\par
\indent
"Semoga Allah membalasmu dengan sesuatu yang lebih baik."\\\\
\par
\noindent
\textbf{Tingkatan Doa dan Sanad}: \textbf{Shahih}: HR. At-Tirmidzi (no.
2035), an-Nasa'I dalam kitab \textit{'Amalul Yaum wal Lailah} (no. 180) dan
Ibnu Hibban (no. 3404). Lihat \textit{Shah\^{i}h al-J\^{a}hmi'ish 
Shagh\^{i}r} (no. 6368) dan \textit{Shah\^{i}h At-Targh\^{i}b wat 
Tarh\^{i}b} (I/571 no. 969).\\
\textbf{Referensi}: Yazid bin Abdul Qadir Jawas. 2016. Kumpulan Do'a dari
Al-Quran dan As-Sunnah yang Shahih. Bogor: Pustaka Imam Asy-Syafi'i.
\index{orang}	
\index{berbuat}
\index{baik}	
\index{kita}
\footnote{Hanifah Atiya Budianto 1417051063 - Jurusan Ilmu Komputer,
Universitas Lampung}
\end{document}