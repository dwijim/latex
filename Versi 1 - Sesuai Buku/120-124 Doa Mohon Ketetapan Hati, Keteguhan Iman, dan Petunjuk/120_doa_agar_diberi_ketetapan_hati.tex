\documentclass[a4paper,12pt]{article}
\usepackage{arabtex} 
\usepackage[bahasa] {babel}
\usepackage{calligra}
\usepackage[top=2cm,left=3cm,right=3cm,bottom=3cm]{geometry}
\title{\Large Doa Agar Diberi Ketetapan Hati}
\author{\calligra Hanifah Atiya Budianto}
\begin{document}
\sffamily
\maketitle 
\fullvocalize
\setcode{arabtex}
\begin{arabtext}
\noindent
al-ll_ahumma mu.sarrifa al-qulu-wbi, .sarrif qulu-wbanA `alY .tA`atika.\\
\end{arabtext}
\noindent
\textbf{Artinya}:
\par
\indent
"Ya Allah, yang megarahkan hati, arahkanlah hati-hati kami untuk taat 
kepada-Mu."{\scriptsize 1}\\
\begin{arabtext}
\noindent
yA muqaliba al-qulu-wbi, _tabbit qalbi-y `alY di-ynika.\\
\end{arabtext}
\noindent
\textbf{Artinya}:
\par
\indent
"Wahai Yang membolak-balikkan hati, teguhkanlah hatiku pada agamu-Mu."
{\scriptsize 2}\\\\
\par
\noindent
\textbf{Tingkatan Doa dan Sanad}:
\begin{enumerate}
\item \textbf{Shahih}: HR. Muslim (no. 2654) dari Abdullah bin Amr bin 
al-Ash r.a.
\item \textbf{Shahih}: HR. At-Tirmidzi (no. 3522), Ahmad (VI/302, 315) dari 
Ummu Salamah r.a., dan al-Hakim (I/525) dari an-Nawas bin Sam'an. 
Dishahihkan dan disepakati oleh adz-Dzahabi. Lihat juga \textit{Shah\^{i}h 
at-Tirmidzi} (III/171), no. 2792). Ummu Salamah berkata: "Doa itu adalah 
doa Nabi SAW. yang paling sering dibaca."
\end{enumerate}
\textbf{Referensi}: Yazid bin Abdul Qadir Jawas. 2016. Kumpulan Do'a dari
Al-Quran dan As-Sunnah yang Shahih. Bogor: Pustaka Imam Asy-Syafi'i.
\index{diberi}	
\index{ketetapan}
\index{hati}
\footnote{Hanifah Atiya Budianto 1417051063 - Jurusan Ilmu Komputer,
Universitas Lampung}
\end{document}