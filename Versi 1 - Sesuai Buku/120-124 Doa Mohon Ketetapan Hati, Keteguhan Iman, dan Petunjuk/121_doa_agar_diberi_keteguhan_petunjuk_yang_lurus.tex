\documentclass[a4paper,12pt]{article}
\usepackage{arabtex} 
\usepackage[bahasa] {babel}
\usepackage{calligra}
\usepackage[top=2cm,left=3cm,right=3cm,bottom=3cm]{geometry}
\title{\Large Doa agar Diberi Keteguhan Petunjuk yang Lurus}
\author{\calligra Hanifah Atiya Budianto}
\begin{document}
\sffamily
\maketitle 
\fullvocalize
\setcode{arabtex}
\begin{arabtext}
\noindent
al-ll_ahumma _tabbitni-y, wA^g`alni-y hAdiyaN mahdiyyaN.\\
\end{arabtext}
\noindent
\textbf{Artinya}:
\par
\indent
"Ya Allah, teguhkanlah diriku, jadikanlah diriku pemberi petunjuk dan 
diberi petunjuk (oleh-Mu)."{\scriptsize 1}\\
\begin{arabtext}
\noindent
al-ll_ahumma ahdini-y wasaddini-y, al-ll_ahumma 'inni-y 'as'aluka al-hudY 
wal-ssadAda.\\
\end{arabtext}
\noindent
\textbf{Artinya}:
\par
\indent
"Ya Allah, berilah petunjuk kepadaku, dan luruskanlah aku. Ya Allah, 
sungguh aku memohon petunjuk dan kelurusan kepada-Mu."{\scriptsize 2}\\\\
\par
\noindent
\textbf{Tingkatan Doa dan Sanad}:
\begin{enumerate}
\item Doa ini diambil dari doa Rasulullah SWT. untuk Jarir. 
\textbf{Shahih}: HR. Al-Bukhari (no. 6333) dan Muslim (no. 2476).
\item \textbf{Shahih}: HR. Muslim (no. 2725).
\end{enumerate}
\textbf{Referensi}: Yazid bin Abdul Qadir Jawas. 2016. Kumpulan Do'a dari
Al-Quran dan As-Sunnah yang Shahih. Bogor: Pustaka Imam Asy-Syafi'i.
\index{diberi}
\index{keteguhan}
\index{petunjuk}
\index{lurus}
\footnote{Hanifah Atiya Budianto 1417051063 - Jurusan Ilmu Komputer,
Universitas Lampung}
\end{document}