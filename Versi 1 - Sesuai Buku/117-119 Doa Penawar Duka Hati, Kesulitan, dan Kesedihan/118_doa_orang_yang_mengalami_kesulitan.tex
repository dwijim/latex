\documentclass[a4paper,12pt]{article}
\usepackage{arabtex} 
\usepackage[bahasa] {babel}
\usepackage{calligra}
\usepackage[top=2cm,left=3cm,right=3cm,bottom=3cm]{geometry}
\title{\Large Doa Orang yang Mengalami Kesulitan}
\author{\calligra Hanifah Atiya Budianto}
\begin{document}
\sffamily
\maketitle 
\fullvocalize
\setcode{arabtex}
\begin{arabtext}
\noindent
al-ll_ahumma lA sahla 'illA mA^ga`altahu sahlaN, wa'anta ta^g`alu al-.hazna
'i_dA ^si'ta sahlaN.\\
\end{arabtext}
\noindent
\textbf{Artinya}:
\par
\indent
"Ya Allah, tidak ada kemudahan kecuali apa yang Engkau jadikan mudah. 
Sedang yang susah bisa Engkau jadikan mudah, apabila Engkau 
menghendakinya."\\\\
\par
\noindent
\textbf{Tingkatan Doa dan Sanad}: \textbf{Shahih}: HR. Ibnu Hibban 
(\textit{at-Ta'l\^{i}q\^{a}tul His\^{a}n} [no. 970], dan 
\textit{Maw\^{a}ridizh Zh\^{a}m-an} [no. 2427]) \textit{Shah\^{i}h 
Maw\^{a}ridizh Zh\^{a}m-an} (II/450 no. 2058) dan Ibnus Sunni dalam 
\textit{'Amalul Yaum wal Lailah} (no. 351). Al-Hafizh berkata: "Hadits ini 
shahih." Lihat \textit{Silsilah Ah\^{a}d\^{i}ts ash-Shah\^{i}hah} (no. 
2886).\\
\textbf{Referensi}: Yazid bin Abdul Qadir Jawas. 2016. Kumpulan Do'a dari
Al-Quran dan As-Sunnah yang Shahih. Bogor: Pustaka Imam Asy-Syafi'i.
\index{sulit}	
\footnote{Hanifah Atiya Budianto 1417051063 - Jurusan Ilmu Komputer,
Universitas Lampung}
\end{document}