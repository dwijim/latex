\documentclass[a4paper,12pt]{article}
\usepackage{arabtex} 
\usepackage[bahasa] {babel}
\usepackage{calligra}
\usepackage[top=2cm,left=3cm,right=3cm,bottom=3cm]{geometry}
\usepackage{xcolor, framed}
\definecolor{shadecolor}{rgb}{0.8,0.8,0.8}
\title{\Large Doa Saat Mengalami Kesusahan, Kesedihan, dan Penawar 
Kedukaan}
\author{\calligra Hanifah Atiya Budianto}
\begin{document}
\sffamily
\maketitle 
\fullvocalize
\setcode{arabtex}
\begin{arabtext}
\noindent
lA 'il_aha 'illA al-ll_ahu al-`a.zi-ymu al-.hali-ymu, lA 'il_aha 'illA 
al-ll_ahu rabbu al-`ar^si al-`a.zi-ymi, lA 'il_aha 'illA al-ll_ahu rabbu 
al-ssamAwAti, warabbu al-'ar.di, warabbu al-`ar^si al-kari-ymi.\\
\end{arabtext}
\noindent
\textbf{Artinya}:
\par
\indent
"Tidak ada ilah yang berhak diibadahi dengan benar melainkan hanya Allah, 
Rabb Yang Mahaagung lagi Maha Penyantun. Tidak ada ilah yang berhak 
diibadahi dengan benar melainkan Allah, Pemilik Arsy yang agung. Tidak ada 
ilah yang berhak diibadahi dengan benar melainkan hanya Allah, Rabb langit 
dan Rabb bumi, Pemilik Arsy yang mulia."{\scriptsize 1}\\
\begin{arabtext}
\noindent
al-ll_ahumma 'inni-y `abduka, wAbnu `abdika, wAbnu 'amatika, nA.siyati-y 
biyadika, mA.diN fiyya .hukmuka, `adluN fiyya qa.dA'uka. 'as'aluka bikulli 
asmiN huwa laka, samma-yta bihi nafsaka, 'a-w 'anzaltahu fi-y kitAbika, 
'a-w `allamtahu 'a.hadaN min _halqika, 'awi asta'_tarta bihi fi-y `ilmi 
al-.ga-ybi `indaka, 'an ta^g`ala al-qur-^Ana rabi-y`a qalbi-y, wanu-wra 
.sadri-y, wa^galA'a .huzni-y, wa_dahAba hammi-y.\\
\end{arabtext}
\noindent
\textbf{Artinya}:
\par
\indent
"Ya Allah, sesungguhnya aku adalah hamba-Mu, anak hamba-Mu (Adam), dan anak
hamba perempuan-Mu (Hawa), ubun-ubunku berada di tangan-Mu, hukum-Mu 
berlaku terhadap diriku dan ketetapan-Mu adil pada diriku. Aku memohon 
kepada-Mu dengan segala Nama yang menjadi milik-Mu, yang Engkau namai 
diri-Mu dengannya, atau yang Engkau turunkan di dalam Kitab-Mu, atau yang 
Engkau ajarkan kepada seorang dari makhluk-Mu, atau yang Engkau rahasiakan 
di dalam ilmu ghaib di sisi-Mu, maka dengannya aku memohon supaya Engkau 
menjadikan al-Qur-an penyejuk bagi hatiku, cahaya bagi dadaku, pelipur bagi
kesedihanku, dan penghilang kesusahanku."
\begin{shaded*}
\noindent
Melainkan Allah akan menghilangkan kesedihannya dan kesusahannya (orang 
yang mengucapkan doa ini) serta menggantikan semuanya itu dengan 
kegembiraan.{\scriptsize 2}
\end{shaded*}
\begin{arabtext}
\noindent
al-ll_ahu, al-ll_ahu rabbi-y, lA 'u^sriku bihi ^say'aN.\\
\end{arabtext}
\noindent
\textbf{Artinya}:
\par
\indent
"Allah, Allah  adalah Rabbku, aku tidak menyekutukan-Nya dengan sesuatu 
apapun."{\scriptsize 3}\\\\
\par
\noindent
\textbf{Tingkatan Doa dan Sanad}:
\begin{enumerate}
\item \textbf{Shahih}: HR. Al-Bukhari (no. 6345, 6346, 7426, 7431), Muslim 
(no. 2730), at-Tirmidzi (no. 3435), Ibnu Majah (no. 3883), dan Ahmad 
(I/228, 259, 268, 280) dari Ibnu Abbas r.a.
\item \textbf{Shahih}: HR. Ahmad (I/391, 452), Ibnu Hibban 
(\textit{at-Ta'l\^{i}q\^{a}tul His\^{a}n} [no. 968]), al-Hakim (I/509), dan
ath-Thabrani dalam \textit{al-Mu'jamul Kab\^{i}r} (X/169-170, no. 352) dari
Abdullah bin Mas'ud r.a. Dihasankan al-Hafizh dalam \textit{Takhr\^{i}j 
al-Adzk\^{a}r}. Dishahihkan Syaikh al-Albani. Lihat \textit{al-Kalimuth 
Thayyib} (hlm. 119, no. 124) dan \textit{Silsilah Ah\^{a}d\^{i}ts 
ash-Shah\^{i}hah} (no. 199).
\item \textbf{Shahih}: HR. Abu Dawud (no. 1525), Ibnu Majah (no. 3882), dan
lihat \textit{Silsilah Ah\^{a}d\^{i}ts ash-Shah\^{i}hah} (no. 2755).
\end{enumerate}
\textbf{Referensi}: Yazid bin Abdul Qadir Jawas. 2016. Kumpulan Do'a dari
Al-Quran dan As-Sunnah yang Shahih. Bogor: Pustaka Imam Asy-Syafi'i.
\index{susah}	
\index{sedih}
\index{duka}
\footnote{Hanifah Atiya Budianto 1417051063 - Jurusan Ilmu Komputer,
Universitas Lampung}
\end{document}