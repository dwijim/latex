\documentclass[a4paper,12pt]{article}
\usepackage{arabtex} 
\usepackage[bahasa] {babel}
\usepackage{calligra}
\usepackage[top=2cm,left=3cm,right=3cm,bottom=3cm]{geometry}
\title{\Large Doa Setelah Wudhu}
\author{\calligra Hanifah Atiya Budianto}
\begin{document}
\sffamily
\maketitle 
\fullvocalize
\setcode{arabtex}
\begin{arabtext}
\noindent
'a^shadu 'an lA 'il_aha 'illA al-ll_ahu wa.hdahu lA ^sari-yka lahu 
wa'a^shadu 'anna mu.hammadaN `abduhu warasu-wluhu.\\
\end{arabtext}
\noindent
\textbf{Artinya}:
\par
\indent
"Aku bersaksi bahwa tidak ada ilah yang berhak diibadahi dengan benar 
kecuali hanya Allah, Yang Maha Esa, tiada sekutu bagi-Nya. Dan aku bersaksi
bahwa Muhammad adalah hamba dan Rasul-Nya." {\scriptsize 1}\\
\begin{arabtext}
\noindent
al-ll_ahumma a^g`alniy mina al-ttawwAbi-yna wA^g`alni-y mina 
al-muta.tahhiri-yna.\\
\end{arabtext}
\noindent
\textbf{Artinya}:
\par
\indent
"Ya Allah jadikanlah aku termasuk orang-orang yang bertaubat dan jadikanlah
aku termasuk orang-orang (yang senang) bersuci." {\scriptsize 2}\\\\
\par
\noindent
\textbf{Tingkatan Doa dan Sanad}:
\begin{enumerate}
\item \textbf{Shahih}: HR. Muslim (I/209-210, no. 234).
\item \textbf{Shahih}: HR. At-Tirmidzi (no. 55). Lihat \textit{Shah\^{i}h 
at-Tirmidzi} (I/8, no. 48). Dishahihkan oleh Syaikh al-Albani.
\end{enumerate}
\textbf{Referensi}: Yazid bin Abdul Qadir Jawas. 2016. Kumpulan Do'a dari
Al-Quran dan As-Sunnah yang Shahih. Bogor: Pustaka Imam Asy-Syafi'i.
\index{wudhu}	
\index{setelah}
\footnote{Hanifah Atiya Budianto 1417051063 - Jurusan Ilmu Komputer,
Universitas Lampung}
\end{document}