\documentclass[a4paper,12pt]{article}
\usepackage{arabtex} 
\usepackage[bahasa] {babel}
\usepackage{calligra}
\usepackage[top=2cm,left=3cm,right=3cm,bottom=3cm]{geometry}
\title{\Large Doa Masuk Masjid}
\author{\calligra Hanifah Atiya Budianto}
\begin{document}
\sffamily
\maketitle 
\fullvocalize
\setcode{arabtex}
\begin{arabtext}
\noindent
'a`u-w_du bi-al-ll_ahi al-`a.zi-ymi, wa biwa^ghihi al-kariymi, wasul.tAnihi 
al-qadiymi, mina al-^s^say.tAni al-rra^gi-ymi.\\
\end{arabtext}
\noindent
\textbf{Artinya}:
\par
\indent
"Aku berlindung kepada Allah Yang Mahaagung, dengan wajah-Nya yang mulia 
dan kekuasaan-Nya yang abadi, dari syaitan yang terkutuk."{\scriptsize 1}\\
\begin{arabtext}
\noindent
bismi al-ll_ahi wAl-.s.salATu wAl-ssalAmu `al_aY rasu-wli al-ll_ahi, 
al-ll_ahumma afta.h li-y 'abwAba ra.hmatika.\\
\end{arabtext}
\noindent
\textbf{Artinya}:
\par
\indent
"Dengan nama Allah, semoga shalawat dan salam tercurah kepada Rasulullah.
{\scriptsize 2} Ya Allah, bukakanlah untukku pintu-pintu rahmat-Mu."
{\scriptsize 3}\\\\
\par
\noindent
\textbf{Tingkatan Doa dan Sanad}:
\begin{enumerate}
\item \textbf{Shahih}: HR. Abu Dawud (no. 466) - \textit{Shah\^{i}h Abi 
Dawud} (I/93), no. 441). Jika dia berucap demikian, syaitan pun berkata: 
"Orang ini terjaga (terlindungi) dari diriku sepanjang hari."
\item \textbf{Shahih}: HR. Abu Dawud (no. 465), lihat \textit{Shah\^{i}h 
al-J\^{a}mi'ish Shagh\^{i}r} (no. 514); Ibnus Sunni dalam \textit{'Amalul 
Yaum wal Lailah} (no. 88). Dinyatakan hasan oleh Syaikh al-Albani dalam 
\textit{al-Kalimuth Thayyib} (hlm. 92, no. 64, catatan kaki no. 52).
\item \textbf{Shahih}: HR. Muslim (no. 713).
\end{enumerate}
\textbf{Referensi}: Yazid bin Abdul Qadir Jawas. 2016. Kumpulan Do'a dari
Al-Quran dan As-Sunnah yang Shahih. Bogor: Pustaka Imam Asy-Syafi'i.
\index{masuk}	
\index{masjid}
\footnote{Hanifah Atiya Budianto 1417051063 - Jurusan Ilmu Komputer,
Universitas Lampung}
\end{document}