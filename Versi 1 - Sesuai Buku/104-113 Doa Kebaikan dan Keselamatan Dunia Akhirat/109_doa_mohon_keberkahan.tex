\documentclass[a4paper,12pt]{article}
\usepackage{arabtex} 
\usepackage[bahasa] {babel}
\usepackage{calligra}
\usepackage[top=2cm,left=3cm,right=3cm,bottom=3cm]{geometry}
\title{\Large Doa Mohon Keberkahan}
\author{\calligra Hanifah Atiya Budianto}
\begin{document}
\sffamily
\maketitle 
\fullvocalize
\setcode{arabtex}
\begin{arabtext}
\noindent
al-ll_ahumma 'ak_tir mAli-y wawaladi-y, wabArik li-y fiymA 'a`.ta-ytani-y, 
(wa'a.til .hayAti-y `alY .tA`atika, wa'a.hsin `amali-y, wA.gfirli-y).\\
\end{arabtext}
\noindent
\textbf{Artinya}:
\par
\indent
"Ya Allah, perbanyaklah harta dan juga anakku, serta berilah berkah 
kepadaku atas apa yang telah Engkau karuniakan kepadaku.{\scriptsize 1}
[Panjangkan kehidupanku pada ketaatan terhadap-Mu, perbaikilah amal 
perbuatanku, dan berikan ampunan kepadaku]."{\scriptsize 2}\\\\
\par
\noindent
\textbf{Tingkatan Doa dan Sanad}:
\begin{enumerate}
\item \textbf{Shahih}: HR. Al-Bukhari (no. 6378-6381), Muslim (no. 2480, 
2481) dari Ummu Sulaim r.a.
\item \textbf{Shahih}: HR. Al-Bukhari dalam \textit{al-Adabul Mufrad} (no. 
653). Dishahihkan oleh al-Albani dalam \textit{Silsilah Ah\^{a}d\^{i}ts 
ash-Shah\^{i}hah} (no. 2241) dan \textit{Shah\^{i}h al-Adabil Mufrad} 
(hlm. 244, no. 508).
\end{enumerate}
\textbf{Referensi}: Yazid bin Abdul Qadir Jawas. 2016. Kumpulan Do'a dari
Al-Quran dan As-Sunnah yang Shahih. Bogor: Pustaka Imam Asy-Syafi'i.
\index{mohon}	
\index{keberkahan}
\footnote{Hanifah Atiya Budianto 1417051063 - Jurusan Ilmu Komputer,
Universitas Lampung}
\end{document}