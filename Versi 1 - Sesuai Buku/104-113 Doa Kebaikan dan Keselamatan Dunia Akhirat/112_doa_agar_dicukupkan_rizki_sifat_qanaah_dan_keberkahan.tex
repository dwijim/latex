\documentclass[a4paper,12pt]{article}
\usepackage{arabtex} 
\usepackage[bahasa] {babel}
\usepackage{calligra}
\usepackage[top=2cm,left=3cm,right=3cm,bottom=3cm]{geometry}
\title{\Large Doa agar Dicukupkan Rizki yang Halal, Sifat Qana'ah, dan
Keberkahan}
\author{\calligra Hanifah Atiya Budianto}
\begin{document}
\sffamily
\maketitle 
\fullvocalize
\setcode{arabtex}
\begin{arabtext}
\noindent
al-ll_ahumma qanni`ni-y bimA razaqtani-y, wabArik li-y fi-yhi, wA_hluf `alY 
kulli .gA'ibaTiN li-y bi_ha-yriN.\\
\end{arabtext}
\noindent
\textbf{Artinya}:
\par
\indent
"Ya Allah, jadikan aku merasa \textit{qana'ah} (cukup, puas, rela) terhadap
segala yang telah Engkau rizkikan kepadaku, dan berilah berkah kepadaku di 
dalamnya dan gantikan bagiku semua yang hilang dariku dengan yang lebih 
baik."\\\\
\par
\noindent
\textbf{Tingkatan Doa dan Sanad}: \textbf{Shahih}: HR. Al-Hakim (I/510) dan
dishahihkannya serta disepakati oleh adz-Dzahabi dari Ibnu Abbas r.a.\\
\textbf{Referensi}: Yazid bin Abdul Qadir Jawas. 2016. Kumpulan Do'a dari
Al-Quran dan As-Sunnah yang Shahih. Bogor: Pustaka Imam Asy-Syafi'i.
\index{diberi}	
\index{rizki}
\index{sifat}	
\index{qanaah}
\index{keberkahan}	
\footnote{Hanifah Atiya Budianto 1417051063 - Jurusan Ilmu Komputer,
Universitas Lampung}
\end{document}