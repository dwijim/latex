\documentclass[a4paper,12pt]{article}
\usepackage{arabtex} 
\usepackage[bahasa] {babel}
\usepackage{calligra}
\usepackage[top=2cm,left=3cm,right=3cm,bottom=3cm]{geometry}
\title{\Large Doa Mohon Ampunan dan Kasih Sayang}
\author{\calligra Hanifah Atiya Budianto}
\begin{document}
\sffamily
\maketitle 
\fullvocalize
\setcode{arabtex}
\begin{arabtext}
\noindent
rabbi a.gfirli-y, watub `alayya, 'innaka 'anta al-ttawwAbu al-.gafu-wru.\\
\end{arabtext}
\noindent
\textbf{Artinya}:
\par
\indent
"Ya Rabbku, ampunilah aku, terimalah taubatku, sesungguhnya Engkau adalah 
Yang Maha Penerima taubat lagi Yang Maha Pengampun."{\scriptsize 1}\\
\begin{arabtext}
\noindent
al-ll_ahumma 'inni-y .zalamtu nafsi-y .zulmaN ka_ti-yraN, walA ya.gfiru 
al-_d_dunu-wba 'illA 'anta, fA.gfir li-y ma.gfiraTaN min `indika, 
wAr.hamni-y, 'innaka 'anta al-.gafu-wru al-rra.hi-ymu.\\
\end{arabtext}
\noindent
\textbf{Artinya}:
\par
"Ya Allah, sesungguhnya aku telah menzhalimi diriku dengan kezhaliman yang 
banyak, dan tidak ada yang dapat mengampuni dosa melainkan Engkau. Oleh 
karena itu ampunilah aku dengan ampunan yang datang dari sisi-Mu, dan 
rahmatilah aku, sesungguhnya Engkau adalah Yang Maha Pengampun lagi Maha 
Penyayang."{\scriptsize 2}\\
\begin{arabtext}
\noindent
al-ll_ahumma 'inni-y 'as'aluka yA Aal-ll_ahu, bi-'annaka al-wA.hidu 
al-'a.hadu al-.s.samadu, alla_di-y lam yalid walam yu-wlad, walam yakun 
lahu kufu-waN 'a.haduN, 'an ta.gfira li-y _dunu-wbi-y, 'innaka 'anta 
al-.gafu-wru al-rra.hi-ymu.\\
\end{arabtext}
\noindent
\textbf{Artinya}:
\par
"Ya Allah, sesungguhnya aku memohon kepada-Mu ya Allah, karena Engkau 
adalah satu-satunya Yang Maha Esa, yang bergantung kepada-Mu seluruh 
makhluk, yang tidak beranak dan tidak pula diperanakkan, serta tidak ada 
seorang pun yang sebanding dengan-Nya, agar Engkau memberikan ampunan 
kepadaku atas dosa-dosaku, sesungguhnya Engkau Maha Pengampun lagi Maha 
Penyayang."{\scriptsize 3}\\
\begin{arabtext}
\noindent
al-ll_ahumma a.gfir li-y _ha.ti-y'ati-y, wa^gahli-y, wa-'isrAfi-y fi-y 
'amri-y, wamA 'anta 'a`lamu bihi minni-y, al-ll_ahumma a.gfir liy haz li-y,
wa^giddi-y, wa_ha.ta'iy, wa`amdi-y, wakulla _d_alika `indi-y.\\
\end{arabtext}
\noindent
\textbf{Artinya}:
\par
"Ya Allah, ampunilah aku dari setiap kesalahanku, setiap kebodohanku, serta
sikap berlebihan dalam urusanku, dan atas segala sesuatu yang lebih Engkau
ketahui daripada diriku ini. Ya Allah, berilah ampunan kepadaku atas canda 
dan keseriusanku, juga kekeliruan dan kesengajaanku, dan semuanya itu ada 
pada diriku."{\scriptsize 4}\\
\begin{arabtext}
\noindent
al-ll_ahumma .tahhir ni-y mina al-_d_dunu-wbi wAl-_ha.tAyA, al-ll_ahumma 
naqqini-y minhA, kamA yunaqqY al-_t_tawbu al-'abya.du mina al-ddanasi, 
al-ll_ahumma .tahhir ni-y bi-al-_t_tal^gi, wAl-baradi, wAl-mA'i al-bAridi.
\\
\end{arabtext}
\noindent
\textbf{Artinya}:
\par
"Ya Allah, sucikanlah aku dari berbagai dosa dan kesalahan. Ya Allah, 
bersihkan diriku darinya sebagaimana dibersihkannya kain putih dari 
kotoran. Ya Allah, sucikanlah diriku dengan salju, embun, dan air yang 
dingin."{\scriptsize 5}\\\\
\par
\noindent
\textbf{Tingkatan Doa dan Sanad}: 
\begin{enumerate}
\item Abdullah bin Umar berkata: "Aku menghitung kalimat yang diucapkan 
Rasulullah: '\textit{Rabbighfirl\^{i} watub 'alayya innaka antat 
taww\^{a}bul ghaf\^{u}r}' dalam satu majelis sebanyak seratus kali." 
\textbf{Hasan Shahih}: HR. Abu Dawud (no. 1516), at-Tirmidzi (no. 3434), 
Ibnu Majah (no. 3814). Lafazhnya milik at-Tirmidzi, dan dia menyatakan: 
"Hadits \textit{hasan shahih gharib}." Lihat \textit{Shah\^{i}h 
al-J\^{a}mi-us Shaghir} (no. 3486) dan \textit{Silsilah Ah\^{a}d\^{i}ts 
ash-Shah\^{i}hah} (no. 556).
\item \textbf{Shahih}: HR. Al-Bukhari (no. 834), Bab "ad-Du'\^{a}' qabla 
Sal\^{a}m" dan Muslim (no. 2705 [48]) dari Abu Bakar ash-Shiddiq r.a. Doa 
ini dibaca setelah tasyahud akhir sebelum salam. 
\item \textbf{Shahih}: HR. An-Nasai dengan lafazhnya (III/52), Ahmad 
(IV/338). Lihat \textit{Shah\^{i}h an-Nasai} (I/279). Pada akhir riwayat, 
Nabi SAW. bersabda: "Allah telah mengampuni dosanya." - Beliau 
mengucapkannya tiga kali. 
\item \textbf{Shahih}: HR. Al-Bukhari (no. 6399)/\textit{Fathul B\^{a}ri} 
(XI/196), dari Abu Musa al-Asy'ari r.a.
\item \textbf{Shahih}: HR. Muslim (no. 476 [204]), an-Nasai (I/198, 199) 
dan at-Tirmidzi (no. 3547) dari Abdullah bin Abi Aufa. Lafazh ini milik 
an-Nasai. Lihat \textit{Shah\^{i}h an-Nasai} (I/86).
\end{enumerate}
\textbf{Referensi}: Yazid bin Abdul Qadir Jawas. 2016. Kumpulan Do'a dari
Al-Quran dan As-Sunnah yang Shahih. Bogor: Pustaka Imam Asy-Syafi'i.
\index{mohon}	
\index{ampun}
\index{kasih}	
\index{sayang}
\footnote{Hanifah Atiya Budianto 1417051063 - Jurusan Ilmu Komputer,
Universitas Lampung}
\end{document}