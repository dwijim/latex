\documentclass[a4paper,12pt]{article}
\usepackage{arabtex} 
\usepackage[bahasa] {babel}
\usepackage{calligra}
\usepackage[top=2cm,left=3cm,right=3cm,bottom=3cm]{geometry}
\title{\Large Doa Mendapatkan Kebaikan Dunia dan Akhirat}
\author{\calligra Hanifah Atiya Budianto}
\begin{document}
\sffamily
\maketitle 
\fullvocalize
\setcode{arabtex}
\begin{arabtext}
\noindent
al-ll_ahumma 'inni-y 'as'aluka al-`AfiyaTi, fiy al-ddunyA wAl-^A_hiraTi.\\
\end{arabtext}
\noindent
\textbf{Artinya}:
\par
\indent
"Ya Allah, sesungguhnya aku memohon kepada-Mu \textit{'afiat} (dijauhkan 
dari petaka) di dunia dan di akhirat."{\scriptsize 1}\\
\begin{arabtext}
\noindent
al-ll_ahumma 'inni-y 'as'aluka al-^gannaTa wa'a`u-w_du bika mina al-nnAri.
\\
\end{arabtext}
\noindent
\textbf{Artinya}:
\par
\indent
"Ya Allah, aku memohon kepada-Mu agar dimasukkan ke dalam Surga dan aku 
berlindung kepada-Mu dari siksa Neraka."{\scriptsize 2}\\
\begin{arabtext}
\noindent
al-ll_ahumma ^AtinA fi-y al-ddunyA .hasanaTaN, wafi-y al-^A_hiraTi 
.hasanaTaN, waqinA `a_dAba al-nnAri.\\
\end{arabtext}
\noindent
\textbf{Artinya}:
\par
\indent
"Ya Allah, berikanlah kebaikan kepada kami di dunia dan kebaikan di 
akhirat. (Ya Allah,) lindungilah kami dari adzab Neraka."{\scriptsize 3}
\\\\
\par
\noindent
\textbf{Tingkatan Doa dan Sanad}:
\begin{enumerate}
\item \textbf{Shahih}: HR. Ahmad (I/209), al-Bukhari dalam 
\textit{al-Adabul Mufard} (no. 726), dan at-Tirmidzi (no. 3514).
\item \textbf{Shahih}: HR. Abu Dawud (no. 792), Ibnu Majah (no. 910), dan 
Ibnu Khuzaimah (no. 725). Dishahihkan oleh Imam Ibnu Khuzaimah, Imam 
an-Nawawi, dan Syaikh al-Albani.
\item \textbf{Shahih}: HR. Al-Bukhari (no. 6389) dan Muslim (no. 2690).
\end{enumerate}
\textbf{Referensi}: Yazid bin Abdul Qadir Jawas. 2016. Kumpulan Do'a dari
Al-Quran dan As-Sunnah yang Shahih. Bogor: Pustaka Imam Asy-Syafi'i.
\index{dapat}	
\index{kebaikan}
\index{dunia}	
\index{akhirat}
\footnote{Hanifah Atiya Budianto 1417051063 - Jurusan Ilmu Komputer,
Universitas Lampung}
\end{document}