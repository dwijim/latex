\documentclass[a4paper,12pt]{article}
\usepackage{arabtex} 
\usepackage[bahasa] {babel}
\usepackage{calligra}
\usepackage[top=2cm,left=3cm,right=3cm,bottom=3cm]{geometry}
\title{\Large Doa Memohon Surga dan Berlindung dari Neraka}
\author{\calligra Hanifah Atiya Budianto}
\begin{document}
\sffamily
\maketitle 
\fullvocalize
\setcode{arabtex}
\begin{arabtext}
\noindent
al-ll_ahumma 'inni-y 'as'aluka al-^gannaTa, wa'a`u-w_dubika mina al-nnAri.
\\
\end{arabtext}
\noindent
\textbf{Artinya}:
\par
\indent
"Ya Allah, sesungguhnya aku memohon Surga kepada-Mu dan aku memohon 
perlindungan kepada-Mu dari Neraka." [Dibaca 3x] {\scriptsize 1}\\
\begin{arabtext}
\noindent
al-ll_ahumma 'inni-y 'as'aluka bi-'anna laka al-.hamda, lA 'il_aha 'illA 
'anta wa.hdaka lA ^sari-yka laka al-mannAnu, yA badi-y`a al-ssamAwAti 
wAl-'ar.di, yA_dA al-^galAli wAl-'ikrAmi, yA .hayyu yA qayyu-wmu, 'inni-y 
'as'aluka (al-^gannaTa, wa'a`u-w_du bika mina al-nnAri).\\
\end{arabtext}
\noindent
\textbf{Artinya}:
\par
\indent
"Ya Allah, sesungguhnya aku memohon kepada-Mu, karena segala puji hanyalah 
bagi-Mu, tidak ada ilah yang berhak untuk diibadahi dengan benar kecuali 
Engkau, tidak ada sekutu bagi-Mu, Yang Maha Pemberi, Pencipta langit dan 
bumi, wahai Rabb Pemilik keagungan serta kemuliaan, wahai Rabb Yang 
Mahahidup lagi Maha Berdiri sendiri, sesungguhnya aku memohon kepada-Mu 
[Surga dan aku berlindung kepada-Mu dari Neraka]."{\scriptsize 2}\\
\begin{arabtext}
\noindent
al-ll_ahumma rabba ^gibrA'iyla, wami-ykA'iyla, warabba 'isrAfi-yla, 
'a`u-w_du bika min .harri al-nnAri wamin `a_dAbi al-qabri.\\
\end{arabtext}
\noindent
\textbf{Artinya}:
\par
\indent
"Ya Allah, Rabb Malaikat Jibril, Mika-il dan Rabb Malaikat Israfril, aku 
berlindung kepada-Mu dari panasnya api Neraka dan adzab kubur."
{\scriptsize 3}\\\\
\par
\noindent
\textbf{Tingkatan Doa dan Sanad}:
\begin{enumerate}
\item \textbf{Shahih}: HR. At-Tirmidzi (no. 2572), an-Nasai (VIII/279), 
Ibnu Majah (no. 4340), Ahmad (III/117, 141, 155), al-Hakim (I/534-535).
\item \textbf{Shahih}: HR. Abu Dawud (no. 1495), an-Nasai (III/52), Ibnu 
Majah (no. 3858), Ahmad (III/158, 245) dan Ibnu Mandah dalam 
\textit{Kitabut Tauhid} (no. 355) dan tambahan dalam kurung miliknya dari 
Anas bin Malik r.a. Dan juga at-Tirmidzi (no. 3475) dari Abdullah bin 
Buraidah al-Aslami r.a. dari ayahnya.
\item \textbf{Hasan}: HR. An-Nasai (VIII/278) dari Aisyah r.a. Lihat 
\textit{Silsilah Ah\^{a}d\^{i}ts ash-Shah\^{i}hah} (no. 1544).
\end{enumerate}
\textbf{Referensi}: Yazid bin Abdul Qadir Jawas. 2016. Kumpulan Do'a dari
Al-Quran dan As-Sunnah yang Shahih. Bogor: Pustaka Imam Asy-Syafi'i.
\index{mohon}
\index{surga}
\index{berlindung}
\index{neraka}
\footnote{Hanifah Atiya Budianto 1417051063 - Jurusan Ilmu Komputer,
Universitas Lampung}
\end{document}