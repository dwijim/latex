\documentclass[a4paper,12pt]{article}
\usepackage{arabtex} 
\usepackage[bahasa] {babel}
\usepackage{calligra}
\usepackage[top=2cm,left=3cm,right=3cm,bottom=3cm]{geometry}
\title{\Large Doa Mendapatkan Kenikmatan}
\author{\calligra Hanifah Atiya Budianto}
\begin{document}
\sffamily
\maketitle 
\fullvocalize
\setcode{arabtex}
\begin{arabtext}
\noindent
al-ll_ahumma matti`ni-y bisam`i-y waba.sari-y, wA^g`alhumA al-wAri_ta 
minni-y, wAn.surni-y `alY man ya.zlimuni-y, wa_hu_d minhu bi_ta'ri-y.\\
\end{arabtext}
\noindent
\textbf{Artinya}:
\par
\indent
"Ya Allah, berikanlah manfaat kepadaku melalui pendengaran dan pandanganku,
jadikanlah keduanya sebagai pewarisku (yakni, jadikanlah keduanya sehat 
sampai mati), (ya Allah) tolonglah aku atas orang yang berbuat zhalim 
terhadapku, dan hukumlah dia sebagai balasanku atas dirinya."\\\\
\par
\noindent
\textbf{Tingkatan Doa dan Sanad}: \textbf{Hasan}: HR. At-Tirmidzi (no. 
3604) dan \textit{Shah\^{i}h at-Tirmidzi} (III/188, no. 2854). Juga 
al-Hakim (I/523), serta dishahihkan olehnya lalu disepakati adz-Dzahabi. 
Sanadnya hasan.\\
\textbf{Referensi}: Yazid bin Abdul Qadir Jawas. 2016. Kumpulan Do'a dari
Al-Quran dan As-Sunnah yang Shahih. Bogor: Pustaka Imam Asy-Syafi'i.
\index{dapat}	
\index{nikmat}
\footnote{Hanifah Atiya Budianto 1417051063 - Jurusan Ilmu Komputer,
Universitas Lampung}
\end{document}