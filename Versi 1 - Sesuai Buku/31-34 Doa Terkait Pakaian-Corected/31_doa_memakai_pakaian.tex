\documentclass[a4paper,12pt]{article}
\usepackage{arabtex} 
\usepackage[bahasa] {babel}
\usepackage{calligra}
\usepackage[top=2cm,left=3cm,right=3cm,bottom=3cm]{geometry}
\title{\Large Doa Memakai Pakaian}
\author{\calligra Hanifah Atiya Budianto}
\begin{document}
\sffamily
\maketitle 
\fullvocalize
\setcode{arabtex}
\begin{arabtext}
\noindent
al-.hamdu lill_ahi alla_di-y kasAni-y h_a_dA al-_tawba warazaqani-yhi min 
.ga-yri .hawliN minni-y walA quwwaTiN.\\
\end{arabtext}
\noindent
\textbf{Artinya}:
\par
\indent
"Segala puji bagi Allah yang memberi aku pakaian ini sebagai rizki 
dari-Nya, tanpa daya dan kekuatan dariku."\\\\
\par
\noindent
\textbf{Tingkatan Doa dan Sanad}: \textbf{Hasan}: HR. Abu Dawud pada Kitab
"al-Lib\^{a}s" (no. 4023), \textit{Shah\^{i}h Abi Dawud} (II/760, no. 3394)
dan selainnya. \\
\textbf{Referensi}: Yazid bin Abdul Qadir Jawas. 2016. Kumpulan Do'a dari
Al-Quran dan As-Sunnah yang Shahih. Bogor: Pustaka Imam Asy-Syafi'i.
\index{pakaian}
\footnote{Hanifah Atiya Budianto 1417051063 - Jurusan Ilmu Komputer,
Universitas Lampung}
\end{document}