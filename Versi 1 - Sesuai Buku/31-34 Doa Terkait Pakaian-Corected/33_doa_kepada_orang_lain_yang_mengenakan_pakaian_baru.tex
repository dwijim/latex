\documentclass[a4paper,12pt]{article}
\usepackage{arabtex} 
\usepackage[bahasa] {babel}
\usepackage{calligra}
\usepackage[top=2cm,left=3cm,right=3cm,bottom=3cm]{geometry}
\title{\Large Doa Kepada Orang Lain yang Mengenakan Pakaian Baru}
\author{\calligra Hanifah Atiya Budianto}
\begin{document}
\sffamily
\maketitle 
\fullvocalize
\setcode{arabtex}
\begin{arabtext}
\noindent
_tubli-y wayu_hlifu al-ll_ahu ta-`AlY.\\
\end{arabtext}
\noindent
\textbf{Artinya}:
\par
\indent
"Semoga engkau dapat memakainya sampai lusuh, dan semoga Allah
\textit{Ta'ala} menggantinya untukmu dengan yang lebih baik." 
{\scriptsize 1}\\
\begin{arabtext}
\noindent
'ilbas jadi-ydaN, wa`i^s .hami-ydaN, wamut ^sahi-ydaN.\\
\end{arabtext}
\noindent
\textbf{Artinya}:
\par
\indent
"Berpakaianlah yang baru, hiduplah dengan terpuji dan matilah dalam keadaan
syahid." {\scriptsize 2}\\\\
\par
\noindent
\textbf{Tingkatan Doa dan Sanad}:
\begin{enumerate}
\item \textbf{Shahih}: HR. Abu Dawud (no. 4020) dan lihat pula kitab
\textit{Shah\^{i}h Abi Dawud} (II/760 no. 3393).
\item \textbf{Shahih}: HR. Ibnu Majah (II/1178, no. 3558), Ahmad (II/89), 
al-Baghawi (XII/41, no. 3112), dan \textit{Shah\^{i}h Ibnu Majah} (II/275, 
no. 2863).
\end{enumerate}
\textbf{Referensi}: Yazid bin Abdul Qadir Jawas. 2016. Kumpulan Do'a dari
Al-Quran dan As-Sunnah yang Shahih. Bogor: Pustaka Imam Asy-Syafi'i.
\index{orang}	
\index{pakaian}
\index{baru}
\footnote{Hanifah Atiya Budianto 1417051063 - Jurusan Ilmu Komputer,
Universitas Lampung}
\end{document}