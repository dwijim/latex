\documentclass[a4paper,12pt]{article}
\usepackage{arabtex} 
\usepackage[bahasa] {babel}
\usepackage{calligra}
\usepackage[top=2cm,left=3cm,right=3cm,bottom=3cm]{geometry}
\title{\Large Doa Mengenakan Pakaian Baru}
\author{\calligra Hanifah Atiya Budianto}
\begin{document}
\sffamily
\maketitle 
\fullvocalize
\setcode{arabtex}
\begin{arabtext}
\noindent
al-ll_ahumma laka al-.hamdu 'anta kasawtani-yhi, 'as'aluka min _ha-yrihi
wa_ha-yri mA .suni`a lahu, wa'a`u-w_du bika min ^sarrihi wa^sarri mA 
.suni`a lahu.\\
\end{arabtext}
\noindent
\textbf{Artinya}:\\
\indent
"Ya Allah, hanya milik-Mu segala puji, Engkaulah yang memberi pakaian ini 
kepadaku. Aku mohon kepada-Mu untuk memperoleh kebaikannya dan kebaikan 
dari tujuan pembuatan pakaian ini. Aku pun berlindung kepada-Mu dari 
keburukannya serta keburukan tujuan dibuatnya pakaian ini."\\\\
\par
\noindent
\textbf{Tingkatan Doa dan Sanad}: \textbf{Shahih}: HR. Abu Dawud (no. 
4020), at-Tirmidzi (no. 1767), al-Hakim (IV/192), dan al-Baghawi (no. 3111)
dari Abu Sa'id al-Khudri r.a. Lihat \textit{Mukhtasar Syam\^{a}-ilit
Tirmidzi} karya Syaikh al-Albani (hlm. 47-48). \\
\textbf{Referensi}: Yazid bin Abdul Qadir Jawas. 2016. Kumpulan Do'a dari
Al-Quran dan As-Sunnah yang Shahih. Bogor: Pustaka Imam Asy-Syafi'i.
\index{pakaian}	
\index{baju}	
\index{baru}
\footnote{Hanifah Atiya Budianto 1417051063 - Jurusan Ilmu Komputer,
Universitas Lampung}
\end{document}