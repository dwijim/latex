\documentclass[a4paper,12pt]{article}
\usepackage{arabtex} 
\usepackage[bahasa] {babel}
\usepackage{calligra}
\usepackage[top=2cm,left=3cm,right=3cm,bottom=3cm]{geometry}
\title{\Large Mengerjakan Thawaf 7 kali Putaran dan Berdoa dengan Doa-doa 
yang Mudah}
\author{\calligra Hanifah Atiya Budianto}
\begin{document}
\sffamily
\maketitle 
\fullvocalize
\par
\indent
Tidak dicontohkan oleh Nabi Shallallahu ‘alaihi wa sallam dan para Sahabat 
beliau mengucapkan bacaan tertentu pada putaran pertama, kedua, ketiga, dan 
seterusnya sampai putaran terakhir.\\
\indent
Yang Nabi Shallallahu ‘alaihi wa sallam contohkan adalah doa antara dua rukun 
pada setiap putaran. Selain itu, kita dianjurkan agar banyak berdzikir, 
membaca al-Quran dan doa, karena thawaf seperti shalat hanya saja dibolehkan 
bicara di dalamnya.
\\
\indent
Dan, bacalah doa serta dzikir yang mudah ketika thawaf.\\\\
\par
\noindent
\textbf{Referensi}: Yazid bin Abdul Qadir Jawas. 2016. Kumpulan Do'a dari
Al-Quran dan As-Sunnah yang Shahih. Bogor: Pustaka Imam Asy-Syafi'i.
\index{thawaf}
\footnote{Hanifah Atiya Budianto 1417051063 - Jurusan Ilmu Komputer,
Universitas Lampung}
\end{document}