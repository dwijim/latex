\documentclass[a4paper,12pt]{article}
\usepackage{arabtex} 
\usepackage[bahasa] {babel}
\usepackage{calligra}
\usepackage[top=2cm,left=3cm,right=3cm,bottom=3cm]{geometry}
\title{\Large Doa Antara Rukun Yamani dan Hajar Aswad}
\author{\calligra Hanifah Atiya Budianto}
\begin{document}
\sffamily
\maketitle 
\fullvocalize
\setcode{arabtex}
\begin{arabtext}
\noindent
rabbana-^A -'a-atinA fiY al-ddunyA .hasanaTaN wafiY al-'a_hiraTi 
.hasanaTaN waqinA `a_dAba al-nnAri).\\
\end{arabtext}
\noindent
\textbf{Artinya}:
\par
\indent
"Wahai Rabb kami, berikanlah kami kebaikan di dunia dan juga kebaikan di 
akhirat, serta lindungilah kami dari siksa api Neraka."\\
\par
\indent
Setiap selesai tawaf tujuh putaran, disunnahkan shalat sunnah dua rakaat 
di belakang Maqam Ibrahim.\\
\indent
Adapun surah yang dibaca setelah Al-F\^{a}thihah pada rakaat pertama 
adalah Al-K\^{a}fir\^{u}n, sedang pada rakaat kedua adalah Al-Ikhl\^{a}sh 
berdasarkan hadits Jabir r.a.\\\\
\par
\noindent
\textbf{Tingkatan Doa dan Sanad}: \textbf{Hasan}: HR. Abu Dawud (no. 1892),
Ahmad (III/411), dan al-Baghawi dalam \textit{Syarhus Sunnah} (VII/128 no. 
1915) dari Abdullah bin as-Sa-ib r.a. Lihat \textit{Shah\^{i}h Abi Dawud} 
(I/354).\\
\textbf{Referensi}: Yazid bin Abdul Qadir Jawas. 2016. Kumpulan Do'a dari
Al-Quran dan As-Sunnah yang Shahih. Bogor: Pustaka Imam Asy-Syafi'i.
\index{rukun}	
\index{yamani}
\index{hajar}	
\index{aswad}
\footnote{Hanifah Atiya Budianto 1417051063 - Jurusan Ilmu Komputer,
Universitas Lampung}
\end{document}