\documentclass[a4paper,12pt]{article}
\usepackage{arabtex} 
\usepackage[bahasa] {babel}
\usepackage{calligra}
\usepackage[top=2cm,left=3cm,right=3cm,bottom=3cm]{geometry}
\title{\Large Bacaan ketika Berada di Atas Bukit Shafa dan Marwah}
\author{\calligra Hanifah Atiya Budianto}
\begin{document}
\sffamily
\maketitle 
\fullvocalize
\setcode{arabtex}
\par
\indent
Dari Jabir r.a., ia berkata: "Ketika Nabi berada dekat dengan bukit Shafa, 
beliau membaca:\\
\begin{arabtext}
\noindent
( 'inna al-.s.safA wa-ulmarwaTa min ^sa`a-^A'iri al-llahi ) ( 'abda'u bimA 
bada'a al-ll_ahu bihi ).\\
\end{arabtext}
\noindent
\textbf{Artinya}:
\par
\indent
\textit{'Sesungguhnya Shafa dan Marwah adalah termasuk syi'ar agama Allah.} 
Aku memulai sa'i dengan apa yang didahulukan Allah.' [\textbf{Dibaca 1x 
ketika naik bukit Shafa}]\\
\par
\indent
Kemudian, beliau mulai sa'i dengan naik bukit Shafa, hingga melihat Ka'bah.
Lalu menghadap kiblat, membaca kalimat tauhid, dan bertakbir 
(\textit{All\^{a}hu Akbar}) tiga kali. {\scriptsize 1} Kemudian beliau 
Shallallahu ‘alaihi wa sallam membaca:\\
\begin{arabtext}
\noindent
lA 'il_aha 'illA al-ll_ahu wa.hdahu lA ^sari-yka lahu, lahu al-mulku, 
walahu al-.hamdu, wahuwa `alaY kulli ^saY'iN qadi-yruN. lA 'il_aha 'illA 
al-ll_ahu wa.hdahu lA ^sari-yka lahu, 'an^gaza wa`dahu, wana.sara `abdahu, 
wahazama al-'a.hzAba wa.hdahu.\\
\end{arabtext}
\noindent
\textbf{Artinya}:
\par
\indent
'Tidak ada ilah yang berhak diibadahi dengan benar selain Allah, Yang Maha 
Esa, tidak ada sekutu bagi-Nya. Bagi-Nya kerajaan dan pujian. Dialah Yang 
Maha kuasa atas segala sesuatu. Tidak ada ilah yang berhak diibadahi dengan
benar selain Allah Yang Maha Esa, Rabb Yang melaksanakan janji-Nya, Yang 
membela hamba-Nya (Muhammad), dan Yang mengalahkan golongan musuh 
sendirian.'\\
\indent
Setelah itu, Rasulullah Shallallahu ‘alaihi wa sallam berdoa lantas 
mengulangi dzikir-dzikir di atas. Beliau membacanya tiga kali.\\
\indent
Dalam hadits Jabir dikatakan bahwa Nabi juga membaca demikian tatkala 
berada di Marwah."{\scriptsize 2}\\
\indent
Sewaktu sa'i, dianjurkan banyak berdzikir, bertasbih, baca al-qur-an, dan 
memanjatkan doa.\\\\
\par
\noindent
\textbf{Tingkatan Doa dan Sanad}:
\begin{enumerate}
\item HR. Al-Baihaqi (III/315).
\item \textbf{Shahih}: HR. Muslim (no. 1218 [147]) dari Jabir bin Abdillah 
r.a., Bab "Hajjatun Nabi".
\end{enumerate}
\textbf{Referensi}: Yazid bin Abdul Qadir Jawas. 2016. Kumpulan Do'a dari
Al-Quran dan As-Sunnah yang Shahih. Bogor: Pustaka Imam Asy-Syafi'i.
\index{bukit}	
\index{shafa}
\index{marwah}	
\footnote{Hanifah Atiya Budianto 1417051063 - Jurusan Ilmu Komputer,
Universitas Lampung}
\end{document}