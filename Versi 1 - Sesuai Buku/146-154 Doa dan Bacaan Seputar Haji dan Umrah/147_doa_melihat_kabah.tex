\documentclass[a4paper,12pt]{article}
\usepackage{arabtex} 
\usepackage[bahasa] {babel}
\usepackage{calligra}
\usepackage[top=2cm,left=3cm,right=3cm,bottom=3cm]{geometry}
\title{\Large Doa Melihat Ka'bah}
\author{\calligra Hanifah Atiya Budianto}
\begin{document}
\sffamily
\maketitle 
\fullvocalize
\setcode{arabtex}
\begin{arabtext}
\noindent
al-ll_ahumma 'anta al-ssalAmu, waminka al-ssalAmu, fa.hayyinA rabbanA 
bi-al-ssalAmi.\\
\end{arabtext}
\noindent
\textbf{Artinya}:
\par
\indent
"Ya Allah, Engkaulah Mahasejahtera, dari Engkau pula kesejahteraan, maka 
kekalkanlah kami, wahai Rabb kami, dalam kesejahteaan".\\\\
\par
\noindent
\textbf{Sanad}: HR. Al-Baihaqi (V/73). Lihat \textit{Man\^{a}sikul Hajji 
wal 'Umrah} (hlm. 20) karya Syaikh Muhammad Nashiruddin al-Albani.\\
\textbf{Referensi}: Yazid bin Abdul Qadir Jawas. 2016. Kumpulan Do'a dari
Al-Quran dan As-Sunnah yang Shahih. Bogor: Pustaka Imam Asy-Syafi'i.
\index{kakbah}	
\footnote{Hanifah Atiya Budianto 1417051063 - Jurusan Ilmu Komputer,
Universitas Lampung}
\end{document}