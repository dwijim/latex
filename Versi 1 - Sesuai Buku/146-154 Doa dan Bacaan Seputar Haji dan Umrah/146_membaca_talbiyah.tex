\documentclass[a4paper,12pt]{article}
\usepackage{arabtex} 
\usepackage[bahasa] {babel}
\usepackage{calligra}
\usepackage[top=2cm,left=3cm,right=3cm,bottom=3cm]{geometry}
\title{\Large Membaca Talbiyah}
\author{\calligra Hanifah Atiya Budianto}
\begin{document}
\sffamily
\maketitle 
\fullvocalize
\setcode{arabtex}
\begin{arabtext}
\noindent
labbayka Aal-ll_ahumma labbayka, labbayka lA ^sari-yka laka labbayka, 'inna 
al-.hamda wAl-nni`maTalaka wAl-mulka lA ^sari-yka laka.\\
\end{arabtext}
\noindent
\textbf{Artinya}:
\par
\indent
"Aku penuhi panggilan-Mu, ya Allah, aku penuhi panggilan-Mu. Aku penuhi 
pangilan-Mu, tiada sekutu bagi-Mu, aku penuhi panggilan-Mu. Sesungguhnya 
pujian dan nikmat adalah milik-Mu, begitu juga kerajaan, tidak ada sekutu 
bagi-Mu".\\\\
\par
\noindent
\textbf{Tingkatan Doa dan Sanad}: \textbf{Shahih}: HR. Al-Bukhari (no. 
1549), \textit{Fathul B\^{a}ri} (III/408), Muslim (no. 1184 [19]).\\
\textbf{Referensi}: Yazid bin Abdul Qadir Jawas. 2016. Kumpulan Do'a dari
Al-Quran dan As-Sunnah yang Shahih. Bogor: Pustaka Imam Asy-Syafi'i.
\index{talbiyah}	
\footnote{Hanifah Atiya Budianto 1417051063 - Jurusan Ilmu Komputer,
Universitas Lampung}
\end{document}