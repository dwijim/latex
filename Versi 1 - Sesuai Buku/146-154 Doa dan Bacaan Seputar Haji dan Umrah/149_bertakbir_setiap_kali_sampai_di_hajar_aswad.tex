\documentclass[a4paper,12pt]{article}
\usepackage{arabtex} 
\usepackage[bahasa] {babel}
\usepackage{calligra}
\usepackage[top=2cm,left=3cm,right=3cm,bottom=3cm]{geometry}
\title{\Large Bertakbir Setiap Kali Sampai di Hajar Aswad}
\author{\calligra Hanifah Atiya Budianto}
\begin{document}
\sffamily
\maketitle 
\fullvocalize
\setcode{arabtex}
\begin{arabtext}
\noindent
.tAfa al-nnabiyyu .sallaY al-ll_ahu `ala-yhi wasallama bi-alba-yti `alY 
ba`i-yriN kullamA 'atY al-rrukna 'a^sAra 'ila-yhi bi^sa-y'iN kAna `indahu 
wakabbara.\\
\end{arabtext}
\noindent
\textbf{Artinya}:
\par
\indent
"Nabi Shallallahu ‘alaihi wa sallam thawaf di Baitullah di atas 
(menunggangi) unta. Setiap datang ke Hajar Aswad (yakni sudut Ka'bah yang 
padanya terdapat Hajar Aswad), beliau memberi isyarat dengan sesuatu yang 
dipegangnya dan bertakbir."{\scriptsize 1}\\
\par
\indent
Yang demikian dilakukan setiap melewati Hajar Aswad. Apabila mampu 
menciumnya, hendaklah dia lakukan. Apabila tidak, cukup dengan disentuh. 
Dan apabila tidak mungkin disentuh, cukuplah berisyarat sambil bertakbir: 
"Allahu akbar". Dan dibolehkan juga membaca "Bismillahi allahu akbar" 
berdasarkan perbuatan Ibnu Umar r.a.{\scriptsize 2}\\\\
\par
\noindent
\textbf{Tingkatan Doa dan Sanad}:
\begin{enumerate}
\item \textbf{Shahih}: HR. Al-Bukhari (no. 1613). Yang dimaksud "sesuatu" 
adalah tongkat. Lihat \textit{Shah\^{i}h al-Bukhari} (no. 1607)
\item HR. Al-Baihaqi (V/79).
\end{enumerate}
\textbf{Referensi}: Yazid bin Abdul Qadir Jawas. 2016. Kumpulan Do'a dari
Al-Quran dan As-Sunnah yang Shahih. Bogor: Pustaka Imam Asy-Syafi'i.
\index{hajar}	
\index{aswad}
\footnote{Hanifah Atiya Budianto 1417051063 - Jurusan Ilmu Komputer,
Universitas Lampung}
\end{document}