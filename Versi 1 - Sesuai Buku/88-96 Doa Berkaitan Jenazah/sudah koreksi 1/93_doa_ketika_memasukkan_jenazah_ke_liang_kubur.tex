\documentclass[a4paper,12pt]{article}
\usepackage{arabtex} 
\usepackage[bahasa] {babel}
\usepackage{calligra}
\usepackage[top=2cm,left=3cm,right=3cm,bottom=3cm]{geometry}
\title{\Large Doa Ketika Memasukkan Jenazah Ke Liang Kubur}
\author{\calligra Hanifah Atiya Budianto}
\begin{document}
\sffamily
\maketitle 
\fullvocalize
\setcode{arabtex}
\begin{arabtext}
\noindent
bismi al-ll_ahi wa`alY sunnaTi rasu-wli al-ll_ahi.\\
\end{arabtext}
\noindent
\textbf{Artinya}:
\par
\indent
"\textit{(Bismillahi wa 'ala Sunnati Rasulillah)} Dengan nama Allah dan 
atas Sunnah Rasulullah."\\\\
\par
\noindent
\textbf{Tingkatan Doa dan Sanad}: \textbf{Shahih}: HR. Abu Dawud (no. 
3213), dan lainnya dengan sanad yang shahih. Dan Ahmad (II/27, 40-41).\\
\textbf{Referensi}: Yazid bin Abdul Qadir Jawas. 2016. Kumpulan Do'a dari
Al-Quran dan As-Sunnah yang Shahih. Bogor: Pustaka Imam Asy-Syafi'i.
\index{memasukkan}	
\index{jenazah}
\index{liang}	
\index{kubur}
\footnote{Hanifah Atiya Budianto 1417051063 - Jurusan Ilmu Komputer,
Universitas Lampung}
\end{document}