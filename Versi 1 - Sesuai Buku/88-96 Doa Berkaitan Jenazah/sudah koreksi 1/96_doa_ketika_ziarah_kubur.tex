\documentclass[a4paper,12pt]{article}
\usepackage{arabtex} 
\usepackage[bahasa] {babel}
\usepackage{calligra}
\usepackage[top=2cm,left=3cm,right=3cm,bottom=3cm]{geometry}
\title{\Large Doa Ketika Ziarah Kubur}
\author{\calligra Hanifah Atiya Budianto}
\begin{document}
\sffamily
\maketitle 
\fullvocalize
\setcode{arabtex}
\begin{arabtext}
\noindent
al-ssalAmu `alaykum 'ahla al-ddiyAri mina al-mu'mini-yna wAl-muslimi-yna, 
wa-'innA 'in ^sA'a al-ll_ahu bikum lA .hiqu-wna, nas'alu al-ll_aha lanA 
walakumu al-`AfiyaTa.\\
\end{arabtext}
\noindent
\textbf{Artinya}:
\par
\indent
"Semoga kesejahteraan terlimpah atas kalian, wahai para penghuni kubur dari
kaum Mukminin dan kaum Muslimin. Dan, \textit{insya Allah} kami menyusul 
kalian. Kami memohon kepada Allah untuk kami dan kamu sekalian, supaya 
diberi keselamatan dari segala apa yang tidak diinginkan."\\\\
\par
\noindent
\textbf{Tingkatan Doa dan Sanad}: \textbf{Shahih}: HR. Muslim (no.  975) 
dan Ibnu Majah (no. 1547) dari Buraidah r.a. Lafazh ini menurut Ibnu Majah.
Diriwayatkan juga oleh Muslim (no. 974 [102, 103]) dari Aisyah r.a. dengan 
ada tambahan.\\
\textbf{Referensi}: Yazid bin Abdul Qadir Jawas. 2016. Kumpulan Do'a dari
Al-Quran dan As-Sunnah yang Shahih. Bogor: Pustaka Imam Asy-Syafi'i.
\index{ziarah}	
\index{kubur}
\footnote{Hanifah Atiya Budianto 1417051063 - Jurusan Ilmu Komputer,
Universitas Lampung}
\end{document}