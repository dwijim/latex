\documentclass[a4paper,12pt]{article}
\usepackage{arabtex} 
\usepackage[bahasa] {babel}
\usepackage{calligra}
\usepackage[top=2cm,left=3cm,right=3cm,bottom=3cm]{geometry}
\title{\Large Doa untuk Jenazah Anak Kecil}
\author{\calligra Hanifah Atiya Budianto}
\begin{document}
\sffamily
\maketitle 
\fullvocalize
\setcode{arabtex}
\begin{arabtext}
\noindent
al-ll_ahumma 'a`i_dhu min `a_dAbi al-qabri.\\
\end{arabtext}
\noindent
\textbf{Artinya}:
\par
\indent
"Ya Allah, lindungilah dia dari siksa kubur."{\scriptsize 1}\\
\begin{arabtext}
\noindent
al-ll_ahumma a^g`alhu lanA fara.taN wasalafaN wa'a^graN.\\
\end{arabtext}
\noindent
\textbf{Artinya}:
\par
\indent
"Ya Allah, jadikanlah kematian anak ini sebagai simpanan pahala dan amal 
baik serta pahala untuk kami."{\scriptsize 2}\\\\
\par
\noindent
\textbf{Tingkatan Doa dan Sanad}:
\begin{enumerate}
\item \textbf{Atsar Shahih}: Diriwayatkan oleh Malik/\textit{al-Muwaththa'}
(I/198 no. 18), Ibnu Abi Syaibah dalam al-Mushannaf (II/217), al-Baihaqi 
(IV/9) dan al-Baghawi dalam \textit{Syarhus Sunnah} (V/357) dari perkataan 
Abu Hurairah r.a. Dishahihkan oleh Imam al-Albani. Lihat takhrij 
\textit{Hidayatur Ruw\^{a}h} (II/213 no. 1631).
\item \textbf{Atsar Shahih}: Diriwayatkan oleh Al-Baghawi dalam 
\textit{Syarhus Sunnah} (V/357), Abdurrazzaq (no. 6588) dari perkataan 
al-Hasan al-Bashri dan al-Bukhari meriwayatkan hadits tersebut secara 
\textit{mu'allaq} dalam kitab \textit{al-Jan\^{a}-iz} baba 65: "Membaca 
\textit{F\^{a}tihatul Kit\^{a}b} atas jenazah."
\end{enumerate}
\textbf{Referensi}: Yazid bin Abdul Qadir Jawas. 2016. Kumpulan Do'a dari
Al-Quran dan As-Sunnah yang Shahih. Bogor: Pustaka Imam Asy-Syafi'i.
\index{jenazah}	
\index{anak}
\index{kecil}
\footnote{Hanifah Atiya Budianto 1417051063 - Jurusan Ilmu Komputer,
Universitas Lampung}
\end{document}