\documentclass[a4paper,12pt]{article}
\usepackage{arabtex} 
\usepackage[bahasa] {babel}
\usepackage{calligra}
\usepackage[top=2cm,left=3cm,right=3cm,bottom=3cm]{geometry}
\title{\Large Doa Memejamkan Mata Jenazah}
\author{\calligra Hanifah Atiya Budianto}
\begin{document}
\sffamily
\maketitle 
\fullvocalize
\setcode{arabtex}
\begin{arabtext}
\noindent
al-ll_ahumma a.gfir lifulAniN (bi-asmihi) wArfa` dara^gatahu fi-y 
al-mahdiyyi-yna, wA_hlufhu fi-y `aqibihi fi-y al-.gAbiri-yna, wA.gfirlanA 
walahu yA rabba al-`Alami-yna, wAfsa.h lahu fi-y qabrihi wanawwirlahu 
fi-yhi.\\
\end{arabtext}
\noindent
\textbf{Artinya}:
\par
\indent
"Ya Allah, ampunilah Fulan (hendaklah ia menyebut namanya), angkatlah 
derajatnya bersama orang-orang yang mendapat petunjuk, berikanlah 
penggantinya bagi orang-orang yang ditinggalkan sesudahnya. Dan ampunilah 
kami dan dia, wahai Rabb sekalian alam. Luaskanlah kuburnya, dan 
berikanlah cahaya di dalamnya."\\\\
\par
\noindent
\textbf{Tingkatan Doa dan Sanad}: \textbf{Shahih}: HR. Muslim (no. 920) 
dari Sahabah Ummu Salamah r.a.\\
\textbf{Referensi}: Yazid bin Abdul Qadir Jawas. 2016. Kumpulan Do'a dari
Al-Quran dan As-Sunnah yang Shahih. Bogor: Pustaka Imam Asy-Syafi'i.
\index{memejamkan}	
\index{mata}
\index{jenazah}
\footnote{Hanifah Atiya Budianto 1417051063 - Jurusan Ilmu Komputer,
Universitas Lampung}
\end{document}