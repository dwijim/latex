\documentclass[a4paper,12pt]{article}
\usepackage{arabtex} 
\usepackage[bahasa] {babel}
\usepackage{calligra}
\usepackage[top=2cm,left=3cm,right=3cm,bottom=3cm]{geometry}
\title{\Large Doa untuk Jenazah Setelah Dimakamkan}
\author{\calligra Hanifah Atiya Budianto}
\begin{document}
\sffamily
\maketitle 
\fullvocalize
\setcode{arabtex}
\begin{arabtext}
\noindent
al-ll_ahumma a.gfirlahu, al-ll_ahumma _tabbithu.\\
\end{arabtext}
\noindent
\textbf{Artinya}:
\par
\indent
"Ya Allah, ampunilah dia. Ya Allah, teguhkanlah dia."\\\\
\par
\noindent
\textbf{Tingkatan Doa dan Sanad}: Seusai memakamkan mayat, Rasulullah 
berdiri tepat di atasnya lalu bersabda: "Mintalah ampun kepada Allah untuk 
saudaramu, dan mohonkan agar dia teguh (ketika ditanya oleh dua Malaikan), 
dan sesungguhnya dia sekarang sedang ditanya." \textbf{Shahih}: HR. Abu 
Dawud (no. 3221) dan al-Hakim (I/370), al-Hakim menshahihkannya dan 
disepakati oleh Imam adz-Dzahabi.\\
\textbf{Referensi}: Yazid bin Abdul Qadir Jawas. 2016. Kumpulan Do'a dari
Al-Quran dan As-Sunnah yang Shahih. Bogor: Pustaka Imam Asy-Syafi'i.
\index{jenazah}	
\index{setelah}
\index{dimakamkan}
\footnote{Hanifah Atiya Budianto 1417051063 - Jurusan Ilmu Komputer,
Universitas Lampung}
\end{document}