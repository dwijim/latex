\documentclass[a4paper,12pt]{article}
\usepackage{arabtex} 
\usepackage[bahasa] {babel}
\usepackage{calligra}
\usepackage[top=2cm,left=3cm,right=3cm,bottom=3cm]{geometry}
\title{\Large Mengajari Orang yang Akan Meninggal Dunia dengan Kalimat 
\textit{L\^{a} il\^{a}ha illall\^{a}h}}
\author{\calligra Hanifah Atiya Budianto}
\begin{document}
\sffamily
\maketitle 
\fullvocalize
\setcode{arabtex}
\begin{arabtext}
\noindent
man kAna ^A_hiru kalAmihi lA 'il_aha 'illA al-ll_ahu da_hala al-^gannaTa.\\
\end{arabtext}
\noindent
\textbf{Artinya}:
\par
\indent "Barang siapa yang akhir perkataannya adalah: '\textit{L\^{a} 
il\^{a}ha illall\^{a}h},' akan masuk ke dalam Surga."{\scriptsize 2}\\\\
\indent Sabda Nabi shallallahu ‘alaihi wa sallam: "Talqini (ajarkanlah) 
orang yang akan meninggal di antara kalian dengan \textit{L\^{a} il\^{a}ha 
illall\^{a}h}."{\scriptsize 3}  Sabda Nabi: "Barang siapa pada akhir 
ucapannya, ketika hendak meninggal '\textit{L\^{a} il\^{a}ha 
illall\^{a}h}', maka ia masuk Surga suatu masa kelak, kendatipun akan 
mengalami musibah sebelum itu yang mungkin menimpanya."{\scriptsize 4}\\
\indent Nabi shallallahu ‘alaihi wa sallam bersabda: "Barang siapa yang 
meninggal dalam keadaan ia mengetahui bahwasanya tidak ada ilah yang berhak 
diibadahi dengan benar kecuali hanya Allah, maka ia akan masuk Surga."
{\scriptsize 5}\\
\indent Sabda Rasulullah shallallahu ‘alaihi wa sallam: "Barang siapa yang 
meninggal dalam keadaan tidak menyekutukan Allah dengan sesuatu pun juga, 
maka ia akan masuk Surga. Dan barang siapa yang meninggal dalam keadaan 
menyekutukan Allah dengan sesuatu, maka ia akan masuk Neraka."
{\scriptsize 6}\\\\
\par
\noindent
\textbf{Tingkatan Doa dan Sanad}:
\begin{enumerate}
\item \textbf{Shahih}: HR. Abu Dawud (no. 3116), Ahmad (V/233, 247) dan 
al-Hakim (I/351, 500) dari Mu'adz bin Jabal r.a. dan lihat 
\textit{Shah\^{i}hul J\^{a}mi'} (no. 6479).
\item \textbf{Shahih}: HR. Muslim (no. 916), Abu Dawud (3117), at-Tirmidzi 
(no. 976), an-Nasai (IV/5), Ibnu Majah (no. 1445).
\item \textbf{Shahih}: HR. Ibnu Hibban (no. 719-\textit{al-Maw\^{a}rid} dan
no. 2993-\textit{at-Ta'liqatul His\^{a}n}), \textit{Shah\^{i}h 
Maw\^{a}ridizh Zham-\^{a}n} (no. 595). Lihat \textit{Ahk\^{a}mul 
Jan\^{a}-iz} (hlm. 19) dan \textit{Irw\^{a}-ul Ghal\^{i}l} (III/150).
\item \textbf{Shahih}: HR. Muslim (no. 26 [143]).
\item \textbf{Shahih}: HR. Muslim (no. 93 [151]).
\end{enumerate}
\textbf{Referensi}: Yazid bin Abdul Qadir Jawas. 2016. Kumpulan Do'a dari
Al-Quran dan As-Sunnah yang Shahih. Bogor: Pustaka Imam Asy-Syafi'i.
\index{mengajari}	
\index{orang}
\index{meninggal}
\index{dunia}
\footnote{Hanifah Atiya Budianto 1417051063 - Jurusan Ilmu Komputer,
Universitas Lampung}
\end{document}