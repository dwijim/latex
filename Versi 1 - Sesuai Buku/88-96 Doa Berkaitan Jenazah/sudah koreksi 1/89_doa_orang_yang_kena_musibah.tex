\documentclass[a4paper,12pt]{article}
\usepackage{arabtex} 
\usepackage[bahasa] {babel}
\usepackage{calligra}
\usepackage[top=2cm,left=3cm,right=3cm,bottom=3cm]{geometry}
\title{\Large Orang yang Kena Musibah}
\author{\calligra Hanifah Atiya Budianto}
\begin{document}
\sffamily
\maketitle 
\fullvocalize
\setcode{arabtex}
\begin{arabtext}
\noindent
'innA li-ll_ahi wa'i-nnA 'ila-yhi rA^gi`u-wna Aal-ll_ahumma '^gurni-y fi-y 
mu.si-ybati-y wa'a_hlif li-y _ha-yraN minhA.\\
\end{arabtext}
\noindent
\textbf{Artinya}:
\par
\indent
"Sesungguhnya kami milik Allah dan kepada-Nya kami akan kembali. Ya Allah, 
berikanlah pahala kepadaku dalam musibahku dan gantikanlah untukku dengan 
yang lebih baik darinya (dari musibahku)."\\\\
\par
\noindent
\textbf{Tingkatan Doa dan Sanad}: \textbf{Shahih}: HR. Muslim (no. 918) 
dari Sahabah Ummu Salamah r.a. Nabi SAW. bersabda: "Tidaklah seorang hamba 
mendapat satu musibah lalu ia mengucapkan (doa di atas) melainkan Allah 
akan membagikan pahala kepadanya dalam musibahnya tersebut serta memberikan
ganti baginya dengan yang lebih baik darinya."\\
\textbf{Referensi}: Yazid bin Abdul Qadir Jawas. 2016. Kumpulan Do'a dari
Al-Quran dan As-Sunnah yang Shahih. Bogor: Pustaka Imam Asy-Syafi'i.
\index{orang}	
\index{kena}
\index{musibah}
\footnote{Hanifah Atiya Budianto 1417051063 - Jurusan Ilmu Komputer,
Universitas Lampung}
\end{document}