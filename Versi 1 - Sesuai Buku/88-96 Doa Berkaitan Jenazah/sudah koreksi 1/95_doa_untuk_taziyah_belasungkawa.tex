\documentclass[a4paper,12pt]{article}
\usepackage{arabtex} 
\usepackage[bahasa] {babel}
\usepackage{calligra}
\usepackage[top=2cm,left=3cm,right=3cm,bottom=3cm]{geometry}
\title{\Large Doa untuk Ta'ziyah (Belasungkawa)}
\author{\calligra Hanifah Atiya Budianto}
\begin{document}
\sffamily
\maketitle 
\fullvocalize
\setcode{arabtex}
\begin{arabtext}
\noindent
'inna li-ll_ahi mA-'a-_ha_da, walahu mA 'a`.tY, wakullu ^sa-y'iN `indahu 
bi-'a^galiN musammaNY, falta.sbir, walta.htasib.\\
\end{arabtext}
\noindent
\textbf{Artinya}:
\par
\indent
"Sesungguhnya merupakan hak Allah mengambil dan memberikan sesuatu. Segala 
sesuatu di sisi-Nya dibatasi dengan ajal yang ditentukan. Oleh karena itu, 
bersabarlah dan carilah ganjaran dari Allah (dengan sebab musibah itu)."\\\\
\par
\noindent
\textbf{Tingkatan Doa dan Sanad}: \textbf{Shahih}: HR. Al-Bukhari (no. 
1284), Muslim (no. 923).\\
\textbf{Referensi}: Yazid bin Abdul Qadir Jawas. 2016. Kumpulan Do'a dari
Al-Quran dan As-Sunnah yang Shahih. Bogor: Pustaka Imam Asy-Syafi'i.
\index{takziyah}	
\index{belasungkawa}
\footnote{Hanifah Atiya Budianto 1417051063 - Jurusan Ilmu Komputer,
Universitas Lampung}
\end{document}