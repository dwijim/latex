\documentclass[a4paper,12pt]{article}
\usepackage{arabtex} 
\usepackage[bahasa] {babel}
\usepackage{calligra}
\usepackage[top=2cm,left=3cm,right=3cm,bottom=3cm]{geometry}
\usepackage{xcolor, framed}
\definecolor{shadecolor}{rgb}{0.8,0.8,0.8}
\title{\Large Doa pada Shalat Jenazah}
\author{\calligra Hanifah Atiya Budianto}
\begin{document}
\sffamily
\maketitle 
\fullvocalize
\setcode{arabtex}
\begin{arabtext}
\noindent
al-ll_ahumma a.gfir lahu, wAr.hamhu, wa`Afihi, wA`fu `anhu, wa'akrim 
nuzulahu, wawassi` mud_halahu, wA.gsilhu bi-almA'i wAl-_t_tal^gi 
wAl-baradi, wanaqqihi mina al-_ha.tAyA kamA naqqa-yta al-_t_tawba 
al-'abya.da mina al-ddanasi, wa'abdilhu dAraN _ha-yraN min dArihi, 
wa'ahlaN _ha-yraN min 'ahlihi, wazaw^gaN _ha-yraN min zaw^gihi, 
wa'ad_hilhu al-^gannaTa, wa-'a`i_dhu min `a_dAbi al-qabri, wamin `a_dAbi 
al-nnAri.\\
\end{arabtext}
\noindent
\textbf{Artinya}:
\par
\indent
"Ya Allah, ampunilah dia (mayit), dan berikanlah rahmat kepadanya, 
selamatkanlah dia (dari siksa kubur), maafkanlah dia dan tempatkanlah di 
tempat yang mulia (Surga), luaskanlah kuburannya, mandikanlah dia dengan 
air, salju, dan air es. Bersihkanlah dia dari segala kesalahan, sebagaimana
Engkau membersihkan baju yang putih dari kotoran. Berikanlah rumah yang 
lebih baik dari rumahnya (di dunia), berikanlah keluarga yang lebih baik 
dari keluarganya (di dunia), istri (atau suami) yang lebih baik daripada 
istri (atau suami) nya, dan masukkanlah dia ke Surga, serta lindungilah dia 
dari siksa kubur dan dari siksa api Neraka.{\scriptsize 1}
\begin{shaded*}
\noindent
Catatan:\\
- Jika jenazahnya perempuan, huruf hu/hi diganti menjadi haa.\\
- Jika jenazah 2 orang atau lebih, huruf hu diganti menjadi hum, huruf hi 
diganti menjadi him.
\end{shaded*}
\begin{arabtext}
\noindent
al-ll_ahumma a.gfir li.hayyinA wamayyitinA, wa^sAhidinA wa.gA'ibinA, 
wa.sa.gi-yrinA wakabi-yrinA, wa_dakarinA wa-'un_tAnA. al-ll_ahumma man 
'a.hya-ytahu minnA fa'a.hyihi `alY al-'islAmi, waman tawaffa-ytahu minnA 
fatawaffahu `alY al-'iymAni, al-ll_ahumma lA ta.hrimnA 'a^grahu walA 
tu.dillanA ba`dahu.\\
\end{arabtext}
\noindent
\textbf{Artinya}:
\par
\indent
"Ya Allah, ampuni orang yang masih hidup di antara kami dan yang sudah 
mati, yang hadir dan yang tidak hadir, yang masih kecil maupun dewasa, 
laki-laki maupun perempuan. Ya Allah, orang yang Engkau hidupkan di antara 
kami, hidupkanlah dengan memegang ajaran Islam, dan yang Engkau wafatkan di
antara kami, maka wafatkanlah dalam keadaan beriman. Ya Allah, jangan 
halangi kami untuk memperoleh pahalanya dan jangan sesatkan kami 
sepeninggalnya."{\scriptsize 2}\\
\begin{arabtext}
\noindent
al-ll_ahumma `abduka wAbnu 'amatika 'i.htA^ga 'ilY ra.hmatika, wa-'anta 
.ganiyyuN `an `a_dAbihi, 'in kAna mu.hsinaN fazid fi-y .hasanAtihi, wa-'in 
kAna musi-y'aN fata^gAwaz `anhu.\\
\end{arabtext}
\noindent
\textbf{Artinya}:
\par
\indent
"Ya Allah, ini (adalah) hamba-Mu, anak hamba perempuan-Mu (Hawa), 
membutuhkan rahmat-Mu, sedang Engkau tidak membutuhkan untuk menyiksanya. 
Jika ia berbuat baik, tambahkanlah dalam amalan baiknya, dan jika dia orang
yang bersalah, maafkanlah kesalahannya [kemudian beliau berdoa dengan apa 
yang Allah kehendaki]."{\scriptsize 3}\\\\
\par
\noindent
\textbf{Tingkatan Doa dan Sanad}:
\begin{enumerate}
\item \textbf{Shahih}: HR. Muslim (no. 963), an-Nasai (IV/73-74), Ahmad 
(VI/23), dan Ibnu Majah (no. 1500) dari Auf bin Malik r.a. Lihat 
\textit{Ahk\^{a}mul Jan\^{a}-iz} (hlm. 157).
\item \textbf{Shahih}: HR. Abu Dawud (no. 3201), at-Tirmidzi (no. 1024), 
Ibnu Majah (no. 1498) dan Ahmad (II/368) dan lainnya. Lihat 
\textit{Ahk\^{a}mul Jan\^{a}-iz} (hlm. 157-158).
\item \textbf{Shahih}: HR. Ath-Thabrani dalam \textit{al-Mu'jamul 
Kab\^{i}r} (XXII/249) tambahan dalam kurung miliknya, dan Al-Hakim (I/359).
Sanadnya shahih. Imam adz-Dzahabi menyetujuinya. Lihat \textit{Ahk\^{a}mul 
Jan\^{a}-iz} (hlm. 159) karya Syaikh al-Albani.
\end{enumerate}
\textbf{Referensi}: Yazid bin Abdul Qadir Jawas. 2016. Kumpulan Do'a dari
Al-Quran dan As-Sunnah yang Shahih. Bogor: Pustaka Imam Asy-Syafi'i.
\index{shalat}	
\index{jenazah}
\footnote{Hanifah Atiya Budianto 1417051063 - Jurusan Ilmu Komputer,
Universitas Lampung}
\end{document}