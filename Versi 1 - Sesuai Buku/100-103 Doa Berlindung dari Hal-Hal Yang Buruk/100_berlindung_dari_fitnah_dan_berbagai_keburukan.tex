\documentclass[a4paper,12pt]{article}
\usepackage{arabtex} 
\usepackage[bahasa] {babel}
\usepackage{calligra}
\usepackage[top=2cm,left=3cm,right=3cm,bottom=3cm]{geometry}
\title{\Large Berlindung dari Fitnah dan Berbagai Keburukan}
\author{\calligra Hanifah Atiya Budianto}
\begin{document}
\sffamily
\maketitle 
\fullvocalize
\setcode{arabtex}
\begin{arabtext}
\noindent
al-ll_ahumma 'inni-y 'a`u-w_du bika min fitnaTi al-nnAri wa`a_dAbi 
al-nnAri, wafitnaTi al-qabri wa`a_dAbi al-qabri, wa^sarri fitnaTi 
al.ginaYi, wa^sarri fitnaTi al-faqri, al-ll_ahumma 'inni-y 'a`u-w_du bika 
min ^sarri fitnaTi al-masi-y.hi al-dda^g^gAli, al-ll_ahumma a.gsil qalbi-y 
bimA'i al-_t_tal^gi wAl-baradi, wanaqqi qalbi-y mina al-_ha.tAyA, kamA 
naqqa-yta al-_t_tawba al-'abya.da mina al-ddanasi, wabA`id bayni-y 
waba-yna _ha.tAyAya, kamA bA`adta ba-yna al-ma^sriqi wAl-ma.gribi. 
al-ll_ahumma 'inni-y 'a`u-w_du bika mina al-kasali, wAl-ma'_tami, 
wAl-ma.grami.\\
\end{arabtext}
\noindent
\textbf{Artinya}:
\par
\indent
"Ya Allah, sesungguhnya aku berlindung kepada-Mu dari fitnah dan adzab 
Neraka, fitnah dan adzab kubur, keburukan fitnah kekayaan dan keburukan 
fitnah kefakiran. Ya Allah, sesungguhnya aku berlindung kepada-Mu dari 
kejahatan fitnah Dajjal. Ya Allah, bersihkanlah hatiku dengan salju dan air
es, serta sucikanlah hatiku dari tiap kesalahan sebagaimana Engkau 
menyucikan pakaian putih dari kotoran. Dan jauhkanlah antara diriku dengan 
kesalahan-kesalahanku itu sebagaimana Engkau menjauhkan timur dan barat. Ya
Allah, sesungguhnya aku berlindung kepada-Mu dari kemalasan, perbuatan 
dosa, dan utang."{\scriptsize 1}\\
\begin{arabtext}
\noindent
al-ll_ahumma 'inni-y 'a`u-w_du bika mina al-`a^gzi, wAl-kasali, wAl-^gubni,
wAl-bu_hli, wAl-harami, wAl-qaswaTi, wAl-.gaflaTi, wAl-`aylaTi, 
wAl-_d_dillaTi, wAl-maskanaTi, wa'a`u-w_du bika mina al-faqri, wAl-kufri, 
wAl-fusu-wqi, wAl-^s^siqAqi, wAl-nnifAqi, wAl-ssum`aTi, wAl-rriyA'i, 
wa'a`u-w_du bika mina al-.s.samami, wAl-bakami, wAl-^gunu-wni, 
wAl-^gu_dAmi, wAl-bara.si, wasayyi -'i al-'asqAmi.\\
\end{arabtext}
\noindent
\textbf{Artinya}:
\par
\indent
"Ya Allah, aku berlindung kepada-Mu dari kelemahan, kemalasan, sifat yang
pengecut, kekikiran, pikun, kekerasan hati, lalai, berat tanggungan,
kehinaan, dan kerendahan. Dan aku berlindung kepada-Mu dari kemiskinan,
kekufuran, kefasikan, perpecahan, kemunafikan, \textit{sum'ah} (amalnya
ingin didengar orang), \textit{riya'} (amalnya ingin dilihat orang) serta
aku berlindung kepada-Mu dari tuli, bisu, gila, sakit lepra, belang, dan
dari keburukan berbagai jenis penyakit."{\scriptsize 2}\\
\begin{arabtext}
\noindent
al-ll_ahumma 'inni-y 'a`u-w_du bika mina al-^gubni, wa'a`u-w_du bika mina
al-bu_hli, wa'a`u-w_du bika min 'an 'uradda 'i-lY 'ar_dali al-`umuri,
wa'a`u-w_du bika min fitnaTi al-ddunyA wa`a_dAbi al-qabri.\\
\end{arabtext}
\noindent
\textbf{Artinya}:
\par
\indent
"Ya Allah, sesungguhnya aku memohon perlindungan kepada-Mu dari sifat yang
pengecut, aku berlindung kepada-Mu dari sifat kikir, dan aku berlindung
kepada-Mu dari dikembalikan kepada umur yang paling hina (pikun), serta aku
berlindung kepada-Mu dari fitnah dunia dan adzab kubur."{\scriptsize 3}\\
\begin{arabtext}
\noindent
al-ll_ahumma qini-y ^sarra nafsi-y, wA`zim li-y `alY 'ar^sadi 'amri-y, 
al-ll_ahumma a.gfirli-y mA 'asrartu wamA 'a`lantu, wamA 'a_h.ta'tu wamA 
`amadtu, wamA `alimtu wamA ^gahiltu.\\
\end{arabtext}
\noindent
\textbf{Artinya}:
\par
\indent
"Ya Allah, lindungi aku dari kejahatan nafsuku dan kuatkan diriku dalam hal
sebaik-baik urusanku. Ya Allah, berilah ampunan kepadaku atas segala yang
aku sembunyikan dan segala yang aku tampakkan, juga atas apa yang tidak aku
sengaja dan yang memang aku sengaja, serta atas apa yang diketahui dan yang 
tidak kuketahui."{\scriptsize 4}\\
\begin{arabtext}
\noindent
al-ll_ahumma 'inni-y 'a`u-w_du bika mina al-`a^gzi, wAl-kasali, wAl-^gubni, 
wAl-harami, wAl-bu_hli, wa'a`u-w_du bika min `a_dAbi al-qabri, wamin 
fitnaTi al-ma.hyA wAl-mamAti.\\
\end{arabtext}
\noindent
\textbf{Artinya}:
\par
\indent
"Ya Allah, sungguh aku berlindung kepada-Mu dari kelemahan, kemalasan, 
sifat pengecut, pikun, dan kekikiran. Aku juga berlindung kepada-Mu dari 
adzab kubur serta fitnah kehidupan dan kematian."{\scriptsize 5}\\\\
\par
\noindent
\textbf{Tingkatan Doa dan Sanad}:
\begin{enumerate}
\item \textbf{Shahih}: HR. Al-Bukhari (no. 6368, 6376, 6377), Muslim (no. 
589 [129]), Ahmad (VI/57, 207), Abu Dawud (no. 1543), at-Tirmidzi (no. 
3495), an-Nasai (I/51, 176, dan VIII/262, 266), al-Hakim (I/541) dan 
lainnya, dari Aisyah r.a.
\item \textbf{Shahih}: HR. Al-Hakim (I/530) dan Ibnu Hibban (no. 2446 - 
\textit{Maw\^{a}riduzh Zh\^{a}m-\^{a}n} dan no. 1019-\textit{at-Ta'liqatul 
His\^{a}n}) dari Anas bin Malik r.a. Lihat \textit{Shah\^{i}hul J\^{a}mi'} 
(no. 1285) dan \textit{Irw\^{a}-ul Ghal\^{i}l} (III/357). Dishahihkan 
al-Hakim dan disetujui adz-Dzahabi.
\item \textbf{Shahih}: HR. Al-Bukhari (no. 2822, 6374) /
\textit{Fathul B\^{a}ri} (XI/181). Doa ini boleh dibaca sebelum atau 
sesudah salam dari shalat wajib. Lihat \textit{Fathu Dzil Jal\^{a}li wal 
Ikr\^{a}m Syarah Bul\^{u}ghul Mar\^{a}m} (III/509-510), Syarah Syaikh 
Utsaimin.
\item \textbf{Shahih}: HR. Ahmad (IV/444), Ibnu Hibban (no. 896 - 
\textit{at-Ta'l\^{i}q\^{a}tul His\^{a}n}) dan al-Hakim (I/510)--Dishahihkan
oleh al-Hakim, dan adz-Dzahabi menyepakatinya. Imam Haitsami berkata dalam 
\textit{Majma'uz Zaw\^{a}-id} (X/181): "Para perawi hadits ini 
\textit{shahih} (terpercaya)."
\item \textbf{Shahih}: HR. Al-Bukhari (no. 2823, 6367), Muslim (no. 2706) 
dari Anas bin Malik r.a.
\end{enumerate}
\textbf{Referensi}: Yazid bin Abdul Qadir Jawas. 2016. Kumpulan Do'a dari
Al-Quran dan As-Sunnah yang Shahih. Bogor: Pustaka Imam Asy-Syafi'i.
\index{berlindung}	
\index{fitnah}
\index{berbagai}	
\index{keburukan}
\footnote{Hanifah Atiya Budianto 1417051063 - Jurusan Ilmu Komputer,
Universitas Lampung}
\end{document}