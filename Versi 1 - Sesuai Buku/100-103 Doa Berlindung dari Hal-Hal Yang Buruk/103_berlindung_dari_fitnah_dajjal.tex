\documentclass[a4paper,12pt]{article}
\usepackage{arabtex} 
\usepackage[bahasa] {babel}
\usepackage{calligra}
\usepackage[top=2cm,left=3cm,right=3cm,bottom=3cm]{geometry}
\title{\Large Berlindung dari Fitnah Dajjal}
\author{\calligra Hanifah Atiya Budianto}
\begin{document}
\sffamily
\maketitle 
\fullvocalize
\setcode{arabtex}
\begin{arabtext}
\noindent
man .hafi.za `a^sra ^AyAtiN min 'awwali su-wraTi alkahfi `u.sima mina 
al-dda^g^gAli.\\
\end{arabtext}
\noindent
\textbf{Artinya}:
\par
\indent
"Barang siapa menghafal sepuluh ayat dari permulaan surah Al-Kahfi, maka ia
terpelihara dari fitnah Dajjal."{\scriptsize 1}\\
\indent
Begitu juga memohon perlindungan kepada Allah dari fitnah ad-Dajjal setelah
tasyahud akhir sebelum salam.{\scriptsize 2}\\\\
\par
\noindent
\textbf{Tingkatan Doa dan Sanad}:
\begin{enumerate}
\item \textbf{Shahih}: HR. Muslim (no. 809), al-Baihaqi (III/249), dan 
al-Hakim (II/368) dari Abu Darda r.a.
\item \textbf{Shahih}: HR. Al-Bukhari (no. 1377) dan Muslim (no. 588).
\end{enumerate}
\textbf{Referensi}: Yazid bin Abdul Qadir Jawas. 2016. Kumpulan Do'a dari
Al-Quran dan As-Sunnah yang Shahih. Bogor: Pustaka Imam Asy-Syafi'i.
\index{berlindung}	
\index{fitnah}
\index{dajjal}
\footnote{Hanifah Atiya Budianto 1417051063 - Jurusan Ilmu Komputer,
Universitas Lampung}
\end{document}