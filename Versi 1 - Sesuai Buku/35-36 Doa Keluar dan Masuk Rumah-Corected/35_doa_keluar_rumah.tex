\documentclass[a4paper,12pt]{article}
\usepackage{arabtex} 
\usepackage[bahasa] {babel}
\usepackage{calligra}
\usepackage[top=2cm,left=3cm,right=3cm,bottom=3cm]{geometry}
\title{\Large Doa Keluar Rumah}
\author{\calligra Hanifah Atiya Budianto}
\begin{document}
\sffamily
\maketitle 
\fullvocalize
\setcode{arabtex}
\begin{arabtext}
\noindent
bismi al-ll_ahi, tawakkaltu `alY al-ll_ahi, lA .hawla walA quwwata 'illA 
bi-al-ll_ahi.\\
\end{arabtext}
\noindent
\textbf{Artinya}:
\par
\indent
"Dengan nama Allah (aku keluar). Aku bertawakal kepada Allah, tidak ada 
daya dan upaya kecuali karena pertolongan Allah semata."{\scriptsize 1}\\
\begin{arabtext}
\noindent
al-ll_ahumma 'innI 'a`uw_dubika 'an 'a.dilla, 'aw 'u.dalla, 'aw 'azilla, 
'aw 'uzalla, 'aw 'a.zlima, 'aw 'u.zlama, 'aw 'a^ghala, 'aw yu^ghala 
`alayya.\\
\end{arabtext}
\noindent
\textbf{Artinya}:
\par
\indent
"Ya Allah, sesungguhnya aku berlindung kepada-Mu, janganlah sampai aku 
sesat atau disesatkan (syaitan atau orang jahat), tergelincir atau 
digelincirkan orang lain, menganiaya atau dianiaya orang lain, dan berbuat 
bodoh atau dibodohi orang lain."{\scriptsize 2}\\\\
\par
\noindent
\textbf{Tingkatan Doa dan Sanad}:
\begin{enumerate}
\item \textbf{Shahih}: HR. Abu Dawud (no. 5095), at-Tirmidzi (no. 3426), 
dari Anas bin Malik r.a. Lihat \textit{Shah\^{i}h at-Tirmidzi} (III/151, 
no.  2724).
\item \textbf{Shahih}: HR. Abu Dawud (no. 5094, at-Tirmidzi (no. 3427), 
an-Nasai (VII/268), Ibnu Majah (no. 3884) dari Ummu Salamah r.a. Lihat 
kitab \textit{Hid\^{a}yatur Ruw\^{a}t} (III/12, no. 2376). Sanad hadits 
ini shahih.
\end{enumerate}
\textbf{Referensi}: Yazid bin Abdul Qadir Jawas. 2016. Kumpulan Do'a dari
Al-Quran dan As-Sunnah yang Shahih. Bogor: Pustaka Imam Asy-Syafi'i.
\index{keluar}	
\index{rumah}	
\footnote{Hanifah Atiya Budianto 1417051063 - Jurusan Ilmu Komputer,
Universitas Lampung}
\end{document}