\documentclass[a4paper,12pt]{article}
\usepackage{arabtex} 
\usepackage[bahasa] {babel}
\usepackage{calligra}
\usepackage[top=2cm,left=3cm,right=3cm,bottom=3cm]{geometry}
\title{\Large Doa supaya Terhindar dari Bahaya Syirik}
\author{\calligra Hanifah Atiya Budianto}
\begin{document}
\sffamily
\maketitle 
\fullvocalize
\setcode{arabtex}
\begin{arabtext}
\noindent
al-ll_ahumma 'innA na`u-w_du bika min 'an nu^srika bika ^sa-y'aN na`lamuhu, 
wanasta.gfiruka limA lA na`lamuhu.\\
\end{arabtext}
\noindent
\textbf{Artinya}:
\par
\indent
"Ya Allah, sungguh kami berlindung kepada-Mu dari menyekutukan-mu, sedang 
kami mengetahuinya dan kami memohon ampunan kepada-Mu atas apa yang kami 
tidak mengetahuinya."\\\\
\par
\noindent
\textbf{Tingkatan Doa dan Sanad}: \textbf{Shahih}: HR. Ahmad (IV/403) dan 
selainnya dari Abu Musa al-Asy'ari r.a. Lihat kitab \textit{Shah\^{i}h 
at-Targh\^{i}b wat Tarh\^{i}b} (I/121-122, no. 36), dan \textit{Shah\^{i}h 
al-Adabul Mufrad} (no. 551).\\
\textbf{Referensi}: Yazid bin Abdul Qadir Jawas. 2016. Kumpulan Do'a dari
Al-Quran dan As-Sunnah yang Shahih. Bogor: Pustaka Imam Asy-Syafi'i.
\index{terhindar}
\index{bahaya}
\index{syirik}
\footnote{Hanifah Atiya Budianto 1417051063 - Jurusan Ilmu Komputer,
Universitas Lampung}
\end{document}