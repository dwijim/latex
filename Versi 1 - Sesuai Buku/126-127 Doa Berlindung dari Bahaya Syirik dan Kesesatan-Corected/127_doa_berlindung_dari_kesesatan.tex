\documentclass[a4paper,12pt]{article}
\usepackage{arabtex} 
\usepackage[bahasa] {babel}
\usepackage{calligra}
\usepackage[top=2cm,left=3cm,right=3cm,bottom=3cm]{geometry}
\title{\Large Doa Berlindung dari Kesesatan}
\author{\calligra Hanifah Atiya Budianto}
\begin{document}
\sffamily
\maketitle 
\fullvocalize
\setcode{arabtex}
\begin{arabtext}
\noindent
al-ll_ahumma laka 'aslamtu, wabika ^Amantu, wa`ala-yka tawakkaltu, 
wa'i-la-yka 'anabtu wabika _hA.samtu, al-ll_ahumma 'inni-y 'a`u-w_du 
bi`izzatika lA 'il_aha 'illA 'anta 'an tu.dillani-y, 'anta al-.hayyu 
alla_di-y lA yamu-wtu, wAl-^ginnu wAl-'insu yamu-wtu-wna.\\
\end{arabtext}
\noindent
\textbf{Artinya}:
\par
\indent
"Ya Allah, kepada-Mulah aku berserah diri, dan kepada-Mulah aku beriman, 
serta kepada-Mulah aku bertawakal, kepada-Mu aku bertaubat dan dengan 
nama-Mu aku membela. Ya Allah, sesungguhnya aku berlindung dengan 
keperkasaan-Mu, tidak ada ilah yang berhak untuk diibadahi hamba dengan 
benar melainkan hanya Engkau, agar Engkau tidak menyesatkan diriku. 
Engkaulah yang Mahahidup dan yang tidak akan pernah mati, sedangkan jin dan
manusia semuanya akan mati."\\\\
\par
\noindent
\textbf{Tingkatan Doa dan Sanad}: \textbf{Shahih}: HR. Al-Bukhari (no. 
7383) dan Muslim (no. 2717) dari Ibnu Abbas r.a. Lafazh ini milik Muslim.\\
\textbf{Referensi}: Yazid bin Abdul Qadir Jawas. 2016. Kumpulan Do'a dari
Al-Quran dan As-Sunnah yang Shahih. Bogor: Pustaka Imam Asy-Syafi'i.
\index{berlindung}
\index{kesesatan}
\footnote{Hanifah Atiya Budianto 1417051063 - Jurusan Ilmu Komputer,
Universitas Lampung}
\end{document}