\documentclass[a4paper,12pt]{article}
\usepackage{arabtex} 
\usepackage[bahasa] {babel}
\usepackage{calligra}
\usepackage[top=2cm,left=3cm,right=3cm,bottom=3cm]{geometry}
\title{\Large Doa Penghilang Kegelisahan dan Rasa Takut serta Menolak
Gangguan Syaitan ketika Tidur}
\author{\calligra Hanifah Atiya Budianto}
\begin{document}
\sffamily
\maketitle 
\fullvocalize
\setcode{arabtex}
\begin{arabtext}
\noindent
'a`u-w_du bikalimAti al-ll_ahi al-ttAmmAti min .ga.dabihi, wa`iqAbihi, 
wa^sarri `ibAdihi, wamin hamazAti al-^s^sayA .ti-yni, wa'an ya.h.duru-wni.
\\
\end{arabtext}
\noindent
\textbf{Artinya}:
\par
\indent
"Aku berlindung dengan perantara kalimat-kalimat Allah yang sempurna dari
murka dan siksa-Nya, serta dari kejahatan hamba-hamba-Nya, dan dari godaan
syaitan-syaitan, juga dari kedatangan mereka kepadaku."{\scriptsize 1}\\
\begin{arabtext}
\noindent
'a`u-w_du bikalimAti al-ll_ahi al-ttAmmAti allati-y lA yu^gA wizuhunna 
barruN walA fA^giruN min ^sarri mA _halaqa, wa_dara'a wabara'a, wamin 
^sarri mA yanzilu mina al-ssamA'i, wamin ^sarri mA ya`ru^gu fi-yhA, wamin 
^sarri mA _dara'a fiy al-'ar.di, wamin ^sarri mA ya_hru^gu minhA, wamin 
^sarri fitani al-llayli wAl-nnahAri, wamin ^sarri kulli .tAriqiN 'illA 
.tAriqaN ya.truqu bi_hayriN yA ra.hm_anu.\\
\end{arabtext}
\noindent
\textbf{Artinya}:
\par
\indent
"Aku berlindung dengan perantara kalimat-kalimat Allah yang sempurna, yang 
tidak akan dapat ditembus oleh orang yang baik maupun yang jahat, dari 
kejahatan apa yang Dia ciptakan, Dia tanamkan dan Dia adakan. Serta dari 
kejahatan yang turun dari langit, dari kejahatan yang naik ke langit, dari 
kejahatan yang ditanamkan ke bumi, dari kejahatan yang keluar dari bumi, 
dari kejahatan fitnah malam dan siang, dan dari kejahatan setiap yang 
datang kecuali apa-apa yang datang dengan membawa kebaikan, wahai Rabb Yang
Maha Pemurah."{\scriptsize 2}\\\\
\par
\noindent
\textbf{Tingkatan Doa dan Sanad}:
\begin{enumerate}
\item \textbf{Shahih}: HR. Abu Dawud (no. 3893), at-Tirmidzi (no. 3528)
Ibnu Sunni dalam \textit{'Amalul Yaum wal Lailah} (no. 748), dan lainnya.
Lihat \textit{Silsilah Ah\^{a}d\^{i}ts ash-Shah\^{i}hah} (no. 264).
\item \textbf{Shahih}: HR. Ahmad (III/419), Ibnu Sunni dalam 
\textit{'Amalul Yaum wal Lailah} (no. 637) dari Abdurrahman bin Khanbasy 
r.a. Diriwayatkan oleh ath-Thabrani dalam \textit{Mu'jamul Ausath} (no. 5411) 
dari al-Khalid bin Walid r.a. Lihat \textit{Silsilah Ah\^{a}d\^{i}ts 
ash-Shah\^{i}hah} (no. 840, 2738, 2995). Sanadnya shahih.\\
\end{enumerate}
\par
\noindent
\textbf{Referensi}: Yazid bin Abdul Qadir Jawas. 2016. Kumpulan Do'a dari
Al-Quran dan As-Sunnah yang Shahih. Bogor: Pustaka Imam Asy-Syafi'i.
\index{hilang}	
\index{gelisah}
\index{takut}	
\index{syaitan}
\index{tidur}	
\footnote{Hanifah Atiya Budianto 1417051063 - Jurusan Ilmu Komputer,
Universitas Lampung}
\end{document}