\documentclass[a4paper,12pt]{article}
\usepackage{arabtex} 
\usepackage[bahasa] {babel}
\usepackage{calligra}
\usepackage[top=2cm,left=3cm,right=3cm,bottom=3cm]{geometry}
\title{\Large Hal-Hal yang Disunnahkan pada Waktu Bermimpi Buruk}
\author{\calligra Hanifah Atiya Budianto}
\begin{document}
\sffamily
\maketitle 
\fullvocalize
\setcode{arabtex}
\par
\indent
Apabila seseorang bermimpi buruk dalam tidurnya, atau dia memimpikan 
sesuatu yang tidak disukainya, maka sebaiknya dia melakukan beberapa hal 
dibawah ini:
\begin{enumerate}
\item Meludah kecil ke arah kiri sebanyak tiga kali.{\scriptsize 1}
\item Meminta perlindungan kepada Allah dari kejahatan syaitan dan
keburukan mimpinya, juga sebanyak tiga kali.{\scriptsize 2}
\item Tidak membicarakan mimpi buruk tersebut kepada orang lain.
{\scriptsize 3}
\item Membalikkan tubuh atau mengubah posisi tidur.{\scriptsize 4}
\item Berdiri dan mengerjakan shalat, jika dia menghendakinya.
{\scriptsize 5}\\\\
\end{enumerate}
\par
\noindent
\textbf{Tingkatan Doa dan Sanad}: 
\begin{enumerate}
\item \textbf{Shahih}: HR. Al-Bukhari (no. 5747), Muslim (no.2261 [2]) dari
Abu Qatadah r.a.
\item \textbf{Shahih}: HR. Muslim (no. 2261 [4]) dari Abu Qatadah.
\item \textbf{Shahih}: HR. Muslim (no. 2261 [3,4]) dari Abu Qatadah dan
(no. 2263) dari Abu Hurairah r.a.
\item \textbf{Shahih}: HR. Muslim (no. 2262).
\item \textbf{Shahih}: HR. Muslim (no. 2263).
\end{enumerate}
\textbf{Referensi}: Yazid bin Abdul Qadir Jawas. 2016. Kumpulan Do'a dari
Al-Quran dan As-Sunnah yang Shahih. Bogor: Pustaka Imam Asy-Syafi'i.
\index{mimpi}	
\index{buruk}
\footnote{Hanifah Atiya Budianto 1417051063 - Jurusan Ilmu Komputer,
Universitas Lampung}
\end{document}