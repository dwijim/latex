\documentclass[a4paper,12pt]{article}
\usepackage{arabtex} 
\usepackage[bahasa] {babel}
\usepackage{calligra}
\usepackage[top=2cm,left=3cm,right=3cm,bottom=3cm]{geometry}
\title{\Large Doa dan Dzikir sebelum Tidur}
\author{\calligra Hanifah Atiya Budianto}
\begin{document}
\sffamily
\maketitle 
\fullvocalize
\setcode{arabtex}
\begin{arabtext}
\noindent
^gama`a kaffayhi _tumma nafa_ta fi-yhimA faqara'a fi-yhimA: (qul huwa 
al-llahu 'a.haduN) (qul 'a`uw_du birabbi al-falaqi) (qul 'a`uw_du birabbi 
al-nnAsi) _tumma yamsa.hu bihimA mA asta.tA`a min ^gasadihi yabda'u bihimA
`alY ra'sihi wawa^ghihi wamA 'aqbala min ^gasadihi.
\end{arabtext}
\noindent
\textbf{Artinya}:\\
\indent
"Rasulullah Shallallahu ‘alaihi wa sallam merapatkan dua telapak tangan 
lantas ditiup dan dibacakan: 
\textit{Qul huwall\^{a}hu ahad} (surah Al-Ikhl\^{a}s), \textit{Qul 
a'\^{u}dzu bi Rabbil falaq} (surah Al-Falaq), \textit{Qul a'\^{u}dzu bi 
Rabbin n\^{a}s} (surah An-N\^{a}s), kemudian mengusap tubuh yang dapat 
dijangkau dengan dua telapak tangan yang dimulai dari kepala, wajah, hingga 
tubuh bagian depan sebanyak 3x."{\scriptsize 1}\\\\
\noindent
- Membaca ayat Kursi.{\scriptsize 2}\\
\begin{arabtext}
\noindent
al-llahu l^A 'il_aha 'ilA huwa al-.haYYu al-qayyu-wmu, lA ta'_hu_duhu 
sinaTuN walA na-wmuN, llahu mA fiY al-ssam_aw_ati wamA fiY al-'ar.di, man 
_dA alla_diY ya^sfa`u `indahu 'illA bi-'i-_dnihi, ya`lamu mA ba-yna 
'aydi-yhim wamA _halfahum walA yu.hi-y.tuwna bi^saY'iN mmin `ilmihi, 'illA 
bimA ^sa-^A'a, wasi`a kursiyyuhu al-ssam_aw_ati wAl-'ar.da, walA ya'uduhu 
.hif.zuhumA, wahuwa al-`aliYYu al-`a.zi-ymu.\\
\end{arabtext}
\noindent
\textbf{Artinya}:\\
\indent
\textit{"Allah, tidak ada ilah (yang berhak diibadahi dengan benar) selain 
Dia. Yang Mahahidup, Yang terus-menerus mengurus (makhluk-Nya), tidak 
mengantuk dan tidak tidur. Milik-Nya apa yang ada di langit dan apa yang ada 
di bumi. Tidak ada yang dapat memberi syafaat di sisi-Nya tanpa izin-Nya. 
Dia mengetahui apa yang di hadapan mereka dan apa yang di belakang mereka, 
dan mereka tidak mengetahui sesuatu apa pun tentang ilmu-Nya melainkan apa 
yang Dia kehendaki. Kursi-Nya meliputi langit dan bumi. Dan Dia tidak merasa 
berat memelihara keduanya, dan Dia Mahatinggi, Mahabesar."} (QS. Al-Baqarah 
[2]: 255)\\\\
\noindent
- Membaca 2 ayat terakhir dari surah al-Baqarah:\\
\begin{arabtext}
\noindent
-'a-amana al-rrasuwlu bima-^A 'unzila 'ila-yhi min rrabbihi, 
wa-al-mu'-minuwna kulluN -'a-amana bi-al-llahi wamal^A'ikatihi, wakutubihi, 
warusulihi, lA nufarriqu ba-yna 'a.hadiN mmin rrusulihi, waqAluW sami`nA 
wa'a.ta`nA, .gufrAnaka rabbanA wa-'ila-yka al-ma.siyru (285). lA yukallifu 
al-llahu nafsaN 'illA wus`ahA lahA mA kasabat wa`ala-yhA mA aktasabat, 
rabbanA lA tu'A _hi_dn^A 'in nnasi-yn^A 'aw'a_h.ta'nA, rabbanA walA ta.hmil 
`ala-yn^A 'i.sraN kamA .hamaltahu, `ala alla_diyna min qablinA, rabbanA 
walA tu.hammilnA mA lA .tAqaTalanA bihi, wa-a`fu `annA wa-a.gfirlanA 
wa-ar.hamn^A, 'anta ma-wl_anA fa-an.surnA `alY al-qa-wmi al-k_afiriyna. 
(286)\\
\end{arabtext}
\noindent
\textbf{Artinya}:\\
\indent
\textit{"Rasul (Muhammad) telah beriman kepada apa (Al-Qur-an) yang 
diturunkan kepadanya dari Rabbnya, demikian pula orang-orang yang beriman. 
Semua beriman kepada Allah, Malaikat-Malaikat-Nya, Kitab-Kitab-Nya dan 
Rasul-Rasul-Nya. (Mereka berkata): 'Kami tidak membeda-bedakan seseorang 
pun dari Rasul-Rasul-Nya,' dan mereka berkata: 'Kami dengar dan kami taat.' 
(Mereka berdoa): 'Ampunilah kami ya Rabb kami dan kepada Engkaulah tempat 
kami kembali.'  Allah tidak membebani seseorang melainkan sesuai dengan 
kesanggupannya. Ia mendapat pahala (dari kebajikan) yang diusahakan dan 
mendapat siksa (dari kejahatan) yang dikerjakannya. (Mereka berdoa): 'Ya 
Rabb kami, janganlah Engkau hukum kami jika kami lupa atau kami melakukan 
kesalahan. Ya Rabb kami, janganlah Engkau bebankan kepada kami beban yang 
berat sebagaimana Engkau bebankan kepada orang-orang sebelum kami. Ya Rabb 
kami, janganlah Engkau pikulkan kepada kami apa yang tidak sanggup kami 
memikulnya. Maafkanlah kami; ampunilah kami; dan rahmatilah kami. Engkaulah 
Pelindung kami, maka tolonglah kami menghadapi orang-orang kafir.'"} 
(QS. Al-Baqarah [2]: 285-286){\scriptsize 3}\\\\
\indent
Dari al-Bara bin Azib, dia berkata: "Pada suatu hari, Rasulullah 
Shallallahu ‘alaihi wa sallam bersabda kepadaku: 'Apabila kamu hendak 
tidur, berwudhulah sebagaimana wudhumu ketika hendak melaksanakan shalat. 
Setelah itu berbaringlah di atas bagian tubuh yang kanan, kemudian bacalah 
olehmu:\\
\begin{arabtext}
\noindent
al-ll_ahumma 'aslamtu tafsi-y 'ila-yka, wawa^g^gahtu wa^ghi-y 'ila-yka, 
wafawwa.dtu  'amri-y 'ila-yka, wa'al^ga'tu .zahri-y 'ila-yka, ra.gbaTaN 
warahbaTaN 'ila-yka, lA mal^ga'a walA man^gA minka 'illA 'ila-yka ^Amantu 
bikitAbika alla_diY 'anzalta wabinabiyyika alla_diY 'arsalta.\\
\end{arabtext}
\noindent
\textbf{Artinya}:\\
\indent
'Ya Allah, aku menyerahkan diriku kepada-Mu, aku juga menghadapkan wajahku 
kepada-Mu, aku menyerahkan semua urusanku kepada-Mu, serta aku selalu 
menyandarkan punggungku kepada-Mu (yakni aku menyandarkan segala urusanku 
kepada-Mu) karena mengharap sekaligus takut kepada-Mu. Sesungguhnya, tidak 
ada tempat untuk berlindung dan menyelamatkan diri dari ancaman-Mu selain 
kepada-Mu. Aku beriman kepada Kitab yang Engkau turunkan dan beriman kepada 
Nabi yang Engkau utus.'"{\scriptsize 4}\\
\begin{arabtext}
\noindent
bi-asmika rabbi-y wa.da`tu ^ganbi-y, wabika 'arfa`uhu, 'in 'amsakta nafsi-y 
fAr.hamhA, wa-'in 'arsaltahA fA.hfa.zhA bimA ta.hfa.zu bihi `ibAdaka 
al-.s.sAli.hi-yna.\\
\end{arabtext}
\noindent
\textbf{Artinya}:\\
\indent
"Dengan nama-Mu, wahai Rabbku, aku meletakkan lambungku (tidur). Dengan 
nama-Mu pula aku bangun. Apabila Engkau mencabut nyawaku, maka berikanlah 
rahmat-Mu padanya. Apabila Engkau membiarkannya hidup maka peliharalah, 
sebagaimana Engkau selalu memelihara hamba-hamba-Mu yang shalih." 
{\scriptsize 5}\\
\begin{arabtext}
\noindent
al-ll_ahumma _halaqta nafsi-y wa'anta tawaffAhA, laka mamAtuhA wama.hyAhA, 
'in 'a.hyaytahA fA.hfa.zhA, wa-'in 'amattahA fA.gfirlahA. al-ll_ahumma 
'inni-y 'as'aluka al-`AfiyaTa.\\
\end{arabtext}
\noindent
\textbf{Artinya}:\\
\indent
"Ya Allah, sesungguhnya Engkau telah menciptakan diriku, dan Engkaulah 
yang akan mematikannya. Mati dan hidupnya hanyalah milik-Mu. Jika Engkau 
menghidupkan jiwaku ini, maka peliharalah ia. Dan jika Engkau mematikannya, 
maka ampunilah ia. Ya Allah, sesungguhnya aku mohon keselamatan kepada-Mu." 
{\scriptsize 6}\\
\begin{arabtext}
\noindent
al-ll_ahumma qini-y `a_dAbaka ya-wma tab`a_tu `ibAdaka.\\
\end{arabtext}
\noindent
\textbf{Artinya}:\\
\indent
"Ya Allah, lindungilah diriku ini dari siksaan-Mu pada hari ketika Engkau 
membangkitkan hamba-hamba-Mu."{\scriptsize 7}\\
\begin{arabtext}
\noindent
bi-asmika al-ll_ahumma 'amu-wtu wa'a.hyA.\\
\end{arabtext}
\noindent
\textbf{Artinya}:\\
\indent
"Dengan nama-Mu, ya Allah, aku mati dan aku hidup."{\scriptsize 8}\\
\begin{arabtext}
\noindent
kAna lA yanAmu .hattY yaqra'a: al^Am-^A tanzi-ylu al-ssa^gdaTa watabAraka 
alla_di-y biyadihi al-mulku.\\
\end{arabtext}
\noindent
\textbf{Artinya}:\\
\indent
"Nabi Shallallahu ‘alaihi wa sallam, apabila hendak tidur, beliau membaca: 
\textit{Alif l\^{a}m m\^{i}m tanz\^{i}l as-Sajdah} (QS. As-Sajdah: 1-30) dan 
\textit{Tab\^{a}rakallazd\^{i} biyadihil mulku} (QS. Al-Mulk [67]: 1-30)." 
{\scriptsize 9}\\\\
\par
\noindent
\textbf{Tingkatan Doa dan Sanad} :
\begin{enumerate}
\item \textbf{Shahih}: HR. Al-Bukhari (no. 5017), Abu Dawud (no. 5056), 
an-Nasai dalam \textit{'Amalul Yaum wal Lailah} (no. 793), at-Tirmidzi 
(no. 3402), dan Ahmad (VI/116). Lihat \textit{Silsilah Ah\^{a}d\^{i}ts 
ash-Shah\^{i}hah} (no. 3104).
\item Yakni ayat ke-255 surah Al-Baqarah. Rasulullah Shallallahu ‘alaihi wa 
sallam bersabda: "Siapa yang membacanya ketika akan tidur, maka dia 
senantiasa dijaga (dilindungi) oleh Allah dan tidak akan didekati oleh 
syaitan sampai Shubuh." \textbf{Shahih}: HR. Al-Bukhari (no. 2311). Lihat 
kitab \textit{Fathul B\^{a}ri} (IV/487).
\item "Siapa membaca dua ayat tersebut pada malam hari, maka keduanya telah 
mencukupinya." \textbf{Shahih}: HR. Al-Bukhari (no. 5051) dan Muslim (no. 
807, 808). \textit{Fathul B\^{a}ri} (IX/94).
\item Nabi Shallallahu ‘alaihi wa sallam bersabda: "Bila kamu mati malam 
itu, kamu mati di atas fitrah (Islam). Jadikan kalimat (dzikir) itu sebagai 
kalimat terakhir yang engkau ucapkan." \textbf{Shahih}: HR. Al-Bukhari (no. 
247, 6311, 6313, 6315, 7488), Muslim (no. 2710), Ahmad (IV/290), Abu Dawud 
(no. 5046), dan at-Tirmidzi (no. 3394). 
\item Rasulullah Shallallahu ‘alaihi wa sallam bersabda: "Apabila seseorang 
di antara kalian bangkit dari tempat tidurnya kemudian ingin tidur lagi, 
hendaknya dia mengibaskan ujung kainnya 3x, dan menyebut nama Allah, karena 
dia tidak tahu apa yang ditinggalkannya di atas tempat tidur setelah 
bangkit. Apabila dia ingin berbaring, maka hendaklah membaca: 
'\textit{Bismika Rabb}i ...'"  \textbf{Shahih}: HR. Al-Bukhari (no. 6320), 
Muslim (no. 2714), at-Tirmidzi (no. 3401), dan an-Nasai dalam kitab 
\textit{'Amalul Yaum wal Lailah} (no. 796). 
\item \textbf{Shahih}: HR. Muslim (no. 2712 [60]) dan Ahmad (II/79). Juga 
Ibnus Sunni dalam \textit{'Amalul Yaum wal Lailah} (no. 721). 
\item Apabila hendak tidur, Nabi meletakkan tangan kanan di bawah pipinya, 
lalu beliau membaca: "\textit{All\^{a}humma qin\^{i}} ..." \textbf{Shahih 
li ghairi}: HR. Al-Bukhari dalam \textit{al-Adabil Mufrad} (no. 1215) dari 
al-Bara bin Azib, at-Tirmidzi (no. 3398) dari Hudzaifah bin Yaman, serta 
Abu Dawud (no. 5045) dari Hafshah r.a. Lihat \textit{Silsilah 
Ah\^{a}d\^{i}ts ash-Shah\^{i}hah} (no. 2754).
\item \textbf{Shahih}: HR. Al-Bukhari (no. 6312, 6324) dari Hudzaifah dan 
Muslim (no. 2711) dari al-Bara bin Azib r.a.
\item \textbf{Shahih}: HR. Al-Bukhari-dalam \textit{al-Adabil Mufrad} (no. 
1207), Ahmad (III/340), ad-Darimi (II/455), dan yang lainnya. Sanad hadits 
ini shahih. Lihat \textit{Silsilah Ah\^{a}d\^{i}ts ash-Shah\^{i}hah} (no. 
585).
\end{enumerate}
\textbf{Referensi}: Yazid bin Abdul Qadir Jawas. 2016. Kumpulan Do'a dari
Al-Quran dan As-Sunnah yang Shahih. Bogor: Pustaka Imam Asy-Syafi'i.
\index{sebelum}	
\index{tidur}
\footnote{Hanifah Atiya Budianto 1417051063 - Jurusan Ilmu Komputer,
Universitas Lampung}
\end{document}