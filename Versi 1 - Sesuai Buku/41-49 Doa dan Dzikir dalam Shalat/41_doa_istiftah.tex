\documentclass[a4paper,12pt]{article}
\usepackage{arabtex} 
\usepackage[bahasa] {babel}
\usepackage{calligra}
\usepackage[top=2cm,left=3cm,right=3cm,bottom=3cm]{geometry}
\title{\Large Doa Istiftah}
\author{\calligra Hanifah Atiya Budianto}
\begin{document}
\sffamily
\maketitle 
\fullvocalize
\setcode{arabtex}
\begin{arabtext}
\noindent
sub.hAnaka al-ll_ahumma wabi.hamdika, watabAraka asmuka, wata`Al_aY 
^gadduka, walA 'il_aha .gayruka.\\
\end{arabtext}
\noindent
\textbf{Artinya}:
\par
\indent
"Mahasuci Engkau ya Allah, aku memuji-Mu, Mahaberkah Nama-Mu. Mahatinggi 
kekayaan dan kebesaran-Mu, tidak ada ilah yang berhak diibadahi dengan 
benar selain Engkau."{\scriptsize 1}\\\\
Atau membaca:\\
\begin{arabtext}
\noindent
al-ll_ahumma bA`id bayni-y wabayna _ha.tAyAya kamA bA`adta bayna 
al-ma^sriqi wAl-ma.gribi, al-ll_ahumma naqqini-y min _ha.tAyAya, kamA 
yunaqqY al-_t_tawbu al-'abya.du mina al-ddanasi, al-ll_ahumma a.gsilni-y 
min _ha.tAyAya bi-al-_t_tal^gi wAl-mA'i wAl-baradi.\\
\end{arabtext}
\noindent
\textbf{Artinya}:
\par
\indent
"Ya Allah, jauhkanlah antara aku dan kesalahan-kesalahanku, sebagaimana 
Engkau menjauhkan antara timur dan barat. Ya Allah, bersihkanlah aku dari 
kesalahan-kesalahanku, sebagaimana  baju putih dibersihkan dari kotoran. Ya
Allah, cucilah aku dari kesalahan-kesalahanku dengan salju, air, dan air 
es."{\scriptsize 2}\\\\
Atau membaca:\\
\begin{arabtext}
\noindent
wa^g^gahtu wa^ghiya lilla_di-y fa.tara al-ssamAwAti wAl-'ar.da .hani-yfaN 
wamA 'anA mina al-mu^sriki-yna, 'inna .salAti-y, wanusuki-y, wama.hyAya, 
wamamAti-y li-ll_ahi rabbi al-`Alami-yna, lA^sari-yka lahu wabi_d_alika 
'umirtu wa'anA mina al-muslimi-yna. al-ll_ahumma 'anta almaliku lA 'il_aha 
'illA 'anta. 'anta rabbi-y wa'anA `abduka, .zalamtu nafsi-y wA`taraftu 
bi_danbi-y fA.gfirli-y _dunu-wbi-y ^gami-y`aN 'innahu lA ya.gfiru 
al-_d_dunu-wba 'illA 'anta. wAhdini-y li-'a.hsani al-'a_hlAqi lA yahdi-y 
li-'a.hsanihA 'illA 'anta, wA.srif `anni-y sayyi'ahA, lA ya.srifu `anni-y 
sayyi'ahA 'illA 'anta, labbayka wasa`dayka, wAl-_hayru kulluhu fi-y 
yadayka, wAl-^s^sarru laysa 'ilayka, 'anAbika wa-'ilayka, tabArakta 
wata`Alayta, 'asta.gfiruka wa'atu-wbu 'ilayka.\\
\end{arabtext}
\noindent
\textbf{Artinya}:
\par
\indent
"Aku menghadapkan wajahku kepada Rabb Pencipta langit dan bumi, dalam 
keadaan lurus dan aku tidak termasuk orang-orang yang musyrik. Sesungguhnya
shalatku, ibadahku, hidupku serta matiku adalah untuk Allah. Rabb alam 
semesta, tidak ada sekutu bagi-Nya. Demikianlah aku diperintah dan bahwa 
aku termasuk orang Muslim. Ya Allah, Engkau adalah Raja, tidak ada ilah 
yang berhak diibadahi dengan benar kecuali Engkau, Engkau Rabbku sedangkan 
aku ini adalah hamba-Mu. Aku menganiaya diriku, aku mengakui dosa-dosaku 
(yang pernah aku lakukan). Oleh karena itu, ampunilah seluruh dosaku, 
sesungguhnya tidak ada yang dapat mengampuni dosa-dosa, kecuali Engkau. 
Tunjukkan aku pada akhlak yang baik (mulia), tidak ada yang dapat 
menunjukkan kepada akhlak yang mulia kecuali Engkau. Hindarkan aku dari 
akhlak yang buruk, tidak ada yang dapat menjauhkanku darinya kecuali 
Engkau. Aku penuhi panggilan-Mu, aku mohon pertolongan-Mu, seluruh kebaikan
berada di kedua tangan-Mu, kejelekan tidak dinisbatkan kepada-Mu. Aku hidup
dengan pertolongan dan rahmat-Mu, dan kepada-Mu (aku kembali). Mahasuci 
Engkau dan Mahatinggi. Aku memohon ampunan dan bertaubat kepada-Mu."
{\scriptsize 3}\\\\
Atau membaca:\\
\begin{arabtext}
\noindent
al-ll_ahumma rabba ^gibrA'iyla, wami-ykA'iyla, wa-'isrAfi-yla fA.tira 
al-ssamAwAti wAl-'ar.di, `Alima al-.gaybi wAl-^s^sahAdaTi, 'anta ta.hkumu 
bayna `ibAdika fi-ymA kAnuW fi-yhi ya_htalifu-wna. ihdini-y limA a_htulifa 
fi-yhi mina al-.haqqi bi-'i_dnika 'innaka tahdi-y man ta^sA'u 'ilY 
.sirA.tiN mustaqi-ymiN.\\
\end{arabtext}
\noindent
\textbf{Artinya}:
\par
\indent
"Ya Allah, Rabb Malaikat Jibril, Mika-il dan Israfil. Pencipta seluruh 
langit dan bumi. Yang Maha Mengetahui semua yang ghaib dan yang nyata. 
Engkau yang memutuskan hukum di antara hamba-hamba-Mu tentang apa-apa yang 
mereka perselisihkan. Dengan izin-Mu tunjukkanlah aku kepada kebenaran 
(yaitu, tetapkan aku di atas kebenaran) dari apa yang mereka perselisihkan.
Sesungguhnya Engkau memberi petunjuk kepada siapa yang Engkau kehendaki ke 
jalan yang lurus."{\scriptsize 4}\\\\
Atau membaca:\\
\begin{arabtext}
\noindent
al-ll_ahumma laka al-.hamdu, 'anta qayyimu al-ssamAwAti wAl-'ar.di waman 
fi-yhinna, walaka al-.hamdu laka mulku al-ssamAwAti wAl-'ar.di waman 
fi-yhinna, walaka al-.hamdu, 'anta nu-wru al-ssamAwAti wAl-'ar.di, walaka 
al-.hamdu, 'anta maliku al-ssamAwAti wAl-'ar.di, walaka al-.hamdu 'anta 
al-.haqqu, wawa`duka al-.haqqu, waliqA'uka .haqquN, waqawluka .haqquN, 
wAl-^gannaTu .haqquN, wAl-nnAru .haqquN, wAl-nnabiyyu-wna .haqquN, 
wamu.hammaduN .sallY al-ll_ahu `ala-yhi wasallama .haqquN, wAl-ssA`aTu 
.haqquN, al-ll_ahumma laka 'aslamtu, wabika ^Amantu, wa`alayka tawakkaltu, 
wa-'ilayka 'anabtu, wabika _hA.samtu, wa-'ilayka .hAkamtu, fA.gfirli-y mA 
qaddamtu wamA-'a_h_hartu, wamA-'asrartu, wamA-'a`lantu, 'anta al-muqaddimu,
wa-'anta al-mu'a_h_hiru, lA 'il_aha 'illA 'anta.\\
\end{arabtext}
\noindent
\textbf{Artinya}:
\par
\indent
"Ya Allah, bagi-Mu segala puji, Engkaulah Pemelihara seluruh langit dan 
bumi, serta segenap makhluk yang ada padanya. Bagi-Mu segala puji, bagi-Mu 
kerajaan langit dan bumi, serta segenap makhluk yang ada padanya. Bagi-Mu 
segala puji, Engkau adalah cahaya langit dan bumi. Bagi-Mu segala puji, 
Engkaulah penguasa langit dan bumi. Bagi-Mu segala puji, Engkaulah Yang 
Mahabenar, janji-Mu benar, pertemuan dengan-Mu adalah benar, firman-Mu 
benar, adanya Surga itu benar, adanya Neraka adalah benar, diutusnya para 
Nabi \textit{'alaihimussalatu wassalam} adalah benar, Nabi Muhammad 
shallallahu ‘alaihi wa sallam adalah benar, dan adanya Kiamat adalah benar. 
Ya Allah, hanya kepada-Mu aku berserah, hanya pada-Mu aku beriman, hanya 
kepada-Mu aku bertawakal, hanya pada-Mulah aku bertaubat, hanya dengan 
(pertolongan)-Mu aku berdebat dan hanya kepada-Mu aku berhukum (mohon 
keputusan). Oleh karena itu, ampunilah dosa-dosaku yang telah lalu dan yang 
akan datang, yang aku lakukan secara sembunyi-sembunyi atau 
terang-terangan. Engkaulah Yang mendahulukan dan mengakhirkan. Tidak ada 
ilah yang berhak diibadahi dengan benar kecuali Engkau."{\scriptsize 5}\\
- Setelah membaca doa istiftah, membaca ta'awwudz:\\
\begin{arabtext}
\noindent
'a`u-w_du bi-al-ll_ahi al-ssami-y`i al-`ali-ymi mina al-^s^sa-y.tAni 
al-rra^gi-ymi min hamzihi wanaf_hihi wanaf_tihi.\\
\end{arabtext}
\noindent
\textbf{Artinya}:
\par
\indent
"Aku berlindung kepada Allah Yang Maha Mendengar lagi Maha Mengetahui dari 
gangguan syaitan yang terkutuk, dari kegilaannya, kesombongannya, dan 
syairnya yang tercela."{\scriptsize 6}\\
- Membaca surah Al-Fatihah.\\
- Mengucapkan "Aamiin" setelah \begin{arabtext} (walA al-.d.da-^Ali-yna)
\end{arabtext}
- Dalam shalat berjamaah, makmum tidak boleh mendahului imam.\\
- Membaca surah sesuai dengan apa yang dicontohkan oleh Rasulullah
shallallahu 'alaihi wa sallam.{\scriptsize 7}\\\\
\par
\noindent
\textbf{Tingkatan Doa dan Sanad}:
\begin{enumerate}
\item \textbf{Shahih}: HR. Muslim (no. 399 [52]) dan ad-Daraquthni 
(I/628-629, no. 1127, 1132) dari Umar bin al-Khathab r.a. secara 
\textit{mauquf}. Diriwayatkan juga oleh ad-Daraquthni (no. 1133) dan 
ath-Thabrani dalam \textit{ad-Du'\^{a}'} (no. 506), dari Anas bin Malik 
r.a. secara marfu'. Sanadnya shahih. Lihat kitab \textit{Silsilah 
ash-Shah\^{i}hah} (no. 2996), \textit{Ashlu Shifatu Shal\^{a}tin Nabi} 
(I/254), serta kitab \textit{Irw\^{a}-ul Ghal\^{i}l} (II/51-53).
\item \textbf{Shahih}: HR. Al-Bukhari (no. 744) Muslim (no. 598 [147]), 
Ibnu Majah (no. 805), an-Nasai (II/129), dan Abu Dawud (no. 781).
\item \textbf{Shahih}: HR. Muslim (no. 771 [201]), Abu Dawud (no. 760), 
an-Nasai (II/130), Ahmad (I/94-95, 102), dan selainnya. Doa ini dibaca saat
shalat wajib dan saat shalat sunnah (\textit{Shifatu Shal\^{a}tin Nabi 
karya Syaikh al-Albani}).                                                  
\item \textbf{Shahih}: HR. Muslim (no. 770 [200]), Abu Dawud (no. 767), 
Ibnu Majah (no. 1357). Nabi membaca doa istiftah ini ketika shalat malam.
\item \textbf{Shahih}: HR. Al-Bukhari (no. 1120, 6317, 7385, 7442, 7499). 
Muslim juga meriwayatkannya dengan ringkas (no. 769 [199]) dari Ibnu Abbas 
r.a. Doa istiftah ini dibaca ketika shalat malam (Tahajud).
\item \textbf{Shahih}: HR. Abu Dawud (no. 775) dan at-Tirmidzi (no. 242). 
Dengan dasar firman Allah dalam surah Fushshilat ayat 36, lihat 
\textit{al-Kalimuth Tahyib} (no. 130), shahih. \textit{Shifatu Shal\^{a}tin
Nabi} (hlm. 95-96) dan \textit{Irw\^{a}-ul Ghal\^{i}l} (II/53-57, no. 342).
\item Lihat kitab \textit{Shifatu Shal\^{a}tin Nabi} karya Syaikh Muhammad 
Nashiruddin al-Albani.
\end{enumerate}
\textbf{Referensi}: Yazid bin Abdul Qadir Jawas. 2016. Kumpulan Do'a dari
Al-Quran dan As-Sunnah yang Shahih. Bogor: Pustaka Imam Asy-Syafi'i.
\index{istiftah}
\footnote{Hanifah Atiya Budianto 1417051063 - Jurusan Ilmu Komputer,
Universitas Lampung}
\end{document}