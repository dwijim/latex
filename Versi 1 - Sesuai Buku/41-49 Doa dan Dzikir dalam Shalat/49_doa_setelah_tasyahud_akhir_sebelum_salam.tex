\documentclass[a4paper,12pt]{article}
\usepackage{arabtex} 
\usepackage[bahasa] {babel}
\usepackage{calligra}
\usepackage[top=2cm,left=3cm,right=3cm,bottom=3cm]{geometry}
\title{\Large Doa setelah Tasyahud Akhir sebelum Salam}
\author{\calligra Hanifah Atiya Budianto}
\begin{document}
\sffamily
\maketitle 
\fullvocalize
\setcode{arabtex}
\begin{arabtext}
\noindent
al-ll_ahumma 'inni-y 'a`u-w_du bika min `a_dAbi ^gahannama, wamin `a_dAbi 
al-qabri, wamin fitnaTi al-ma.hyA wAl-mamAti, wamin ^sarri fitnaTi 
al-masi-y.hi al-dda^g^gAli.\\
\end{arabtext}
\noindent
\textbf{Artinya}:
\par
\indent
"Ya Allah, sungguh aku berlindung kepada-Mu dari siksa Neraka Jahannam, 
siksa kubur, fitnah kehidupan dan fitnah setelah mati, serta dari kejahatan
fitnah al-Masih ad-Dajjal."{\scriptsize 1}\\
Atau membaca:\\
\begin{arabtext}
\noindent
al-ll_ahumma 'inni-y 'a`u-w_du bika min `a_dA bi al-qabri, wa'a`u-w_du bika
min fitnnaTi al-masiy.hi al-dda^g^gAli, wa'a`u-w_du bika min fitnaTi 
al-ma.hyA wAl-mamAti. al-ll_ahumma 'inni-y 'a`u-w_du bika mina al-ma'_tami 
wAl-ma.grami.\\
\end{arabtext}
\noindent
\textbf{Artinya}:
\par
\indent
"Ya Allah, sesungguhnya aku berlindung kepada-Mu dari siksa kubur. Aku 
berlindung kepada-Mu dari fitnah al-Masih ad-Dajjal. Aku juga berlindung 
kepada-Mu dari fitnah kehidupan dan fitnah sesudah mati. Ya Allah, 
sesungguhnya aku berlindung kepada-Mu dari perbuatan dosa dan dari 
utang."{\scriptsize 2}\\
\begin{arabtext}
\noindent
al-ll_ahumma 'inni-y .zalamtu nafsi-y .zulmaN ka_ti-yraN, walA ya.gfiru 
al-ddunu-wba 'illA 'anta, fA.gfir li-y ma.gfiraTaN min `indika, wAr.hamni-y
'innaka 'anta al-.gafu-wru al-rra.hi-ymu.\\
\end{arabtext}
\noindent
\textbf{Artinya}:
\par
\indent
"Ya Allah, sesungguhnya aku banyak menganiaya diriku, dan tidak ada yang 
dapat mengampuni dosa-dosa kecuali Engkau. Oleh karena itulah, ampunilah 
dosa-dosaku dengan ampunan dari sisi Engkau, dan berilah rahmat kepadaku. 
Sesungguhnya Engkau Maha Pengampun lagi Maha Penyayang."{\scriptsize 3}\\
\begin{arabtext}
\noindent
al-ll_ahumma 'inni-y 'as'aluka yA Aal-ll_ahu bi-'annaka al-wA.hidu 
al-'a.hadu al-.s.samadu alla_di-y lam yalid walam yu-wlad walam yakun lahu 
kufuwaN 'a.haduN, 'an ta.gfirali-y _dunu-wbi-y 'innaka 'anta al-.gafu-wru 
al-rra.hi-ymu.\\
\end{arabtext}
\noindent
\textbf{Artinya}:
\par
\indent
"Ya Allah, sesungguhnya aku memohon kepada-Mu. Ya Allah, dengan bersaksi 
Engkau adalah Rabb Yang Maha Esa, Mahatunggal yang tidak membutuhkan 
sesuatu, tapi segala sesuatu yang butuh kepada-Mu, tidak beranak dan tidak 
diperanakan (tidak mempunyai ibu maupun bapak), tidak seorang pun yang 
menyamai-Mu, aku memohon agar Engkau mengampuni dosa-dosaku. Sesungguhnya 
Engkau Maha Pengampun lagi Maha Penyayang."{\scriptsize 4}\\
\begin{arabtext}
\noindent
al-ll_ahumma 'inni-y 'as'aluka bi-'anna laka al-.hamda lA 'il_aha 'anta 
wa.hdaka lA ^sari-yka laka, al-mannAnu, yA badi-y`a al-ssamAwAti wAl-'ar.di 
yA _dAl^galAli wAl-'ikrAmi, yA.hayyu yA qayyu-wmu 'inni-y 'as'aluka 
(al-^gannaTa wa'a`u-w_du bika mina al-nnAri).\\
\end{arabtext}
\noindent
\textbf{Artinya}:
\par
\indent
"Ya Allah, sesungguhnya aku memohon kepada-Mu. Sesungguhnya bagi-Mu segala 
pujian, tidak ada ilah yang berhak diibadahi dengan benar kecuali Engkau 
Yang Maha Esa, tiada sekutu bagi-Mu, Mahapemberi nikmat, Pencipta langit 
dan bumi tanpa contoh sebelumnya. Wahai Rabb Yang memiliki keagungan dan 
kemuliaan, wahai Rabb Yang Mahahidup, Yang berdiri sendiri (mengurusi 
makhluk-Nya) sesungguhnya aku mohon kepada-Mu agar dimasukkan [ke Surga dan
aku berlindung kepada-Mu dari siksa Neraka]."{\scriptsize 5}\\\\
\par
\noindent
\textbf{Tingkatan Doa dan Sanad}:
\begin{enumerate}
\item \textbf{Shahih}: HR. Muslim (no. 588 [128]) dari Sahabat Abu Hurairah 
r.a.
\item \textbf{Shahih}: HR. Al-Bukhari (no. 832) dan Muslim (no. 589 [129]), 
dan an-Nasai (III/56-57) dari Aisyah r.a.
\item \textbf{Shahih}: HR. Al-Bukhari (no. 834, 6326, 7387, 7388), dan 
Muslim (no. 2705 [48]) dari Sahabat Abu Bakar ash-Shiddiq r.a.
\item \textbf{Shahih}: HR. An-Nasai (III/52)-lafazh ini ialah miliknya-dan 
Ahmad (IV/338) dari Mihjan bin al-Adru r.a. Dinyatakan shahih oleh Syaikh 
al-Albani dalam \textit{Shah\^{i}h an-Nasai} (I/279, no. 1234).
\item Sabda Rasulullah Shallallahu ‘alaihi wa sallam: "Sesungguhnya dia 
meminta kepada Allah dengan nama-Nya yang teragung (\textit{ismullabi 
a'zham}). Apabila ia minta kepada Allah maka akan dipenuhi dan apabila ia 
berdoa maka akan dikabulkan." Shahih: HR. Abu Dawud (no. 1495), an-Nasai 
(III/52), Ibnu Majah (no. 3858) Ahmad (III/158, 245) dan Ibnu Mandah dalam 
\textit{Kitabut Tauhid} (no. 355). Dan tambahan dalam kurung miliknya dari 
Anas bin Malik r.a.
\end{enumerate}
\textbf{Referensi}: Yazid bin Abdul Qadir Jawas. 2016. Kumpulan Do'a dari
Al-Quran dan As-Sunnah yang Shahih. Bogor: Pustaka Imam Asy-Syafi'i.
\index{tasyahud}
\index{akhir}
\index{salam}	
\footnote{Hanifah Atiya Budianto 1417051063 - Jurusan Ilmu Komputer,
Universitas Lampung}
\end{document}