\documentclass[a4paper,12pt]{article}
\usepackage{arabtex} 
\usepackage[bahasa] {babel}
\usepackage{calligra}
\usepackage[top=2cm,left=3cm,right=3cm,bottom=3cm]{geometry}
\title{\Large Doa Ruku'}
\author{\calligra Hanifah Atiya Budianto}
\begin{document}
\sffamily
\maketitle 
\fullvocalize
\setcode{arabtex}
\begin{arabtext}
\noindent
sub.hAna rabbiya al-`a.zi-ymi.\\
\end{arabtext}
\noindent
\textbf{Artinya}:
\par
\indent
"Mahasuci Rabbku Yang Mahaagung." [Dibaca 3x].{\scriptsize 1}\\
\par
\indent
Atau membaca:
\begin{arabtext}
\noindent
sub.hAnaka al-ll_ahumma rabbanA wabi.hamdika Aal-ll_ahumma a.gfirli-y.\\
\end{arabtext}
\noindent
\textbf{Artinya}:
\par
\indent
"Mahasuci Engkau, ya Allah! Rabb kami, dan dengan memuji-Mu. Ya Allah, 
ampunilah dosaku."{\scriptsize 2}\\
\par
\indent
Atau membaca:
\begin{arabtext}
\noindent
sub.hAna _diy al-^gabaru-wti wAl-malaku-wti wAl-kibriyA'i wAl-`a.zamaTi.\\
\end{arabtext}
\noindent
\textbf{Artinya}:
\par
\indent
"Mahasuci Dia (Allah) Yang memiliki keperkasaan, kerajaan, kebesaran dan 
keagungan."{\scriptsize 3}\\
\par
\indent
Atau membaca:
\begin{arabtext}
\noindent
subbu-w.huN quddu-wsuN, rabbu al-malA'ikaTi wAl-rru-w.hi.\\
\end{arabtext}
\noindent
\textbf{Artinya}:
\par
\indent
"Engkau Rabb Yang Mahasuci (dari kekurangan dan yang tidak layak bagi 
kebesaran-Mu), Rabb seluruh Malaikat dan Jibril."{\scriptsize 4}\\\\
\par
\noindent
\textbf{Tingkatan Doa dan Sanad}: 
\begin{enumerate}
\item \textbf{Shahih}: HR. Ahmad (V/382, 394), Abu Dawud (no. 871), 
an-Nasai (II/190), at-Tirmidzi (no. 262) dan Ibnu Majah (no. 888), lafazh 
ini miliknya. Lihat \textit{Irw\^{a}-ul Ghal\^{i}l} (no. 333 dan 334).
\item \textbf{Shahih}: HR. Al-Bukhari (no. 794, 817) dan Muslim (no. 484).
\item \textbf{Shahih}: HR. Abu Dawud (no. 873), an-Nasai (II/191), dan 
sanadnya shahih.
\item \textbf{Shahih}: HR. Muslim (no. 487), Abu Dawud (no. 872), an-Nasai 
(II/191), dan Ahmad (VI/35).
\end{enumerate}
\textbf{Referensi}: Yazid bin Abdul Qadir Jawas. 2016. Kumpulan Do'a dari
Al-Quran dan As-Sunnah yang Shahih. Bogor: Pustaka Imam Asy-Syafi'i.
\index{ruku}
\footnote{Hanifah Atiya Budianto 1417051063 - Jurusan Ilmu Komputer,
Universitas Lampung}
\end{document}