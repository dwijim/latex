\documentclass[a4paper,12pt]{article}
\usepackage{arabtex} 
\usepackage[bahasa] {babel}
\usepackage{calligra}
\usepackage[top=2cm,left=3cm,right=3cm,bottom=3cm]{geometry}
\title{\Large Membaca Shalawat{\scriptsize 1} Nabi setelah Tasyahud}
\author{\calligra Hanifah Atiya Budianto}
\begin{document}
\sffamily
\maketitle 
\fullvocalize
\setcode{arabtex}
\begin{arabtext}
\noindent
al-ll_ahumma .salli `alY mu.hammadiN wa`alY ^Ali mu.hammadiN, kamA 
.sallayta `alY 'ibrAhi-yma wa`alY ^Ali 'ibrAhi-yma, 'innaka .hami-yduN 
ma^gi-yduN, al-ll_ahumma bArik `alY mu.hammadiN wa`alY ^Ali mu.hammadiN, 
kamA bArakta `alY 'ibrAhi-yma wa`alY ^Ali 'ibrAhi-yma, 'innaka .hami-yduN
ma^gi-yduN.\\
\end{arabtext}
\noindent
\textbf{Artinya} :
\par
\indent
"Ya Allah, berikanlah shalawat kepada Nabi Muhammad beserta keluarga 
Muhammad, sebagaimana Engkau telah memberikan shalawat kepada Ibrahim dan 
keluarga Ibrahim. Sesungguhnya Engkau Maha Terpuji lagi Mahamulia. 
Berikanlah berkah kepada Muhammad dan keluarga Muhammad sebagaimana Engkau 
telah memberi berkah kepada Ibrahim beserta keluarga Ibrahim. Sesungguhnya 
Engkau Maha Terpuji lagi Mahamulia."{\scriptsize 2}\\
Atau membaca:\\
\begin{arabtext}
\noindent
al-ll_ahumma .salli `alaY mu.hammadiN wa`alaY 'azwA^gihi wa_durriyyatihi, 
kamA .salla-yta `alaY ^Ali 'ibrAhi-yma, wabArik `alaY mu.hammadiN wa`alaY 
'azwA^gihi wa_durriyyatihi, kamA bArakta `alaY ^Ali 'ibrAhi-yma, 'innaka 
.hamiyduN ma^gi-yduN.\\
\end{arabtext}
\noindent
\textbf{Artinya}:
\par
\indent
"Ya Allah, berikanlah shalawat kepada Muhammad, istri-istri dan 
keturunannya, sebagaimana Engkau telah memberikan shalawat kepada keluarga 
Nabi Ibrahim. Berikanlah berkah kepada Muhammad, istri-istri dan 
keturunannya, sebagaimana Engkau telah memberikan berkah kepada keluarga 
Ibrahim. Sesungguhnya Engkau Maha Terpuji lagi Mahamulia."{\scriptsize 3}\\
\begin{arabtext}
\noindent
al-ll_ahumma .salli `alaY mu.hammadiN wa`alaY ^Ali mu.hammadiN, kamA 
.salla-yta `alaY ^Ali 'ibrAhi-yma, wabArik `alaY mu.hammadiN wa`alaY ^Ali 
mu.hammadiN, kamA bArakta `alaY ^Ali 'ibrAhi-yma fiy al-`Alami-yna, 'innaka 
.hami-yduN ma^gi-yduN.\\
\end{arabtext}
\noindent
\textbf{Artinya}:
\par
\indent
"Ya Allah, berikanlah shalawat kepada Muhammad dan keluarga Muhammad 
sebagaimana Engkau telah memberi shalawat kepada keluarga Ibrahim. Dan 
berkahilah Muhammad dan keluarga Muhammad sebagaimana Engkau telah 
memberkahi keluarga Ibrahim atas sekalian alam, sesungguhnya Engkau Maha 
Terpuji (lagi) Mahamulia."{\scriptsize 4}\\\\
\par
\noindent
\textbf{Tingkatan Doa dan Sanad}:
\begin{enumerate}
\item Tidak ada tambahan lafazh "sayyidinaa" dalam shalawat dan tidak ada 
satu pun riwayat yang shahih dari Nabi Shallallahu ‘alaihi wa sallam, dan 
lafazh ini pun tidak diucapkan oleh para Sahabat radhiyallahu 'anhum.
\item \textbf{Shahih}: HR. Al-Bukhari (no. 3370)/\textit{Fathul B\^{a}ri} 
(VI/408), Muslim (no. 406), Abu Dawud (no. 976, 977, 978), at-Tirmidzi (no. 
483), an-Nasai (III/47-48), Ahmad (IV/243-244), Ibnu Majah (no. 904), dan 
selainnya dari Ka'ab bin Ujrah r.a.
\item \textbf{Shahih}: HR. Malik dalam \textit{al-Muwaththa'} (I/152, no. 
66), al-Bukhari (no. 3369)/\textit{Fathul Bari} (VI/407), Muslim (no. 407 
[69]), Abu Dawud (no. 979), dan lainnya. Lafazh tersebut diriwayatkan oleh 
Muslim dari Abu Humaid as-Sa'idi r.a.
\item \textbf{Shahih}: HR. Malik dalam \textit{al-Muwaththa'} (I/152, no. 
67), Muslim (no. 405 [65]), Abu Dawud (no. 980), Ahmad (IV/118, V/273-274), 
at-Tirmidzi (no. 3220), an-Nasai (III/45), \textit{'Amalul Yaum wal Lailah} 
(no. 48), dan selainnya dari Abu Mas'ud al-Anshari r.a.
\end{enumerate}
\textbf{Referensi}: Yazid bin Abdul Qadir Jawas. 2016. Kumpulan Do'a dari
Al-Quran dan As-Sunnah yang Shahih. Bogor: Pustaka Imam Asy-Syafi'i.
\index{shalawat}	
\index{nabi}
\index{tasyahud}
\footnote{Hanifah Atiya Budianto 1417051063 - Jurusan Ilmu Komputer,
Universitas Lampung}
\end{document}