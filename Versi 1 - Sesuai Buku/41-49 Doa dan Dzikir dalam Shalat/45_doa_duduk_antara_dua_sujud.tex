\documentclass[a4paper,12pt]{article}
\usepackage{arabtex} 
\usepackage[bahasa] {babel}
\usepackage{calligra}
\usepackage[top=2cm,left=3cm,right=3cm,bottom=3cm]{geometry}
\title{\Large Doa Duduk Antara Dua Sujud}
\author{\calligra Hanifah Atiya Budianto}
\begin{document}
\sffamily
\maketitle 
\fullvocalize
\setcode{arabtex}
\begin{arabtext}
\noindent
rabbi a.gfirli-y, rabbi a.gfirli-y.\\
\end{arabtext}
\noindent
\textbf{Artinya}:
\par
\indent
"Wahai Rabbku, ampunilah dosaku, wahai Rabbku, ampunilah dosaku."
{\scriptsize 1}\\
\par
\indent
Atau membaca:
\begin{arabtext}
\noindent
al-ll_ahumma a.gfirli-y wAr.hamni-y wA^gburni-y wArfa`ni-y wAhdini-y 
wa`Afini-y wArzuqni-y.\\
\end{arabtext}
\noindent
\textbf{Artinya}:
\par
\indent
"Ya Allah, ampunilah dan sayangilah aku, cukupilah kekuranganku, angkatlah 
derajatku, berilah petunjuk kepadaku, selamatkanlah aku, dan berikanlah aku
rizki (yang halal)."{\scriptsize 2}\\\\
\par
\noindent
\textbf{Tingkatan Doa dan Sanad}: 
\begin{enumerate}
\item \textbf{Shahih}: HR. Abu Dawud (no. 874). Lihat \textit{Shah\^{i}h 
Ibni Majah} (no. 731).
\item \textbf{Shahih}: HR. At-Tirmidzi (no. 284), Abu Dawud (no. 850), Ibnu
Majah (no. 898). Lihat \textit{Shah\^{i}h Tirmidzi} (I/90, no. 233), 
\textit{Shah\^{i}h Abi Dawud} (I/160, no. 756), dan \textit{Shah\^{i}h Ibni 
Majah} (I/148, no. 732) dengan lafazh \textit{"rabbi"}, \textit{Shifatu 
Shal\^{a}tin Nabi} karya Syaikh al-Albani.
\end{enumerate}
\textbf{Referensi}: Yazid bin Abdul Qadir Jawas. 2016. Kumpulan Do'a dari
Al-Quran dan As-Sunnah yang Shahih. Bogor: Pustaka Imam Asy-Syafi'i.
\index{duduk}	
\index{antara}
\index{dua}	
\index{sujud}
\footnote{Hanifah Atiya Budianto 1417051063 - Jurusan Ilmu Komputer,
Universitas Lampung}
\end{document}