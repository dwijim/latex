\documentclass[a4paper,12pt]{article}
\usepackage{arabtex} 
\usepackage[bahasa] {babel}
\usepackage{calligra}
\usepackage[top=2cm,left=3cm,right=3cm,bottom=3cm]{geometry}
\title{\Large Doa Sujud}
\author{\calligra Hanifah Atiya Budianto}
\begin{document}
\sffamily
\maketitle 
\fullvocalize
\setcode{arabtex}
\begin{arabtext}
\noindent
sub.hAna rabbiya al-'a `l_aY.\\
\end{arabtext}
\noindent
\textbf{Artinya}:
\par
\indent
"Mahasuci Rabbku, Yang Mahatinggi (dari segala kekurangan dan hal yang 
tidak layak)." [Dibaca 3x]{\scriptsize 1}\\
\indent Atau membaca:
\begin{arabtext}
\noindent
sub.hA naka al-ll_ahumma rabbanA wa-bi.hamdika Aal-ll_ahumma a.gfirli-y.\\
\end{arabtext}
\noindent
\textbf{Artinya}:
\par
\indent
"Mahasuci Engkau, ya Allah. Rabb kami, dan dengan memuji-Mu. Ya Allah, 
ampunilah dosaku."{\scriptsize 2}\\
\indent Atau membaca:
\begin{arabtext}
\noindent
subbu-w.huN quddu-wsuN, rabbu al-malA'ikaTi wAl-rru-w.hi.\\
\end{arabtext}
\noindent
\textbf{Artinya}:
\par
\indent
"Engkau Rabb Yang Mahasuci (dari kekurangan dan hal yang tidak layak bagi 
kebesaran-Mu) Rabb para Malaikat dan Jibril."{\scriptsize 3}\\
\indent Atau membaca:
\begin{arabtext}
\noindent
sub.hAna _diy al-^gabaru-wti wAl-malaku-wti wAl-kibriyA'i wAl-`a.zamaTi.\\
\end{arabtext}
\noindent
\textbf{Artinya}:
\par
\indent
"Mahasuci (Allah), Rabb Yang memiliki keperkasaan, kerajaan, kebesaran, 
dan keagungan."{\scriptsize 4}\\\\
\par
\noindent
\textbf{Tingkatan Doa dan Sanad}:
\begin{enumerate}
\item \textbf{Shahih}: HR. Ahmad (V/382, 394), Abu Dawud (no. 871), 
an-Nasai (II/190), at-Tirmidzi (no. 262), Ibnu Majah (no. 888). Lihat 
\textit{Irw\^{a}-ul Ghal\^{i}l} (no. 333, 334).
\item \textbf{Shahih}: HR. Al-Bukhari (no. 794, 817) dan Muslim (no. 484).
\item \textbf{Shahih}: HR. Muslim (no. 487)-\textit{Syarah Muslim} 
(IV/204-205).
\item \textbf{Shahih}: HR. Abu Dawud (no. 873), an-Nasai, dan Ahmad. 
Dishahihkan oleh Syaikh al-Albani dalam kitab \textit{Shah\^{i}h Abi 
Dawud} (I/166).
\end{enumerate}
\textbf{Referensi}: Yazid bin Abdul Qadir Jawas. 2016. Kumpulan Do'a dari
Al-Quran dan As-Sunnah yang Shahih. Bogor: Pustaka Imam Asy-Syafi'i.
\index{sujud}
\footnote{Hanifah Atiya Budianto 1417051063 - Jurusan Ilmu Komputer,
Universitas Lampung}
\end{document}