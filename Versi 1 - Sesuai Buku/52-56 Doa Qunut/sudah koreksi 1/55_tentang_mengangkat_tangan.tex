\documentclass[a4paper,12pt]{article}
\usepackage{arabtex} 
\usepackage[bahasa] {babel}
\usepackage{calligra}
\usepackage[top=2cm,left=3cm,right=3cm,bottom=3cm]{geometry}
\title{\Large Tentang Mengangkat Tangan}
\author{\calligra Hanifah Atiya Budianto}
\begin{document}
\sffamily
\maketitle 
\fullvocalize
\setcode{arabtex}
\par
\indent
Disunnahkan mengangkat tangan baik dalam qunut Nazilah maupun dalam qunut 
Witir berdasarkan dalil hadits-hadits yang sanadnya shahih dan atsar dari 
sahabat.{\scriptsize 1}\\
\indent Adapun mengusap wajah sesudah qunut atau berdoa, tidak ada satu pun
riwayat yang shahih. Maka, perbuatan ini adalah \textbf{bid'ah}.
{\scriptsize 2}\\
\indent Imam al-Baihaqi juga menjelaskan bahwa tidak ada seorang pun dari 
ulama Salafush Shalih yang mengusap wajah sesudah doa qunut dalam shalat.
{\scriptsize 3}\\\\
\par
\noindent
\textbf{Tingkatan Doa dan Sanad}: 
\begin{enumerate}
\item Lihat \textit{Sunanul Kubra lil Baihaqi} (II/211-212, III/39-41) dan 
\textit{Irwa-ul Ghalil} (II/163-164).
\item Lihat \textit{Irwa-ul Ghalil} (II/181). Lihat pula kitab 
\textit{Shahih Kitab al-Adzkar wa Dha'ifuhu} (hlm. 960-962).
\item \textit{Sunan al-Baihaqi} (II/212).
\end{enumerate}
\textbf{Referensi}: Yazid bin Abdul Qadir Jawas. 2016. Kumpulan Do'a dari
Al-Quran dan As-Sunnah yang Shahih. Bogor: Pustaka Imam Asy-Syafi'i.
\index{mengangkat}	
\index{tangan}
\index{shalat}	
\index{qunut}
\index{nazilah}	
\index{doa}
\index{nazilah}	
\footnote{Hanifah Atiya Budianto 1417051063 - Jurusan Ilmu Komputer,
Universitas Lampung}
\end{document}