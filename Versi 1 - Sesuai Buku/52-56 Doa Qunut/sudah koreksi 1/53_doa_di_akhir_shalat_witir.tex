\documentclass[a4paper,12pt]{article}
\usepackage{arabtex} 
\usepackage[bahasa] {babel}
\usepackage{calligra}
\usepackage[top=2cm,left=3cm,right=3cm,bottom=3cm]{geometry}
\title{\Large Doa di Akhir Shalat Witir}
\author{\calligra Hanifah Atiya Budianto}
\begin{document}
\sffamily
\maketitle 
\fullvocalize
\setcode{arabtex}
\begin{arabtext}
\noindent
al-ll_ahumma 'inni-y 'a`u-w_du biri.dAka min sa_ha.tika, wabimu`AfAtika min 
`uqu-wbatika, wa'a`u-w_du bika minka, lA 'u.h.si-y _tanA'aN `alayka, 'anta 
kamA 'a_tnayta `alY nafsika.\\
\end{arabtext}
\noindent
\textbf{Artinya}:
\par
\indent
"Ya Allah, sesungguhnya aku berlindung dengan keridhaan-Mu dari 
kemurkaan-Mu, dengan keselamatan-Mu dari hukuman-Mu, dan berlindung 
kepada-Mu dari siksaan-Mu. Aku tidaklah mampu menghitung pujian dan 
sanjungan kepada-Mu, Engkau adalah sebagaimana Engkau menyanjung/memuji 
diri-Mu sendiri."{\scriptsize 1}\\
\begin{arabtext}
\noindent
sub.hAna al-maliki al-qudduwsi, sub.hAna al-maliki al-qudduwsi, sub.hAna 
al-maliki al-qudduwsi.\\
\end{arabtext}
\noindent
\textbf{Artinya}:
\par
\indent
"Mahasuci Allah Raja Yang Mahasuci, Mahasuci Allah Raja Yang Mahasuci, 
Mahasuci Allah Raja Yang Mahasuci." [Nabi mengangkat dan memanjangkan 
suaranya pada ucapan yang ketiga] {\scriptsize 2}\\\\
\par
\noindent
\textbf{Tingkatan Doa dan Sanad}: 
\begin{enumerate}
\item \textbf{Shahih}: HR. Abu Dawud (no. 1427), at-Tirmidzi (no. 3566), 
Ibnu Majah (no. 1179), an-Nasai (III/249), Ahmad (I/98, I/96, 118, 150). 
Lihat \textit{Shah\^{i}h at-Tirmidzi} (III/180), \textit{Shah\^{i}h Ibni 
Majah} (I/194), \textit{Irw\^{a}-ul Ghal\^{i}l} (II/175), dan 
\textit{Shah\^{i}h Kit\^{a}b al-Adzk\^{a}r} (I/255-256, no. 246/184).
\item \textbf{Shahih}: Abu Dawud (no. 1430), an-Nasai (III/245), dan Ahmad 
(V/123), Ibnu Hibban (no. 2441-\textit{at-Ta'liqatul His\^{a}n}), Ibnus 
Sunni (no. 706), serta al-Baghawi dalam \textit{Syarhus Sunnah} (IV/98, 
no. 972). Lihat juga \textit{Shah\^{i}h Kit\^{a}b al-Adzk\^{a}r} (I/255) 
dan \textit{Z\^{a}dul Ma'\^{a}d} (I/337).
\end{enumerate}
\textbf{Referensi}: Yazid bin Abdul Qadir Jawas. 2016. Kumpulan Do'a dari
Al-Quran dan As-Sunnah yang Shahih. Bogor: Pustaka Imam Asy-Syafi'i.
\index{akhir}	
\index{sesudah}
\index{shalat}	
\index{witir}
\footnote{Hanifah Atiya Budianto 1417051063 - Jurusan Ilmu Komputer,
Universitas Lampung}
\end{document}