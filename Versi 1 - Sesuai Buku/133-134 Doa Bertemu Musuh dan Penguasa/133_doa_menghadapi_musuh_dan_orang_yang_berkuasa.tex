\documentclass[a4paper,12pt]{article}
\usepackage{arabtex} 
\usepackage[bahasa] {babel}
\usepackage{calligra}
\usepackage[top=2cm,left=3cm,right=3cm,bottom=3cm]{geometry}
\title{\Large Doa Menghadapi Musuh dan Orang yang Berkuasa}
\author{\calligra Hanifah Atiya Budianto}
\begin{document}
\sffamily
\maketitle 
\fullvocalize
\setcode{arabtex}
\begin{arabtext}
\noindent
al-ll_ahumma 'innA na^g`aluka fi-y nu.hu-wrihim wana`u-w_du bika min 
^suru-wrihim.\\
\end{arabtext}
\noindent
\textbf{Artinya}:
\par
\indent
"Ya Allah, sungguh kami menjadikan Engkau di leher mereka (agar kekuatan 
mereka tidak berdaya saat berhadapan dengan kami). Dan kami berlindung 
kepada-Mu dari kejelekan mereka."{\scriptsize 1}\\
\begin{arabtext}
\noindent
al-ll_ahumma 'anta `a.dudi-y, wa'anta na.si-yri-y, bika 'a.hu-wlu, wabika 
'a.su-wlu, wabika 'uqAtilu.\\
\end{arabtext}
\noindent
\textbf{Artinya}:
\par
\indent
"Ya Allah, Engkau adalah Penolongku. Engkau adalah Pembelaku. Dan dengan 
pertolongan-Mu aku bergerak, dengan bantuan-Mu aku menyergap, dengan 
pertolongan-Mu pula aku berperang."{\scriptsize 2}\\
\begin{arabtext}
\noindent
al-ll_ahumma munzila al-kitAbi sari-y`a al-.hisAbi, ihzimi al-'a.hzAba, 
al-ll_ahumma ahzimhum wazalzilhum.\\
\end{arabtext}
\noindent
\textbf{Artinya}:
\par
\indent
"Ya Allah, Yang menurunkan Kitab suci, Yang menghisab perbuatan manusia 
dengan cepat. Kalahkanlah golongan musuh. Ya Allah, cerai beraikanlah dan 
goncangkanlah mereka."{\scriptsize 3}\\
\begin{arabtext}
\noindent
( .hasbunA al-llahu wani`ma al-waki-ylu )\\
\end{arabtext}
\noindent
\textbf{Artinya}:
\par
\indent
\textit{"Cukuplah Allah menjadi Penolong kami. Dan Dia adalah sebaik-baik 
Pelindung."} (QS. Ali 'Imran [3]: 173).{\scriptsize 4}\\\\
\par
\noindent
\textbf{Tingkatan Doa dan Sanad}:
\begin{enumerate}
\item \textbf{Shahih}: HR. Abu Dawud (no. 1537), an-Nasai dalam 
\textit{'Amalul Yaum wal Lailah} (no. 606), dan al-Hakim (II/142). 
Dishahihkan oleh al-Hakim dan disepakati oleh adz-Dzahabi.
\item \textbf{Shahih}: HR. Abu Dawud (no. 2632) dan at-Tirmidzi (no. 3584) 
dari Anas r.a. Lihat \textit{Shah\^{i}h at-Tirmidzi} (III/183) dan 
\textit{al-Kalimuth Thayyib} (hlm. 120, no. 126).     
\item \textbf{Shahih}: HR. Al-Bukhari (no. 2933, 4115), Muslim (no.  1742 
[21]), at-Tirmidzi (no. 1678) dan Ibnu Majah (no. 2796).
\item Kalimat ini diucapkan oleh Nabi Ibrahim a.s. ketika dilemparkan ke 
dalam api, dan juga diucapkan Nabi Muhammad SAW. ketika orang-orang 
berkata: "\textit{Sesungguhnya manusia telah mengumpulkan pasukan untuk 
menyerangmu}." (QS. Ali 'Imran [3]: 173). \textbf{Shahih}: HR. Al-Bukhari 
(no. 4563, 4564).
\end{enumerate}
\textbf{Referensi}: Yazid bin Abdul Qadir Jawas. 2016. Kumpulan Do'a dari
Al-Quran dan As-Sunnah yang Shahih. Bogor: Pustaka Imam Asy-Syafi'i.
\index{menghadapi}	
\index{musuh}
\index{orang}
\index{berkuasa}
\footnote{Hanifah Atiya Budianto 1417051063 - Jurusan Ilmu Komputer,
Universitas Lampung}
\end{document}