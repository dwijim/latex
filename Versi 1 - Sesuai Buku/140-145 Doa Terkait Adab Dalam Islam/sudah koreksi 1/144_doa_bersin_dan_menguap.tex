\documentclass[a4paper,12pt]{article}
\usepackage{arabtex} 
\usepackage[bahasa] {babel}
\usepackage{calligra}
\usepackage[top=2cm,left=3cm,right=3cm,bottom=3cm]{geometry}
\title{\Large Doa Bersin dan Menguap}
\author{\calligra Hanifah Atiya Budianto}
\begin{document}
\sffamily
\maketitle 
\fullvocalize
\setcode{arabtex}
\begin{arabtext}
\noindent
'i_dA `a.tasa 'a.hadukum falyaqul: al-.hamduli-ll_ahi, walyaqul lahu 
'a_hu-whu 'aw.sA.hibuhu: yar.hamuka al-ll_ahu, fa'i-_dA qala lahu: 
yar.hamuka al-ll_ahu falyaqul: yahdi-ykumu al-ll_ahu wayu.sli.hu bAlakum.\\
\end{arabtext}
\noindent
\textbf{Artinya}:
\par
\indent
"Apabila salah seorang di antara kalian bersin, hendaklah ia berkata: 
\textit{Alhamdulill\^{a}h} 'Segala puji bagi Allah,' lantas saudara atau 
temannya berkata: \textit{yarkhamukall\^{a}h} 'Semoga Allah memberikan 
rahmat kepada-Mu.' Apabila teman atau saudaranya berkata demikian, bacalah:
\textit{yahdiykumull\^{a}h wayushlikhu balakum} 'Semoga Allah memberi 
petunjuk kepadamu dan memperbaiki keadaanmu."{\scriptsize 1}\\
\begin{arabtext}
\noindent
'inna al-ll_aha yu.hibbu al-`u.tAsa wayakrahu al-tta_tA'uba, fa'i-_dA 
`a.tasa 'a.hadukum wa.hamida al-ll_aha kAna .haqqaN `alY kulli muslimiN 
sami`ahu 'an yaqu-wla lahu :  yar.hamuka al-ll_ahu, wa'ammA al-tta_tA'ubu 
fa'i-nnamA huwa min al-^s^say.tAni,  fa'i-_dA ta_tA'a ba 'a.hadukum 
falyaruddahu mA asta.tA`a, fa'i-nna 'a.hadakum 'i_dA ta_tA'aba .da.hika 
minhu al-^s^say.tAnu.\\
\end{arabtext}
\noindent
\textbf{Artinya}:
\par
\indent
"Sesungguhnya Allah menyukai bersin dan membenci menguap. Apabila salah 
seorang dari kalian bersin dan memuji Allah (mengucapkan 
\textit{Alhamdulill\^{a}h}), maka hendaklah setiap Muslim yang mendengarnya
berkata kepada orang yang bersin: '\textit{Yarhamukall\^{a}h} (artinya, 
semoga Allah merahmatimu).' Adapun menguap itu datangnya dari syaitan. Maka
apabila salah seorang dari kalian menguap, hendaklah ia berusaha untuk 
menahan semampunya. Sebab syaitan akan tertawa tatkala salah seorang dari 
kalian menguap." {\scriptsize 2}\\
\par
\indent
Dari Abu Musa al-Asy'ari r.a., dia bertutur: "Aku mendengar Rasulullah 
Shallallahu ‘alaihi wa sallam bersabda:\\
\begin{arabtext}
\noindent
'i_dA `a.tasa 'a.hadukum fa.hamida al-ll_aha, fa^sammitu-whu, fa'i-n lam 
ya.hmadi al-ll_aha, falA tu^sammitu-whu.\\
\end{arabtext}
\noindent
\textbf{Artinya}:
\par
\indent
'Jika salah seorang dari kalian bersin kemudian mengucapkan 
\textit{Alhamdulill\^{a}h}, hendaklah kalian membacakan \textit{tasymit} 
baginya (yaitu ucapan \textit{Yarhamukall\^{a}h}). Sedangkan jika dia tidak
mengucapkan \textit{Alhamdulill\^{a}h}, maka janganlah kamu membacakan 
\textit{tasymit} baginya.'"{\scriptsize 3}\\\\
\par
\noindent
\textbf{Tingkatan Doa dan Sanad}:
\begin{enumerate}
\item \textbf{Shahih}: HR. Al-Bukhari (no. 6224) dari Sahabat Abu Hurairah 
r.a.
\item HR. Al-Bukhari (no. 6226). Lihat \textit{Fathul B\^{a}ri} (X/611 no. 
6226).
\item \textbf{Shahih}: HR. Muslim (no. 2992).
\end{enumerate}
\textbf{Referensi}: Yazid bin Abdul Qadir Jawas. 2016. Kumpulan Do'a dari
Al-Quran dan As-Sunnah yang Shahih. Bogor: Pustaka Imam Asy-Syafi'i.
\index{bersin}	
\index{menguap}
\footnote{Hanifah Atiya Budianto 1417051063 - Jurusan Ilmu Komputer,
Universitas Lampung}
\end{document}