\documentclass[a4paper,12pt]{article}
\usepackage{arabtex} 
\usepackage[bahasa] {babel}
\usepackage{calligra}
\usepackage[top=2cm,left=3cm,right=3cm,bottom=3cm]{geometry}
\title{\Large Lafazh Shalawat dan Salam kepada Nabi yang Ringkas}
\author{\calligra Hanifah Atiya Budianto}
\begin{document}
\sffamily
\maketitle 
\fullvocalize
\setcode{arabtex}
\begin{arabtext}
\begin{enumerate}
\item .sallY al-ll_ahu `alayhi wasallama
\item `alayhi al-.s.salATu wAl-ssalAmu
\item al-ll_ahumma .salli wasallim `alayhi
\end{enumerate}
\end{arabtext}
\par
\indent Imam an-Nawawi r.a. berkata: "Apablia seorang dari kalian 
bershalawat kepada Nabi, hendaklah dia menggabungkan antara shalawat dan 
salam, tidak boleh mengucapkan \textit{Shallallahu'alayhi} saja atau hanya 
mengucapkan \textit{'alayhissalam}.{\scriptsize 1}\\
\indent Ibnu Shalah r.a. berkata: "Sebaiknya penulis hadits dan para 
penuntut ilmu menulis shalawat serta salam kepada Nabi (dengan atau secara 
lengkap), dan saat menyebutnya jangan merasa bosan mengulang-ulangnya, 
karena amal yang demikian sangat besar manfaatnya yang segera diperoleh 
bagi mereka. Adapun bagi siapa saja yang lalai darinya, maka ia tercegah 
mendapat pahala besar, dan hendaklah dia tidak memotong dan tidak 
menyingkat shalawat ketika menuliskannya."{\scriptsize 2}\\
\indent Yang perlu kita perhatikan dalam hal ini, yakni mengucapkan 
shalawat kepada Nabi SAW, bahwa tidak boleh seseorang membuat 
shalawat-shalawat yang tidak dicontohkan oleh beliau. Sebab amalan tersebut
merupakan ibadah, sedangkan dasar ibadah dalam Islam adalah 
\textit{ittiba'} (mencontoh Rasulullah SAW.)\\
\indent \textbf{Di antara contoh shalawat yang diucapkan kaum Muslimin 
namun tidak ada contohnya dari Nabi kita atau bid'ah adalah shalawat 
al-Fatih dan yang lainnya.}\\
\indent \textbf{Adapun di antara contoh buku atau risalah yang berisikan 
shalawat bid'ah seperti Dal\^{a}-ilul Khair\^{a}t wa Syaw\^{a}riqul 
Anw\^{a}r F\^{i} Dzikris Shal\^{a}ti alan Nabiyyil Mukht\^{a}r}. 
{\scriptsize 3}\\\\
\noindent
\textbf{Sanad}:
\begin{enumerate}
\item Shah\^{i}h al-Adzk\^{a}r (I/325).
\item \textit{'Ulumul Had\^{i}ts} karya Ibnu Shalah (hlm. 124). Lihat 
nukilan tersebut dalam kitab \textit{al-B\^{a}'itsul Hatsits}: 
\textit{Syarh Ikhtishar 'Ul\^{u}mil Had\^{i}ts} karya Ibnu Katsir, dengan 
\textit{syarh} Ahmad Muhammad Syakir. Lihat pula kitab \textit{Fadhlush 
Shal\^{a}h 'alan Nabi} karya Syaikh Abdul Muhsin al-Abbad al-Badr (hlm. 
15).
\item Lihat \textit{Kutub allati Hadzdzara Minhal Ulama'} karya Syaikh 
Masyhur Hasan Alu Salman, serta kitab \textit{Fadhlush Shal\^{a}h 'alan} 
Nabi (hlm. 18-21) karya Syaikh Abdul Muhsin al-Abbad al-Badr.
\end{enumerate}
\textbf{Referensi}: Yazid bin Abdul Qadir Jawas. 2016. Kumpulan Do'a dari
Al-Quran dan As-Sunnah yang Shahih. Bogor: Pustaka Imam Asy-Syafi'i.
\index{shalawat}	
\index{salam}
\index{nabi}	
\footnote{Hanifah Atiya Budianto 1417051063 - Jurusan Ilmu Komputer,
Universitas Lampung}
\end{document}