\documentclass[a4paper,12pt]{article}
\usepackage{arabtex} 
\usepackage[bahasa] {babel}
\usepackage{calligra}
\usepackage[top=2cm,left=3cm,right=3cm,bottom=3cm]{geometry}
\title{\Large Menyebarkan Salam}
\author{\calligra Hanifah Atiya Budianto}
\begin{document}
\sffamily
\maketitle 
\fullvocalize
\setcode{arabtex}
\begin{arabtext}
\noindent
.qala rasu-wlu al-ll_ahi .sallY al-ll_ahu `ala-yhi wasallama : lA 
tad_hulu-wna al-^gannaTa .hattY tu'minuW, walA tu'minuW .hattY ta.hAbuW, 
'awalA 'adullukum `alY ^sa-y'iN 'i_dA fa`altumu-whu ta.hAbabtum? 'af^suW 
al-ssalAma baynakum.\\
\end{arabtext}
\noindent
\textbf{Artinya}:
\par
\indent
Rasulullah shallallahu ‘alaihi wa sallam bersabda: "Kalian tidak akan masuk 
Surga hingga beriman, kalian tidak akan beriman secara sempurna hingga 
saling mencintai. Maukah kalian aku tunjukkan sesuatu yang apabila kalian 
melakukannya, maka kalian akan saling mencintai? Sebarkan (ucapkanlah) 
salam di antara kalian (apabila bertemu)."{\scriptsize 1}\\
\begin{arabtext}
\noindent
`an `abdi al-ll_ahi bni `amriN-w ....: 'anna ra^gulaN sa'ala al-nnabiyya 
.sallY al-ll_ahu `ala-yhi wasallama : 'ayyu al-'islAmi _hayruN? .qala : 
tu.t`imu al-.t.ta`Ama, wata.qra'u al-ssalAma `alY man `arafta waman lam 
ta`rif.\\
\end{arabtext}
\noindent
\textbf{Artinya}:
\par
\indent
Dari Abdullah bin Amr r.a., bahwa ada laki-laki bertanya kepada Nabi 
shallallahu ‘alaihi wa sallam: "Mana ajaran Islam yang lebih baik?" Beliau 
menjawab: "Hendaklah kamu memberi makan (kaum fakir-miskin), mengucapkan 
salam baik kepada orang yang kamu kenal maupun kepada orang yang tidak kamu 
kenal."{\scriptsize 2}\\\\
\par
\noindent
\textbf{Shahih}: 
\textbf{Tingkatan Doa dan Sanad}:
\begin{enumerate}
\item \textbf{Shahih}: HR. Muslim (no. 54), Abu Awanah (I/30), Ahmad 
(II/391, 442), Abu Dawud (no. 5193), dan Ibnu Majah (no. 3692) dari Abu 
Hurairah r.a.
\item \textbf{Shahih}: HR. Al-Bukhari (no. 12, 28), \textit{Fathul 
B\^{a}ri} (I/55), dan Muslim (no. 39).
\end{enumerate}
\textbf{Referensi}: Yazid bin Abdul Qadir Jawas. 2016. Kumpulan Do'a dari
Al-Quran dan As-Sunnah yang Shahih. Bogor: Pustaka Imam Asy-Syafi'i.
\index{salam}		
\footnote{Hanifah Atiya Budianto 1417051063 - Jurusan Ilmu Komputer,
Universitas Lampung}
\end{document}