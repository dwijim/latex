\documentclass[a4paper,12pt]{article}
\usepackage{arabtex} 
\usepackage[bahasa] {babel}
\usepackage{calligra}
\usepackage[top=2cm,left=3cm,right=3cm,bottom=3cm]{geometry}
\title{\Large Amalan dan Doa pada Hari Jum'at}
\author{\calligra Hanifah Atiya Budianto}
\begin{document}
\sffamily
\maketitle 
\fullvocalize
\setcode{arabtex}
\par
\indent
\begin{enumerate}
\item Tidak boleh mengkhususkan malam Jum'at dari malam-malam yang lain 
dengan ibadah yang tertentu. Tidak boleh juga mengkhususkan hari Jum'at 
dengan puasa yang tertentu, terkecuali pada waktu yang biasa seseorang 
puasa (yang bertepatan dengan hari Jum'at). {\scriptsize 1}
\item Tidak boleh mengkhususkan bacaan dzikir dan doa tertentu, juga 
membaca surah-surah tertentu baik pada malam maupun hari Jum'at 
{\scriptsize 2} secara umum, kecuali yang memang disyariatkan.
\item Amal-amal yang disyariatkan dan disunnahkan pada hari Jum'at, 
adalah:
\begin{description}
\item[a]Memperbanyak bacaan shalawat kepada Nabi Shallallahu ‘alaihi wa 
sallam. {\scriptsize 3}
\item[b]Membaca surah Al-Kahfi.\\
\indent Dari Abu Sa'id al-Khudri r.a., ia berkata, bahwa Rasulullah 
Shallallahu ‘alaihi wa sallam bersabda: 
\begin{arabtext}
\noindent
man qara'a su-wraTa al-kahfi yawma al-^gumu`aTi 'a.dA'a lahu mina 
al-nnu-wri mA bayna al-^gumu`atayni.\\
\end{arabtext}
\noindent
\textbf{Artinya}:
\par
\indent
"Barang siapa membaca surah Al-Kahfi pada hari Jum'at akan diberikan cahaya
baginya di antara dua Jum'at."{\scriptsize 4}\\
\item[c]Memperbanyak doa.\\
\indent Nabi Shallallahu ‘alaihi wa sallam bersabda: "Pada hari Jum'at ada satu waktu yang bila seorang Muslim shalat dan minta kepada Allah, maka akan dikabulkan". Lalu beliau mengisyaratkan bahwa waktunya sedikit.{\scriptsize 5}\\
\indent Dalam riwayat lain diterangkan: "Waktu (terkabulnya doa itu) antara
duduk imam hingga selesai shalat".{\scriptsize 6}\\
\indent Dalam riwayat lainnya Jabir r.a. bahwa Rasulullah Shallallahu ‘alaihi wa sallam bersabda: 
"Carilah (waktu dikabulkannya doa) di akhir waktu sesudah Ashar pada hari 
Jumat".{\scriptsize 7}\\
\item[d]Shalat Jum'at berjamaah\\
\indent Amal paling utama dan wajib pada hari ini adalah shalat Jum'at 
berjamaah, bersama kaum Muslimin, dan Nabi Shallallahu ‘alaihi wa sallam memerintahkan mandi sebelumnya.\\
\indent Adapun berbagai keutamaan serta kewajiban pada hari Jum'at bisa 
dilihat pada kitab \textit{Z\^{a}dul Ma'\^{a}d} (I/364-440) karya Imam Ibnu
Qayyim al-Jauziyyah.\\\\
\end{description}
\end{enumerate}
\par
\noindent
\textbf{Tingkatan Doa dan Sanad}:
\begin{enumerate}
\item \textbf{Shahih}: HR. Muslim (no. 1144 [148]).
\item Misalnya membaca surah Yasin, Al-W\^{a}qi'ah, dan Ar-Rahm\^{a}n atau 
wirid-wirid tertentu yang tidak ada satu pun riwayat shahih tentangnya.
\item \textbf{Hasan}: HR. Abu Dawud (no. 1047), an-Nasai (III/91-92). 
Disunnahkan membaca shalawat kepada Nabi pada malam dan hari Jum'at. HR. 
Al-Baihaqi (III/249) dari Anas r.a. Lihat \textit{Silsilah Ah\^{a}d\^{i}ts 
ash-Shah\^{i}hah} (no. 1407).
\item \textbf{Shahih}: HR. Al-Hakim (II/368) dan al-Baihaqi (III/249). 
Dishahihkan oleh Syaikh al-Albani dalam kitab \textit{Irw\^{a}-ul 
Ghal\^{i}l} (no. 626). Ada riwayat lain dari Abu Sa'id al-Khudri r.a., dia 
berkata: "Barang siapa membaca surah Al-Kahfi pada malam Jum'at ....". HR. 
Ad-Darimi (II/454), sanadnya \textit{mauquf shahih}. Lihat 
\textit{Ah\^{a}ditsul Jumu'ah} karya Syaikh Abdul Quddus. Imam asy-Syafi'i 
menyatakan: "Aku menyukai surah Al-Kahfi untuk dibaca pada malam Jum'at". 
Lihat \textit{Shah\^{i}h al-Adzk\^{a}r} (I/449). Dengan demikian, 
disunnahkan bagi kita membaca surah Al-Kahfi pada malam dan hari Jum'at.
\item \textbf{Shahih}: HR. Al-Bukhari (no. 935). Dari Sahabat Abu Hurairah 
r.a.
\item \textbf{Shahih}: HR. Muslim (no. 853). Dari Abdullah bin Umar r.a.
\item \textbf{Shahih}: HR. Abu Dawud (no. 1048), an-Nasai (III/99-100).
\end{enumerate}
\textbf{Referensi}: Yazid bin Abdul Qadir Jawas. 2016. Kumpulan Do'a dari
Al-Quran dan As-Sunnah yang Shahih. Bogor: Pustaka Imam Asy-Syafi'i.
\index{jumat}	
\footnote{Hanifah Atiya Budianto 1417051063 - Jurusan Ilmu Komputer,
Universitas Lampung}
\end{document}