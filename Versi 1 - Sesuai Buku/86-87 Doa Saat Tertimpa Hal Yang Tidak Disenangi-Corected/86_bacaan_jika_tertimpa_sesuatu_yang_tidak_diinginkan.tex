\documentclass[a4paper,12pt]{article}
\usepackage{arabtex} 
\usepackage[bahasa] {babel}
\usepackage{calligra}
\usepackage[top=2cm,left=3cm,right=3cm,bottom=3cm]{geometry}
\title{\Large Bacaan jika Tertimpa Sesuatu yang Tidak Diinginkan}
\author{\calligra Hanifah Atiya Budianto}
\begin{document}
\sffamily
\maketitle 
\fullvocalize
\setcode{arabtex}
\begin{arabtext}
\noindent
qaddara al-ll_ahu wamA ^sA'a fa`ala.\\
\end{arabtext}
\noindent
\textbf{Artinya}:
\par
\indent
"Allah sudah menakdirkan segala sesuatu dan Dia berbuat menurut apa yang 
Dia kehendaki."\\
\begin{arabtext}
\noindent
qadaru al-ll_ahi wamA ^sA'a fa`ala.\\
\end{arabtext}
\noindent
\textbf{Artinya}:
\par
\indent
Boleh juga diucapkan: "Ini adalah takdir Allah dan Dia berbuat menurut apa 
yang Dia kehendaki."\\\\
\par
\noindent
\textbf{Tingkatan Doa dan Sanad}: \textbf{Shahih}: HR. Muslim (no. 2664 
[34]).\\
\textbf{Referensi}: Yazid bin Abdul Qadir Jawas. 2016. Kumpulan Do'a dari
Al-Quran dan As-Sunnah yang Shahih. Bogor: Pustaka Imam Asy-Syafi'i.
\index{tertimpa}	
\index{sesuatu}
\index{tidak}	
\index{diinginkan}
\footnote{Hanifah Atiya Budianto 1417051063 - Jurusan Ilmu Komputer,
Universitas Lampung}
\end{document}