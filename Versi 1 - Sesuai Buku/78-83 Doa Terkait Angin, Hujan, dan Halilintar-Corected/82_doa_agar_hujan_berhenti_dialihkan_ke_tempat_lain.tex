\documentclass[a4paper,12pt]{article}
\usepackage{arabtex} 
\usepackage[bahasa] {babel}
\usepackage{calligra}
\usepackage[top=2cm,left=3cm,right=3cm,bottom=3cm]{geometry}
\title{\Large Doa Agar Hujan Berhenti (Dialihkan  ke tempat lain)}
\author{\calligra Hanifah Atiya Budianto}
\begin{document}
\sffamily
\maketitle 
\fullvocalize
\setcode{arabtex}
\begin{arabtext}
\noindent
al-ll_ahumma .hawAla-ynA walA `alaynA, al-ll_ahumma `alY al'AkAmi
wAl-.z.zirAbi, wabu.tu-wni al-'awdiyaTi wamanAbiti al-^s^sa^gari.\\
\end{arabtext}
\noindent
\textbf{Artinya}:
\par
\indent
"Ya Allah, turunkanlah hujan di sekitar kami, bukan untuk merusak kami. Ya 
Allah, turunkanlah hujan ke daratan tinggi, beberapa anak bukit, perut 
lembah dan beberapa tanah yang menumbuhkan pepohonan."\\\\
\par
\noindent
\textbf{Tingkatan Doa dan Sanad}: \textbf{Shahih}: HR. Al-Bukhari (no. 1013,
1014), Muslim (no. 897) dari Anas bin Malik r.a.\\
\textbf{Referensi}: Yazid bin Abdul Qadir Jawas. 2016. Kumpulan Do'a dari
Al-Quran dan As-Sunnah yang Shahih. Bogor: Pustaka Imam Asy-Syafi'i.
\index{hujan}	
\index{berhenti}
\footnote{Hanifah Atiya Budianto 1417051063 - Jurusan Ilmu Komputer,
Universitas Lampung}
\end{document}