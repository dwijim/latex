\documentclass[a4paper,12pt]{article}
\usepackage{arabtex} 
\usepackage[bahasa] {babel}
\usepackage{calligra}
\usepackage[top=2cm,left=3cm,right=3cm,bottom=3cm]{geometry}
\title{\LARGE Dzikir Setelah Hujan}
\author{\calligra Hanifah Atiya Budianto}
\begin{document}
\sffamily
\maketitle 
\fullvocalize
\setcode{arabtex}
\begin{arabtext}
\noindent
mu.tirnA bifa.dli al-ll_ahi wara.hmatihi.\\
\end{arabtext}
\noindent
\textbf{Artinya}:
\par
\indent
"Kita diberi hujan karena karunia dan rahmat Allah."\\\\
\par
\noindent
\textbf{Tingkatan Doa dan Sanad}: \textbf{Shahih}: HR. Al-Bukhari (no. 846,
1038), Muslim (no. 71). Tidak boleh menisbatkan hujan kepada bintang, karena
datangnya hujan itu dengan sebab rahmat Allah, bukan karena bintang. Orang 
yang menisbatkan demikian telah kufur kepada Allah.\\
\textbf{Referensi}: Yazid bin Abdul Qadir Jawas. 2016. Kumpulan Do'a dari
Al-Quran dan As-Sunnah yang Shahih. Bogor: Pustaka Imam Asy-Syafi'i.
\index{setelah}	
\index{hujan}
\footnote{Hanifah Atiya Budianto 1417051063 - Jurusan Ilmu Komputer,
Universitas Lampung}
\end{document}