\documentclass[a4paper,12pt]{article}
\usepackage{arabtex} 
\usepackage[bahasa] {babel}
\usepackage{calligra}
\usepackage[top=2cm,left=3cm,right=3cm,bottom=3cm]{geometry}
\title{\LARGE Dzikir Ketika Mendengar Halilintar}
\author{\calligra Hanifah Atiya Budianto}
\begin{document}
\sffamily
\maketitle 
\fullvocalize
\setcode{arabtex}
\begin{arabtext}
\noindent
sub.haana a-lla_di-y yusabbi.hu al-rra`du bi.hamdihi wAl-malA'ikaTu min
_hi-yfatihi.\\
\end{arabtext}
\noindent
\textbf{Artinya}:
\par
\indent
"Mahasuci Allah yang halilintar bertasbih dengan memuji-Nya, begitu juga 
para Malaikat, karena takut kepada-Nya."\\\\
\par
\noindent
\textbf{Tingkatan Doa dan Sanad}: \textbf{Shahih}: \textit{Al-Muwaththa}'
(II/757, no. 26), al-Bukhari dalam \textit{al-Adabul Mufrad} (no. 723),
\textit{Shah\^{i}h al-Adabul Mufrad} (no. 556), al-Baihaqi (III/362),
\textit{al-Kalimuth Thayyib} (no. 157). Syaikh al-Albani berkata: "Hadits 
di atas mauquf sanadnya shahih ," dari Ibnu az-Zubair r.a. \\
\textbf{Referensi}: Yazid bin Abdul Qadir Jawas. 2016. Kumpulan Do'a dari
Al-Quran dan As-Sunnah yang Shahih. Bogor: Pustaka Imam Asy-Syafi'i.
\index{halilintar}	
\footnote{Hanifah Atiya Budianto 1417051063 - Jurusan Ilmu Komputer,
Universitas Lampung}
\end{document}