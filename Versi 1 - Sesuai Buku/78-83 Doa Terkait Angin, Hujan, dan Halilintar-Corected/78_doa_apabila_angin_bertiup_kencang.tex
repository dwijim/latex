\documentclass[a4paper,12pt]{article}
\usepackage{arabtex} 
\usepackage[bahasa] {babel}
\usepackage{calligra}
\usepackage[top=2cm,left=3cm,right=3cm,bottom=3cm]{geometry}
\title{\Large Doa Apabila Angin Bertiup Kencang}
\author{\calligra Hanifah Atiya Budianto}
\begin{document}
\sffamily
\maketitle 
\fullvocalize
\setcode{arabtex}
\begin{arabtext}
\noindent
al-ll_ahumma 'inni-y 'as'aluka _ha-yrahA, wa'a`u-w_du bika min ^sarrihA.\\
\end{arabtext}
\noindent
\textbf{Artinya}:
\par
\indent
"Ya Allah, sesungguhnya aku mohon kepada-Mu kebaikan angin ini, dan aku 
berlindung kepada-Mu dari kejelekannya."{\scriptsize 1}\\
\begin{arabtext}
\noindent
al-ll_ahumma 'inni-y 'as'aluka _ha-yrahA, wa_ha-yra mA fi-yhA, wa_ha-yra 
mA 'ursilat bihi, wa'a`u-w_du bika min ^sarrihA, wa^sarri mA fi-yhA, 
wa^sarri mA 'ursilat bihi.\\
\end{arabtext}
\noindent
\textbf{Artinya}:
\par
\indent
"Ya Allah, sungguh kepada-Mu aku memohon kebaikan angin ini, kebaikan apa-
apa yang ada padanya dan kebaikan tujuan angin ini dihembuskan. Aku 
berlindung kepada-Mu dari kejelekan angin ini, kejelekan apa-apa yang ada 
padanya dan kejelekan tujuan angin ini dihembuskan."{\scriptsize 2}\\\\
\par
\noindent
\textbf{Tingkatan Doa dan Sanad}:
\begin{enumerate}
\item \textbf{Shahih}: HR. Abu Dawud (no. 5097), Ibnu Majah (no. 3727), dan
lihat \textit{Shah\^{i}h al-Adzk\^{a}r} (no. 521/381).
\item \textbf{Shahih}: HR. Muslim (no. 899 [15]) dan at-Tirmidzi (no. 3449)
dari Aisyah r.a.
\end{enumerate}
\textbf{Referensi}: Yazid bin Abdul Qadir Jawas. 2016. Kumpulan Do'a dari
Al-Quran dan As-Sunnah yang Shahih. Bogor: Pustaka Imam Asy-Syafi'i.
\index{angin}	
\index{bertiup}
\index{kencang}
\footnote{Hanifah Atiya Budianto 1417051063 - Jurusan Ilmu Komputer,
Universitas Lampung}
\end{document}