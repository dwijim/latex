\documentclass[a4paper,12pt]{article}
\usepackage{arabtex} 
\usepackage[bahasa] {babel}
\usepackage[top=1cm,left=1.5cm,right=1cm,bottom=2cm]{geometry}
\usepackage{xcolor, framed}
\definecolor{shadecolor}{rgb}{0.8,0.8,0.8}
\begin{document}
\sffamily
\fullvocalize
\setcode{arabtex}
\noindent 
2------------------------------Doa agar Termasuk Golongan Orang yang Beriman
\begin{arabtext}
\noindent
rabbi hab liY .hukmaN wa'al.hiqniY bi-al-.s.s_ali.hiyna $\odot$ wa-a^g`al 
lliY lisAna .sidqiN fiY  al-'a_hiriyna $\odot$ wa-a^g`alniY min wara_taTi 
^gannaTi al-nna`iymi $\odot$ walA tu_hziniY yawma yub`a_tuwna $\odot$
\end{arabtext}
\noindent
\textbf{Artinya}:\\
\indent
"Ya Rabbku, berikanlah kepadaku ilmu dan masukkanlah aku ke dalam golongan 
orang-orang yang shalih, dan jadikanlah aku buah tutur yang baik bagi 
orang-orang (yang datang) kemudian, dan jadikanlah aku termasuk orang yang 
mewarisi Surga yang penuh kenikmatan ..., dan janganlah Engkau hinakan aku 
pada hari mereka dibangkitkan." (QS. Asy-Syu'ar\^{a}' [26]: 83-85 dan 87)\\
\begin{arabtext}
\noindent
rabbana-^A -'a-amannA fa-uktubnA ma`a al-^s^s_ahidiyna
\end{arabtext}
\noindent
\textbf{Artinya}:\\
\indent
"Ya Rabb, kami telah beriman, maka catatlah kami bersama orang-orang yang 
menjadi saksi (atas kebenaran al-Qur-an dan kenabian Muhammad)." 
(QS. Al-M\^{a}-idah [5]: 83).\\\\
\par
\noindent 
3------------------------------Doa Agar Amal Ibadah dan Taubat Diterima
\begin{arabtext}
\noindent
rabbanaa taqabbal minna'A 'innaka 'anta al-ssamiy`u al-`aliymu $\odot$ 
watub `alayna'A 'innaka 'anta al-ttawwaabu al-rra.hiymu $\odot$ 
\end{arabtext}
\noindent
\textbf{Artinya}:\\
\indent
"Ya Rabb kami, terimalah (amal) dari kami. Sungguh, Engkaulah Yang Maha 
Mendengar, Maha Mengetahui. Dan terimalah taubat kami. Sungguh, Engkaulah 
Yang Maha Penerima taubat, Maha Penyayang." (QS. Al-Baqarah [2]: 127, 128).\\
\par
\noindent 
4------------------------------Doa Mohon Diberi Keturunan yang Shalih
\begin{arabtext}
\noindent
rabbi lA ta_darniY fardaN wa'anta _hayru al-w_ari_tiyna
\end{arabtext}
\noindent
\textbf{Artinya}:\\
\indent
"Ya Rabbku, janganlah Engkau biarkan aku hidup seorang diri (tanpa 
keturunan) dan Engkaulah ahli waris yang terbaik." (QS. Al-Anbiy\^{a}' 
[21]: 89).\\
\begin{arabtext}
\noindent
rabbi hab liY mina al-.s.s_ali.hiyna
\end{arabtext}
\noindent
\textbf{Artinya}:\\
\indent
"Ya Rabbku, anugerahkanlah kepadaku (seorang anak) yang termasuk orang yang
shalih." (QS. Ash-Sh\^{a}ff\^{a}t [37]: 100).\\
\begin{arabtext}
\noindent
rabbi hab liY min lladunka _durriyyaTaN .tayyibaTaN 'innaka samiy`u 
al-ddu`a-^A'i
\end{arabtext}
\noindent
\textbf{Artinya}:\\
\indent
"Ya Rabbku, berilah aku keturunan yang baik dari sisi-Mu. Sesungguhnya 
Engkau Maha Mendengar doa." (QS. Ali 'Imran [3]: 38).\\
\begin{arabtext}
\noindent
rabbanA hab lanA min 'azw_a^ginA wa_durriyy_atinA qurraTa 'a`yuniN 
wa-u^g`alnA lilmuttaqi-yna 'imAmaN
\end{arabtext}
\noindent
\textbf{Artinya}:\\
\indent
"Ya Rabb kami, anugerahkanlah kepada kami pasangan kami dan keturunan kami 
sebagai penyenang hati (kami), dan jadikanlah kami pemimpin bagi 
orang-orang yang bertakwa." (QS. Al-Furq\^{a}n [25]: 74).\\\\\\
\par
\noindent 
5------------------------------Doa Memohon Ketetapan bagi Diri Sendiri dan Keluarga dalam Mendirikan Shalat
\begin{arabtext}
\noindent
rabbi a^g`alniY muqiyma al-.s.sal_awTi wamin _durriyyatiY, rabbanaa 
wataqabbal du`a'A'i
\end{arabtext}
\noindent
\textbf{Artinya}:\\
\indent
"Ya Rabbku, jadikanlah aku dan anak cucuku orang-orang yang tetap 
melaksanakan shalat. Ya Rabb kami, perkenankanlah doaku." (QS. Ibrahim
[14]: 40)\\\\
\par
\noindent 
6------------------------------Doa Mohon Ampun bagi Orang Tua dan Kaum Mukmin
\begin{arabtext}
\noindent
rabbanA a.gfir liY waliw_alidayya walilmu' miniyna yawma yaquwmu al-.hisAbu
\end{arabtext}
\noindent
\textbf{Artinya}:\\
\indent
"Ya Rabb kami, ampunilah aku dan kedua ibu bapakku dan semua orang yang 
beriman pada hari diadakan perhitungan (hari Kiamat)." (QS. Ibrahim [14]: 
41)\\
\begin{arabtext}
\noindent
rabbanA a.gfir lanA wali-'i_hw_aninA alla_diyna sabaquwnA bi-al-'iym_ani 
walA ta^g`al fiY quluwbinA .gillaN llilla_diyna -'a-amanuW rabbana-^A 
'innaka ra -'u-wfuN rra.hiymuN
\end{arabtext}
\noindent
\textbf{Artinya}:\\
\indent
"Ya Rabb kami, ampunilah kami dan saudara-saudara kami yang telah beriman 
lebih dahulu dari kami, dan janganlah Engkau tanamkan kedengkian dalam hati
kami terhadap orang-orang yang beriman. Ya Tuhan kami, Sungguh, Engkau Maha
Penyantun, Maha Penyayang." (QS. Al-Hasyr [59]: 10).\\
\begin{arabtext}
\noindent
rabbi a.gfir liY waliw_alidaYYa waliman da_hala baytiYa mu'minaN 
walilmu'miniyna wa-al-mu'min_ati
\end{arabtext}
\noindent
\textbf{Artinya}:\\
\indent
"Ya Rabbku, ampunilah aku, ibu bapakku, orang yang masuk ke rumahku dengan
beriman dan semua orang yang beriman laki-laki dan perempuan." (QS. Nuh 
[71]: 28).\\\\
\par
\noindent 
7------------------------------Doa Agar Ditambahkan Ilmu
\begin{arabtext}
\noindent
rabbi zidniY `ilmaN
\end{arabtext}
\noindent
\textbf{Artinya}:\\
\indent
"Ya Rabbku, tambahkanlah ilmu kepadaku." (QS. Thaha [20]: 14).\\\\
\par
\noindent 
8------------------------------Doa Supaya Disempurnakan Cahaya Diri
\begin{arabtext}
\noindent
rabban'A 'atmim lanA nuwranA wa-a.gfir lana'A 'innaka `al_aY kulli ^saY'iN 
qadiyruN
\end{arabtext}
\noindent
\textbf{Artinya}:\\
\indent
"Ya Rabb kami, sempurnakanlah untuk kami cahaya kami dan ampunilah kami; 
Sungguh, Engkau Mahakuasa atas segala sesuatu." (QS. At-Tahrim [66]: 8).\\\\
\par
\noindent 
9------------------------------Doa Berlindung Dari Orang Yang Zhalim
\begin{arabtext}
\noindent
rabbi na^g^giniY mina al-qawmi al-.z.z_alimiyna
\end{arabtext}
\noindent
\textbf{Artinya}:\\
\indent
"Ya Rabbku, selamatkanlah aku dari orang-orang yang zhalim itu." (QS. 
Al-Qashash [28]: 21).\\
\begin{arabtext}
\noindent
rabbanA lA ta^g`alnA ma`a al-qawmi al-.z.z_alimiyna
\end{arabtext}
\noindent
\textbf{Artinya}:\\
\indent
"Ya Rabb kami, janganlah Engkau tenpatkan kami bersama orang-orang yang 
zhalim itu." (QS. Al-A'r\^{a}f [7]: 47).\\
\begin{arabtext}
\noindent
rabbi an.surniY `alaY al-qawmi al-mufsidiyna
\end{arabtext}
\noindent
\textbf{Artinya}:\\
\indent
"Ya Rabbku, tolonglah aku (dengan menimpakan azab) atas golongan yang 
berbuat kerusakan itu." (QS. Al-'Ankab\^{u}t [29]: 30).\\\\
\par
\noindent 
10------------------------------Doa Bertawakal Kepada Allah
\begin{arabtext}
\noindent
rabbanA `alayka tawakkalnA wa'i-layka 'anabnA wa'i-layka al-ma.siyru
\end{arabtext}
\noindent
\textbf{Artinya}:\\
\indent
"Ya Rabb kami, hanya kepada Engkau kami bertawakal dan hanya kepada Engkau 
kami bertobat dan hanya kepada Engkaulah kami kembali." (QS. Al-Mumtahanah 
[60]: 4).\\
\begin{arabtext}
\noindent
.hasbiYa al-llahu la-^A 'il_aha 'illA huwa `alayhi tawakkaltu wahuwa rabbu
al-`ar^si al-`a.ziymi
\end{arabtext}
\noindent
\textbf{Artinya}:\\
\indent
"Cukuplah Allah bagiku; tidak ada ilah selain Dia. Hanya kepada-Nya aku
bertawakal, dan Dia adalah Rabb yang memiliki 'Arsy (singgasana) yang 
agung." (QS. At-Taubah [9]: 129).\\\\
\par
\noindent 
13------------------------------Doa Memohon Kebaikan Dunia dan Akhirat
\begin{arabtext}
\noindent
rabbana'A -'a-atinaa fiY alddunyaa .hasanaTaN wafiY al-'a _hiraTi .hasanaTaN 
waqinaa `a_daaba alnnaari
\end{arabtext}
\noindent
\textbf{Artinya}:\\
\indent
"Ya Rabb kami, berilah kami kebaikan di dunia dan kebaikan di akhirat, dan 
lindungilah kami dari azab Neraka." (QS. Al-Baqarah [2]: 201).\\\\
\par
\noindent 
14------------------------------Doa Supaya Hati Ditetapkan dalam Hidayah
\begin{arabtext}
\noindent
rabbanaa lA tuzi.g quluwbanaa ba`da 'i_d hadaytanaa wahab lanaa min 
lladunka ra.hmaTaN 'innaka 'anta al-wahhaabu
\end{arabtext}
\noindent
\textbf{Artinya}:\\
\indent
"Ya Rabb kami, janganlah Engkau condongkan hati kami kepada kesesatan 
setelah Engkau berikan petunjuk kepada kami, dan karuniakanlah kepada kami 
rahmat dari sisi-Mu, dan sesungguhnya Engkau Maha Pemberi." (QS. Ali 
'Imran [3]: 8).\\\\
\par
\noindent 
15------------------------------Doa Bagi Keamanan Negeri dan Berlindung dari Syirik
\begin{arabtext}
\noindent
rabbi a_h`al h_a_daa albalada -'a-aminaN waa^gnubniY wabaniyya 'an nna`buda 
al-'a.snaama
\end{arabtext}
\noindent
\textbf{Artinya}:\\
\indent
"Ya Rabb, jadikanlah negeri ini, negeri yang aman, dan jauhkanlah aku 
beserta anak cucuku agar tidak menyembah berhala." (QS. Ibrahim [14]: 35)\\\\
\par
\noindent 
16------------------------------Doa Berlindung dari Neraka
\begin{arabtext}
\noindent
rabbanaa a.srif `annaa `a_daaba ^gahannama 'inna `a_daabahaa kaana 
.garaa-maN $\odot$ 'innahaa sa'A'at mustaqarraN wamuqaamaN
\end{arabtext}
\noindent
\textbf{Artinya}:\\
\indent
"'Ya Rabb kami, jauhkanlah azab Jahanam dari kami, karena sesungguhnya 
azabnya itu membuat kebinasaan yang kekal,' sungguh, Jahanam itu 
seburuk-buruk tempat menetap dan tempat kediaman." (QS. Al-Furqan [25]: 
65-66).\\\\
\par
\noindent 
17----------------------Doa Berlindung dari Syaitan
\begin{arabtext}
\noindent
rabbi 'a`uw_du bika min hamaz_ati al-^s^say_a.tiyni $\odot$ wa'a`uw_du 
bika rabbi 'an ya.h.duruwni
\end{arabtext}
\noindent
\textbf{Artinya}:\\
\indent
"Ya Rabbku, aku berlindung kepada Engkau dari bisikan-bisikan syaitan, dan 
aku berlindung (pula) kepada Engkau ya Rabbku, agar mereka tidak mendekati 
aku." (QS. Al-Mu'min\^{u}n [23]: 97-98).\\\\
\par
\noindent 
18-----------------------Doa Berlindung dari Fitnah Kemenangan Orang Kafir
\begin{arabtext}
\noindent
rabbanA lA ta^g`alnA fitnaTaN llilla_diyna kafaruW wa-a.gfir lanA rabban^A 
'innaka 'anta al-`aziyzu al-.hakiymu
\end{arabtext}
\noindent
\textbf{Artinya}:\\
\indent
"Ya Rabb kami, janganlah Engkau jadikan kami (sasaran) fitnah bagi 
orang-orang kafir. Dan ampunilah kami, ya Rabb kami. Sesungguhnya Engkau 
yang Mahaperkasa, Mahabijaksana." (QS. Al-Mumtahanah [60]: 5).\\
\begin{arabtext}
\noindent
rabbanA lA ta^g`alnA fitnaTaN llilqawmi al-.z.z_alimiyna $\odot$ 
wana^g^ginA bira.hmatika mina al-qawmi al-k_afiriyna
\end{arabtext}
\noindent
\textbf{Artinya}:\\
\indent
"Ya Rabb kami, janganlah Engkau jadikan kami (sasaran) fitnah bagi kaum 
yang zalim, dan selamatkanlah kami dengan rahmat-Mu dari orang-orang 
kafir." (QS. Yunus [10]: 85-86).\\\\
\par
\noindent 
19---------------------Doa Meminta Pertolongan dan Perlindungan Dari Orang Kafir
\begin{arabtext}
\noindent
.hasbunaa al-llahu wani`ma al-wakiylu
\end{arabtext}
\noindent
\textbf{Artinya}:\\
\indent
"Cukuplah Allah (menjadi penolong) bagi kami dan Dia sebaik-baik 
pelindung." (QS. Ali 'Imran [3]: 173).\\\\
\par
\noindent 
20----------------------Doa Mohon Kesabaran dan Diwafatkan Dalam Keadaan Muslim
\begin{arabtext}
\noindent
rabbana'A 'afri.g `alaynaa .sabraN watawaffanaa muslimiyna
\end{arabtext}
\noindent
\textbf{Artinya}:\\
\indent
"Ya Rabb kami, limpahkan kesabaran kepada kami dan matikanlah kami dalam 
keadaan muslim (berserah diri kepada-Mu)." (QS. Al-A'r\^{a}f [7]: 126).\\\\
\par
\noindent 
23------------------------------Hal-Hal yang Disunnahkan pada Waktu Bermimpi Buruk
\par
\indent
Apabila seseorang bermimpi buruk dalam tidurnya, atau dia memimpikan 
sesuatu yang tidak disukainya, maka sebaiknya dia melakukan bebrapa hal 
dibawah ini:
\begin{enumerate}
\item Meludah kecil ke arah kiri sebanyak tiga kali.{\scriptsize 1}
\item Meminta perlindungan kepada Allah dari kejahatan syaitan dan
keburukan mimpinya, juga sebanyak tiga kali.{\scriptsize 2}
\item Tidak membicarakan mimpi buruk tersebut kepada orang lain.
{\scriptsize 3}
\item Membalikkan tubuh atau mengubah posisi tidur.{\scriptsize 4}
\item Berdiri dan mengerjakan shalat, jika dia menghendakinya.
{\scriptsize 5}\\
\end{enumerate}
\par
\noindent
\textbf{Tingkatan Doa dan Sanad}: 
\begin{enumerate}
\item \textbf{Shahih}: HR. Al-Bukhari (no. 5747), Muslim (no.2261 [2]) dari
Abu Qatadah r.a.
\item \textbf{Shahih}: HR. Muslim (no. 2261 [4]) dari Abu Qatadah r.a.
\item \textbf{Shahih}: HR. Muslim (no. 2261 [3,4]) dari Abu Qatadah r.a dan
(no. 2263) dari Abu Hurairah r.a.
\item \textbf{Shahih}: HR. Muslim (no. 2262).
\item \textbf{Shahih}: HR. Muslim (no. 2263).\\\\
\end{enumerate}
\par
\noindent 
24---------------------Doa Penghilang Kegelisahan dan Rasa Takut serta Menolak
Gangguan Syaitan ketika Tidur
\begin{arabtext}
\noindent
'a`u-w_du bikalimAti al-ll_ahi al-ttAmmAti min .ga.dabihi, wa`iqAbihi, 
wa^sarri `ibAdihi, wamin hamazAti al-^s^sayA .ti-yni, wa'an ya.h.duru-wni.
\\
\end{arabtext}
\noindent
\textbf{Artinya}:
\par
\indent
"Aku berlindung dengan perantara kalimat-kalimat Allah yang sempurna dari
murka dan siksa-Nya, serta dari kejahatan hamba-hamba-Nya, dan dari godaan
syaitan-syaitan, juga dari kedatangan mereka kepadaku."{\scriptsize 1}\\
\begin{arabtext}
\noindent
'a`u-w_du bikalimAti al-ll_ahi al-ttAmmAti allati-y lA yu^gA wizuhunna 
barruN walA fA^giruN min ^sarri mA _halaqa, wa_dara'a wabara'a, wamin 
^sarri mA yanzilu mina al-ssamA'i, wamin ^sarri mA ya`ru^gu fi-yhA, wamin 
^sarri mA _dara'a fiy al-'ar.di, wamin ^sarri mA ya_hru^gu minhA, wamin 
^sarri fitani al-llayli wAl-nnahAri, wamin ^sarri kulli .tAriqiN 'illA 
.tAriqaN ya.truqu bi_hayriN yA ra.hm_anu.\\
\end{arabtext}
\noindent
\textbf{Artinya}:
\par
\indent
"Aku berlindung dengan perantara kalimat-kalimat Allah yang sempurna, yang 
tidak akan dapat ditembus oleh orang yang baik maupun yang jahat, dari 
kejahatan apa yang Dia ciptakan, Dia tanamkan dan Dia adakan. Serta dari 
kejahatan yang turun dari langit, dari kejahatan yang naik ke langit, dari 
kejahatan yang ditanamkan ke bumi, dari kejahatan yang keluar dari bumi, 
dari kejahatan fitnah malam dan siang, dan dari kejahatan setiap yang 
datang kecuali apa-apa yang datang dengan membawa kebaikan, wahai Rabb Yang
Maha Pemurah."{\scriptsize 2}\\
\par
\noindent
\textbf{Tingkatan Doa dan Sanad}:
\begin{enumerate}
\item \textbf{Shahih}: HR. Abu Dawud (no. 3893), at-Tirmidzi (no. 3528)
Ibnu Sunni dalam \textit{'Amalul Yaum wal Lailah} (no. 748), dan lainnya.
Lihat \textit{Silsilah Ah\^{a}d\^{i}ts ash-Shah\^{i}hah} (no. 264).
\item \textbf{Shahih}: HR. Ahmad (III/419), Ibnu Sunni dalam 
\textit{'Amalul Yaum wal Lailah} (no. 637) dari Abdurrahman bin Khanbasy 
r.a. Diriwayatkan 
oleh ath-Thabrani dalam \textit{Mu'jamul Ausath} (no. 5411) dari al-Khalid 
bin Walid r.a. Lihat \textit{Silsilah Ah\^{a}d\^{i}ts ash-Shah\^{i}hah} 
(no. 840, 2738, 2995). Sanadnya shahih.\\\\
\end{enumerate}
\par
\noindent 
25------------------------------Dzikir Apabila Membalikkan Tubuh ketika Tidur Malam
\begin{arabtext}
\noindent
lA 'il_aha 'illA al-ll_ahu al-wA.hidu al-qahhAru, rabbu al-ssamAwAti
wAl'ar.di wamA baynahumA al-`aziyzu al-.gaffAru. \\ \\
\end{arabtext}
\noindent
\textbf{Artinya}:
\par
\indent
"Tidak ada ilah yang berhak diibadahi dengan benar kecuali Allah Yang Maha
Esa, Mahaperkasa, Rabb Yang menguasai langit dan bumi dan apa yang berada 
di antara keduanya, Yang Mahamulia lagi Maha Pengampun." \\
\par
\noindent
\textbf{Tingkatan Doa dan Sanad}: Beliau membaca dzikir ini ketika 
membalikkan tubuh dari satu sisi ke sisi lain pada malam hari.
\textbf{Shahih}: HR. Al-Hakim (I/540) - dishahihkan olehnya dan disetujui
adz-Dzahabi - Ibnu Hibban \textit{Maw\^{a}ridizh Zham-\^{a}n} (no. 2358). 
Lihat \textit{Shah\^{i}h Maw\^{a}ridizh Zham-\^{a}n} (no. 2003) dan
\textit{Silsilah Ah\^{a}d\^{i}ts ash-Shah\^{i}hah} (no. 2066).\\\\
\par
\noindent 
26------------------------------Doa Bangun Tidur
\begin{arabtext}
\noindent
al-.hamdu li-ll_ahi a-lla_di-y 'a.hyaanaa ba`damA 'amAtanA wa'i-layhi 
al-nnu^su-wru. \\ 
\end{arabtext}
\noindent
\textbf{Artinya}:
\par
\indent
"Segala puji bagi Allah, yang telah menghidupkan kami setelah kami 
ditidurkan-Nya dan kepada-Nya kami dibangkitkan."\\
\par
\noindent
\textbf{Tingkatan Doa dan Sanad}: \textbf{Shahih}: HR. Al-Bukhari (no. 6312,
 6324) dari Hudzaifah dan Muslim (no. 2711) dari al-Bara.\\\\
\par
\noindent 
27------------------------------Doa Masuk WC
\begin{arabtext}
\noindent
( bismi al-ll_ahi ) al-ll_ahumma 'inni-y 'a`u-w_du bika mina al-_hubu_ti
wAl-_habA'i_ti.\\
\end{arabtext}
\noindent
\textbf{Artinya}:
\par
\indent
"Dengan nama Allah. Ya Allah, sungguh aku berlindung kepada-Mu dari godaan
syaitan yang laki-laki dan syaitan yang perempuan."\\
\par
\noindent
\textbf{Tingkatan Doa dan Sanad}: \textbf{Shahih}: HR. Al-Bukhari (no. 142)
dan Muslim (no. 375), juga at-Tirmidzi (no. 606), Ibnu Majah (297, 298).
Adapun tambahan \textit{Bismill\^{a}h} pada awal hadits, lihat
\textit{Fathul B\^{a}ri} (I/244). Dishahihkan oleh Syaikh al-Albani dalam
\textit{Irw\^{a}-ul Ghal\^{i}l}. \\\\
\par
\noindent 
28------------------------------Doa Keluar WC
\begin{arabtext}
\noindent
.gufrAnaka.\\
\end{arabtext}
\noindent
\textbf{Artinya}:
\par
\indent
"Aku mohon ampunan kepada-Mu."\\
\par
\noindent
\textbf{Tingkatan Doa dan Sanad}: \textbf{Shahih}: HR. Abu Dawud (no. 30),
at-Tirmidzi (no. 7), Ibnu Majah (no. 300), Ahmad (VI/155), al-Hakim (I/158)
dari Aisyah r.a. Dishahihkan oleh al-Hakim dan yang lainnya.\\\\
\par
\noindent 
29------------------------------Doa Sebelum Wudhu
\begin{arabtext}
\noindent
bismi al-ll_ahi.\\
\end{arabtext}
\noindent
\textbf{Artinya}:
\par
\indent
"Dengan nama Allah (aku berwudhu)."\\
\par
\noindent
\textbf{Tingkatan Doa dan Sanad}: \textbf{Shahih}: HR. Abu Dawud (no. 101),
Ibnu Majah (no. 399). Lihat \textit{Irw\^{a}-ul Ghal\^{i}l} (I/122), 
\textit{Shah\^{i}h Sunan Abi Dawud} (no. 90). \\
\par
\noindent 
30------------------------------Doa Setelah Wudhu
\begin{arabtext}
\noindent
'a^shadu 'an lA 'il_aha 'illA al-ll_ahu wa.hdahu lA ^sari-yka lahu 
wa'a^shadu 'anna mu.hammadaN `abduhu warasu-wluhu.\\
\end{arabtext}
\noindent
\textbf{Artinya}:
\par
\indent
"Aku bersaksi bahwa tidak ada ilah yang berhak diibadahi dengan benar 
kecuali hanya Allah, Yang Maha Esa, tiada sekutu bagi-Nya. Dan aku bersaksi
bahwa Muhammad adalah hamba dan Rasul-Nya." {\scriptsize 1}\\
\begin{arabtext}
\noindent
al-ll_ahumma a^g`alniy mina al-ttawwAbi-yna wA^g`alni-y mina 
al-muta.tahhiri-yna.\\
\end{arabtext}
\noindent
\textbf{Artinya}:
\par
\indent
"Ya Allah jadikanlah aku termasuk orang-orang yang bertaubat dan jadikanlah
aku termasuk orang-orang (yang senang) bersuci." {\scriptsize 2}\\
\par
\noindent
\textbf{Tingkatan Doa dan Sanad}:
\begin{enumerate}
\item \textbf{Shahih}: HR. Muslim (I/209-210, no. 234).
\item \textbf{Shahih}: HR. At-Tirmidzi (no. 55). Lihat \textit{Shah\^{i}h 
at-Tirmidzi} (I/8, no. 48). Dishahihkan oleh Syaikh al-Albani.\\\\
\end{enumerate}
\par
\noindent 
31------------------------------Doa Memakai Pakaian
\begin{arabtext}
\noindent
al-.hamdu lill_ahi alla_di-y kasAni-y h_a_dA al-_tawba warazaqani-yhi min 
.ga-yri .hawliN minni-y walA quwwaTiN.\\
\end{arabtext}
\noindent
\textbf{Artinya}:
\par
\indent
"Segala puji bagi Allah yang memberi aku pakaian ini sebagai rizki 
dari-Nya, tanpa daya dan kekuatan dariku."\\
\par
\noindent
\textbf{Tingkatan Doa dan Sanad}: \textbf{Hasan}: HR. Abu Dawud pada Kitab
"al-Lib\^{a}s" (no. 4023), \textit{Shah\^{i}h Abi Dawud} (II/760, no. 3394)
dan selainnya. \\\\
\par
\noindent 
32------------------------------Doa Mengenakan Pakaian Baru
\begin{arabtext}
\noindent
al-ll_ahumma laka al-.hamdu 'anta kasawtani-yhi, 'as'aluka min _ha-yrihi
wa_ha-yri mA .suni`a lahu, wa'a`u-w_du bika min ^sarrihi wa^sarri mA 
.suni`a lahu.\\
\end{arabtext}
\noindent
\textbf{Artinya}:\\
\indent
"Ya Allah, hanya milik-Mu segala puji, Engkaulah yang memberi pakaian ini 
kepadaku. Aku mohon kepada-Mu untuk memperoleh kebaikannya dan kebaikan 
dari tujuan pembuatan pakaian ini. Aku pun berlindung kepada-Mu dari 
keburukannya serta keburukan tujuan dibuatnya pakaian ini."\\
\par
\noindent
\textbf{Tingkatan Doa dan Sanad}: \textbf{Shahih}: HR. Abu Dawud (no. 
4020), at-Tirmidzi (no. 1767), al-Hakim (IV/192), dan al-Baghawi (no. 3111)
dari Abu Sa'id al-Khudri r.a. Lihat \textit{Mukhtasar Syam\^{a}-ilit
Tirmidzi} karya Syaikh al-Albani (hlm. 47-48). \\\\
\par
\noindent 
33------------------------------Doa Kepada Orang Lain yang Mengenakan Pakaian Baru
\begin{arabtext}
\noindent
_tubli-y wayu_hlifu al-llahu ta-`AlY.\\
\end{arabtext}
\noindent
\textbf{Artinya}:
\par
\indent
"Semoga engkau dapat memakainya sampai lusuh, dan semoga Alla
\textit{Ta'ala} menggantinya untukmu dengan yang lebih baik." 
{\scriptsize 1}\\
\begin{arabtext}
\noindent
'ilbas jadi-ydaN, wa`i^s .hami-ydaN, wamut ^sahi-ydaN.\\
\end{arabtext}
\noindent
\textbf{Artinya}:
\par
\indent
"Berpakaianlah yang baru, hiduplah dengan terpuji dan matilah dalam keadaan
syahid." {\scriptsize 2}\\
\par
\noindent
\textbf{Tingkatan Doa dan Sanad}:
\begin{enumerate}
\item \textbf{Shahih}: HR. Abu Dawud (no. 4020) dan lihat pula kitab
\textit{Shah\^{i}h Abi Dawud} (II/760 no. 3393).
\item \textbf{Shahih}: HR. Ibnu Majah (II/1178, no. 3558), Ahmad (II/89), 
al-Baghawi (XII/41, no. 3112), dan \textit{Shah\^{i}h Ibnu Majah} (II/275, 
no. 2863).\\\\
\end{enumerate}
\par
\noindent 
34------------------------------Dzikir Meletakkan Pakaian
\begin{arabtext}
\noindent
bismi al-ll_ahi.\\
\end{arabtext}
\noindent
\textbf{Artinya}:
\par
\indent
"Dengan nama Allah."\\
\par
\noindent
\textbf{Tingkatan Doa dan Sanad}: \textbf{Shahih}: HR. Ath-Thabrani dalam
\textit{Mu'jam al-Ausath} (no. 2525) dari Anas r.a. \textit{Shah\^{i}h 
al-J\^{a}mi'ish Shagh\^{i}r} (no. 3610). \\\\
\par
\noindent 
35------------------------------Doa Keluar Rumah
\begin{arabtext}
\noindent
bismi al-ll_ahi, tawakkaltu `alY al-ll_ahi, lA .hawla walA quwwata 'illA 
bi-al-ll_ahi.\\
\end{arabtext}
\noindent
\textbf{Artinya}:
\par
\indent
"Dengan nama Allah (aku keluar). Aku bertawakal kepada Allah, tidak ada 
daya dan upaya kecuali karena pertolongan Allah semata."{\scriptsize 1}\\
\begin{arabtext}
\noindent
al-ll_ahumma 'innI 'a`uw_dubika 'an 'a.dilla, 'aw 'u.dalla, 'aw 'azilla, 
'aw 'uzalla, 'aw 'a.zlima, 'aw 'u.zlama, 'aw 'a^ghala, 'aw yu^ghala 
`alayya.\\
\end{arabtext}
\noindent
\textbf{Artinya}:
\par
\indent
"Ya Allah, sesungguhnya aku berlindung kepada-Mu, janganlah sampai aku 
sesat atau disesatkan (syaitan atau orang jahat), tergelincir atau 
digelincirkan orang lain, menganiaya atau dianiaya orang lain, dan berbuat 
bodoh atau dibodohi orang lain."{\scriptsize 2}\\
\par
\noindent
\textbf{Tingkatan Doa dan Sanad}:
\begin{enumerate}
\item \textbf{Shahih}: HR. Abu Dawud (no. 5095), at-Tirmidzi (no. 3426), 
dari Anas bin Malik r.a. Lihat \textit{Shah\^{i}h at-Tirmidzi} (III/151, 
no.  2724).
\item \textbf{Shahih}: HR. Abu Dawud (no. 5094, at-Tirmidzi (no. 3427), 
an-Nasai (VII/268), Ibnu Majah (no. 3884) dari Ummu Salamah r.a. Lihat 
kitab \textit{Hid\^{a}yatur Ruw\^{a}t} (III/12, no. 2376). Sanad hadits 
ini shahih.\\\\
\end{enumerate}
\par
\noindent 
36------------------------------Doa Masuk Rumah
\begin{arabtext}
\noindent
bismi al-ll_ahi.\\
\end{arabtext}
\noindent
\textbf{Artinya}:
\par
\indent
"Dengan nama Allah (aku masuk)."\\
\par
\noindent
\textbf{Tingkatan Doa dan Sanad}: \textbf{Shahih}: HR. Muslim (III/1598, 
no. 2018 [103]). \\\\
\par
\noindent 
37------------------------------Doa Pergi ke Masjid
\begin{arabtext}
\noindent
al-ll_ahumma a^g`al fi-y qalbi-y nu-wraN , wafi-y lisAni-y nu-wraN, 
wA^g`al fi-y sam`iy nuwraN wA^g`al fi-y ba.sari-y nu-wraN, wA^g`al min 
_halfi-y nu-wraN, wamin 'amAmi-y nu-wraN wA^g`al min fawqi-y nu-wraN, wamin
ta.hti-y nu-wraN, al-ll_ahumma 'a`.tini-y nu-wraN.\\
\end{arabtext}
\noindent
\textbf{Artinya}:
\par
\indent
"Ya Allah, jadikanlah cahaya pada hatiku, cahaya pada lidahku, cahaya pada 
pendengaranku, cahaya dari belakangku, cahaya dari hadapanku, cahaya dari 
atasku, serta cahaya dari bawahku. Ya Allah, berikanlah padaku cahaya."\\
\par
\noindent
\textbf{Tingkatan Doa dan Sanad}: \textbf{Shahih}: HR. Muslim (no. 763 
[191])-\textit{Syarah Muslim} (V/51)-Lafazh ini miliknya-diriwayatkan oleh 
Imam al-Bukhari (no. 6216). Dan al-Hafizh Ibnu Hajar al-Asqalani 
menyebutkan doa ini dalam \textit{Fathul B\^{a}ri} (XI/116) dengan banyak 
tambahan di dalamnya. Untuk lebih jelasnya, lihat kitab tersebut.\\\\
\par
\noindent 
38------------------------------Doa Masuk Masjid
\begin{arabtext}
\noindent
'a`u-w_du bi-al-ll_ahi al-`a.zi-ymi, wa biwa^ghihi al-kariymi, wasul.tAnihi 
al-qadiymi, mina al-^s^say.tAni al-rra^gi-ymi.\\
\end{arabtext}
\noindent
\textbf{Artinya}:
\par
\indent
"Aku berlindung kepada Allah Yang Mahaagung, dengan wajah-Nya yang mulia 
dan kekuasaan-Nya yang abadi, dari syaitan yang terkutuk."{\scriptsize 1}\\
\begin{arabtext}
\noindent
bismi al-ll_ahi wAl-.s.salATu wAl-ssalAmu `al_aY rasu-wli al-ll_ahi, 
al-ll_ahumma afta.h li-y 'abwAba ra.hmatika.\\
\end{arabtext}
\noindent
\textbf{Artinya}:
\par
\indent
"Dengan nama Allah, semoga shalawat dan salam tercurah kepada Rasulullah.
{\scriptsize 2} Ya Allah, bukakanlah untukku pintu-pintu rahmat-Mu."
{\scriptsize 3}\\
\par
\noindent
\textbf{Tingkatan Doa dan Sanad}:
\begin{enumerate}
\item \textbf{Shahih}: HR. Abu Dawud (no. 466) - \textit{Shah\^{i}h Abi 
Dawud} (I/93), no. 441). Jika dia berucap demikian, syaitan pun berkata: 
"Orang ini terjaga (terlindungi) dari diriku sepanjang hari."
\item \textbf{Shahih}: HR. Abu Dawud (no. 465), lihat \textit{Shah\^{i}h 
al-J\^{a}mi'ish Shagh\^{i}r} (no. 514); Ibnus Sunni dalam \textit{'Amalul 
Yaum wal Lailah} (no. 88). Dinyatakan hasan oleh Syaikh al-Albani dalam 
\textit{al-Kalimuth Thayyib} (hlm. 92, no. 64, catatan kaki no. 52).
\item \textbf{Shahih}: HR. Muslim (no. 713).\\\\
\end{enumerate}
\par
\noindent 
39------------------------------Doa Keluar Masjid
\begin{arabtext}
\noindent
bismi al-ll_ahi waal-.s.salATu wAl-ssalAmu `al_aY rasu-wli  al-ll_ahi, 
Aal-ll_ahumma 'inniy 'as'aluka min fa.dlika, Aal-ll_ahumma a`.simniy mina 
al-^s^say.taani al-rra^giymi.\\
\end{arabtext}
\noindent
\textbf{Artinya}:
\par
\indent
"Dengan nama Allah, semoga shalawat dan salam selalu terlimpahkan  
Rasulullah. Ya Allah, sesungguhnya aku memohon kepada-Mu karunia-Mu, Ya 
Allah, lindungilah aku dari godaan syaitan yang terkutuk."\\
\par
\noindent
\textbf{Tingkatan Doa dan Sanad}: \textbf{Shahih}: HR. Muslim (no. 713), 
dan Ibnus Sunni dalam \textit{'Amalul Yaum wal Lailah} (no. 88). Adapun 
tambahan: (Allahumma….) adalah dari Ibnu Majah (no. 773). 
\textit{Shah\^{i}h Ibnu Majah} (no. 627).\\\\
\par
\noindent 
40------------------------------Doa ketika Mendengar Adzan
\indent
Terdapat lima hal yang disunnahkan ketika adzan dikumandangkan:
\begin{enumerate}
\item Menjawab adzan seperti apa yang diucapkan muadzin, kecuali pada 
lafazh: \begin{arabtext} (.hayya `alaY al-.s.salATi) \end{arabtext} dan 
lafazh: \begin{arabtext} (.hayya `alaY al-falA.hi) \end{arabtext}, maka 
kita mengucapkan:
\begin{arabtext}
\noindent
lA .hawla walA quwwaTa 'illA bi-al-ll_ahi.\\
\end{arabtext}
\noindent
\textbf{Artinya}:
\par
\indent
"Tidak ada daya dan kekuatan kecuali dengan pertolongan Allah."
{\scriptsize 1}
\item Setelah muadzin selesai adzan, maka kita mengucapkan:{\scriptsize 2}
\begin{arabtext}
\noindent
(wa'anA) 'a^shadu 'an lA 'i-l_aha 'illA al-ll_ahu wa.hdahu lA^sari-yka lahu 
wa ('a^shadu) 'anna mu.hammadaN `abduhu warasu-wluhu, ra.di-ytu 
bi-al-ll_ahi rabbaN, wabimu.hammadiN rasu-wlaN wabi-al-'i-slA-mi di-ynaN.\\
\end{arabtext}
\noindent
\textbf{Artinya}:
\par
\indent
"Dan aku bersaksi bahwa tidqak ada ilah yang berhak diibadahi dengan benar 
melainkan Allah Yang Esa, tidak ada sekutu bagi-Nya, dan aku pun bersaksi 
bahwa Muhammad adalah hamba-Nya dan Rasul-Nya, aku ridha Allah sebagai 
Rabb, Muhammad sebagai Rasul dan Islam sebagai agama(ku)."{\scriptsize 3}
\item Membaca shalawat kepada Nabi Muhammad \textit{Shallallahu ‘alaihi wa 
sallam}.{\scriptsize 4}
\item Membaca doa setelah adzan:
\begin{arabtext}
\noindent
al-ll_ahumma rabba h_a_dihi al-dda`waTi al-ttAmmaTi, wAl-.s.salATi 
al-qA'imaTi, ^Ati mu.hammadaN ni al-wasi-ylaTa wAl-fa.di-ylaTa, wAb`a_thu 
maqAmaN ma.hmu-wdA-ni alla_di-y wa`adtahu.\\
\end{arabtext}
\noindent
\textbf{Artinya}:
\par
\indent
"Ya Allah, Rabb Pemilik panggilan yang sempurna (adzan) ini dan shalat 
(wajib) yang didirikan. Berikanlah \textit{al-wasilah} (derajat di Surga), 
dan keutamaan kepada Muhammad \textit{Shallallahu ‘alaihi wa sallam}. Dan 
bangkitkanlah beliau sehingga bisa menempati maqam terpuji yang telah 
Engkau janjikan."{\scriptsize 5}
\item Berdoa untuk diri sendiri dengan doa yang dikehendaki antara adzan 
dan iqamah, sebab doa pada saat itu dikabulkan oleh Allah.
\begin{arabtext}
\noindent
al-ddu`A'u lA yuraddu bayna al-'a _dAni wAl-'iqAmaTi.\\
\end{arabtext}
\noindent
\textbf{Artinya}:
\par
\indent
"Tidak ditolah doa antara adzan dan iqamah."{\scriptsize 6}\\
\end{enumerate}
\par
\noindent
\textbf{Tingkatan Doa dan Sanad} :
\begin{enumerate}
\item "Barang siapa menjawab adzan dengan ikhlas dari hatinya, ia akan 
masuk Surga. "Lihat \textit{Syarah Muslim} (IV/85-86 no. 385). Dan apabila 
seorang muadzin mengucapkan: \begin{arabtext} (al-.s.salATu _ha-yruN mina 
al-nna-wmi) \end{arabtext}, maka hendaklah dijawab seperti itu juga.
\item Ada yang berpendapat bahwa dzikir ini dibaca setelah muadzin membaca 
syahadat. Lihat kitab \textit{ats-Tsamar al-Mustath\^{a}b f\^{i} Fiqhis 
Sunnah wal Kit\^{a}b} (hlm. I/172-185) karya Syaikh al-Albani, 
\textit{Maus\^{u}'ah al-Fiqhiyyah al-Muyassarah f\^{i} Fiqhil Kit\^{a}b was
Sunnah al-Muthahhara}h (hlm. 371) karya Husain al-Audah al-Awayisyah, 
\textit{Shah\^{i}h al-W\^{a}bilish Shayyib} (hlm. 184), dan 
\textit{Tash-h\^{i}hud Du'\^{a}'} (hlm. 370-372).
\item \textbf{Shahih}: HR. Muslim (no. 386), at-Tirmidzi (no. 210), Abu 
Dawud (no. 525), an-Nasai (II/26), Ibnu Majah (no. 721), Ahmad (I/181), 
Ibnu Khuzaimah (no. 421) dan yang lainya dari Sa'ad bin Abi Waqqash r.a.
\item \textbf{Shahih}: HR. Muslim (no. 384), an-Nasai (II/25-26), Abu Dawud
(no. 523), Ibnu Khuzaimah (no. 418), Ahmad (II/168), Al-Baihaqi (I/409-410)
dari Abdullah bin Amr bin al-Ash r.a.
\item \textbf{Shahih}: HR. Al-Bukhari (no. 614)--Lihat 
\textit{Fathul B\^{a}ri} (II/94)--Abu Dawud (no. 529), at-Tirmidzi (no. 
211), an-Nasai (II/26-27), Ibnu Majah (no. 722). Adapun tambahan 
\begin{arabtext} ('innaka lA tu_hlifu al-mi-y`Adu) \end{arabtext} adalah 
\textbf{lemah}, jadi ia tidak boleh diamalkan. Lihat kitab 
\textit{Irw\^{a}-ul Ghal\^{i}l} (I/260, 261). Tidak ada juga tambahan: 
\begin{arabtext} (wAl-ddara^gaTa al-rrafi-y`aTa) (yA 'ar.hama 
al-rra.himi-yna) \end{arabtext} karena tidak ada asalnya.
\item \textbf{Shahih}: HR. Abu Dawud (no. 521), at-Tirmidzi (no. 212, 
3595), Ahmad (III/119, 155, 225), an-Nasai dalam \textit{'Amalul Yaum wal 
Lailah} (no. 67, 68, 69), Ibnu Khuzaimah (no. 425-427). Lihat penjelasan 
Ibnu Qayyim tentang lima hal ini dalam \textit{Shah\^{i}h al-Wabilish 
Shayyib} (hlm. 182-185), \textit{Z\^{a}dul Ma'\^{a}d} (II/391-392).\\\\
\end{enumerate}
\par
\noindent 
41------------------------------Doa Istiftah
\begin{arabtext}
\noindent
sub.hAnaka al-ll_ahumma wabi.hamdika, watabAraka asmuka, wata`Al_aY 
^gadduka, walA 'il_aha .gayruka.\\
\end{arabtext}
\noindent
\textbf{Artinya}:
\par
\indent
"Mahasuci Engkau ya Allah, aku memuji-Mu, Mahaberkah Nama-Mu. Mahatinggi 
kekayaan dan kebesaran-Mu, tidak ada ilah yang berhak diibadahi dengan 
benar selain Engkau."{\scriptsize 1}\\\\
Atau membaca:\\
\begin{arabtext}
\noindent
al-ll_ahumma bA`id bayni-y wabayna _ha.tAyAya kamA bA`adta bayna 
al-ma^sriqi wAl-ma.gribi, al-ll_ahumma naqqini-y min _ha.tAyAya, kamA 
yunaqqY al-_t_tawbu al-'abya.du mina al-ddanasi, al-ll_ahumma a.gsilni-y 
min _ha.tAyAya bi-al-_t_tal^gi wAl-mA'i wAl-baradi.\\
\end{arabtext}
\noindent
\textbf{Artinya}:
\par
\indent
"Ya Allah, jauhkanlah antara aku dan kesalahan-kesalahanku, sebagaimana 
Engkau menjauhkan antara timur dan barat. Ya Allah, bersihkanlah aku dari 
kesalahan-kesalahanku, sebagaimana  baju putih dibersihkan dari kotoran. Ya
Allah, cucilah aku dari kesalahan-kesalahanku dengan salju, air, dan air 
es."{\scriptsize 2}\\\\
Atau membaca:\\
\begin{arabtext}
\noindent
wa^g^gahtu wa^ghiya lilla_di-y fa.tara al-ssamAwAti wAl-'ar.da .hani-yfaN 
wamA 'anA mina al-mu^sriki-yna, 'inna .salAti-y, wanusuki-y, wama.hyAya, 
wamamAti-y li-ll_ahi rabbi al-`Alami-yna, lA^sari-yka lahu wabi_d_alika 
'umirtu wa'anA mina al-muslimi-yna. al-ll_ahumma 'anta almaliku lA 'il_aha 
'illA 'anta. 'anta rabbi-y wa'anA `abduka, .zalamtu nafsi-y wA`taraftu 
bi_danbi-y fA.gfirli-y _dunu-wbi-y ^gami-y`aN 'innahu lA ya.gfiru 
al-_d_dunu-wba 'illA 'anta. wAhdini-y li-'a.hsani al-'a_hlAqi lA yahdi-y 
li-'a.hsanihA 'illA 'anta, wA.srif `anni-y sayyi'ahA, lA ya.srifu `anni-y 
sayyi'ahA 'illA 'anta, labbayka wasa`dayka, wAl-_hayru kulluhu fi-y 
yadayka, wAl-^s^sarru laysa 'ilayka, 'anAbika wa-'ilayka, tabArakta 
wata`Alayta, 'asta.gfiruka wa'atu-wbu 'ilayka.\\
\end{arabtext}
\noindent
\textbf{Artinya}:
\par
\indent
"Aku menghadapkan wajahku kepada Rabb Pencipta langit dan bumi, dalam 
keadaan lurus dan aku tidak termasuk orang-orang yang musyrik. Sesungguhnya
shalatku, ibadahku, hidupku serta matiku adalah untuk Allah. Rabb alam 
semesta, tidak ada sekutu bagi-Nya. Demikianlah aku diperintah dan bahwa 
aku termasuk orang Muslim. Ya Allah, Engkau adalah Raja, tidak ada ilah 
yang berhak diibadahi dengan benar kecuali Engkau, Engkau Rabbku sedangkan 
aku ini adalah hamba-Mu. Aku menganiaya diriku, aku mengakui dosa-dosaku 
(yang pernah aku lakukan). Oleh karena itu, ampunilah seluruh dosaku, 
sesungguhnya tidak ada yang dapat mengampuni dosa-dosa, kecuali Engkau. 
Tunjukkan aku pada akhlak yang baik (mulia), tidak ada yang dapat 
menunjukkan kepada akhlak yang mulia kecuali Engkau. Hindarkan aku dari 
akhlak yang buruk, tidak ada yang dapat menjauhkanku darinya kecuali 
Engkau. Aku penuhi panggilan-Mu, aku mohon pertolongan-Mu, seluruh kebaikan
berada di kedua tangan-Mu, kejelekan tidak dinisbatkan kepada-Mu. Aku hidup
dengan pertolongan dan rahmat-Mu, dan kepada-Mu (aku kembali). Mahasuci 
Engkau dan Mahatinggi. Aku memohon ampunan dan bertaubat kepada-Mu."
{\scriptsize 3}\\\\
Atau membaca:\\
\begin{arabtext}
\noindent
al-ll_ahumma rabba ^gibrA'iyla, wami-ykA'iyla, wa-'isrAfi-yla fA.tira 
al-ssamAwAti wAl-'ar.di, `Alima al-.gaybi wAl-^s^sahAdaTi, 'anta ta.hkumu 
bayna `ibAdika fi-ymA kAnuW fi-yhi ya_htalifu-wna. ihdini-y limA a_htulifa 
fi-yhi mina al-.haqqi bi-'i_dnika 'innaka tahdi-y man ta^sA'u 'ilY 
.sirA.tiN mustaqi-ymiN.\\
\end{arabtext}
\noindent
\textbf{Artinya}:
\par
\indent
"Ya Allah, Rabb Malaikat Jibril, Mika-il dan Israfil. Pencipta seluruh 
langit dan bumi. Yang Maha Mengetahui semua yang ghaib dan yang nyata. 
Engkau yang memutuskan hukum di antara hamba-hamba-Mu tentang apa-apa yang 
mereka perselisihkan. Dengan izin-Mu tunjukkanlah aku kepada kebenaran 
(yaitu, tetapkan aku di atas kebenaran) dari apa yang mereka perselisihkan.
Sesungguhnya Engkau memberi petunjuk kepada siapa yang Engkau kehendaki ke 
jalan yang lurus."{\scriptsize 4}\\\\
Atau membaca:\\
\begin{arabtext}
\noindent
al-ll_ahumma laka al-.hamdu, 'anta qayyimu al-ssamAwAti wAl-'ar.di waman 
fi-yhinna, walaka al-.hamdu laka mulku al-ssamAwAti wAl-'ar.di waman 
fi-yhinna, walaka al-.hamdu, 'anta nu-wru al-ssamAwAti wAl-'ar.di, walaka 
al-.hamdu, 'anta maliku al-ssamAwAti wAl-'ar.di, walaka al-.hamdu 'anta 
al-.haqqu, wawa`duka al-.haqqu, waliqA'uka .haqquN, waqawluka .haqquN, 
wAl-^gannaTu .haqquN, wAl-nnAru .haqquN, wAl-nnabiyyu-wna .haqquN, 
wamu.hammaduN .sallY al-ll_ahu `ala-yhi wasallama .haqquN, wAl-ssA`aTu 
.haqquN, al-ll_ahumma laka 'aslamtu, wabika ^Amantu, wa`alayka tawakkaltu, 
wa-'ilayka 'anabtu, wabika _hA.samtu, wa-'ilayka .hAkamtu, fA.gfirli-y mA 
qaddamtu wamA-'a_h_hartu, wamA-'asrartu, wamA-'a`lantu, 'anta al-muqaddimu,
wa-'anta al-mu'a_h_hiru, lA 'il_aha 'illA 'anta.\\
\end{arabtext}
\noindent
\textbf{Artinya}:
\par
\indent
"Ya Allah, bagi-Mu segala puji, Engkaulah Pemelihara seluruh langit dan 
bumi, serta segenap makhluk yang ada padanya. Bagi-Mu segala puji, bagi-Mu 
kerajaan langit dan bumi, serta segenap makhluk yang ada padanya. Bagi-Mu 
segala puji, Engkau adalah cahaya langit dan bumi. Bagi-Mu segala puji, 
Engkaulah penguasa langit dan bumi. Bagi-Mu segala puji, Engkaulah Yang 
Mahabenar, janji-Mu benar, pertemuan dengan-Mu adalah benar, firman-Mu 
benar, adanya Surga itu benar, adanya Neraka adalah benar, diutusnya para 
Nabi \textit{'alaihimussalatu wassalam} adalah benar, Nabi Muhammad 
shallallahu ‘alaihi wa sallam adalah benar, dan adanya Kiamat adalah benar. 
Ya Allah, hanya kepada-Mu aku berserah, hanya pada-Mu aku beriman, hanya 
kepada-Mu aku bertawakal, hanya pada-Mulah aku bertaubat, hanya dengan 
(pertolongan)-Mu aku berdebat dan hanya kepada-Mu aku berhukum (mohon 
keputusan). Oleh karena itu, ampunilah dosa-dosaku yang telah lalu dan yang 
akan datang, yang aku lakukan secara sembunyi-sembunyi atau 
terang-terangan. Engkaulah Yang mendahulukan dan mengakhirkan. Tidak ada 
ilah yang berhak diibadahi dengan benar kecuali Engkau."{\scriptsize 5}\\
- Setelah membaca doa istiftah, membaca ta'awwudz:\\
\begin{arabtext}
\noindent
'a`u-w_du bi-al-ll_ahi al-ssami-y`i al-`ali-ymi mina al-^s^sa-y.tAni 
al-rra^gi-ymi min hamzihi wanaf_hihi wanaf_tihi.\\
\end{arabtext}
\noindent
\textbf{Artinya}:
\par
\indent
"Aku berlindung kepada Allah Yang Maha Mendengar lagi Maha Mengetahui dari 
gangguan syaitan yang terkutuk, dari kegilaannya, kesombongannya, dan 
syairnya yang tercela."{\scriptsize 6}\\
- Membaca surah Al-Fatihah.\\
- Mengucapkan "Aamiin" setelah \begin{arabtext} (walA al-.d.da-^Ali-yna)
\end{arabtext}
- Dalam shalat berjamaah, makmum tidak boleh mendahului imam.\\
- Membaca surah sesuai dengan apa yang dicontohkan oleh Rasulullah
shallallahu 'alaihi wa sallam.{\scriptsize 7}\\
\par
\noindent
\textbf{Tingkatan Doa dan Sanad}:
\begin{enumerate}
\item \textbf{Shahih}: HR. Muslim (no. 399 [52]) dan ad-Daraquthni 
(I/628-629, no. 1127, 1132) dari Umar bin al-Khathab r.a. secara 
\textit{mauquf}. Diriwayatkan juga oleh ad-Daraquthni (no. 1133) dan 
ath-Thabrani dalam \textit{ad-Du'\^{a}'} (no. 506), dari Anas bin Malik 
r.a. secara marfu'. Sanadnya shahih. Lihat kitab \textit{Silsilah 
ash-Shah\^{i}hah} (no. 2996), \textit{Ashlu Shifatu Shal\^{a}tin Nabi} 
(I/254), serta kitab \textit{Irw\^{a}-ul Ghal\^{i}l} (II/51-53).
\item \textbf{Shahih}: HR. Al-Bukhari (no. 744) Muslim (no. 598 [147]), 
Ibnu Majah (no. 805), an-Nasai (II/129), dan Abu Dawud (no. 781).
\item \textbf{Shahih}: HR. Muslim (no. 771 [201]), Abu Dawud (no. 760), 
an-Nasai (II/130), Ahmad (I/94-95, 102), dan selainnya. Doa ini dibaca saat
shalat wajib dan saat shalat sunnah (\textit{Shifatu Shal\^{a}tin Nabi 
karya Syaikh al-Albani}).                                                  
\item \textbf{Shahih}: HR. Muslim (no. 770 [200]), Abu Dawud (no. 767), 
Ibnu Majah (no. 1357). Nabi membaca doa istiftah ini ketika shalat malam.
\item \textbf{Shahih}: HR. Al-Bukhari (no. 1120, 6317, 7385, 7442, 7499). 
Muslim juga meriwayatkannya dengan ringkas (no. 769 [199]) dari Ibnu Abbas 
r.a. Doa istiftah ini dibaca ketika shalat malam (Tahajud).
\item \textbf{Shahih}: HR. Abu Dawud (no. 775) dan at-Tirmidzi (no. 242). 
Dengan dasar firman Allah dalam surah Fushshilat ayat 36, lihat 
\textit{al-Kalimuth Tahyib} (no. 130), shahih. \textit{Shifatu Shal\^{a}tin
Nabi} (hlm. 95-96) dan \textit{Irw\^{a}-ul Ghal\^{i}l} (II/53-57, no. 342).
\item Lihat kitab \textit{Shifatu Shal\^{a}tin Nabi} karya Syaikh Muhammad 
Nashiruddin al-Albani.\\\\
\end{enumerate}
\par
\noindent 
42------------------------------Doa Ruku'
\begin{arabtext}
\noindent
sub.hAna rabbiya al-`a.zi-ymi.\\
\end{arabtext}
\noindent
\textbf{Artinya}:
\par
\indent
"Mahasuci Rabbku Yang Mahaagung." [Dibaca 3x].{\scriptsize 1}\\
\par
\indent
Atau membaca:
\begin{arabtext}
\noindent
sub.hAnaka al-ll_ahumma rabbanA wabi.hamdika Aal-ll_ahumma a.gfirli-y.\\
\end{arabtext}
\noindent
\textbf{Artinya}:
\par
\indent
"Mahasuci Engkau, ya Allah! Rabb kami, dan dengan memuji-Mu. Ya Allah, 
ampunilah dosaku."{\scriptsize 2}\\
\par
\indent
Atau membaca:
\begin{arabtext}
\noindent
sub.hAna _diy al-^gabaru-wti wAl-malaku-wti wAl-kibriyA'i wAl-`a.zamaTi.\\
\end{arabtext}
\noindent
\textbf{Artinya}:
\par
\indent
"Mahasuci Dia (Allah) Yang memiliki keperkasaan, kerajaan, kebesaran dan 
keagungan."{\scriptsize 3}\\
\par
\indent
Atau membaca:
\begin{arabtext}
\noindent
subbu-w.huN quddu-wsuN, rabbu al-malA'ikaTi wAl-rru-w.hi.\\
\end{arabtext}
\noindent
\textbf{Artinya}:
\par
\indent
"Engkau Rabb Yang Mahasuci (dari kekurangan dan yang tidak layak bagi 
kebesaran-Mu), Rabb seluruh Malaikat dan Jibril."{\scriptsize 4}\\
\par
\noindent
\textbf{Tingkatan Doa dan Sanad}: 
\begin{enumerate}
\item \textbf{Shahih}: HR. Ahmad (V/382, 394), Abu Dawud (no. 871), 
an-Nasai (II/190), at-Tirmidzi (no. 262) dan Ibnu Majah (no. 888), lafazh 
ini miliknya. Lihat \textit{Irw\^{a}-ul Ghal\^{i}l} (no. 333 dan 334).
\item \textbf{Shahih}: HR. Al-Bukhari (no. 794, 817) dan Muslim (no. 484).
\item \textbf{Shahih}: HR. Abu Dawud (no. 873), an-Nasai (II/191), dan 
sanadnya shahih.
\item \textbf{Shahih}: HR. Muslim (no. 487), Abu Dawud (no. 872), an-Nasai 
(II/191), dan Ahmad (VI/35).\\\\
\end{enumerate}
\par
\noindent 
43------------------------------Doa Bangkit dari Ruku'
\begin{arabtext}
\noindent
sami`a al-ll_ahu liman .hamidahu.\\
\end{arabtext}
\noindent
\textbf{Artinya}:
\par
\indent
"Allah Maha Mendengar pujian orang yang memuji-Nya."\\
\begin{arabtext}
\noindent
rabbanA walaka al-.hamdu, .hamdaN ka_ti-yraN .tayyibaN mubArakaN fi-yhi.\\
\end{arabtext}
\noindent
\textbf{Artinya}:
\par
\indent
"Wahai Rabb kami, bagi-Mu segala puji, aku memuji-Mu dengan pujian yang 
banyak, yang baik dan penuh dengan berkah."\\
\par
\indent
Atau membaca:\\
\begin{arabtext}
\noindent
rabbanA laka al-.hamdu mil'a al-ssamAwAti wamil'a al-'ar.di wamil'a mA 
^si'ta min ^say'iN ba`du. 'ahla al-_t_tanA'i wAl-ma^gdi, 'a.haqqu mAqAla 
al-`abdu, wakullunA laka `abduN. al-ll_ahumma lA mAni`a limA 'a`.tayta, 
walA mu`.tiya limA mana`ta, walA yanfa`u _dA al-^gaddi minka al-^gaddu.\\
\end{arabtext}
\noindent
\textbf{Artinya}:
\par
\indent
"Wahai Rabb kami, bagi-Mu segala pujian (kami memuji-Mu dengan) pujian 
sepenuh langit dan sepenuh bumi, sepenuh apa yang Engkau kehendaki setelah 
itu. Wahai Rabb yang layak dipuji dan diagungkan, Yang paling benar 
dikatakan oleh seorang hamba dan kami seluruhnya adalah hamba-Mu. Ya Allah,
tidak ada yang akan dapat menghalangi apa yang Engkau berikan dan tidak ada
yang dapat memberi apa yang Engkau halangi, tidak bermanfaat kekayaan dan 
kemuliaan bagi pemilik keduanya dari adzab-Mu."\\
\par
\noindent
\textbf{Tingkatan Doa dan Sanad}:
\begin{enumerate}
\item \textbf{Shahih}: HR. Al-Bukhari (no. 795)/\textit{Fathul B\^{a}ri} 
(II/282).
\item \textbf{Shahih}: HR. Al-Bukhari (no. 799)/\textit{Fathul B\^{a}ri} 
(II/284).
\item \textbf{Shahih}: HR. Muslim (no. 477 [205]), Abu Awanah (II/176), 
Abu Dawud (no. 847) dari Abu Sa'id al-Khudri r.a.\\\\
\end{enumerate}
\par
\noindent 
44------------------------------Doa Sujud
\begin{arabtext}
\noindent
sub.hAna rabbiya al-'a `l_aY.\\
\end{arabtext}
\noindent
\textbf{Artinya}:
\par
\indent
"Mahasuci Rabbku, Yang Mahatinggi (dari segala kekurangan dan hal yang 
tidak layak)." [Dibaca 3x]{\scriptsize 1}\\
\indent Atau membaca:
\begin{arabtext}
\noindent
sub.hA naka al-ll_ahumma rabbanA wa-bi.hamdika Aal-ll_ahumma a.gfirli-y.\\
\end{arabtext}
\noindent
\textbf{Artinya}:
\par
\indent
"Mahasuci Engkau, ya Allah. Rabb kami, dan dengan memuji-Mu. Ya Allah, 
ampunilah dosaku."{\scriptsize 2}\\
\indent Atau membaca:
\begin{arabtext}
\noindent
subbu-w.huN quddu-wsuN, rabbu al-malA'ikaTi wAl-rru-w.hi.\\
\end{arabtext}
\noindent
\textbf{Artinya}:
\par
\indent
"Engkau Rabb Yang Mahasuci (dari kekurangan dan hal yang tidak layak bagi 
kebesaran-Mu) Rabb para Malaikat dan Jibril."{\scriptsize 3}\\
\indent Atau membaca:
\begin{arabtext}
\noindent
sub.hAna _diy al-^gabaru-wti wAl-malaku-wti wAl-kibriyA'i wAl-`a.zamaTi.\\
\end{arabtext}
\noindent
\textbf{Artinya}:
\par
\indent
"Mahasuci (Allah), Rabb Yang memiliki keperkasaan, kerajaan, kebesaran, 
dan keagungan."{\scriptsize 4}\\
\par
\noindent
\textbf{Tingkatan Doa dan Sanad}:
\begin{enumerate}
\item \textbf{Shahih}: HR. Ahmad (V/382, 394), Abu Dawud (no. 871), 
an-Nasai (II/190), at-Tirmidzi (no. 262), Ibnu Majah (no. 888). Lihat 
\textit{Irw\^{a}-ul Ghal\^{i}l} (no. 333, 334).
\item \textbf{Shahih}: HR. Al-Bukhari (no. 794, 817) dan Muslim (no. 484).
\item \textbf{Shahih}: HR. Muslim (no. 487)-\textit{Syarah Muslim} 
(IV/204-205).
\item \textbf{Shahih}: HR. Abu Dawud (no. 873), an-Nasai, dan Ahmad. 
Dishahihkan oleh Syaikh al-Albani dalam kitab \textit{Shah\^{i}h Abi 
Dawud} (I/166).\\\\
\end{enumerate}
\par
\noindent 
45------------------------------Doa Duduk Antara Dua Sujud
\begin{arabtext}
\noindent
rabbi a.gfirli-y, rabbi a.gfirli-y.\\
\end{arabtext}
\noindent
\textbf{Artinya}:
\par
\indent
"Wahai Rabbku, ampunilah dosaku, wahai Rabbku, ampunilah dosaku."
{\scriptsize 1}\\
\par
\indent
Atau membaca:
\begin{arabtext}
\noindent
al-ll_ahumma a.gfirli-y wAr.hamni-y wA^gburni-y wArfa`ni-y wAhdini-y 
wa`Afini-y wArzuqni-y.\\
\end{arabtext}
\noindent
\textbf{Artinya}:
\par
\indent
"Ya Allah, ampunilah dan sayangilah aku, cukupilah kekuranganku, angkatlah 
derajatku, berilah petunjuk kepadaku, selamatkanlah aku, dan berikanlah aku
rizki (yang halal)."{\scriptsize 2}\\
\par
\noindent
\textbf{Tingkatan Doa dan Sanad}: 
\begin{enumerate}
\item \textbf{Shahih}: HR. Abu Dawud (no. 874). Lihat \textit{Shah\^{i}h 
Ibni Majah} (no. 731).
\item \textbf{Shahih}: HR. At-Tirmidzi (no. 284), Abu Dawud (no. 850), Ibnu
Majah (no. 898). Lihat \textit{Shah\^{i}h Tirmidzi} (I/90, no. 233), 
\textit{Shah\^{i}h Abi Dawud} (I/60, no. 756), dan \textit{Shah\^{i}h Ibni 
Majah} (I/148, no. 732) dengan lafazh \textit{"rabbi"}, \textit{Shifatu 
Shal\^{a}tin Nabi} karya Syaikh al-Albani.\\\\
\end{enumerate}
\par
\noindent 
46------------------------------Doa Sujud Tilawah
\begin{arabtext}
\noindent
sa^gada wa^ghiya lilla_di-y _halaqahu wa^saqqa sam`ahu waba.sarahu, 
bi.hawlihi waquwwatihi (fatabAraka al-llahu 'a.hsanu al-_h_aliqiyna).\\
\end{arabtext}
\noindent
\textbf{Artinya}:
\par
\indent
"Wajahku bersujud kepada Rabb yang menciptakannya, yang telah membelah 
pendengarannya dan penglihatannya dengan daya dan kekuatan-Nya, maka 
Mahasuci Allah. Sebaik-baik Pencipta."{\scriptsize 1}\\
\begin{arabtext}
\noindent
al-ll_ahumma aktub li-y bihA `indaka 'a^graN, wa.da` `anni-y bihA wizraN, 
wA^g`alhA li-y `indaka _du_hraN, wataqabbalhA minni-y kamA taqabbaltahA min 
`abdika dAwuda.\\
\end{arabtext}
\noindent
\textbf{Artinya}:
\par
\indent
"Ya Allah, tulislah untukku dengan sujudku pahala di sisi-Mu dan ampuni 
dosaku dengannya, serta jadikanlah ia simpanan untukku di sisi-Mu, dan juga
terimalah sujudku sebagaimana Engkau menerimanya dari hamba-Mu, Dawud."
{\scriptsize 2}\\
\par
\noindent
\textbf{Tingkatan Doa dan Sanad}:
\begin{enumerate}
\item Nabi SAW. mengucapkan dalam sujud al-Qur'an (sujud tilawah) pada 
waktu malam, yakni beliau mengucapkan (berkali-kali): "\textit{Sajada
wajh\^{i}...}".\\
\textbf{Shahih}: HR. Abu Dawud (no. 1414), At-Tirmidzi (no. 580), An-Nasai 
(II/222), Ahmad (VI/30-31), dan al-Hakim (I/220) dari Aisyah r.a. Hadits 
ini dishahihkan oleh Imam At-Tirmidzi, al-Hakim, an-Nawawi, adz-Dzahabi, 
Syaikh al-Albani, dan dihasankan oleh al-Hafizh Ibnu Hajar al-Asqalani 
dalam \textit{Nat\^{a}-ijul Afk\^{a}r} (II/116-118). Lihat 
\textit{Shah\^{i}h at-Tirmidzi} (I/80, no. 474), \textit{Shah\^{i}h Sunan 
Abi Dawud} (V/157-158, no. 1273), dan \textit{Shah\^{i}h al-Adzk\^{a}r} 
(no. 150/122). Adapun tambahan di dalam kurung \begin{arabtext}(fatabAraka 
al-ll_ahu 'a.hsanu al_haliqiyna)\end{arabtext} "Mahasuci Allah: sebaik-baik
Pencipta." Diriwayatkan oleh al-Hakim (I/220). Tambahan ini dishahihkan oleh
Al-Hakim, juga oleh adz-Dzahabi dan an-Nawawi.\\
\item \textbf{Hasan}: HR. At-Tirmidi (no. 579 dan no. 3424), 
\textit{Shah\^{i}h at-Tirmidzi} (I/180 no. 473), dan al-Hakim (I/220). 
At-Tirmidzi mengatakan hasan. Menurut al-Hakim, hadits tersebut shahih. Dan
adz-Dzahabi sependapat sengannya.\\\\
\end{enumerate}
\par
\noindent 
47------------------------------Doa Tasyahud
\begin{arabtext}
\noindent
al-tta.hiyyAtu al-mubArakAtu al-.s.salawAtu al-.t.tayyibAtu li-ll_ahi, 
al-ssalAmu `alayka 'ayyuhA alnnabiyyu wara.hmaTu al-ll_ahi wabarakAtuhu, 
al-ssalAmu `alaynA wa`alY `ibAdi al-ll_ahi al-.s.sA li.hi-yna, 'a^shadu 
'an lA 'il_aha 'illA al-ll_ahu, wa'a^shadu 'anna mu.hammadaN rasuwlu 
al-ll_ahi.\\
\end{arabtext}
\noindent
\textbf{Artinya}:
\par
\indent
"Semua kesejahteraan, kerajaan, dan kekekalan segala yang diberkahi; semua 
doa untuk mengagungkan Allah; dan seluruh perkataan yang baik dan amal 
shalih hanyalah milik Allah. Semoga kesejahteraan, rahmat, dan karunia 
Allah tercurah untukmu wahai Nabi wahai Nabi. Semoga kesejahteraan 
diberikan kepada kami dan hamba-hamba Allah yang shalih. Aku bersaksi bahwa
tidak ada ilah yang berhak diibadahi dengan benar selain Allah dan aku 
bersaksi bahwasanya Muhammad adalah utusan Allah."{\scriptsize 1}\\
\par
\indent
Atau membaca:
\begin{arabtext}
\noindent
al-tta.hiyyAtu li-ll_ahi, wAl-.s.salawAtu wAl-.t.tayyibAtu, al-ssalAmu 
`alayka 'ayyuhA alnnabiyyu wara.hmaTu al-ll_ahu wabarakAtuhu, al-ssalAmu 
`alaynA wa`alY `ibAdi al-ll_ahi al-.s.sAli.hiyna, 'a^shadu 'an lA 'il_aha 
'illA al-ll_ahu, wa'a^shadu 'anna mu.hammadaN `abduhu warasu-wluhu.\\
\end{arabtext}
\noindent
\textbf{Artinya}:
\par
\indent
"Semua kesejahteraan, kerajaan, dan kekekalan; semua doa untuk mengagungkan
Allah; dan seluruh perkataan yang baik dan amal shalih hanyalah milik Allah
tercurah kepadamu Nabi. Semoga keselamatan dicurahkan kepada kami semua dan
hamba-hamba Allah yang shalih. Aku bersaksi bahwa tidak ada ilah yang 
berhak diibadahi dengan benar selain Allah semata, tidak ada sekutu 
bagi-Nya dan aku bersaksi bahwa Muhammad adalah hamba dan Rasul-Nya."
{\scriptsize 2}\\
\par
\noindent
\textbf{Tingkatan Doa dan Sanad}:
\begin{enumerate}
\item \textbf{Shahih}: HR. Muslim (no. 403 [60]), Abu Awanah (II/228), dari
Abdullah bin Abbas; bahwasanya ia berkata: "Rasulullah SAW. mengajarkan 
kepada kami tasyahud sebagaimana mengajarkan surah dari al-Qur-an."
\item \textbf{Shahih}: HR. Al-Bukhari (no. 831, 835, 1202) dan Muslim (no. 
402 [55]).\\\\
\end{enumerate}
\par
\noindent 
48------------------------------Membaca Shalawat{\scriptsize 1} Nabi setelah Tasyahud
\begin{arabtext}
\noindent
al-ll_ahumma .salli `alY mu.hammadiN wa`alY ^Ali mu.hammadiN, kamA 
.sallayta `alY 'ibrAhi-yma wa`alY ^Ali 'ibrAhi-yma, 'innaka .hami-yduN 
ma^gi-yduN, al-ll_ahumma bArik `alY mu.hammadiN wa`alY ^Ali mu.hammadiN, 
kamA bArakta `alY 'ibrAhi-yma wa`alY ^Ali 'ibrAhi-yma, 'innaka .hami-yduN
ma^gi-yduN.\\
\end{arabtext}
\noindent
\textbf{Artinya} :
\par
\indent
"Ya Allah, berikanlah shalawat kepada Nabi Muhammad beserta keluarga 
Muhammad, sebagaimana Engkau telah memberikan shalawat kepada Ibrahim dan 
keluarga Ibrahim. Sesungguhnya Engkau Maha Terpuji lagi Mahamulia. 
Berikanlah berkah kepada Muhammad dan keluarga Muhammad sebagaimana Engkau 
telah memberi berkah kepada Ibrahim beserta keluarga Ibrahim. Sesungguhnya 
Engkau Maha Terpuji lagi Mahamulia."{\scriptsize 2}\\
Atau membaca:\\
\begin{arabtext}
\noindent
al-ll_ahumma .salli `alaY mu.hammadiN wa`alaY 'azwA^gihi wa_durriyyatihi, 
kamA .salla-yta `alaY ^Ali 'ibrAhi-yma, wabArik `alaY mu.hammadiN wa`alaY 
'azwA^gihi wa_durriyyatihi, kamA bArakta `alaY ^Ali 'ibrAhi-yma, 'innaka 
.hamiyduN ma^gi-yduN.\\
\end{arabtext}
\noindent
\textbf{Artinya}:
\par
\indent
"Ya Allah, berikanlah shalawat kepada Muhammad, istri-istri dan 
keturunannya, sebagaimana Engkau telah memberikan shalawat kepada keluarga 
Nabi Ibrahim. Berikanlah berkah kepada Muhammad, istri-istri dan 
keturunannya, sebagaimana Engkau telah memberikan berkah kepada keluarga 
Ibrahim. Sesungguhnya Engkau Maha Terpuji lagi Mahamulia."{\scriptsize 3}\\
\begin{arabtext}
\noindent
al-ll_ahumma .salli `alaY mu.hammadiN wa`alaY ^Ali mu.hammadiN, kamA 
.salla-yta `alaY ^Ali 'ibrAhi-yma, wabArik `alaY mu.hammadiN wa`alaY ^Ali 
mu.hammadiN, kamA bArakta `alaY ^Ali 'ibrAhi-yma fiy al-`Alami-yna, 'innaka 
.hami-yduN ma^gi-yduN.\\
\end{arabtext}
\noindent
\textbf{Artinya}:
\par
\indent
"Ya Allah, berikanlah shalawat kepada Muhammad dan keluarga Muhammad 
sebagaimana Engkau telah memberi shalawat kepada keluarga Ibrahim. Dan 
berkahilah Muhammad dan keluarga Muhammad sebagaimana Engkau telah 
memberkahi keluarga Ibrahim atas sekalian alam, sesungguhnya Engkau Maha 
Terpuji (lagi) Mahamulia."{\scriptsize 4}\\
\par
\noindent
\textbf{Tingkatan Doa dan Sanad}:
\begin{enumerate}
\item Tidak ada tambahan lafazh "sayyidinaa" dalam shalawat dan tidak ada 
satu pun riwayat yang shahih dari Nabi Shallallahu ‘alaihi wa sallam, dan 
lafazh ini pun tidak diucapkan oleh para Sahabat….
\item \textbf{Shahih}: HR. Al-Bukhari (no. 3370)/\textit{Fathul B\^{a}ri} 
(VI/408), Muslim (no. 406), Abu Dawud (no. 976, 977, 978), at-Tirmidzi (no. 
483), an-Nasai (III/47-48), Ahmad (IV/243-244), Ibnu Majah (no. 904), dan 
selainnya dari Ka'ab bin Ujrah r.a.
\item \textbf{Shahih}: HR. Malik dalam \textit{al-Muwaththa'} (I/152, no. 
66), al-Bukhari (no. 3369)/\textit{Fathul Bari} (VI/407), Muslim (no. 407 
[69]), Abu Dawud (no. 979), dan lainnya. Lafazh tersebut diriwayatkan oleh 
Muslim dari Abu Humaid as-Sa'idi r.a.
\item \textbf{Shahih}: HR. Malik dalam \textit{al-Muwaththa'} (I/152, no. 
67), Muslim (no. 405 [65]), Abu Dawud (no. 980), Ahmad (IV/118, V/273-274), 
at-Tirmidzi (no. 3220), an-Nasai (III/45), \textit{'Amalul Yaum wal Lailah} 
(no. 48), dan selainnya dari Abu Mas'ud al-Anshari r.a.\\\\
\end{enumerate}
\par
\noindent 
49------------------------------Doa setelah Tasyahud Akhir sebelum Salam
\begin{arabtext}
\noindent
al-ll_ahumma 'inni-y 'a`u-w_du bika min `a_dAbi ^gahannama, wamin `a_dAbi 
al-qabri, wamin fitnaTi al-ma.hyA wAl-mamAti, wamin ^sarri fitnaTi 
al-masi-y.hi al-dda^g^gAli.\\
\end{arabtext}
\noindent
\textbf{Artinya}:
\par
\indent
"Ya Allah, sungguh aku berlindung kepada-Mu dari siksa Neraka Jahannam, 
siksa kubur, fitnah kehidupan dan fitnah setelah mati, serta dari kejahatan
fitnah al-Masih ad-Dajjal."{\scriptsize 1}\\
Atau membaca:\\
\begin{arabtext}
\noindent
al-ll_ahumma 'inni-y 'a`u-w_du bika min `a_dA bi al-qabri, wa'a`u-w_du bika
min fitnnaTi al-masiy.hi al-dda^g^gAli, wa'a`u-w_du bika min fitnaTi 
al-ma.hyA wAl-mamAti. al-ll_ahumma 'inni-y 'a`u-w_du bika min al-ma'_tami 
wAl-ma.grami.\\
\end{arabtext}
\noindent
\textbf{Artinya}:
\par
\indent
"Ya Allah, sesungguhnya aku berlindung kepada-Mu dari siksa kubur. Aku 
berlindung kepada-Mu dari fitnah al-Masih ad-Dajjal. Aku juga berlindung 
kepada-Mu dari fitnah kehidupan dan fitnah sesudah mati. Ya Allah, 
sesungguhnya aku berlindung kepada-Mu dari perbuatan dosa dan dari 
utang."{\scriptsize 2}\\
\begin{arabtext}
\noindent
al-ll_ahumma 'inni-y .zalamtu nafsi-y .zulmaN ka_ti-yraN, walA ya.gfiru 
al-ddunu-wba 'illA 'anta, fA.gfir li-y ma.gfiraTaN min `indika, wAr.hamni-y
'innaka 'anta al-.gafu-wru al-rra.hi-ymu.\\
\end{arabtext}
\noindent
\textbf{Artinya}:
\par
\indent
"Ya Allah, sesungguhnya aku banyak menganiaya diriku, dan tidak ada yang 
dapat mengampuni dosa-dosa kecuali Engkau. Oleh karena itulah, ampunilah 
dosa-dosaku dengan ampunan dari sisi Engkau, dan berilah rahmat kepadaku. 
Sesungguhnya Engkau Maha Pengampun lagi Maha Penyayang."{\scriptsize 3}\\
\begin{arabtext}
\noindent
al-ll_ahumma 'inni-y 'as'aluka yA Aal-ll_ahu bi-'annaka al-wA.hidu 
al-'a.hadu al-.s.samadu alla_di-y lam yalid walam yu-wlad walam yakun lahu 
kufuwaN 'a.haduN, 'an ta.gfirali-y _dunu-wbi-y 'innaka 'anta al-.gafu-wru 
al-rra.hi-ymu.\\
\end{arabtext}
\noindent
\textbf{Artinya}:
\par
\indent
"Ya Allah, sesungguhnya aku memohon kepada-Mu. Ya Allah, dengan bersaksi 
Engkau adalah Rabb Yang Maha Esa, Mahatunggal yang tidak membutuhkan 
sesuatu, tapi segala sesuatu yang butuh kepada-Mu, tidak beranak dan tidak 
diperanakan (tidak mempunyai ibu maupun bapak), tidak seorang pun yang 
menyamai-Mu, aku memohon agar Engkau mengampuni dosa-dosaku. Sesungguhnya 
Engkau Maha Pengampun lagi Maha Penyayang."{\scriptsize 4}\\
\begin{arabtext}
\noindent
al-ll_ahumma 'inni-y 'as'aluka bi-'anna laka al-.hamda lA 'il_aha 'anta 
wa.hdaka lA ^sari-yka laka, al-mannAnu, yA badi-y`a al-ssamAwAti wAl-'ar.di 
yA _dAl^galAli wAl-'ikrAmi, yA.hayyu yA qayyu-wmu 'inni-y 'as'aluka 
(al-^gannaTa wa'a`u-w_du bika mina al-nnAri).\\
\end{arabtext}
\noindent
\textbf{Artinya}:
\par
\indent
"Ya Allah, sesungguhnya aku memohon kepada-Mu. Sesungguhnya bagi-Mu segala 
pujian, tidak ada ilah yang berhak diibadahi dengan benar kecuali Engkau 
Yang Maha Esa, tiada sekutu bagi-Mu, Mahapemberi nikmat, Pencipta langit 
dan bumi tanpa contoh sebelumnya. Wahai Rabb Yang memiliki keagungan dan 
kemuliaan, wahai Rabb Yang Mahahidup, Yang berdiri sendiri (mengurusi 
makhluk-Nya) sesungguhnya aku mohon kepada-Mu agar dimasukkan [ke Surga dan
aku berlindung kepada-Mu dari siksa Neraka]."{\scriptsize 5}\\
\par
\noindent
\textbf{Tingkatan Doa dan Sanad}:
\begin{enumerate}
\item \textbf{Shahih}: HR. Muslim (no. 588 [128]) dari Sahabat Abu Hurairah 
r.a.
\item \textbf{Shahih}: HR. Al-Bukhari (no. 832) dan Muslim (no. 589 [129]), 
dan an-Nasai (III/56-57) dari Aisyah r.a.
\item \textbf{Shahih}: HR. Al-Bukhari (no. 834, 6326, 7387, 7388), dan 
Muslim (no. 2705 [48]) dari Sahabat Abu Bakar ash-Shiddiq r.a.
\item \textbf{Shahih}: HR. An-Nasai (III/52)-lafazh ini ialah miliknya-dan 
Ahmad (IV/338) dari Mihjan bin al-Adru r.a. Dinyatakan shahih oleh Syaikh 
al-Albani dalam \textit{Shah\^{i}h an-Nasai} (I/279, no. 1234).
\item Sabda Rasulullah Shallallahu ‘alaihi wa sallam: "Sesungguhnya dia 
meminta kepada Allah dengan nama-Nya yang teragung (\textit{ismullabi 
a'zham}). Apabila ia minta kepada Allah maka akan dipenuhi dan apabila ia 
berdoa maka akan dikabulkan." Shahih: HR. Abu Dawud (no. 1495), an-Nasai 
(III/52), Ibnu Majah (no. 3858) Ahmad (III/158, 245) dan Ibnu Mandah dalam 
\textit{Kitabut Tauhid} (no. 355). Dan tambahan dalam kurung miliknya dari 
Anas bin Malik r.a.\\\\
\end{enumerate}
\par
\noindent 
52------------------------------Doa Qunut Witir
\begin{arabtext}
\noindent
al-ll_ahumma ahdini-y fi-yman hadayta, wa`Afini-y fi-yman `Afa-yta, 
watawallaniy fi-yman tawalla-yta, wabArik li-y fi-ymA 'a`.tayta, waqini-y 
^sarrimA qa.dayta, fa-'innaka taq.di-y walA yuq.dY `alayka, wa-'innahu lA 
ya_dillu man wAlayta, (walA ya`izzu man `Adayta), tabArakta rabbanA 
wata`Alayta.\\
\end{arabtext}
\noindent
\textbf{Artinya}:
\par
\indent
"Ya Allah, berikanlah aku petunjuk sebagaimana orang yang telah Engkau beri
petunjuk, berilah aku perlindungan (dari penyakit dan apa yang tidak 
disukai) sebagaimana orang yang telah Engkau lindungi, tolonglah aku 
sebagaimana orang-orang yang Engkau tolong. Berikanlah berkah terhadap 
apa-apa yang telah Engkau berikan kepadaku, jauhkanlah aku dari kejelekan 
apa yang Engkau telah takdirkan, sesungguhnya Engkaulah yang menjatuhkan 
hukum, dan tidak ada orang yang memberikan hukuman kepada-Mu. Dan 
sesungguhnya orang yang Engkau bela tidak akan terhina, dan tidak akan 
mulia orang yang Engkau musuhi. Mahasuci Engkau, wahai Rabb kami  
Mahatinggi."\\
\par
\noindent
\textbf{Tingkatan Doa dan Sanad}: \textbf{Shahih}: HR. Abu Dawud (no. 
1425), at-Tirmidzi (no. 464), Ibnu Majah (no. 1178), an-Nasai (III/248), 
Ahmad (I/199; 200), al-Baihaqi (II/209, 497-498). Sedang doa yang terdapat 
di dalam kurung menurut riwayat al-Baihaqi. Hadits ini diriwayatkan dari 
al-Hasan bin Ali: "Nabi SAW. mengajarkanku beberapa kalimat yang dapat aku 
baca dalam shalat Witir ...." Lihat \textit{Shah\^{i}h at-Tirmidzi} 
(I/144), \textit{Shah\^{i}h Ibni Majah} (I/194), \textit{Irw\^{a}-ul 
Ghal\^{i}l} (II/172), dan \textit{Shah\^{i}h Kit\^{a}b al-Adzk\^{a}r} 
(I/176-177, no. 155/125). Sanadnya shahih.\\\\
\par
\noindent 
53------------------------------Doa di Akhir Shalat Witir
\begin{arabtext}
\noindent
al-ll_ahumma 'inni-y 'a`u-w_du biri.dAka min sa_ha.tika, wabimu`AfAtika min 
`uqu-wbatika, wa'a`u-w_du bika minka, lA 'u.h.si-y _tanA'aN `alayka, 'anta 
kamA 'a_tnayta `alY nafsika.\\
\end{arabtext}
\noindent
\textbf{Artinya}:
\par
\indent
"Ya Allah, sesungguhnya aku berlindung dengan keridhaan-Mu dari 
kemurkaan-Mu, dengan keselamatan-Mu dari hukuman-Mu, dan berlindung 
kepada-Mu dari siksaan-Mu. Aku tidaklah mampu menghitung pujian dan 
sanjungan kepada-Mu, Engkau aalah sebagaimana Engkau menyanjung/memuji 
diri-Mu sendiri."{\scriptsize 1}\\
\begin{arabtext}
\noindent
sub.hAna al-maliki al-qudduwsi, sub.hAna al-maliki al-qudduwsi, sub.hAna 
al-maliki al-qudduwsi.\\
\end{arabtext}
\noindent
\textbf{Artinya}:
\par
\indent
"Mahasuci Allah Raja Yang Mahasuci, Mahasuci Allah Raja Yang Mahasuci, 
Mahasuci Allah Raja Yang Mahasuci." [Nabi mengangkat dan memanjangkan 
suaranya pada ucapan yang ketiga]{\scriptsize 2}\\
\par
\noindent
\textbf{Tingkatan Doa dan Sanad}: 
\begin{enumerate}
\item \textbf{Shahih}: HR. Abu Dawud (no. 1427), at-Tirmidzi (no. 3566), 
Ibnu Majah (no. 1179), an-Nasai (III/249), Ahmad (I/98, I/96, 118, 150). 
Lihat \textit{Shah\^{i}h at-Tirmidzi} (III/180), \textit{Shah\^{i}h Ibni 
Majah} (I/194), \textit{Irw\^{a}-ul Ghal\^{i}l} (II/175), dan 
\textit{Shah\^{i}h Kit\^{a}b al-Adzk\^{a}r} (I/255-256, no. 246/184).
\item \textbf{Shahih}: Abu Dawud (no. 1430), an-Nasai (III/245), dan Ahmad 
(V/123), Ibnu Hibban (no. 2441-\textit{at-Ta'liqatul His\^{a}n}), Ibnus 
Sunni (no. 706), serta al-Baghawi dalam \textit{Syarhus Sunnah} (IV/98, 
no. 972). Lihat juga \textit{Shah\^{i}h Kit\^{a}b al-Adzk\^{a}r} (I/255) 
dan \textit{Z\^{a}dul Ma'\^{a}d} (I/337).\\\\
\end{enumerate}
\par
\noindent 
55------------------------------Tentang Mengangkat Tangan
\par
\indent
Disunnahkan mengangkat tangan baik dalam qunut Nazilah maupun dalam qunut 
Witir berdasarkan dalil hadits-hadits yang sanadnya shahih dan atsar dari 
sahabat.{\scriptsize 1}\\
\indent Adapun mengusap wajah sesudah qunut atau berdoa, tidak ada satu pun
riwayat yang shahih. Maka, perbuatan ini adalah \textbf{bid'ah}.
{\scriptsize 2}\\
\indent Imam al-Baihaqi juga menjelaskan bahwa tidak ada seorang pun dari 
ulama Salafush Shalih yang mengusap wajah sesudah doa qunut dalam shalat.
{\scriptsize 3}\\
\par
\noindent
\textbf{Tingkatan Doa dan Sanad}: 
\begin{enumerate}
\item Lihat \textit{Sunanul Kubra lil Baihaqi} (II/211-212, III/39-41) dan 
\textit{Irwa-ul Ghalil} (II/163-164).
\item Lihat \textit{Irwa-ul Ghalil} (II/181). Lihat pula kitab 
\textit{Shahih Kitab al-Adzkar wa Dha'ifuhu} (hlm. 960-962).
\item \textit{Sunan al-Baihaqi} (II/212).\\\\
\end{enumerate}
\par
\noindent 
57------------------------------Doa Shalat Istikharah
\par
\indent
Jabir bin Abdillah menuturkan: "Rasulullah shallallahu ‘alaihi wa sallam 
mengajari kami shalat Istikharahuntuk memutuskan segala sesuatu sebagaimana 
mengajari  surah Al-Qur-an." Beliau pun bersabda: "Apabila seseorang di 
antara kalian mempunyai satu rencana untuk mengerjakan sesuatu, hendaknya 
ia melakukan shalat sunnah (Istikharah) dua rakaat, kemudian bacalah doa 
ini:\\
\begin{arabtext}
\noindent
al-ll_ahumma 'inniy 'asta_hi-yruka bi`ilmika, wa-'astaqdiruka biqudratika, 
wa-'as'aluka min fa.dlika al`a.ziymi, fa-'innaka taqdiru walA 'aqdiru, 
wata`lamu walA 'a`lamu, wa'anta `allAmu al.guyu-wbi. al-ll_ahumma 'in kunta 
ta`lamu 'anna h_a_dA al-'amra (wayusammiy .hA^gatahu) _hayruN li-y fi-y 
di-yni-y, wama`A^si-y, wa`AqibaTi 'amri-y ('aw qala : `A^gili 'amri-y 
wa-^A^gilihi) fAqdurhu li-y wayassirhu li-y, _tumma bArik li-y fi-yhi, 
wa-'in kunta ta`lamu 'anna h_a_dA al-'amra ^sarruN li-y fi-y di-yni-y, 
wama`A^si-y, wa-`AqibaTi 'amri-y ('awqAla : `A^gili 'amri-y wa-^A^gilihi) 
fA.srifhu `anni-y wA.srifni-y `anhu, wAqdurliya al-_hayra .hay_tu kAna, 
_tumma 'ar.dini-y bihi.\\
\end{arabtext}
\noindent
\textbf{Artinya}:
\par
\indent
'Ya Allah, sesungguhnya aku meminta pilihan yang tepat kepada-Mu dengan 
ilmu-Mu, dan aku memohon kekuasaan kepada-Mu (untuk mengatasi masalahku) 
dengan kemahakuasaan-Mu. Aku mohon kepada-Mu sesuatu dari anugerah-Mu Yang 
Mahaagung, sesungguhnya Engkau Mahakuasa, sedangkan aku tidak kuasa, 
Engkau mengetahui, sedangkan aku tidak mengetahui dan Engkaulah yang Maha 
Mengetahui perihal yang ghaib. Ya Allah, apabila Engkau mengetahui bahwa 
urusan ini (hendaknya menyebutkan masalahnya) lebih baik dalam agamaku, 
kehidupanku, dan akibatnya terhadap diriku, baik di dunia atau di akhirat, 
maka takdirkanlah untukku, dan mudahkan jalannya, kemudian berilah 
keberkahan. Akan tetapi apabila Engkau mengetahui bahwa urusan ini membawa 
keburukan bagiku dalam agamaku, kehidupanku, dan akibatnya terhadap diriku,
baik di dunia atau di akhirat, maka singkirkan urusan tersebut, dan jauhkan
aku darinya, serta takdirkanlah bagiku kebaikan di mana saja kebaikan 
berada, kemudian jadikanlah aku ridha dalam menerimanya."\\
\par
\indent
Tidak akan menyesal orang yang beristikharah kepada al-Khaliq (Allah) dan 
bermusyawarahlah dengan orang-orang Mukmin serta berhati-hati menangani 
persoalannya. Allah SWT. berfirman:\\
\begin{arabtext}
\noindent
wa^sAwirhum fiY al'amri fa-'i_dA `azamta fatawakkal `alaY al-llahi .... 
(109)\\
\end{arabtext}
\noindent
\textbf{Artinya}:
\par
\indent
\textit{"Dan bermusyawarahlah dengan mereka (para Sahabat) dalam urusan itu 
(peperangan, perekonomian, politik, dan lain-lain). Bila kamu telah 
membulatkan tekad, bertakwakallah kepada Allah ...." (QS. Ali 'Imran 
[3]: 159)}\\
\par
\noindent
\textbf{Tingkatan Doa dan Sanad}: \textbf{Shahih}: HR. Al-Bukhari (no. 
1162, 6382, 7390), Abu Dawud (no. 1538), at-Tirmidzi (no. 480), an-Nasai 
(VI/80), dan Ibnu Majah (no. 1383).\\\\
\par
\noindent 
59------------------------------Doa Kepada Pengantin
\begin{arabtext}
\noindent
baaraka al-ll_ahu laka wabaaraka `alayka wa^gama`a baynakumaa fiy _hayriN.\\
\end{arabtext}
\noindent
\textbf{Artinya}:
\par
\indent
"Semoga Allah memberimu berkah serta memberkahi atas pernikahanmu, dan 
semoga Dia mengumpulkan kalian berdua dalam kebaikan."\\
\par
\noindent
\textbf{Tingkatan Doa dan Sanad}: \textbf{Shahih}: HR. Abu Dawud (no. 
2130), at-Tirmidzi (no. 1091), Ahmad (II/381), ad-Darimi (II/134), Ibnu 
Majah (no. 1905), dan al-Hakim (II/183). Sanadnya shahih. Lihat 
\textit{\^{A}d\^{a}buz Zif\^{a}f} (hlm. 175).\\\\
\par
\noindent 
60------------------------------Doa Pengantin Pria kepada Istri
\par
\indent
"Apabila seseorang di antara kalian menikah dengan wanita atau membeli 
hamba sahaya, maka peganglah ubun-ubunnya, lalu bacalah Bismillah serta 
doakanlah dengan ucapan doa berikut:\\
\begin{arabtext}
\noindent
al-ll_ahumma 'inni-y 'as'aluka _hayrahaa, wa_hayra maa ^gabaltahaa `alayhi, 
wa'a`u-wbika min ^sarrihaa, wa^sarri maa ^gabaltahaa `alayhi.\\
\end{arabtext}
\noindent
\textbf{Artinya}:
\par
\indent
'Ya Allah, sesungguhnya aku mohon kepada-Mu kebaikannya dan kebaikan 
tabi'atnya (wataknya). Dan aku mohon perlindungan kepada-Mu dari 
keburukannya dan keburukan tabiatnya.'\\
\par
\indent
Apabila seseorang membeli unta, hendaklah dipegang puncak punuknya, 
kemudian berkata seperti itu."\\
\par
\noindent
\textbf{Tingkatan Doa dan Sanad}: \textbf{Shahih}: HR. Abu Dawud (no. 
2160), Ibnu Majah (no. 1918), al-Hakim (II/185) dan al-Baihaqi (VIII/148). 
Lihat \textit{Shah\^{i}h Ibni Majah} (I/324) dan \textit{\^{A}d\^{a}buz 
Zif\^{a}f fis Sunnah al-Muthahharah} (hlm. 92-93).\\\\
\par
\noindent 
61------------------------------Doa Sebelum Jima' (Bersetubuh)
\begin{arabtext}
\noindent
bismi al-ll_ahi, al-ll_ahumma ^gannibnaa al-^s^say.taana wa^gannibi 
al-^s^say.taana maa razaqtanaa.\\
\end{arabtext}
\noindent
\textbf{Artinya}:
\par
\indent
"Dengan nama Allah, ya Allah, jauhkanlah kami dari syaitan dan jauhkanlah 
syaitan agar tidak mengganggu apa (anak) yang Engkau rizkikan kepada kami."
\\
\par
\noindent
\textbf{Tingkatan Doa dan Sanad}: \textbf{Shahih}: HR. Al-Bukhari (no. 141,
3271, 5165, 6388) dan Muslim (no. 1434) dari Ibnu Abbas. Sabda Nabi SAW.: 
"Apabila mereka ditakdirkan mendapatkan anak, maka anak itu tidak akan 
diganggu (dibahayakan) oleh syaitan selama-lamanya."\\\\
\par
\noindent 
62------------------------------Doa Sebelum Makan
\par
\indent
Apabila seseorang di antara kamu makan makanan, hendaklah membaca:\\
\begin{arabtext}
\noindent
bismi al-ll_ahi.\\
\end{arabtext}
\noindent
\textbf{Artinya}:
\par
\indent
"Dengan nama Alah (aku makan)."\
\par
\indent
Adapun apabila lupa membaca pada permulaannya, hendaklah membaca:\\
\begin{arabtext}
\noindent
bismi al-ll_ahi fi-y 'awwalihi wa'A_hirihi.\\
\end{arabtext}
\noindent
\textbf{Artinya}:
\par
\indent
"Dengan nama Allah (aku makan) di awal dan di akhirnya."{\scriptsize 1}\\
\par
\indent
Atau membaca:
\begin{arabtext}
\noindent
bismi al-ll_ahi 'awwalahu wa'A_hirahu.\\
\end{arabtext}
\noindent
\textbf{Artinya}:
\par
\indent
"Dengan nama Allah (aku makan), awal dan akhirnya."\\
\par
\noindent
\textbf{Tingkatan Doa dan Sanad}:
\begin{enumerate}
\item \textbf{Shahih}: HR. Abu Dawud (no. 3767), At-Tirmidzi (no. 1858), 
dan Shahih At-Tirmidzi (II/67).\\\\
\end{enumerate}
\par
\noindent 
63------------------------------Doa Sesudah Makan
\begin{arabtext}
\noindent
al-.hamdu li-ll_ahi alla_di-y 'a.t`amaniy h_a_dA warazaqani-yhi min 
.ga-yri .hawliN minni-y walA quwwaTiN.\\
\end{arabtext}
\noindent
\textbf{Artinya}:
\par
\indent
"Segala puji bagi Allah yang telah memberi makanan ini kepadaku dan yang 
telah memberi rizki kepadaku tanpa daya dan kekuatan dariku."
{\scriptsize 1}\\
\begin{arabtext}
\noindent
al-.hamdu lill_ahi .hamdaN ka_ti-yraN .tayyibaN mubArakaN fi-yhi, .ga-yra
makfiyyiN walA muwadda-`iN, walA musta.gnaN_A `anhu rabbanA.\\
\end{arabtext}
\noindent
\textbf{Artinya}:
\par
\indent
"Segala puji bagi Allah (aku memuji-Nya) dengan pujian yang banyak, yang 
baik dan penuh berkah, yang senantiasa dibutuhkan, diperlukan dan tidak 
bisa ditinggalkan (pengharapan kepada-Nya) wahai Rabb kami."{\scriptsize 2}
\\
\par
\noindent
\textbf{Tingkatan Doa dan Sanad}:
\begin{enumerate}
\item \textbf{Shahih}: HR. Abu Dawud (no. 4023), at-Tirmidzi (no. 3458),
Ibnu Majah (no. 3285), Ibnus Sunni (no. 467), Ahmad (III/439) dan al-Hakim 
(I/507; IV/192), Lihat \textit{Irw\^{a}-ul Ghal\^{i}l} (no. 1989).
\item \textbf{Shahih}: HR. Al-Bukhari (no. 5458), Abu Dawud (no. 3849), 
Ahmad (V/252, 256), at-Tirmidzi (no. 3456), Ibnus Sunni dalam 
\textit{'Amalul Yaum wal Lailah} (no. 468, 484), al-Baghawi dalam 
\textit{Syarhus Sunnah} (no. 2828) dari Abu Umamah al-Bahili r.a.\\\\
\end{enumerate}
\par
\noindent 
64------------------------------Doa kepada Orang yang telah Memberi Makan dan Minum
\begin{arabtext}
\noindent
al-ll_ahumma 'a.t`im man 'a.t`amani-y waasqi man saqAni-y.\\
\end{arabtext}
\noindent
\textbf{Artinya}:
\par
\indent
"Ya Allah, berikanlah makan kepada orang yang memberi makan kepadaku dan 
berikanlah minum kepada orang yang memberi minum kepadaku."\\
\par
\noindent
\textbf{Tingkatan Doa dan Sanad}: \textbf{Shahih}: HR. Muslim (no.  2055 
[174]), dan Ahmad (VI/2-5). \\\\
\par
\noindent 
65------------------------------Doa Tamu kepada Tuan Rumah yang Menghidangkan Makanan
\begin{arabtext}
\noindent
al-ll_ahumma bArik lahum fi-ymaa razaqtahum, wA.gfir lahum wAr.hamhum.\\
\end{arabtext}
\noindent
\textbf{Artinya}:
\par
\indent
"Ya Allah, berikanlah berkah terhadap apa yang Engkau rizkikan kepada 
mereka, ampunilah mereka dan rahmatilah mereka."\\
\par
\noindent
\textbf{Tingkatan Doa dan Sanad}: \textbf{Shahih}: HR. Muslim (no. 2042 
[146]), Abu Dawud (no. 3729), at-Tirmidzi (no. 3576) dan lainnya. \\\\
\par
\noindent 
66------------------------------Doa Ketika Berbuka bagi Orang yang Berpuasa
\begin{arabtext}
\noindent
_dahaba al-.z.zama'u, waabtallati al-`uru-wqu, wa_tabata al-'a^gru 'in 
^sA'a al-ll_ahu.\\
\end{arabtext}
\noindent
\textbf{Artinya}:
\par
\indent
"Telah hilang rasa haus, dan urat-urat telah basah serta pahala telah 
tetap, \textit{insya Allah}."\\
\par
\noindent
\textbf{Tingkatan Doa dan Sanad}: \textbf{Hasan}: HR. Abu Dawud (no. 2357),
ad-Daraquthni (II/401, no. 2247), al-Hakim (I/422). Lihat 
\textit{Irw\^{a}'ul Ghal\^{i}l} (IV/39, no. 920), dan \textit{Shah\^{i}h 
Abi Dawud} (III/449, no. 2066).\\\\
\par
\noindent 
67------------------------------Doa Ketika Berbuka Puasa Di Rumah Orang Lain
\begin{arabtext}
\noindent
'af.tara `indakumu al-.s.sA'imu-wna, wa'akala .ta`Amakumu al-'abrAru, 
wa.sallat `ala-ykumu al-malA'ikaTu.\\
\end{arabtext}
\noindent
\textbf{Artinya}:
\par
\indent
"Orang-orang yang berpuasa telah berbuka di tempat kalian, dan orang-orang 
yang baik telah makan makananmu, dan para Malaikat mendoakan (rahmat / 
kebaikan) untuk kalian."\\
\par
\noindent
\textbf{Tingkatan Doa dan Sanad}: \textbf{Shahih}: HR. Abu Dawud (no. 
3854), an-Nasa'I dalam Kitab \textit{'Amalul Yaum wal Lailah} (no. 298, 
299), Ibnu Sunni dalam \textit{'Amalul Yaum wal Lailah} (no. 482), Ahmad 
(III/118, 138). Doa ini boleh juga dibaca setelah makan di rumah orang 
lain. Lihat \textit{\^{A}d\^{a}buz Zif\^{a}f} (hlm. 170-171).\\\\
\par
\noindent 
68------------------------------Doa Bagi Orang Lain yang Berbuat Baik kepada Kita
\begin{arabtext}
\noindent
^gazAka al-ll_ahu _ha-yraN.\\
\end{arabtext}
\noindent
\textbf{Artinya}:
\par
\indent
"Semoga Allah membalasmu dengan sesuatu yang lebih baik."\\
\par
\noindent
\textbf{Tingkatan Doa dan Sanad}: \textbf{Shahih}: HR. At-Tirmidzi (no.
2035), an-Nasa'I dalam kitab \textit{'Amalul Yaum wal Lailah} (no. 180) dan
Ibnu Hibban (no. 3404). Lihat \textit{Shah\^{i}h al-J\^{a}hmi'ish 
Shagh\^{i}r} (no. 6368) dan \textit{Shah\^{i}h At-Targh\^{i}b wat 
Tarh\^{i}b} (I/571 no. 969).\\\\
\par
\noindent 
69------------------------------Doa Musafir kepada Orang yang Ditinggalkan
\begin{arabtext}
\noindent
'astawdi `ukumu al-ll_aha a-lla_di-y lA ta.diy`u wadA'i`uhu.\\
\end{arabtext}
\noindent
\textbf{Artinya}:
\par
\indent
"Aku menitipkan kalian kepada Allah yang tidak akan hilang titipan-Nya."
\\
\par
\noindent
\textbf{Tingkatan Doa dan Sanad}: \textbf{Shahih}: HR. Ahmad (II/403) dan 
Ibnu Majah (no. 2825), An-Nasa'i dalam \textit{'Amalul Yaum wal Lailah} 
(no. 512), Ibnu Sunni dalam \textit{'Amalul Yaum wal Lailah} (no. 505), dan
ath-Thabrani dalam kitab \textit{ad-Du'\^{a}'} (no. 820)-lafazh ini milik 
Ibnu Sunni.\\\\
\par
\noindent 
70------------------------------Doa Orang Mukmin kepada Orang yang Bepergian
\begin{arabtext}
\noindent
'astawdi`u al-ll_aha di-ynaka wa'amAnataka wa_hawAti-yma `amalika.\\
\end{arabtext}
\noindent
\textbf{Artinya}:
\par
\indent
"Aku menitipkan agamamu, amanatmu, dan kesudahan amal perbuatanmu kepada 
Allah."\\
\par
\noindent
\textbf{Tingkatan Doa dan Sanad}: \textbf{Shahih}: HR. Ahmad (II/7), Abu 
Dawud (no. 2600), al-Hakim (I/442), dan at-Tirmidzi (no. 3443) dari Ibnu 
Umar r.a. Lihat \textit{Silsilah Ah\^{a}d\^{i}ts ash-Shah\^{i}hah} (no. 
14).\\\\
\par
\noindent 
71------------------------------Doa Naik Kendaraan
\begin{arabtext}
\noindent
bismi al-ll_ahi, al-.hamduli-ll_ahi ( sub.h_ana alla_diY sa_h_hara lanA 
h_a_dA wamA kunnA lahu muqrini-yna wa-'inn^A 'il_aY rabbinA 
lamunqalibuwna ) al-.hamduli-ll_ahi, al-.hamduli-ll_ahi, 
al-.hamduli-ll_ahi, al-ll_ahu 'akbaru, al-ll_ahu 'akbaru, al-ll_ahu 
'akbaru, sub.hAnaka 'inni-y .zalamtu nafsi-y fA.gfirli-y, fa'i-nnahu lA 
ya.gfiru al-_d_dunu-wba 'illA 'anta.\\
\end{arabtext}
\noindent
\textbf{Artinya}:
\par
\indent
"Dengan nama Allah, segala puji bagi Allah, \textit{Mahasuci Rabb yang 
menundukkan kendaraan ini untuk kami, padahal kami sebelumnya tidak mampu 
menguasainya. Dan sesungguhnya kami akan kembali kepada Rabb kami (di hari 
Kiamat)}. Segala puji bagi Allah (3x), Allah Mahabesar (3x), Mahasuci 
Eangkau. Ya Allah, sesungguhnya aku menganiaya diriku, maka ampunilah aku.
Sesungguhnya tidak ada yang dapat mengampuni dosa-dosa kecuali Engkau."\\
\par
\noindent
\textbf{Tingkatan Doa dan Sanad}: \textbf{Shahih}: HR. Abu Dawud (no. 
2602), at-Tirmidzi (no. 3446), Shahih Abi Dawud (II/493 no. 2267), dan 
Shahih at-Tirmidzi (III/156, no. 2742).\\\\
\par
\noindent 
72------------------------------Doa Safar
\begin{arabtext}
\noindent
al-ll_ahu 'akbaru, al-ll_ahu 'akbaru, al-ll_ahu 'akbaru, (sub.h_ana 
alla_diY sa_h_hara lanA h_a_dA wamA kunnA lahu muqri ni-yna wa-'inna-^A 
'il_aY rabbinA lamunqalibuwna) al-ll_ahumma 'innA nas'aluka fi-y safarinA 
h_a_dA al-birru wAl-ttaqwY, wamina al-`amali mAtar.dY, al-ll_ahumma 
hawwin `ala-ynA safaranA h_a_dA wA.twi `annA bu`dahu, al-ll_ahumma 'anta 
al-.s.sA.hibu fi-y al-ssafari wAl-_hali-yfaTu fi-y al-'ahli, al-ll_ahumma 
'inni-y 'a`u-w_dubika min wa-`_tA'i al-ssafari waka-^AbaTi al-man.zari 
wasu-w'i al-munqalabi fi-y almAli wAl-'ahli.\\
\end{arabtext}
\noindent
\textbf{Artinya}:
\par
\indent
"Allah Mahabesar (3x). \textit{Mahasuci Rabb yang menundukkan kendaraan ini
untuk kami, sedang sebelumnya kami tidak mampu menguasainya. Dan 
sesungguhnya kami akan kembali kepada Rabb kami (di hari Kiamat)}. Ya 
Allah, sesungguhnya kami memohon kepada-Mu kebaikan dan takwa dalam 
perjalanan ini, kami mohon perbuatan yang Engkau ridhai. Ya Allah, 
mudahkanlah perjalanan ini untuk kami, dan dekatkan jaraknya. Ya Allah, 
Engkaulah Pendampingku dalam bepergian dan yang mengurusi keluarga(ku). Ya 
Allah, sesungguhnya aku berlindung kepada-Mu dari kesulitan dalam bepergian,
pemandangan yang menyedihkan dan kepulangan yang buruk dalam harta dan 
keluarga."\\
\par
\indent Sekembalinya dari safar, maka baca doa di atas dan ditambah dengan:
\begin{arabtext}
\noindent
^Ayibu-wna tA'ibu-wna `Abidu-wna lirabbinA .hAmidu-wna.\\
\end{arabtext}
\noindent
\textbf{Artinya}:
\par
\indent
"Kami kembali dengan bertaubat, tetap beribadah dan selalu memuji Rabb 
kami."\\
\par
\noindent
\textbf{Tingkatan Doa dan Sanad}: \textbf{Shahih}: HR. Muslim (no. 1342) 
dari Ibnu Umar r.a.\\\\
\par
\noindent 
73------------------------Disunnahkan bagi Musafir agar Bertakbir pada Jalan Mendaki 
dan Bertasbih ketika Menurun
\begin{arabtext}
\noindent
`an ^gAbiribni `abdi al-ll_ahi : kunnA 'i_dA .sa`idnA kabbarnA, wa 'i_dA 
nazalnA sabba_hnA.\\
\end{arabtext}
\noindent
\textbf{Artinya}:
\par
\indent
Dari Jabir bin Abdillah r.a, ia berkata: "Kami membaca takbir apabila 
berjalan naik, dan kami membaca tasbih apabila berjalan menurun."\\
\par
\noindent
\textbf{Tingkatan Doa dan Sanad}: \textbf{Shahih}: HR. Al-Bukhari (no. 
2993)/\textit{Fathul B\^{a}ri} (VI/135).\\\\
\par
\noindent 
74------------------------------Doa Musafir Menjelang Subuh
\begin{arabtext}
\noindent
samma`a sAmi`uN bi.hamdi al-ll_ahi, wa.husni balA'ihi `ala-ynaa. rabbanA 
.sA.hibnaa, wa'af.dil `ala-ynaa `A'i_daN bi-al-ll_ahi mina al-nnaari.\\
\end{arabtext}
\noindent
\textbf{Artinya}:
\par
\indent
"Semoga ada yang memperdengarkan/menyaksikan pujian kami kepada Allah (atas
nikmat) dan cobaan-Nya yang baik bagi kami. Wahai Rabb kami, dampingilah 
kami (periharalah kami) dan berikanlah karunia kepada kami dengan 
berlindung kepada Allah dari Api Neraka."\\
\par
\noindent
\textbf{Tingkatan Doa dan Sanad}: \textbf{Shahih}: HR. Muslim (no. 2718)
- \textit{Syarh an-Nawawi} (XVII/39)-dan Abu Dawud (no.  5086). Lihat 
\textit{Silsilah Ah\^{a}d\^{i}ts ash-Shah\^{i}hah} (no. 2638). \\\\
\par
\noindent 
75------------------------------Doa Apabila Singgah di Suatu Tempat dalam Safar atau Selainnya
\begin{arabtext}
\noindent
'a`u-w_du bikalimAti al-ll_ahi al-ttAmmAti min ^sarri mA _halaqa.\\
\end{arabtext}
\noindent
\textbf{Artinya}:
\par
\indent
"Aku berlindung dengan kalimat-kalimat Allah yang sempurna, dari kejahatan 
apa yang diciptakan-Nya."\\
\par
\noindent
\textbf{Tingkatan Doa dan Sanad}: \textbf{Shahih}: HR. Muslim (no. 2708 
[53]), at-Tirmidzi (no. 3437), Ibnu Majah (no. 3547), Ahmad (VI/377) dan 
lainnya. Nabi SAW. bersabda: "Barang siapa yang menempati (atau singgah) di
suatu tempat kemudian mengucapkan (doa di atas), maka tidak ada sesuatu pun
yang bisa membahayakannya, sampai dia meninggalkan tempat tersebut."\\\\
\par
\noindent 
76------------------------------Doa Masuk Desa atau Kota
\begin{arabtext}
\noindent
al-ll_ahumma rabba al-ssamAwAti al-ssab`i wamA 'a.zlalna, warabba 
al-'a-r.di-yna al-ssab`i wamA 'aqlalna, warabba al-^s^sayA .ti-yni wamA 
'a.dlalna, warabba al-rriyA.hi wamA _dara-yna. fa'i-nnA nas'aluka _ha-yra 
h_a_dihi al-qaryaTi wa_ha-yra 'ahlihA, wa_ha-yra mA fi-yhA, wana`u-w_dubika
min ^sarrihA wa^sarri 'ahlihA wa^sarri mA fi-yhA.\\
\end{arabtext}
\noindent
\textbf{Artinya}:
\par
\indent
"Ya Allah, Rabb tujuh langit dan apa yang dinaunginya, Rabb tujuh bumi dan 
apa yang diatasnya, Rabb yang menguasai syaitan-syaitan dan apa yang mereka
sesatkan, Rabb yang menguasai angin dan apa yang dihembuskannya. Kami mohon
kepada-Mu kebaikan desa/kota ini, kebaikan penduduknya dan apa yang ada di 
dalamnya. Kami berlindung kepada-Mu dari keburukan desa/kota ini, keburukan
penduduknya dan apa yang ada di dalamnya."\\
\par
\noindent
\textbf{Tingkatan Doa dan Sanad}: \textbf{Shahih}: HR. An-Nasai dalam 
\textit{Sunanul Kubra} (no. 8775, 8776) dan \textit{'Amalul Yaum wal 
Lailah} (no. 547, 548). Ibnus Sunni dalam \textit{'Amalul Yaum wal Lailah} 
(524), Ibnu Khuzaimah (no. 2565), al-Hakim (II/100), dan lainnya dari 
Shuhaib bin Amr r.a. Al-Hakim menilai Hadits shahih. Imam adz-Dzahabi 
menyetujuinya. Lihat \textit{Shah\^{i}h al-Kalimith Thayyib} (no. 179), 
\textit{Silsilah Ah\^{a}d\^{i}ts ash-Shah\^{i}hah} (no. 2759), serta 
\textit{Shah\^{i}h al-Adzk\^{a}r} (no. 617/450).\\\\
\par
\noindent 
77------------------------------Doa Masuk Pasar
\begin{arabtext}
\noindent
lA 'il_aha 'illA al-ll_ahu wa.hdahu lA ^sari-yka lahu, lahu al-mulku, 
walahu al-.hamdu,yu.hyi-y wayumi-ytu, wahuwa .hayyuN lA yamu-wtu, biyadihi 
al-_ha-yru, wahuwa `alY kulli ^say'iN qadi-yruN.\\
\end{arabtext}
\noindent
\textbf{Artinya}:
\par
\indent
"Tidak ada Ilah yang berhak diibadahi dengan benar melainkan hanya Allah, 
Yang Maha Esa, tiada sekutu bagi-Nya. Bagi-Nya kerajaan, bagi-Nya segala 
puji. Dialah Yang Menghidupkan dan Yang Mematikan. Dialah Yang Hidup, tidak
akan mati. Di tangan-Nya kebaikan, Dialah Yang Mahakuasa atas segala 
sesuatu."\\
\par
\noindent
\textbf{Tingkatan Doa dan Sanad}: \textbf{Hasan}: HR. At-Tirmidzi (no. 3428
dan 3429), Ibnu Majah (no. 2235), al-Hakim (I/538). Lihat takhrij hadits 
ini dalam \textit{Shah\^{i}h al-W\^{a}hbilish Shayyib} (hlm. 250-255).\\\\
\par
\noindent 
78------------------------------Doa Apabila Angin Bertiup Kencang
\begin{arabtext}
\noindent
al-ll_ahumma 'inni-y 'as'aluka _ha-yrahA, wa'a`u-w_du bika min ^sarrihA.\\
\end{arabtext}
\noindent
\textbf{Artinya}:
\par
\indent
"Ya Allah, sesungguhnya aku mohon kepada-Mu kebaikan angin ini, dan aku 
berlindung kepada-Mu dari kejelekannya."{\scriptsize 1}\\
\begin{arabtext}
\noindent
al-ll_ahumma 'inni-y 'as'aluka _ha-yrahA, wa_ha-yra mA fi-yhA, wa_ha-yra 
mA 'ursilat bihi, wa'a`u-w_du bika min ^sarrihA, wa^sarri mA fi-yhA, 
wa^sarri mA 'ursilat bihi.\\
\end{arabtext}
\noindent
\textbf{Artinya}:
\par
\indent
"Ya Allah, sungguh kepada-Mu aku memohon kebaikan angin ini, kebaikan apa-
apa yang ada padanya dan kebaikan tujuan angin ini dihembuskan. Aku 
berlindung kepada-Mu dari kejelekan angin ini, kejelekan apa-apa yang ada 
padanya dan kejelekan tujuan angin ini dihembuskan."{\scriptsize 2}\\
\par
\noindent
\textbf{Tingkatan Doa dan Sanad}:
\begin{enumerate}
\item \textbf{Shahih}: HR. Abu Dawud (no. 5097), Ibnu Majah (no. 3727), dan
lihat \textit{Shah\^{i}h al-Adzk\^{a}r} (no. 521/381).
\item \textbf{Shahih}: HR. Muslim (no. 899 [15]) dan at-Tirmidzi (no. 3449)
dari Aisyah r.a.\\\\
\end{enumerate}
\par
\noindent 
79------------------------------Doa Minta Hujan
\begin{arabtext}
\noindent
al-ll_ahumma asqinA .ga-y_taN mu.gi-y_taN mari-y'aN mari-y`aN, nA fi`aN 
.ga-yra .dArriN, `A^gilaN .ga-yra ^A^giliN.\\
\end{arabtext}
\noindent
\textbf{Artinya}:
\par
\indent
"Ya Allah, berilah kami hujan yang merata, yang menyegarkan tubuh dan 
menyuburkan tanaman, bermanfaat, tidak berbahaya. Kami mohon hujan dengan 
segera, tidak ditunda-tunda."{\scriptsize 1}\\
\begin{arabtext}
\noindent
al-ll_ahumma 'a.gi_tnA, al-ll_ahumma 'a.gi_tnA, al-ll_ahumma 'a.gitnA.\\
\end{arabtext}
\noindent
\textbf{Artinya}:
\par
\indent
"Ya Allah, turunkanlah hujan kepada kami. Ya Allah, turunkanlah hujan 
kepada kami. Ya Allah, turunkanlah hujan kepada kami."{\scriptsize 2}\\
\par
\noindent
\textbf{Tingkatan Doa dan Sanad}:
\begin{enumerate}
\item \textbf{Shahih}: HR. Abu Dawud (no. 1169), dinyatakan shahih oleh 
al-Albani dalam \textit{Shah\^{i}h Abi Dawud} (I/216). Dalam riwayat lain 
bahwa Nabi mengangkat kedua tangannya ketika minta hujan, namun tidak 
melewati kepalanya (riwayat Abu Dawud, no. 1168) sehingga terlihat kedua 
ketiak dan telapak tangannya ke arah bumi (Abu Dawud, no. 1171). Lihat 
\textit{Shah\^{i}h al-Bukhari} (no. 1030, 1031) dan \textit{Shahih Muslim} 
(no. 895, 896).
\item \textbf{Shahih}: HR. Al-Bukhari (no. 1014) dan Muslim (no. 897 [8]) 
dari Anas bin Malik r.a.\\\\
\end{enumerate}
\par
\noindent 
80------------------------------Doa Apabila Turun Hujan
\begin{arabtext}
\noindent
al-ll_ahumma .sayyibaN nAfi`aN.\\
\end{arabtext}
\noindent
\textbf{Artinya}:
\par
\indent
"Ya Allah, turunkanlah hujan yang bermanfaat (untuk manusia, tanaman, dan 
binatang)."\\
\par
\noindent
\textbf{Tingkatan Doa dan Sanad}: \textbf{Shahih}: HR. Al-Bukhari (no. 
1032), \textit{Fathul B\^{a}ri} (II/518).\\\\
\par
\noindent 
81------------------------------Dzikir Setelah Hujan
\begin{arabtext}
\noindent
mu.tirnA bifa.dli al-ll_ahi wara.hmatihi.\\
\end{arabtext}
\noindent
\textbf{Artinya}:
\par
\indent
"Kita diberi hujan karena karunia dan rahmat Allah."\\
\par
\noindent
\textbf{Tingkatan Doa dan Sanad}: \textbf{Shahih}: HR. Al-Bukhari (no. 846,
1038), Muslim (no. 71). Tidak boleh menisbatkan hujan kepada bintang, karena
datangnya hujan itu dengan sebab rahmat Allah, bukan karena bintang. Orang 
yang menisbatkan demikian telah kufur kepada Allah.\\\\
\par
\noindent 
82------------------------------Doa Agar Hujan Berhenti (Dialihkan  ke tempat lain)
\begin{arabtext}
\noindent
al-ll_ahumma .hawAla-ynA walA `alaynA, al-ll_ahumma `alY al'AkAmi
wAl-.z.zirAbi, wabu.tu-wni al-'awdiyaTi wamanAbiti al-^s^sa^gari.\\
\end{arabtext}
\noindent
\textbf{Artinya}:
\par
\indent
"Ya Allah, turunkanlah hujan di sekitar kami, bukan untuk merusak kami. Ya 
Allah, turunkanlah hujan ke daratan tinggi, beberapa anak bukit, perut 
lembah dan beberapa tanah yang menumbuhkan pepohonan."\\
\par
\noindent
\textbf{Tingkatan Doa dan Sanad}: \textbf{Shahih}: HR. Al-Bukhari (no. 1013,
1014), Muslim (no. 897) dari Anas bin Malik r.a.\\\\
\par
\noindent 
83------------------------------Dzikir Ketika Mendengar Halilintar
\begin{arabtext}
\noindent
sub.haana a-lla_di-y yusabbi.hu al-rra`du bi.hamdihi wAl-malA'ikaTu min
_hi-yfatihi.\\
\end{arabtext}
\noindent
\textbf{Artinya}:
\par
\indent
"Mahasuci Allah yang halilintar bertasbih dengan memuji-Nya, begitu juga 
para Malaikat, karena takut kepada-Nya."\\
\par
\noindent
\textbf{Tingkatan Doa dan Sanad}: \textbf{Shahih}: \textit{Al-Muwaththa}'
(II/757, no. 26), al-Bukhari dalam \textit{al-Adabul Mufrad} (no. 723),
\textit{Shah\^{i}h al-Adabul Mufrad} (no. 556), al-Baihaqi (III/362),
\textit{al-Kalimuth Thayyib} (no. 157). Syaikh al-Albani berkata: "Hadits 
di atas mauquf sanadnya shahih ," dari Ibnu az-Zubair r.a. \\\\
\par
\noindent 
84------------------------------Doa Melihat Hilal (Awal Bulan Hijriyyah)
\begin{arabtext}
\noindent
al-ll_ahumma 'ahillahu `ala-ynaa bilyumni wAl-'i-ymAni, wAl-ssalAmaTi 
wAl-'i-slAmi, rabbiY warabbuka al-ll_ahu.\\
\end{arabtext}
\noindent
\textbf{Artinya}:
\par
\indent
"Ya Allah, tampakkan bulan itu kepada kami dengan membawa keberkahan dan 
keimanan, keselamatan dan Islam. Rabbku dan Rabbmu (wahai bulan sabit) 
adalah Allah."\\
\par
\noindent
\textbf{Tingkatan Doa dan Sanad}: \textbf{Shahih}: HR. At-Tirmidzi (no. 
3451), Ahmad (I/162), dan al-Hakim (IV/285) dari Thalhah bin Ubaidillah 
r.a. Diriwayatkan oleh ad-Darimi (II/3-4) dan Ibnu Hibban ( no. 885 -
\textit{at-Ta'l\^{i}q\^{a}tul His\^{a}n}) dari Ibnu Umar r.a. Lihat 
\textit{Silsilah Ah\^{a}d\^{i}ts ash-Shah\^{i}hah} (no. 1816) dan 
\textit{al-Ikhb\^{a}r bima La Yashihhu min Ah\^{a}ditsil Adzk\^{a}r} 
(hlm. 282).\\\\
\par
\noindent 
85------------------------------Doa Melihat Putik Buah
\begin{arabtext}
\noindent
al-ll_ahumma bArik lanA fi-y _tamarinA wabArik lanA fi-y madi-ynatinA 
wabArik lanA fi-y .saa`inA wabArik lanA fi-y muddinA.\\
\end{arabtext}
\noindent
\textbf{Artinya}:
\par
\indent
"Ya Allah, berkahilah buah-buahan kami, berkahilah kota kami, berkahilah 
takaran makanan kami dan berkahilah pada \textit{mudd} (ukuran sepenuh dua 
telapak tangan) kami."\\
\par
\noindent
\textbf{Tingkatan Doa dan Sanad}: \textbf{Shahih}: HR. Muslim (no. 1373).\\\\
\par
\noindent 
86----------------------Bacaan jika Tertimpa Sesuatu yang Tidak Diinginkan
\begin{arabtext}
\noindent
qaddara al-ll_ahu wamA ^sA'a fa`ala.\\
\end{arabtext}
\noindent
\textbf{Artinya}:
\par
\indent
"Allah sudah menakdirkan segala sesuatu dan Dia berbuat menurut apa yang 
Dia kehendaki."\\
\begin{arabtext}
\noindent
qadaru al-ll_ahi wamA ^sA'a fa`ala.\\
\end{arabtext}
\noindent
\textbf{Artinya}:
\par
\indent
Boleh juga diucapkan: "Ini adalah takdit Allah dan Dia berbuat menurut apa 
yang Dia kehendaki."\\
\par
\noindent
\textbf{Tingkatan Doa dan Sanad}: \textbf{Shahih}: HR. Muslim (no. 2664 
[34]).\\\\
\par
\noindent 
87------------------------------Doa Apabila Melihat Orang yang Mengalami Cobaan
\begin{arabtext}
\noindent
al-.hamdu li-ll_ahi a-lla_di-y `AfAni-y mimmA abtalAka bihi wafa.d.dalani-y 
`alY ka_ti-yriN mimman _halaqa taf.di-ylaN.\\
\end{arabtext}
\noindent
\textbf{Artinya}:
\par
\indent
"Segala puji bagi Allah yang telah menyelamatkan aku dari musibah yang 
Allah timpakan kepadamu. Dan Allah telah memberi kemuliaan kepadaku 
melebihi orang banyak."\\
\par
\noindent
\textbf{Tingkatan Doa dan Sanad}: \textbf{Shahih}: HR. At-Tirmidzi (no. 
3431), Ibnu Majah (no. 3892) dan lihat \textit{Silsilah Ah\^{a}d\^{i}ts 
ash-Shah\^{i}hah} (no. 602). \\\\
\par
\noindent 
88----------------Mengajari Orang yang Akan Meninggal Dunia dengan Kalimat 
\textit{L\^{a} il\^{a}ha illall\^{a}h}
\begin{arabtext}
\noindent
man kAna ^A_hiru kalAmihi lA 'il_aha 'illA al-ll_ahu da_hala al-^gannaTa.\\
\end{arabtext}
\noindent
\textbf{Artinya}:
\par
\indent "Barang siapa yang akhir perkataannya adalah: '\textit{L\^{a} 
il\^{a}ha illall\^{a}h},' akan masuk ke dalam Surga."{\scriptsize 2}\\\\
\indent Sabda Nabi shallallahu ‘alaihi wa sallam: "Talqini (ajarkanlah) 
orang yang akan meninggal di antara kalian dengan \textit{L\^{a} il\^{a}ha 
illall\^{a}h}."{\scriptsize 3}  Sabda Nabi: "Barang siapa pada akhir 
ucapannya, ketika hendak meninggal '\textit{L\^{a} il\^{a}ha 
illall\^{a}h}', maka ia masuk Surga suatu masa kelak, kendatipun akan 
mengalami musibah sebelum itu yang mungkin menimpanya."{\scriptsize 4}\\
\indent Nabi shallallahu ‘alaihi wa sallam bersabda: "Barang siapa yang 
meninggal dalam keadaan ia mengetahui bahwasanya tidak ada ilah yang berhak 
diibadahi dengan benar kecuali hanya Allah, maka ia akan masuk Surga."
{\scriptsize 5}\\
\indent Sabda Rasulullah shallallahu ‘alaihi wa sallam: "Barang siapa yang 
meninggal dalam keadaan tidak menyekutukan Allah dengan sesuatu pun juga, 
maka ia akan masuk Surga. Dan barang siapa yang meninggal dalam keadaan 
menyekutukan Allah dengan sesuatu, maka ia akan masuk Neraka."
{\scriptsize 6}\\
\par
\noindent
\textbf{Tingkatan Doa dan Sanad}:
\begin{enumerate}
\item \textbf{Shahih}: HR. Abu Dawud (no. 3116), Ahmad (V/233, 247) dan 
al-Hakim (I/351, 500) dari Mu'adz bin Jabal r.a. dan lihat 
\textit{Shah\^{i}hul J\^{a}mi'} (no. 6479).
\item \textbf{Shahih}: HR. Muslim (no. 916), Abu Dawud (3117), at-Tirmidzi 
(no. 976), an-Nasai (IV/5), Ibnu Majah (no. 1445).
\item \textbf{Shahih}: HR. Ibnu Hibban (no. 719-\textit{al-Maw\^{a}rid} dan
no. 2993-\textit{at-Ta'liwatul His\^{a}n}), \textit{Shah\^{i}h 
Maw\^{a}ridizh Zham-\^{a}n} (no. 595). Lihat \textit{Ahk\^{a}mul 
Jan\^{a}-iz} (hlm. 19) dan \textit{Irw\^{a}-ul Ghal\^{i}l} (III/150).
\item \textbf{Shahih}: HR. Muslim (no. 26 [143]).
\item \textbf{Shahih}: HR. Muslim (no. 93 [151]).\\\\
\end{enumerate}
\par
\noindent 
89------------------------------Orang yang Kena Musibah
\begin{arabtext}
\noindent
'innA li-ll_ahi wa'i-nnA 'ila-yhi rA^gi`u-wna Aal-ll_ahumma '^gurni-y fi-y 
mu.si-ybati-y wa'a_hlif li-y _ha-yraN minhA.\\
\end{arabtext}
\noindent
\textbf{Artinya}:
\par
\indent
"Sesungguhnya kami milik Allah dan kepada-Nya kami akan kembali. Ya Allah, 
berikanlah pahala kepadaku dalam musibahku dan gantikanlah untukku dengan 
yang lebih baik darinya (dari musibahku)."\\
\par
\noindent
\textbf{Tingkatan Doa dan Sanad}: \textbf{Shahih}: HR. Muslim (no. 918) 
dari Sahabah Ummu Salamah r.a. Nabi SAW. bersabda: "Tidaklah seorang hamba 
mendapat satu musibah lalu ia mengucapkan (doa di atas) melainkan Allah 
akan membagikan pahala kepadanya dalam musibahnya tersebut serta memberikan
ganti baginya dengan yang lebih baik darinya."\\\\
\par
\noindent 
90------------------------Doa Memejamkan Mata Jenazah
\begin{arabtext}
\noindent
al-ll_ahumma a.gfir lifulAniN (bi-asmihi) wArfa` dara^gatahu fi-y 
al-mahdiyyi-yna, wA_hlufhu fi-y `aqibihi fi-y al-.gAbiri-yna, wA.gfirlanA 
walahu yA rabba al-`Alami-yna, wAfsa.h lahu fi-y qabrihi wanawwirlahu 
fi-yhi.\\
\end{arabtext}
\noindent
\textbf{Artinya}:
\par
\indent
"Ya Allah, ampunilah Fulan (hendaklah ia menyebut namanya), angkatlah 
derajatnya bersama orang-orang yang mendapat petunjuk, berikanlah 
penggantinya bagi orang-orang yang ditinggalkan sesudahnya. Dan ampunilah 
kami dan dia, wahai Rabb sekalian alam. Luaskanlah kuburnya, dan 
sberikanlah cahaya di dalamnya."\\
\par
\noindent
\textbf{Tingkatan Doa dan Sanad}: \textbf{Shahih}: HR. Muslim (no. 920) 
dari SahabaH Ummu Salamah r.a.\\\\
\par
\noindent 
91------------------------------Doa pada Shalat Jenazah
\begin{arabtext}
\noindent
al-ll_ahumma a.gfir lahu, wAr.hamhu, wa`Afihi, wA`fu `anhu, wa'akrim 
nuzulahu, wawassi` mud_halahu, wA.gsilhu bi-almA'i wAl-_t_tal^gi 
wAl-baradi, wanaqqihi mina al-_ha.tAyA kamA naqqa-yta al-_t_tawba 
al-'abya.da mina al-ddanasi, wa'abdilhu dAraN _ha-yraN min dArihi, 
wa'ahlaN _ha-yraN min 'ahlihi, wazaw^gaN _ha-yraN min zaw^gihi, 
wa'ad_hilhu al-^gannaTa, wa-'a`i_dhu min `a_dAbi al-qabri, wamin `a_dAbi 
al-nnAri.\\
\end{arabtext}
\noindent
\textbf{Artinya}:
\par
\indent
"Ya Allah, ampunilah dia (mayit), dan berikanlah rahmat kepadanya, 
selamatkanlah dia (dari siksa kubur), maafkanlah dia dan tempatkanlah di 
tempat yang mulia (Surga), luaskanlah kuburannya, mandikanlah dia dengan 
air, salju, dan air es. Bersihkanlah dia dari segala kesalahan, sebagaimana
Engkau membersihkan baju putih dari kotoran. Berikanlah rumah yang lebih 
baik dari rumahnya (di dunia), berikanlah keluarga yang lebih baik dari 
keluarganya (di dunia), istri (atau suami) yang lebih baik daripada istri 
(atau suami) nya, dan masukkanlah dia ke Surga, serta lindungilah dia dari 
siksa kubur dan dari siksa api Neraka.{\scriptsize 1}
\begin{shaded*}
\noindent
Catatan:\\
- Jika jenazahnya perempuan, huruf hu/hi diganti menjadi haa.\\
- Jika jenazah 2 orang atau lebih, huruf hu diganti menjadi hum, huruf hi 
diganti menjadi him.
\end{shaded*}
\begin{arabtext}
\noindent
al-ll_ahumma a.gfir li.hayyinA wamayyitinA, wa^sAhidinA wa.gA'ibinA, 
wa.sa.gi-yrinA wakabi-yrinA, wa_dakarinA wa-'un_tAnA. al-ll_ahumma man 
'a.hya-ytahu minnA fa'a.hyihi `alY al-'islAmi, wamin tawaffa-ytahu minnA 
fatawaffahu `alY al-'iymAni, al-ll_ahumma lA ta.hrimnA 'a^grahu walA 
tu.dillanA ba`dahu.\\
\end{arabtext}
\noindent
\textbf{Artinya}:
\par
\indent
"Ya Allah, ampuni orang yang masih hidup di antara kami dan yang sudah 
mati, yang takdir dan yang tidak takdir, yang masih kecil maupun dewasa, 
laki-laki maupun perempuan. Ya Allah, orang yang Engkau hidupkan di antara 
kami, hidupkanlah dengan memegang ajaran Islam, dan yang Engkau wafatkan di
antara kami, maka wafatkanlah dalam keadaan beriman. Ya Allah, jangan 
halangi kami untuk memperoleh pahalanya dan jangan sesatkan kami 
sepeninggalnya."{\scriptsize 2}\\
\begin{arabtext}
\noindent
al-ll_ahumma `abduka wAbnu 'amatika 'i.htA^ga 'ilY ra.hmatika, wa-'anta 
.ganiyyuN `an `a_dAbihi, 'in kAna mu.hsinaN fazid fi-y .hasanAtihi, wa-'in 
kAna musi-y'aN fata^gAwaz `anhu.\\
\end{arabtext}
\noindent
\textbf{Artinya}:
\par
\indent
"Ya Allah, ini (adalah) hamba-Mu, anak hamba perempuan-Mu (Hawa), 
membutuhkan rahmat-Mu, sedang Engkau tidak membutuhkan untuk menyiksanya. 
Jika ia berbuat baik, tambahkanlah dalam amalan baiknya, dan jika dia orang
yang bersalah, maafkanlah kesalahannya [kemudian beliau berdoa dengan apa 
yang Allah kehendaki]."{\scriptsize 3}\\
\par
\noindent
\textbf{Tingkatan Doa dan Sanad}:
\begin{enumerate}
\item \textbf{Shahih}: HR. Muslim (no. 963), an-Nasai (IV/73-74), Ahmad 
(VI/23), dan Ibnu Majah (no. 1500) dari Auf bin Malik r.a. Lihat 
\textit{Ahk\^{a}mul Jan\^{a}-iz} (hlm. 157).
\item \textbf{Shahih}: HR. Abu Dawud (no. 3201), at-Tirmidzi (no. 1024), 
Ibnu Majah (no. 1498) dan Ahmad (II/368) dan lainnya. Lihat 
\textit{Ahk\^{a}mul Jan\^{a}-iz} (hlm. 157-158).
\item \textbf{Shahih}: HR. Ath-Thabrani dalam \textit{al-Mu'jamul 
Kab\^{i}r} (XXII/249) tambahan dalam kurung miliknya, dan Al-Hakim (I/359).
Sanadnya shahih. Imam adz-Dzahabi menyetujuinya. Lihat \textit{Ahk\^{a}mul 
Jan\^{a}-iz} (hlm. 159) karya Syaikh al-Albani.\\\\
\end{enumerate}
\par
\noindent 
92------------------------------Doa untuk Jenazah Anak Kecil
\begin{arabtext}
\noindent
al-ll_ahummA 'a`i_dhu min `a_dAbi al-qabri.\\
\end{arabtext}
\noindent
\textbf{Artinya}:
\par
\indent
"Ya Allah, lindungilah dia dari siksa kubur."{\scriptsize 1}\\
\begin{arabtext}
\noindent
al-ll_ahumma a^g`alhu lanA fara.taN wasalafaN wa'a^graN.\\
\end{arabtext}
\noindent
\textbf{Artinya}:
\par
\indent
"Ya Allah, jadikanlah kematian anak ini sebagai simpanan pahala dan amal 
baik serta pahala untuk kami."{\scriptsize 2}\\
\par
\noindent
\textbf{Tingkatan Doa dan Sanad}:
\begin{enumerate}
\item \textbf{Atsar Shahih}: Diriwayatkan oleh Malik/\textit{al-Muwaththa'}
(I/198 no. 18), Ibnu Abi Syaibah dalam al-Mushannaf (II/217), al-Baihaqi 
(IV/9) dan al-Baghawi dalam \textit{Syarhus Sunnah} (V/357) dari perkataan 
Abu Hurairah r.a. Dishahihkan oleh Imam al-Albani. Lihat takhrij 
\textit{Hidayatur Ruw\^{a}h} (II/213 no. 1631).
\item \textbf{Atsar Shahih}: Diriwayatkan oleh Al-Baghawi dalam 
\textit{Syarhus Sunnah} (V/357), Abdurrazzaq (no. 6588) dari perkataan 
al-Hasan al-Bashri dan al-Bukhari meriwayatkan hadits tersebut secara 
\textit{mu'allaq} dalam kitab \textit{al-Jan\^{a}-iz} baba 65: "Membaca 
\textit{F\^{a}tihatul Kit\^{a}b} atas jenazah."\\\\
\end{enumerate}
\par
\noindent 
93------------------------------Doa Ketika Memasukkan Jenazah Ke Liang Kubur
\begin{arabtext}
\noindent
bismi al-ll_ahi wa`alY sunnaTi rasu-wli al-ll_ahi.\\
\end{arabtext}
\noindent
\textbf{Artinya}:
\par
\indent
"\textit{(Bismillahi wa 'ala Sunnati Rasulillah)} Dengan nama Allah dan 
atas Sunnah Rasulullah."\\
\par
\noindent
\textbf{Tingkatan Doa dan Sanad}: \textbf{Shahih}: HR. Abu Dawud (no. 
3213), dan lainnya dengan sanad yang shahih. Dan Ahmad (II/27, 40-41).\\\\
\par
\noindent 
94------------------------------Doa untuk Jenazah Setelah Dimakamkan
\begin{arabtext}
\noindent
al-ll_ahumma a.gfirlahu, al-ll_ahumma _tabbithu.\\
\end{arabtext}
\noindent
\textbf{Artinya}:
\par
\indent
"Ya Allah, ampunilah dia. Ya Allah, teguhkanlah dia."\\
\par
\noindent
\textbf{Tingkatan Doa dan Sanad}: Seusai memakamkan mayat, Rasulullah 
berdiri tepat di atasnya lalu bersabda: "Mintalah ampun kepada Allah untuk 
saudaramu, dan mohonkan agar dia teguh (ketika ditanya oleh dua Malaikan), 
dan sesungguhnya dia sekarang sedang ditanya." \textbf{Shahih}: HR. Abu 
Dawud (no. 3221) dan al-Hakim (I/370), al-Hakim menshahihkannya dan 
disepakati oleh Imam adz-Dzahabi.\\\\
\par
\noindent 
95------------------------------Doa untuk Ta'ziyah (Belasungkawa)
\begin{arabtext}
\noindent
'inna li-ll_ahi mA-'a-_ha_du, walahu mA 'a`.tY, wakullu ^sa-y'iN `indahu 
bi-'a^galiN musammaNY, falta.sbir, walta.htasib.\\
\end{arabtext}
\noindent
\textbf{Artinya}:
\par
\indent
"Sesungguhnya merupakan hak Allah mengambil dan memberikan sesuatu. Segala 
sesuatu di sisi-Nya dibatasi dengan ajal yang ditentukan. Oleh karena itu, 
bersabarlah dan carilah ganjaran dari Allah (dengan sebab musibah itu)."\\
\par
\noindent
\textbf{Tingkatan Doa dan Sanad}: \textbf{Shahih}: HR. Al-Bukhari (no. 
1284), Muslim (no. 923).\\\\
\par
\noindent 
96------------------------------Doa Ketika Ziarah Kubur
\begin{arabtext}
\noindent
al-ssalAmu `alaykum 'ahla al-ddiyAri minA al-mu'mini-yna wAl-muslimi-yna, 
wa-'innA 'in ^sA'a al-ll_ahu bikum lA .hiqu-wna, nas'alu al-ll_aha lanA 
walakumu al-`AfiyaTa.\\
\end{arabtext}
\noindent
\textbf{Artinya}:
\par
\indent
"Semoga kesejahteraan terlimpah atas kalian, wahai para penghuni kubur dari
kaum Mukminin dan kaum Muslimin. Dan, \textit{insya Allah} kami menyusul 
kalian. Kami memohon kepada Allah untuk kami dan kamu sekalian, supaya 
diberi keselamatan dari segala apa yang tidak diinginkan."\\
\par
\noindent
\textbf{Tingkatan Doa dan Sanad}: \textbf{Shahih}: HR. Muslim (no.  975) 
dan Ibnu Majah (no. 1547) dari Buraidah r.a. Lafazh ini menurut Ibnu Majah.
Diriwayatkan juga oleh Muslim (no. 974 [102, 103]) dari Aisyah r.a. dengan 
ada tambahan.\\\\
\par
\noindent 
97--------------------Doa Berlindung terhadap Berbagai Kesusahan, Kesengsaraan, 
dan Hilangnya Kenikmatan
\begin{arabtext}
\noindent
al-ll_ahumma 'inni-y 'a`u-w_du bika min zawAli ni`matika, wata.hawwuli 
`Afiyatika, wafu^gA'aTi niqmatika, wa^gami-y`i sa_ha.tika.\\
\end{arabtext}
\noindent
\textbf{Artinya}:
\par
\indent
"Ya Allah, sesungguhnya aku berlindung kepada-Mu dari hilangnya nikmat-Mu, 
berubahnya afiat (kesejahteraan) dari-Mu, dari hukuman-Mu yang datang 
tiba-tiba, dan dari seluruh kemarahan-Mu."{\scriptsize 1}\\
\begin{arabtext}
\noindent
al-ll_ahumma 'inni-y 'a`u-w_du bika mina al-faqri, wAl-qillaTi, 
wAl-_d_dillaTi, wa'a`uw_du bika min 'an 'a.zlima 'uw 'u.zlama.\\
\end{arabtext}
\noindent
\textbf{Artinya}:
\par
\indent
"Ya Allah, sesungguhnya aku berlindung kepada-Mu dari kefakiran, 
kekurangan, serta kehinaan, dan aku pun berlindung kepada-Mu dari 
menzhalimi ataupun dizhalimi orang lain."{\scriptsize 2}\\
\begin{arabtext}
\noindent
al-ll_ahumma 'inni-y 'a`u-w_du bika mina al-^gu-w`i, fa'i-nnahu bi'sa 
al-.d.da^gi-y`u, wa'a`u-w_du bika mina al-_hiyAnaTi, fa'i-nnahA bi'sati 
al-bi.tAnaTu.\\
\end{arabtext}
\noindent
\textbf{Artinya}:
\par
\indent
"Ya Allah, sesungguhnya aku berlindung kepada-Mu dari kelaparan, karena ia 
adalah seburuk-buruk teman berbaring. Aku juga berlindung kepada-Mu dari 
pengkhianatan, karena ia merupakan seburuk-buruk kawan."{\scriptsize 3}\\
\par
\noindent
\textbf{Tingkatan Doa dan Sanad}:
\begin{enumerate}
\item \textbf{Shahih}: HR. Muslim (no. 2739 [96]) dan Abu Dawud (no. 1545) 
dari Abdullah bin Umar r.a.
\item \textbf{Shahih}: HR. An-Nasai (VIII/261), Abu Dawud (no. 1544) dari 
Abu Hurairah r.a.
\item \textbf{Shahih}: HR. Abu Dawud (no. 1547), an-Nasai (VIII/263), serta
Ibnu Majah (no. 3354). Lihat \textit{Shah\^{i}h an-Nasai} (III/1112, no. 
5051).\\\\
\end{enumerate}
\par
\noindent 
98------------------------------Doa Diselamatkan dari Bencana dan Kehinaan
\begin{arabtext}
\noindent
al-ll_ahumma 'inni-y 'a`u-w_du bika min ^gahdi al-balA'i, wadaraki 
al-^s^saqA'i, wasu-w'i al-qa.dA'i, wa^samAtaTi al-'a`dA'i.\\
\end{arabtext}
\noindent
\textbf{Artinya}:
\par
\indent
"Ya Allah, sesungguhnya aku berlindung kepada-Mu dari susahnya bala 
(bencana) tertimpa kesengsaraan, keburukan qadha, dan kegembiraan para 
musuh."\\
\par
\noindent
\textbf{Tingkatan Doa dan Sanad}: \textbf{Shahih}: HR. Al-Bukhari (no. 
6347, 6616), Muslim (no. 2707).\\\\
\par
\noindent 
99------------------------------Doa Berlindung dari Kebinasaan dan Kehancuran
\begin{arabtext}
\noindent
al-ll_ahumma 'inni-y 'a`u-w_du bika mina al-ttaraddi-y, wAl-hadmi, 
wAl-.garaqi, wAl-.hari-yqi, wa'a`u-w_du bika 'an yata_habba.taniya 
al-^s^sa-y.tAnu `inda al-ma-wti, wa'a`u-w_du bika 'an 'amu-wta fi-y 
sabi-ylika mudbiraN, wa'a`u-w_du bika 'an 'amu-wta ladi-y.gaN.\\
\end{arabtext}
\noindent
\textbf{Artinya}:
\par
\indent
"Ya Allah, sesungguhnya aku berlindung kepada-Mu dari kebinasaan (jatuh 
dari tempat yang tinggi), kehancuran (tertimpa sesuatu), tenggelam, 
kebakaran dan aku berlindung kepada-Mu dari dikuasai syaitan pada saat 
menjelang mati, dan aku berlindung kepada-Mu dari mati dalam keadaan 
berpaling dari jalan-Mu, dan aku berlindung kepada-Mu dari mati dalam 
keadaan tersengat."\\
\par
\noindent
\textbf{Tingkatan Doa dan Sanad}: \textbf{Shahih}: HR. An-Nasai (VIII/282), 
Abu Dawud (no. 1552) dari Abu Yasar r.a dan \textit{Shah\^{i}h an-Nasai} 
(III/1123, no. 5104). Lafazh ini milik an-Nasai.\\\\
\par
\noindent 
100------------------------------Berlindung dari Fitnah dan Berbagai Keburukan
\begin{arabtext}
\noindent
al-ll_ahumma 'inni-y 'a`u-w_du bika min fitnaTi al-nnAri wa`a_dAbi 
al-nnAri, wafitnaTi al-qabri wa`a_dAbi al-qabri, wa^sarri fitnaTi 
al.ginaYi, wa^sarri fitnaTi al-faqri, al-ll_ahumma 'inni-y 'a`u-w_du bika 
min ^sarri fitnaTi al-masi-y.hi al-dda^g^gAli, al-ll_ahumma a.gsil qalbi-y 
bimA'i al-_t_tal^gi wAl-baradi, wanaqqi qalbi-y mina al-_ha.tAyA, kamA 
naqqa-yta al-_t_tawba al-'abya.da mina al-ddanasi, wabA`id bayni-y 
waba-yna _ha.tAyAya, kamA bA`adta ba-yna al-ma^sriqi wAl-ma.gribi. 
al-ll_ahumma 'inni-y 'a`u-w_du bika mina al-kasali, wAl-ma'_tami, 
wAl-ma.grami.\\
\end{arabtext}
\noindent
\textbf{Artinya}:
\par
\indent
"Ya Allah, sesungguhnya aku berlindung kepada-Mu dari fitnah dan adzab 
Neraka, fitnah dan adzab kubur, keburukan fitnah kekayaan dan keburukan 
fitnah kefakiran. Ya Allah, sesungguhnya aku berlindung kepada-Mu dari 
kejahatan fitnah Dajjal. Ya Allah, bersihkanlah hatiku dengan salju dan air
es, serta sucikanlah hatiku dari tiap kesalahan sebagaimana Engkau 
menyucikan pakaian putih dari kotoran. Dan jauhkanlah antara diriku dengan 
kesalahan-kesalahanku itu sebagaimana Engkau menjauhkan timur dan barat. Ya
Allah, sesungguhnya aku berlindung kepada-Mu dari kemalasan, perbuatan 
dosa, dan utang."{\scriptsize 1}\\
\begin{arabtext}
\noindent
al-ll_ahumma 'inni-y 'a`u-w_du bika mina al-`a^gzi, wAl-kasali, wAl-^gubni,
wAl-bu_hli, wAl-harami, wAl-qaswaTi, wAl-.gaflaTi, wAl-`aylaTi, 
wAl-_d_dillaTi, wAl-maskanaTi, wa'a`u-w_du bika mina al-faqri, wAl-kufri, 
wAl-fusu-wqi, wAl-^s^siqAqi, wAl-nnifAqi, wAl-ssum`aTi, wAl-rriyA'i, 
wa'a`u-w_du bika mina al-.s.samami, wAl-bakami, wAl-^gunu-wni, 
wAl-^gu_dAmi, wAl-bara.si, wasayyi -'i al-'asqAmi.\\
\end{arabtext}
\noindent
\textbf{Artinya}:
\par
\indent
"Ya Allah, aku berlindung kepada-Mu dari kelemahan, kemalasan, sifat yang
pengecut, kekikiran, pikun, kekerasan hati, lalai, berat tanggungan,
kehinaan, dan kerendahan. Dan aku berlindung kepada-Mu dari kemiskinan,
kekufuran, kefasikan, perpecahan, kemunafikan, \textit{sum'ah} (amalnya
ingin didengar orang), \textit{riya'} (amalnya ingin dilihat orang) serta
aku berlindung kepada-Mu dari tuli, bisu, gila, sakit lepra, belang, dan
dari keburukan berbagai jenis penyakit."{\scriptsize 2}\\
\begin{arabtext}
\noindent
al-ll_ahumma 'inni-y 'a`u-w_du bika mina al-^gubni, wa'a`u-w_du bika mina
al-bu_hli, wa'a`u-w_du bika min 'an 'uradda 'i-lY 'ar_dali al-`umuri,
wa'a`u-w_du bika min fitnaTi al-ddunyA wa`a_dAbi al-qabri.\\
\end{arabtext}
\noindent
\textbf{Artinya}:
\par
\indent
"Ya Allah, sesungguhnya aku memohon perlindungan kepada-Mu dari sifat yang
pengecut, aku berlindung kepada-Mu dari sifat kikir, dan aku berlindung
kepada-Mu dari dikembalikan kepada umur yang paling hina (pikun), serta aku
berlindung kepada-Mu dari fitnah dunia dan adzab kubur."{\scriptsize 3}\\
\begin{arabtext}
\noindent
al-ll_ahumma qini-y ^sarra nafsi-y, wA`zim li-y `alY 'ar^sadi 'amri-y, 
al-ll_ahumma a.gfirli-y mA 'asrartu wamA 'a`lantu, wamA 'a_h.ta'tu wamA 
`amadtu, wamA `alimtu wamA ^gahiltu.\\
\end{arabtext}
\noindent
\textbf{Artinya}:
\par
\indent
"Ya Allah, lindungi aku dari kejahatan nafsuku dan kuatkan diriku dalam hal
sebaik-baik urusanku. Ya Allah, berilah ampunan kepadaku atas segala yang
aku sembunyikan dan segala yang aku tampakkan, juga atas apa yang tidak aku
sengaja dan yang memang aku sengaja, serta apa yang diketahui dan yang 
tidak kuketahui."{\scriptsize 4}\\
\begin{arabtext}
\noindent
al-ll_ahumma 'inni-y 'a`u-w_du bika mina al-`a^gzi, wAl-kasali, wAl-^gubni, 
wAl-harami, wAl-bu_hli, wa'a`u-w_du bika min `a_dAbi al-qabri, wamin 
fitnaTi al-ma.hyA wAl-mamAti.\\
\end{arabtext}
\noindent
\textbf{Artinya}:
\par
\indent
"Ya Allah, sungguh aku berlindung kepada-Mu dari kelemahan, kemalasan, 
sifat pengecut, pikun, dan kekikiran. Aku juga berlindung kepada-Mu dari 
adzab kubur serta fitnah kehidupan dan kematian."{\scriptsize 5}\\
\par
\noindent
\textbf{Tingkatan Doa dan Sanad}:
\begin{enumerate}
\item \textbf{Shahih}: HR. Al-Bukhari (no. 6368, 6376, 6377), Muslim (no. 
589 [129]), Ahmad (VI/57, 207), Abu Dawud (no. 1543), at-Tirmidzi (no. 
3495), an-Nasai (I/51, 176, dan VIII/262, 266), al-Hakim (I/541) dan 
lainnya, dari Aisyah r.a.
\item \textbf{Shahih}: HR. Al-Hakim (I/530) dan Ibnu Hibban (no. 2446 - 
\textit{Maw\^{a}riduzh Zh\^{a}m-\^{a}n} dan no. 1019-\textit{at-Ta'liqatul 
His\^{a}n}) dari Anas bin Malik r.a. Lihat \textit{Shah\^{i}hul J\^{a}mi'} 
(no. 1285) dan \textit{Irw\^{a}-ul Ghal\^{i}l} (III/357). Dishahihkan 
al-Hakim dan disetujui adz-Dzahabi.
\item \textbf{Shahih}: HR. Al-Bukhari (no. 2822, 6374) /
\textit{Fathul B\^{a}ri} (XI/181). Doa ini boleh dibaca sebelum atau 
sesudah salam dari shalat wajib. Lihat \textit{Fathu Dzil Jal\^{a}li wal 
Ikr\^{a}m Syarah Bul\^{u}ghul Mar\^{a}m} (III/509-510), Syarah Syaikh 
Utsaimin.
\item \textbf{Shahih}: HR. Ahmad (IV/444), Ibnu Hibban (no. 896 - 
\textit{at-Ta'l\^{i}q\^{a}tul His\^{a}n}) dan al-Hakim (I/510)--Dishahihkan
oleh al-Hakim, dan adz-Dzahabi menyepakatinya. Imam Haitsami berkata dalam 
\textit{Majma'uz Zaw\^{a}-id} (X/181): "Para perawi hadits ini 
\textit{shahih} (terpercaya)."
\item \textbf{Shahih}: HR. Al-Bukhari (no. 2823, 6367), Muslim (no. 2706) 
dari Anas bin Malik r.a.\\\\
\end{enumerate}
\par
\noindent 
101------------------------------Berlindung dari Segala Penyakit
\begin{arabtext}
\noindent
al-ll_ahumma 'inni-y 'a`u-w_du bika mina al-bara.si, wAl-^gunu-wni, 
wAl-^gu_dAmi, wamin sayyi -'i al-'asqAmi.\\
\end{arabtext}
\noindent
\textbf{Artinya}:
\par
\indent
"Ya Allah, sungguh aku berlindung kepada-Mu dari penyakit belang, gila,
lepra, dan dari keburukan segala macam penyakit."\\
\par
\noindent
\textbf{Tingkatan Doa dan Sanad}: \textbf{Shahih}: HR. Abu Dawud (no. 
1554), an-Nasai (VIII / 270), Ahmad (III/192), Ibnu Hibban (no. 1013 - 
\textit{at-Ta'liqatul His\^{a}n}) dan lainnya, dari Anas bin Malik r.a.
Lihat \textit{Shah\^{i}h al-J\^{a}mi-us Shagh\^{i}r} (no. 1281).\\\\
\par
\noindent 
102------------------------------Berlindung dari Berbuat Buruk
\begin{arabtext}
\noindent
al-ll_ahumma 'inni-y 'a`u-w_du bika min ^sarri sam`i-y, wamin ^sarri 
ba.sari-y, wamin ^sarri lisAni-y, wamin ^sarri qalbi-y, wamin ^sarri 
miniyyi-y.\\
\end{arabtext}
\noindent
\textbf{Artinya}:
\par
\indent
"Ya Allah, sungguh aku berlindung kepada-Mu dari keburukan yang ada di 
pendengaranku, kejahatan pengelihatanku, keburukan lidahku, keburukan 
hatiku, dan keburukan air maniku."{\scriptsize 1}\\
\begin{arabtext}
\noindent
al-ll_ahumma ^gannibni-y munkarAti al-'a_hlAqi, wAl-'ahwA'i, wAl-'a`mAli, 
wAl-'adwA'i.\\
\end{arabtext}
\noindent
\textbf{Artinya}:
\par
\indent
"Ya Allah, jauhkanlah aku dari berbagai kemunkaran akhlak, hawa nafsu, amal 
perbuatan, dan, segala macam penyakit."{\scriptsize 2}\\
\begin{arabtext}
\noindent
al-ll_ahumma 'inni-y 'a`u-w_dubika min ^sarri mA `amiltu, wamin ^sarri mA 
lam 'a`mal.\\
\end{arabtext}
\noindent
\textbf{Artinya}:
\par
\indent
"Ya Allah, sesungguhnya aku berlindung kepada-Mu dari keburukan apa yang 
telah aku kerjakan dan dari keburukan apa yang belum aku kerjakan."
{\scriptsize 3}\\
\par
\noindent
\textbf{Tingkatan Doa dan Sanad}:
\begin{enumerate}
\item \textbf{Shahih}: HR. Abu Dawud (no. 1551), at-Tirmidzi (no. 3492), 
an-Nasai (VIII/259-260) dari Syakal bin Humaid r.a. Lihat 
\textit{Shah\^{i}h al-J\^{a}mi-us Shaghir} (no. 1292).
\item \textbf{Shahih}: HR. Ibnu Hibban (no. 956-\textit{at-Ta'liqatul 
His\^{a}n}) al-Hakim (I/532), dan dia berkata: "Hadits ini shahih sesuai 
syarat Muslim." Dishahihkannya dan disepakati oleh adz-Dzahabi. Lihat 
\textit{Shah\^{i}h al-Adzk\^{a}r} (1187/938).
\item \textbf{Shahih}: HR. Muslim (no. 2716) dan selainnya.\\\\
\end{enumerate}
\par
\noindent 
103------------------------------Berlindung dari Fitnah Dajjal
\begin{arabtext}
\noindent
man .hafi.za `a^sra ^AyAtiN min 'awwali su-wraTi alkahfi `u.sima mina 
al-dda^g^gAli.\\
\end{arabtext}
\noindent
\textbf{Artinya}:
\par
\indent
"Barang siapa menghafal sepuluh ayat dari permulaan surah Al-Kahfi, maka ia
terpelihara dari fitnah Dajjal."{\scriptsize 1}\\
\indent
Begitu juga memohon perlindungan kepada Allah dari fitnah ad-Dajjal setelah
tasyahud akhir sebelum salam.{\scriptsize 2}\\
\par
\noindent
\textbf{Tingkatan Doa dan Sanad}:
\begin{enumerate}
\item \textbf{Shahih}: HR. Muslim (no. 809), al-Baihaqi (III/249), dan 
al-Hakim (II/368) dari Abu Darda r.a.
\item \textbf{Shahih}: HR. Al-Bukhari (no. 1377) dan Muslim (no. 588).\\\\
\end{enumerate}
\par
\noindent 
104------------------------------Doa untuk Keselamatan
\begin{arabtext}
\noindent
al-ll_ahumma a.gfirli-y, wAhdini-y, wArzuqni-y, wa`Afini-y, 'a`u-w_du 
bi-al-ll_ahi min .di-yqi al-maqAmi ya-wma al-qiyAmaTi.\\
\end{arabtext}
\noindent
\textbf{Artinya}:
\par
\indent
"Ya Allah , ampunilah aku, berikanlah petunjuk kepadaku, karuniakanlah 
rizki kepadaku, berikan keselamatan bagiku. Aku berlindung kepada Allah 
SWT. dari kesempitan tempat berdiri kelak pada hari Kiamat."\\
\par
\noindent
\textbf{Tingkatan Doa dan Sanad}: \textbf{Hasan Shahih}: HR. Abu Dawud (no.
766), an-Nasai (III/209), Ibnu Majah (no. 1356), dan yang lainnya. Lihat
\textit{Shah\^{i}h Sunan Abi Dawud} (III/352-353, no. 742).\\\\
\par
\noindent 
105------------------------------Doa Mendapatkan Kebaikan Dunia dan Akhirat
\begin{arabtext}
\noindent
al-ll_ahumma 'inni-y 'as'aluka al-`AfiyaTi, fiy al-ddunyA wAl-^A_hiraTi.\\
\end{arabtext}
\noindent
\textbf{Artinya}:
\par
\indent
"Ya Allah, sesungguhnya aku memohon kepada-Mu \textit{'afiat} (dijauhkan 
dari petaka) di dunia dan di akhirat."{\scriptsize 1}\\
\begin{arabtext}
\noindent
al-ll_ahumma 'inni-y 'as'aluka al-^gannaTa wa'a`u-w_du bika mina al-nnAri.
\\
\end{arabtext}
\noindent
\textbf{Artinya}:
\par
\indent
"Ya Allah, aku memohon kepada-Mu agar dimasukkan ke dalam Surga dan aku 
berlindung kepada-Mu dari siksa Neraka."{\scriptsize 2}\\
\begin{arabtext}
\noindent
al-ll_ahumma ^AtinA fi-y al-ddunyA .hasanaTaN, wafi-y al-^A_hiraTi 
.hasanaTaN, waqinA `a_dAba al-nnAri.\\
\end{arabtext}
\noindent
\textbf{Artinya}:
\par
\indent
"Ya Allah, berikanlah kebaikan kepada kami di dunia dan kebaikan di 
akhirat. (Ya Allah,) lindungilah kami dari adzab Neraka."{\scriptsize 3}
\\
\par
\noindent
\textbf{Tingkatan Doa dan Sanad}:
\begin{enumerate}
\item \textbf{Shahih}: HR. Ahmad (I/209), al-Bukhari dalam 
\textit{al-Adabul Mufard} (no. 726), dan at-Tirmidzi (no. 3514).
\item \textbf{Shahih}: HR. Abu Dawud (no. 792), Ibnu Majah (no. 910), dan 
Ibnu Khuzaimah (no. 725). Dishahihkan oleh Imam Ibnu Khuzaimah, Imam 
an-Nawawi, dan Syaikh al-Albani.
\item \textbf{Shahih}: HR. Al-Bukhari (no. 6389) dan Muslim (no. 2690).\\\\
\end{enumerate}
\par
\noindent 
106------------------------------Doa Mohon Diperbaiki Urusan Dunia dan Akhirat
\begin{arabtext}
\noindent
al-ll_ahumma 'a.sli.h li-y di-yni-y alla_di-y huwa `i.smaTu 'amri-y, 
wa'a.sli.h li-y dunyAya allati-y fi-yhA ma`A ^si-y, wa'a.sli.h li-y 
^A_hirati-y Aallati-y fi-yhA ma`Adi-y, wA^g`ali al-.hayATa ziyAdaTaN li-y 
fi-y kulli _ha-yriN, wA^g`ali al-mawta rA.haTaN li-y min kulli ^sarriN.\\
\end{arabtext}
\noindent
\textbf{Artinya}:
\par
\indent
"Ya Allah, perbaikilah agamaku bagiku yang ia merupakan benteng pelindung 
bagi urusanku. Dan perbaikilah duniaku bagiku, yang ia menjadi tempat 
hidupku. Serta perbaikilah akhiratku yang ia menjadi tempat kembaliku. 
Jadikanlah kehidupan ini sebagai tambahan bagiku dalam setipa kebaikan, 
serta jadikanlah kematian sebagai kebebasan bagiku dari segala kejahatan."
\\
\par
\noindent
\textbf{Tingkatan Doa dan Sanad}: \textbf{Shahih}: HR. Muslim (no. 2720) 
dari Abu Hurairah.\\\\
\par
\noindent 
107------------------------------Doa untuk Kebaikan Diri
\begin{arabtext}
\noindent
al-ll_ahumma a.gfir li-y, wAr.hamni-y, wAhdini-y, wa`Afini-y, wArzuqni-y.\\
\end{arabtext}
\noindent
\textbf{Artinya}:
\par
\indent
"Ya Allah, ampuni dan sayangilah aku, berikan petunjuk kepadaku, limpahkan 
\textit{'afiat} (kesejahteraan hidup) kepadaku, serta karuniakanlah rizki 
kepadaku."\\
\par
\noindent
\textbf{Tingkatan Doa dan Sanad}: \textbf{Shahih}: HR. Muslim (no. 2696, 
2697), Ibnu Majah (no. 3845), Ahmad (III/472, VI/394).\\\\
\par
\noindent 
108------------------------------Doa Mendapatkan Kenikmatan
\begin{arabtext}
\noindent
al-ll_ahumma matti`ni-y bisam`i-y waba.sari-y, wA^g`alhumA al-wAri_ta 
minni-y, wAn.surni-y `alY man ya.zlimuni-y, wa_hu_d minhu bi_ta'ri-y.\\
\end{arabtext}
\noindent
\textbf{Artinya}:
\par
\indent
"Ya Allah, berikanlah manfaat kepadaku melalui pendengaran dan pandanganku,
jadikanlah keduanya sebagai pewarisku (yakni, jadikanlah keduanya sehat 
sampai mati), (ya Allah) tolonglah aku atas orang yang berbuat zhalim 
terhadapku, dan hukumlah dia sebagai balasanku atas dirinya."\\
\par
\noindent
\textbf{Tingkatan Doa dan Sanad}: \textbf{Hasan}: HR. At-Tirmidzi (no. 
3604) dan \textit{Shah\^{i}h at-Tirmidzi} (III/188, no. 2854). Juga 
al-Hakim (I/523), serta dishahihkan olehnya lalu disepakati adz-Dzahabi. 
Sanadnya hasan.\\\\
\par
\noindent 
109------------------------------Doa Mohon Keberkahan
\begin{arabtext}
\noindent
al-ll_ahumma 'ak_tir mAli-y wawaladi-y, wabArik li-y fiymA 'a`.ta-ytani-y, 
(wa'a.til .hayAti-y `alY .tA`atika, wa'a.hsin `amali-y, wA.gfirli-y).\\
\end{arabtext}
\noindent
\textbf{Artinya}:
\par
\indent
"Ya Allah, perbanyaklah harta dan juga anakku, serta berilah berkah 
kepadaku atas apa yang telah Engkau karuniakan kepadaku.{\scriptsize 1}
[Panjangkan kehidupanku pada ketaatan terhadap-Mu, perbaikilah amal 
perbuatanku, dan berikan ampunan kepadaku]."{\scriptsize 2}\\
\par
\noindent
\textbf{Tingkatan Doa dan Sanad}:
\begin{enumerate}
\item \textbf{Shahih}: HR. Al-Bukhari (no. 6378-6381), Muslim (no. 2480, 
2481) dari Ummu Sulaim r.a.
\item \textbf{Shahih}: HR. Al-Bukhari dalam \textit{al-Adabul Mufrad} (no. 
653). Dishahihkan oleh al-Albani dalam \textit{Silsilah Ah\^{a}d\^{i}ts 
ash-Shah\^{i}hah} (no. 2241) dan \textit{Shah\^{i}h al-Adabil Mufrad} 
(hlm. 244, no. 508).\\\\
\end{enumerate}
\par
\noindent 
110------------------------------Doa agar Diberi Kemudahan ketika Dihisab
\begin{arabtext}
\noindent
al-ll_ahumma .hAsibni-y .hisAbaN yasi-yraN.\\
\end{arabtext}
\noindent
\textbf{Artinya}:
\par
\indent
"Ya Allah, hisablah diriku dengan hisab yang mudah."\\
\par
\noindent
\textbf{Tingkatan Doa dan Sanad}: \textbf{Shahih}: HR. Ahmad (VI/48) dan 
Al-Hakim (I/255).  Dia berkata: "Hadits ini shahih sesuai syarat Muslim." 
Dan disepakati oleh Imam adz-Dzahabi.\\\\
\par
\noindent 
111------------------------Doa Memohon Surga dan Berlindung dari Neraka
\begin{arabtext}
\noindent
al-ll_ahumma 'inni-y 'as'aluka al-^gannaTa, wa'a`u-w_dubika mina al-nnAri.
\\
\end{arabtext}
\noindent
\textbf{Artinya}:
\par
\indent
"Ya Allah, sesungguhnya aku memohon Surga kepada-Mu dan aku memohon 
perlindungan kepada-Mu dari Neraka." [Dibaca 3x] {\scriptsize 1}\\
\begin{arabtext}
\noindent
al-ll_ahumma 'inni-y 'as'aluka bi-'anna laka al-.hamdu, lA 'il_aha 'illA 
'anta wa.hdaka lA ^sari-yka laka al-mannAnu, yA badi-y.ha al-ssamAwAti 
wAl-'ar.di, yA_dA al-^galAli wAl-'ikrAmi, yA .hayyu yA qayyu-wmu, 'inni-y 
'as'aluka (al-^gannaTa, wa'a`u-w_du bika mina al-nnAri).\\
\end{arabtext}
\noindent
\textbf{Artinya}:
\par
\indent
"Ya Allah, sesungguhnya aku memohon kepada-Mu, karena segala puji hanyalah 
bagi-Mu, tidak ada ilah yang berhak untuk diibadahi dengan benar kecuali 
Engkau, tidak ada sekutu bagi-Mu, Yang Maha Pemberi, Pencipta langit dan 
bumi, wahai Rabb Pemilik keagungan serta kemuliaan, wahai Rabb Yang 
Mahahidup lagi Maha Berdiri sendiri, sesungguhnya aku memohon kepada-Mu 
[Surga dan aku berlindung kepada-Mu dari Neraka]."{\scriptsize 2}\\
\begin{arabtext}
\noindent
al-ll_ahumma rabba ^gibrA'iyla, wami-ykA'iyla, warabba 'isrAfi-yla, 
'a`u-w_du bika min .harri al-nnAri wamin `a_dAbi al-qabri.\\
\end{arabtext}
\noindent
\textbf{Artinya}:
\par
\indent
"Ya Allah, Rabb Malaikat Jibril, Mika-il dan Rabb Malaikat Israfril, aku 
berlindung kepada-Mu dari panasnya api Neraka dan adzab kubur."
{\scriptsize 3}\\
\par
\noindent
\textbf{Tingkatan Doa dan Sanad}:
\begin{enumerate}
\item \textbf{Shahih}: HR. At-Tirmidzi (no. 2572), an-Nasai (VIII/279), 
Ibnu Majah (no. 4340), Ahmad (III/117, 141, 155), al-Hakim (I/534-535).
\item \textbf{Shahih}: HR. Abu Dawud (no. 1495), an-Nasai (III/52), Ibnu 
Majah (no. 3858), Ahmad (III/158, 245) dan Ibnu Mandah dalam 
\textit{Kitabut Tauhid} (no. 355) dan tambahan dalam kurung miliknya dari 
Anas bin Malik r.a. Dan juga at-Tirmidzi (no. 3475) dari Abdullah bin 
Buraidah al-Aslami r.a. dari ayahnya.
\item \textbf{Hasan}: HR. An-Nasai (VIII/278) dari Aisyah r.a. Lihat 
\textit{Silsilah Ah\^{a}d\^{i}ts ash-Shah\^{i}hah} (no. 1544).\\\\
\end{enumerate}
\par
\noindent 
112------------------------------Doa agar Diberi Rizki yang Halal, Sifat Qana'ah, dan
Keberkahan
\begin{arabtext}
\noindent
al-ll_ahumma qanni`ni-y bimA razaqtani-y, wabArik li-y fi-yhi, wA_hluf `alY 
kulli .gA'ibaTiN li-y bi_ha-yriN.\\
\end{arabtext}
\noindent
\textbf{Artinya}:
\par
\indent
"Ya Allah, jadikan aku merasa \textit{qana'ah} (cukup, puas, rela) terhadap
segala yang telah Engkau rizkikan kepadaku, dan berilah berkah kepadaku di 
dalamnya dan gantikan bagiku semua yang hilang dariku dengan yang lebih 
baik."\\
\par
\noindent
\textbf{Tingkatan Doa dan Sanad}: \textbf{Shahih}: HR. Al-Hakim (I/510) dan
dishahihkannya serta disepakati oleh adz-Dzahabi dari Ibnu Abbas r.a.\\\\
\par
\noindent 
113------------------------------Doa Mohon Ampunan dan Kasih Sayang
\begin{arabtext}
\noindent
rabbi a.gfirli-y, watub `alayya, 'innaka 'anta al-ttawwAbu al-.gafu-wru.\\
\end{arabtext}
\noindent
\textbf{Artinya}:
\par
\indent
"Ya Rabbku, ampunilah aku, terimalah taubatku, sesungguhnya Engkau adalah 
Yang Maha Penerima taubat lagi Yang Maha Pengampun."{\scriptsize 1}\\
\begin{arabtext}
\noindent
al-ll_ahumma 'inni-y .zalamtu nafsi-y .zulmaN ka_ti-yraN, walA ya.gfiru 
al-_d_dunu-wba 'illA 'anta, fA.gfir li-y ma.gfiraTaN min `indika, 
wAr.hamni-y, 'innaka 'anta al-.gafu-wru al-rra.hi-ymu.\\
\end{arabtext}
\noindent
\textbf{Artinya}:
\par
"Ya Allah, sesungguhnya aku telah menzhalimi diriku dengan kezhaliman yang 
banyak, dan tidak ada yang dapat mengampuni dosa melainkan Engkau. Oleh 
karena itu ampunilah aku dengan ampunan yang datang dari sisi-Mu, dan 
rahmatilah aku, sesungguhnya Engkau adalah Yang Maha Pengampun lagi Maha 
Penyayang."{\scriptsize 2}\\
\begin{arabtext}
\noindent
al-ll_ahumma 'inni-y 'as'aluka yA Aal-ll_ahu, bi-'annaka al-wA.hidu 
al-'a.hadu al-.s.samadu, alla_di-y lam yalid walam yu-wlad, walam yakun 
lahu kufu-waN 'a.haduN, 'an ta.gfira li-y _dunu-wbi-y, 'innaka 'anta 
al-.gafu-wru al-rra.hi-ymu.\\
\end{arabtext}
\noindent
\textbf{Artinya}:
\par
"Ya Allah, sesungguhnya aku memohon kepada-Mu ya Allah, karena Engkau 
adalah satu-satunya Yang Maha Esa, yang bergantung kepada-Mu seluruh 
makhluk, yang tidak beranak dan tidak pula diperanakkan, serta tidak ada 
seorang pun yang sebanding dengan-Nya, agar Engkau memberikan ampunan 
kepadaku atas dosa-dosaku, sesungguhnya Engkau Maha Pengampun lagi Maha 
Penyayang."{\scriptsize 3}\\
\begin{arabtext}
\noindent
al-ll_ahumma a.gfir li-y _ha.ti-y'ati-y, wa^gahli-y, wa-'isrAfi-y fi-y 
'amri-y, wamA 'anta 'a`lamu bihi minni-y, al-ll_ahumma a.gfir liy haz li-y,
wa^giddi-y, wa_ha.ta'i-y, wa`amdi-y, wakulla _d_alika `indi-y.\\
\end{arabtext}
\noindent
\textbf{Artinya}:
\par
"Ya Allah, ampunilah aku dari setiap kesalahanku, setiap kebodohanku, serta
sikap berlebihan dalam urusanku, dan atas segala sesuatu yang lebih Engkau
ketahui daripada diriku ini. Ya Allah, berilah ampunan kepadaku atas canda 
dan keseriusanku, juga kekeliruan dan kesengajaanku, dan semuanya itu ada 
pada diriku."{\scriptsize 4}\\
\begin{arabtext}
\noindent
al-ll_ahumma .tahhir ni-y mina al-_d_dunu-wbi wAl-_ha.tAyA, al-ll_ahumma 
naqqini-y minhA, kamA yunaqqY al-_t_tawbu al-'abya.du mina al-ddanasi, 
al-ll_ahumma .tahhir ni-y bi-al-_t_tal^gi, wAl-baradi, wAl-mA'i al-bAridi.
\\
\end{arabtext}
\noindent
\textbf{Artinya}:
\par
"Ya Allah, sucikanlah aku dari berbagai dosa dan kesalahan. Ya Allah, 
bersihkan diriku darinya sebagaimana dibersihkannya kain putih dari 
kotoran. Ya Allah, sucikanlah diriku dengan salju, embun, dan air yang 
dingin."{\scriptsize 5}\\
\par
\noindent
\textbf{Tingkatan Doa dan Sanad}: 
\begin{enumerate}
\item Abdullah bin Umar berkata: "Aku menghitung kalimat yang diucapkan 
Rasulullah: '\textit{Rabbighfirl\^{i} watub 'alayya innaka antat 
taww\^{a}bul ghaf\^{u}r}' dalam satu majelis sebanyak seratus kali." 
\textbf{Hasan Shahih}: HR. Abu Dawud (no. 1516), at-Tirmidzi (no. 3434), 
Ibnu Majah (no. 3814). Lafazhnya milik at-Tirmidzi, dan dia menyatakan: 
"Hadits \textit{hasan shahih gharib}." Lihat \textit{Shah\^{i}h 
al-J\^{a}mi-us Shaghir} (no. 3486) dan \textit{Silsilah Ah\^{a}d\^{i}ts 
ash-Shah\^{i}hah} (no. 556).
\item \textbf{Shahih}: HR. Al-Bukhari (no. 834), Bab "ad-Du'\^{a}' qabla 
Sal\^{a}m" dan Muslim (no. 2705 [48]) dari Abu Bakar ash-Shiddiq r.a. Doa 
ini dibaca setelah tasyahud akhir sebelum salam. 
\item \textbf{Shahih}: HR. An-Nasai dengan lafazhnya (III/52), Ahmad 
(IV/338). Lihat \textit{Shah\^{i}h an-Nasai} (I/279). Pada akhir riwayat, 
Nabi SAW. bersabda: "Allah telah mengampuni dosanya." - Beliau 
mengucapkannya tiga kali. 
\item \textbf{Shahih}: HR. Al-Bukhari (no. 6399)/\textit{Fathul B\^{a}ri} 
(XI/196), dari Abu Musa al-Asy'ari r.a.
\item \textbf{Shahih}: HR. Muslim (no. 476 [204]), an-Nasai (I/198, 199) 
dan at-Tirmidzi (no. 3547) dari Abdullah bin Abi Aufa. Lafazh ini milik 
an-Nasai. Lihat \textit{Shah\^{i}h an-Nasai} (I/86).\\\\
\end{enumerate}
\par
\noindent 
114------------------------------Doa Agar Terhindar dari Segala Kejahatan
\begin{arabtext}
\noindent
al-ll_ahumma rabba al-ssamAwAti (al-ssab`i), warabba al-'ar.di, warabba 
al-`ar^si al-`a.zi-ymi, rabbanA warabba kulli ^sa-y'iN, fAliqa al-.habbi 
wAl-nnawY, wamunzila al-ttawrATi wAl-'in^gi-yli wAl-furqAti, 'a`u-w_du 
bika min ^sarri kulli ^sa-y'iN 'anta ^A_hi_duN binA.siyatihi, al-ll_ahumma 
'anta al-'awwalu fala-ysa qablaka ^sa-y'uN, wa-'anta al-^A_hiru fala-ysa 
ba`daka ^sa-y'uN, wa'anta al-.z.zAhiru fala-ysa fa-wqaka ^sa-y'uN, wa'anta 
al-bA.tinu fala-ysa du-wnaka ^sa-y'uN, 'iq.di `annA al-dda-yna, wa'a.gninA 
mina al-faqri.\\
\end{arabtext}
\noindent
\textbf{Artinya}:
\par
\indent
“Ya Allah, Rabb langit [yang tujuh] dan Rabb bumi, Rabb Arsy yang agung, 
Rabb kami dan Rabb segala sesuatu, Pembelah biji serta benih, Rabb yang 
menurunkan Taurat, Injil, dan \textit{al-Furqan} (al-Qur’an), aku 
berlindung kepada-Mu dari kejahatan segala yang ubun-ubunnya Engkau pegang.
Ya Allah, Engkau yang paling pertama, tidak ada sesuatu pun sebelum-Mu, 
Engkau adalah yang paling akhir, tidak ada sesuatu pun setelah-Mu. Engkau 
adalah yang zhahir, tidak ada sesuatu pun yang mengungguli-Mu, dan Engkau 
adalah yang bathin, tidak ada sesuatu pun yang tersembunyi dari-Mu, 
lunasilah hutang kami dan cukupkanlah kami dari kefakiran (kemiskinan).”
\\
\par
\noindent
\textbf{Tingkatan Doa dan Sanad}: \textbf{Shahih}: HR. Muslim (no. 2713) 
dari Abu Hurairah r.a. Doa ini dibaca juga ketika hendak tidur.\\\\
\par
\noindent 
115-----------------------Doa Berlindung dari Teman dan Tetangga yang Jahat
\begin{arabtext}
\noindent
al-ll_ahumma 'inni-y 'a`u-w_du bika min ^gAri al-ssu-w'i fi-y dAri 
al-muqAmaTi, fa-'inna ^gAra al-bAdiyaTi yata.hawwalu.\\
\end{arabtext}
\noindent
\textbf{Artinya}:
\par
\indent
"Ya Allah, sesungguhnya aku berlindung kepada-Mu dari tetangga yang jahat 
di tempat tinggal tetapku, karena sungguh tetangga orang-orang Badui (desa)
itu berpindah-pindah."{\scriptsize 1}\\
\begin{arabtext}
\noindent
al-ll_ahumma 'inni-y 'a`u-w_du bika min yawmi al-ssu-w'i, wamin la-ylaTi 
al-ssu-w'i, wamin sA`aTi al-ssu-w'i, wamin .sA.hibi al-ssu-w'i, 
wamin ^gAri al-ssu-w'i fi-y dAri al-muqAmaTi.\\
\end{arabtext}
\noindent
\textbf{Artinya}:
\par
\indent
"Ya Allah, sesungguhnya aku berlindung kepada-Mu dari hari yang buruk, 
malam yang buruk, saat yang buruk, teman yang jahat, dan tetangga yang 
jahat di tempat tinggal tetapku."{\scriptsize 2}\\
\par
\noindent
\textbf{Tingkatan Doa dan Sanad}:
\begin{enumerate}
\item \textbf{Hasan}: HR. Al-Hakim (1/532) - lalu dishahihkannya dan 
disepakati oleh azd-Dzahabi, An-Nasa-i (VIII/274), dan al-Bukhari dalam 
\textit{al-Adabul Mutrad} (no. 117). Lihat \textit{Shah\^{i}hul J\^{a}mi}' 
(no. 1290). 
\item \textbf{Hasan}: HR. Ath-Thabrani, dalam \textit{al-Mu'jamul 
Kab\^{i}r} (XVII/294, no. 810). Imam al-Haitsami berkata dalam kitabnya 
\textit{Majma'uz Zaw\^{a}-id} (X/144): "\textit{Rijal} (perawi) hadits ini 
shahih." Lihat juga \textit{Silsilah Ah\^{a}d\^{i}ts ash-Shah\^{i}hah} (no.
1443).\\\\
\end{enumerate}
\par
\noindent 
116----------------------Doa Diberi Kebahagiaan dan Terhindar dari Kesengsaraan
\begin{arabtext}
\noindent
al-ll_ahumma laka al-.hamdu kulluhu, al-ll_ahumma lA qAbi.da limA basa.tta,
walA bAsi.ta limA qaba.dta, walA hAdiya liman 'a.dlalta, walA mu.dilla 
liman hada-yta, walA mu`.tiya limA mana`ta, walA mAni`a limA 'a`.tayta, 
walA muqarriba limA bA`adta, walA mubA`ida limA qarrabta, al-ll_ahumma 
absu.t `ala-ynA min barakAtika, wara.hmatika, wafa.dlika, warizqika, 
al-ll_ahumma 'inni-y 'as'aluka al-nna`i-yma al-muqi-yma, alla_di-y lA 
ya.hu-wlu walA yazu-wlu, al-ll_ahumma 'inni-y 'as'aluka al-nna`i-yma ya-wma
al-`a-ylaTi, wAl-'amna yawma al-_hawfi, al-ll_ahumma 'inni-y `A'i_duN bika 
min ^sarri mA 'a`.ta-ytanA, wa^sarri mA mana`tanA, al-ll_ahumma .habbib 
'ila-ynA al-'i-ymAna, wazayyinhu fi-y qulu-wbinA, wakarrih 'ila-ynA 
al-kufra, wAl-fusu-wqa, wAl-`i.syAna, wA^g`alnA mina al-rrA^sidi-yna, 
al-ll_ahumma tawaffanA muslimi-yna, wa-'a.hyinA muslimi-yna, wa-'al.hiqnA
bi-al-.s.sAli.hi-yna, .ga-yra _hazAyA walA maftu-wni-yna, al-ll_ahumma 
qAtili al-kafaraTa alla_di-yna yuka_d_dibu-wna rusulaka, waya.suddu-wna 
`an sabi-ylika, wA^g`al `alayhim ri^gzaka wa-`a_dAbaka, al-ll_ahumma qAtili
al-kafaraTa alla_di-yna 'uwtuW al-kitAba, 'il_aha al.haqqi (^Ami-yn).\\
\end{arabtext}
\noindent
\textbf{Artinya}:
\par
\indent
"Ya Allah, segala puji hanya bagi-Mu. Ya Allah, tidak ada yang dapat 
menahan apa yang telah Engkau lapangkan dan tidak ada yang dapat 
melapangkan apa yang Engkau tahan, tidak ada yang dapat memberikan petunjuk
kepada orang yang telah Engkau sesatkan, dan tidak ada yang dapat 
menyesatkan orang yang telah Engkau beri petunjuk, tidak ada yang dapat 
memberi apa yang telah Engkau cegah, dan tidak ada yang dapat mencegah apa 
yang Engkau berikan, tidak ada yang dapat mendekatkan apa yang telah Engkau
jauhkan, dan tidak ada pula yang dapat menjauhkan apa yang telah Engkau 
dekatkan. Ya Allah, lapangkanlah keberkahan, juga rahmat, karunia, beserta 
rizki-Mu kepada kami. Ya Allah, sesungguhnya aku memohon kepada-Mu 
kenikmatan yang abadi yang tidak akan berubah dan tidak pula lenyap. Ya 
Allah, sesungguhnya aku memohon kenikmatan pada hari kesengsaraan, dan 
keamanan pada hari ketakutan. Ya Allah, sungguh aku berlindung kepada-Mu 
dari kejelekan apa yang Engkau berikan kepada kami dan kejelekan apa yang 
Engkau cegah dari sisi kami. Ya Allah, jadikan kami cinta terhadap 
keimanan. Hiasilah ia dalam hati kami dan tanamkanlah kebencian kepada kami
terhadap kekufuran, kefasikan, dan kemaksiatan, serta jadikanlah kami 
termasuk orang-orang yang mengikuti jalan yang lurus. Ya Allah, wafatkan 
dan hidupkanlah kami dalam keadaan Muslim, dan pertemukan kami dengan 
orang-orang yang shalih dalam keadaan tidak terhina dan tidak pula 
terfitnah. Ya Allah, perangilah orang-orang kafir yang mendustakan 
Rasul-Rasul-Mu dan menghadang jalan-Mu, timpakan kepada mereka siksaan 
serta adzab. Ya Allah, perangilah orang-orang kafir yang telah diberi 
al-Kitab, wahai Ilah Yang Mahabenar (kabulkanlah, ya Allah)."{\scriptsize 
1}\\
\begin{arabtext}
\noindent
al-ll_ahumma bi`ilmika al-.ga-yba, waqudratika `alY al-_halqi, 'a.hyini-y 
mA `alimta al-.hayATa _ha-yraN li-y, watawaffani-y 'i_dA `alimta al-wafATa 
_ha-yraN li-y, al-ll_ahumma wa-'as'aluka _ha^syataka fiy al-.ga-ybi 
wAl-^s^sahAdaTi, wa-'as'aluka kalimaTa al-.haqqi fiy al-rri.dA 
wAl-.ga.dabi, wa-'as'aluka al-qa.sda fiy al-faqri wAl-.ginY, wa-'as'aluka 
na`i-ymaN lA yanfadu, wa-'as'aluka qurraTa `a-yniN lA tanqa.ti`u, 
wa-'as'aluka al-rri.dA ba`da al-qa.dA'i, wa'as'aluka barda al-`ay^si ba`da 
al-ma-wti, wa-'as'aluka laddaTa al-nna.zari 'ilY wa^ghika, wAl-^s^sa-wqa 
'ilY liqA'ika, fi-y .ga-yri .darrA'a mu.dirraTiN, walA fitnaTiN 
mu.dillaTiN, al-ll_ahumma zayyinnA bizi-ynaTi al-'iymAni, wA^g`alnA hudATaN
muhtadi-yna.\\
\end{arabtext}
\noindent
\textbf{Artinya}:
\par
\indent
"Ya Allah, dengan pengetahuan-Mu terhadap yang ghaib dan kekuasaan-Mu atas 
semua makhluk, hidupkanlah aku jika Engkau mengetahui kehidupan itu lebih 
baik bagiku, dan matikanlah aku jika Engkau mengetahui kematian itu lebih 
baik bagiku. Ya Allah, dan aku mohon rasa takut kepada-Mu baik dalam 
keadaan sembunyi maupun ketika terang-terangan. Dan aku pun memohon 
kepada-Mu perkataan yang benar baik dalam keadaan senang maupun dalam 
keadaan marah. Aku mohon kepada-Mu kesederhanaan baik saat dalam keadaan 
fakir maupun saat dalam keadaan kaya. Aku memohon kepada-Mu nikmat yang 
tidak pernah habis. Dan aku memohon kepada-Mu penyejuk hati yang tidak 
pernah putus. Aku mohon kepada-Mu kerelaan menerima segala hal setelah 
ditetapkan. Aku memohon kepada-Mu ketenteraman hidup setelah kematian. Dan 
aku memohon pula kepada-Mu kenikmatan memandang wajah-Mu, juga kerinduan 
untuk bertemu dengan-Mu, bukan ketika dalam keadaan kesusahan yang 
membinasakan dan cobaan yang menyesatkan. Ya Allah, hiasilah kami dengan 
hiasan iman dan jadikan kami termasuk orang-orang yang memberi petunjuk dan
diberi petunjuk."{\scriptsize 2}\\
\begin{arabtext}
\noindent
al-ll_ahumma a.hfa.zni-y bi-al-'islAmi qA'imaN, wA.hfa.zni-y bi-al-'islAmi 
qA`idaN, wA.hfa.zni-y bi-al-'islAmi rAqidaN, walA tu^smit bi-y `aduwwaN 
walA .hAsidaN. al-ll_ahumma 'inni-y 'as'aluka min kulli _ha-yriN 
_hazA'inuhu biyadika, wa-'a-`u-w_du bika min kulli ^sarriN _hazA'inuhu 
biyadika.\\
\end{arabtext}
\noindent
\textbf{Artinya}:
\par
\indent
"Ya Allah, peliharalah aku dengan Islam ini ketika sedang berdiri, 
peliharalah aku dengan Islam ini saat sedang duduk, dan peliharalah aku 
dengan Islam ini dalam keadaan tidur. Dan janganlah Engkau jadikan musuh 
dan orang yang dengki gembira karena kedukaanku. Ya Allah, sungguh aku 
memohon segala kebaikan yang tiap perbendaharaannya ada di tangan-Mu, dan 
aku berlindung kepada-Mu dari segala kejahatan yang tiap perbendaharaannya 
ada di tangan-Mu."{\scriptsize 3}\\
\par
\noindent
\textbf{Tingkatan Doa dan Sanad}: 
\begin{enumerate}
\item \textbf{Shahih}: HR. Ahmad dengan lafazhnya (III/424), al-Hakim 
(I/507)-yang dalam kurung miliknya (III/23-24)-al-Bukhari dalam 
\textit{al-Adabul Mufrad} (no. 699). Dishahihkan oleh Syaikh al-Albani 
dalam \textit{Takhr\^{i}j Fiqhis S\^{i}rah} (hlm. 284) dan 
\textit{Shah\^{i}h al-Adabil Mufrad} (no. 541).
\item \textbf{Shahih}: HR. An-Nasai (III/54-55), Ahmad (IV/264), dan 
al-Hakim (I/524) dan lainnya dari Ammar bin Yasir r.a. Sanadnya 
\textit{jayyid}. Lihat \textit{Shah\^{i}h al-J\^{a}mi-us Shagh\^{i}r} (no. 
1301). Lafazh doa ini boleh juga dibaca setelah tasyahud sebelum salam. 
Lihat \textit{Shah\^{i}h al-Kalimith Thayyib} (no. 106) Pasal 16, dan 
\textit{Shifatu Shal\^{a}tin Nabi} (hlm. 184) karya Syaikh Muhammad 
Nashiruddin al-Albani.
\item \textbf{Hasan}: HR. Al-Hakim (I/525), dan dishahihkannya lalu 
disepakati oleh adz-Dzahabi. Lihat \textit{Shah\^{i}hul J\^{a}mi'} (no. 
1260), serta 
\textit{Silsilah Ah\^{a}d\^{i}ts ash-Shah\^{i}hah} (IV/54, no. 1540). 
Sanadnya hasan.\\\\
\end{enumerate}
\par
\noindent 
117------------------------------Doa Menghadapi Kesulitan
\begin{arabtext}
\noindent
lA-'il_aha 'illA 'anta, sub.hAnaka, 'inni-y kuntu mina al-.z.zAlimi-yna.\\
\end{arabtext}
\noindent
\textbf{Artinya}:
\par
\indent
"Tidak ada ilah yang berhak diibadahi dengan benar melainkan hanya Engkau. 
Mahasuci Engkau, sesungguhnya aku termasuk orang-orang yang zhalim".
{\scriptsize 1}\\
\begin{arabtext}
\noindent
al-ll_ahumma ra.hmataka 'ar^gu-w, falA takilni-y 'ilY nafsi-y .tarfaTa 
`a-yniN, wa'a.sli.h li-y ^sa'ni-y kullahu, lA 'il_aha 'illA 'anta.\\
\end{arabtext}
\noindent
\textbf{Artinya}:
\par
\indent
"Ya Allah, rahmat-Mu yang selalu aku harapkan, maka janganlah Engkau 
serahkan urusanku kepada diriku meski hanya sekejap mata, dan perbaikilah 
urusanku semuanya, tidak ada ilah yang berhak diibadahi dengan benar selain
Engkau".{\scriptsize 2}\\
\par
\noindent
\noindent
\textbf{Tingkatan Doa dan Sanad}:
\begin{enumerate}
\item \textbf{Shahih}: HR. At-Tirmidzi (no. 3505) dan al-Hakim (I/505) dan 
lainnya, dishahihkan oleh al-Hakim dan disepakati oleh adz-Dzahabi. Lihat 
\textit{Shah\^{i}h al-J\^{a}mi-us Shagh\^{i}r} (no. 3383) dengan lafazh 
(yang artinya): "Doa Dzun Nun (Nabi Yunus), ketika dia berdoa di dalam 
perut ikan paus adalah: 'Tidak ada ilah yang berhak diibadahi dengan benar 
melainkan hanya Engkau. Mahasuci Engkau, sesungguhnya aku termasuk 
orang-orang yang zhalim.' Sesungguhnya tidak ada seorang Muslim pun yang 
memanjatkan doa dengan kalimat tersebut dalam suatu hal apa pun, melainkan 
Allah akan mengabulkan untuknya".
\item \textbf{Hasan}: HR. Abu Dawud (no. 5090) dan Ahmad (V/42). Dihasankan
oleh Syaikh al-Albani dan selainnya. Lihat kitab \textit{Shah\^{i}h 
al-Adabil Mufrad} (no. 539) dan \textit{Shah\^{i}h al-Adzk\^{a}r} 
(351/251).\\\\
\end{enumerate}
\par
\noindent 
118------------------------------Doa Orang yang Mengalami Kesulitan
\begin{arabtext}
\noindent
al-ll_ahumma lA sahla 'illA mA^ga`altahu sahlaN, wa'anta ta^g`alu al-.hazna
'i_dA ^si'ta sahlaN.\\
\end{arabtext}
\noindent
\textbf{Artinya}:
\par
\indent
"Ya Allah, tidak ada kemudahan kecuali apa yang Engkau jadikan mudah. 
Sedang yang susah bisa Engkau jadikan mudah, apabila Engkau 
menghendakinya."\\
\par
\noindent
\textbf{Tingkatan Doa dan Sanad}: \textbf{Shahih}: HR. Ibnu Hibban 
(\textit{at-Ta'l\^{i}q\^{a}tul His\^{a}n} [no. 970], dan 
\textit{Maw\^{a}ridizh Zh\^{a}m-an} [no. 2427]) \textit{Shah\^{i}h 
Maw\^{a}ridizh Zh\^{a}m-an} (II/450 no. 2058) dan Ibnus Sunni dalam 
\textit{'Amalul Yaum wal Lailah} (no. 351). Al-Hafizh berkata: "Hadits ini 
shahih." Lihat \textit{Silsilah Ah\^{a}d\^{i}ts ash-Shah\^{i}hah} (no. 
2886).\\\\
\par
\noindent 
119------------------------------Doa Saat Mengalami Kesusahan, Kesedihan, dan Penawar 
Kedukaan
\begin{arabtext}
\noindent
lA 'il_aha 'illA al-ll_ahu al-`a.zi-ymu al-.hali-ymu, lA 'il_aha 'illA 
al-ll_ahu rabbu al-`ar^si al-`a.zi-ymi, lA 'il_aha 'illA al-ll_ahu rabbu 
al-ssamAwAti, warabbu al-'ar.di, warabbu al-`ar^si al-kari-ymi.\\
\end{arabtext}
\noindent
\textbf{Artinya}:
\par
\indent
"Tidak ada ilah yang berhak diibadahi dengan benar melainkan hanya Allah, 
Rabb Yang Mahaagung lagi Maha Penyantun. Tidak ada ilah yang berhak 
diibadahi dengan benar melainkan Allah, Pemilik Arsy yang agung. Tidak ada 
ilah yang berhak diibadahi dengan benar melainkan hanya Allah, Rabb langit 
dan Rabb bumi, Pemilik Arsy yang mulia."{\scriptsize 1}\\
\begin{arabtext}
\noindent
al-ll_ahumma 'inni-y `abduka, wAbnu `abdika, wAbnu 'amatika, nA.siyati-y 
biyadika, mA.diN fiyya .hukmuka, `adluN fiyya qa.dA'uka. 'as'aluka bikulli 
asmiN huwa laka, samma-yta bihi nafsaka, 'a-w 'anzaltahu fi-y kitAbika, 
'a-w `allamtahu 'a.hadaN min _halqika, 'awi asta'_tarta bihi fi-y `ilmi 
al-.ga-ybi `indaka, 'an ta^g`ala al-qur-^Ana rabi-y`a qalbi-y, wanu-wra 
.sadri-y, wa^galA'a .huzni-y, wa_dahAba hammi-y.\\
\end{arabtext}
\noindent
\textbf{Artinya}:
\par
\indent
"Ya Allah, sesungguhnya aku adalah hamba-Mu, anak hamba-Mu (Adam), dan anak
hamba perempuan-Mu (Hawa), ubun-ubunku berada di tangan-Mu, hukum-Mu 
berlaku terhadap diriku dan ketetapan-Mu adil pada diriku. Aku memohon 
kepada-Mu dengan segala Nama yang menjadi milik-Mu, yang Engkau namai 
diri-Mu dengannya, atau yang Engkau turunkan di dalam Kitab-Mu, atau yang 
Engkau ajarkan kepada seorang dari makhluk-Mu, atau yang Engkau rahasiakan 
di dalam ilmu ghaib di sisi-Mu, maka dengannya aku memohon sepaya Engkau 
menjadikan al-Qur-an penyejuk bagi hatiku, cahaya bagi dadaku, pelipur bagi
kesedihanku, dan penghilang kesusahanku."
\begin{shaded*}
\noindent
Melainkan Allah akan menghilangkan kesedihannya dan kesusahannya (orang 
yang mengucapkan doa ini) serta menggantikan semuanya itu dengan 
kegembiraan.{\scriptsize 2}
\end{shaded*}
\begin{arabtext}
\noindent
al-ll_ahu, al-ll_ahu rabbi-y, lA 'u^sriku bihi ^say'aN.\\
\end{arabtext}
\noindent
\textbf{Artinya}:
\par
\indent
"Allah, Allah  adalah Rabbku, aku tidak menyekutukan-Nya dengan sesuatu 
apapun."{\scriptsize 3}\\
\par
\noindent
\textbf{Tingkatan Doa dan Sanad}:
\begin{enumerate}
\item \textbf{Shahih}: HR. Al-Bukhari (no. 6345, 6346, 7426, 7431), Muslim 
(no. 2730), at-Tirmidzi (no. 3435), Ibnu Majah (no. 3883), dan Ahmad 
(I/228, 259, 268, 280) dari Ibnu Abbas r.a.
\item \textbf{Shahih}: HR. Ahmad (I/391, 452), Ibnu Hibban 
(\textit{at-Ta'l\^{i}q\^{a}tul His\^{a}n} [no. 968]), al-Hakim (I/509), dan
ath-Thabrani dalam \textit{al-Mu'jamul Kab\^{i}r} (X/169-170, no. 352) dari
Abdullah bin Mas'ud r.a. Dihasankan al-Hafizh dalam \textit{Takhr\^{i}j 
al-Adzk\^{a}r}. Dishahihkan Syaikh al-Albani. Lihat \textit{al-Kalimuth 
Thayyib} (hlm. 119, no. 124) dan \textit{Silsilah Ah\^{a}d\^{i}ts 
ash-Shah\^{i}hah} (no. 199).
\item \textbf{Shahih}: HR. Abu Dawud (no. 1525), Ibnu Majah (no. 3882), dan
lihat \textit{Silsilah Ah\^{a}d\^{i}ts ash-Shah\^{i}hah} (no. 2755).\\\\
\end{enumerate}
\par
\noindent 
120------------------------------Doa Agar Diberi Ketetapan Hati
\begin{arabtext}
\noindent
al-ll_ahumma mu.sarrifa al-qulu-wbi, .sarrif qulu-wbanA `alY .tA`atika.\\
\end{arabtext}
\noindent
\textbf{Artinya}:
\par
\indent
"Ya Allah, yang megarahkan hati, arahkanlah hati-hati kami untuk taat 
kepada-Mu."{\scriptsize 1}\\
\begin{arabtext}
\noindent
yA muqaliba al-qulu-wbi, _tabbit qalbi-y `alY di-ynika.\\
\end{arabtext}
\noindent
\textbf{Artinya}:
\par
\indent
"Wahai Yang membolak-balikkan hati, teguhkanlah hatiku pada agamu-Mu."
{\scriptsize 2}\\
\par
\noindent
\textbf{Tingkatan Doa dan Sanad}:
\begin{enumerate}
\item \textbf{Shahih}: HR. Muslim (no. 2654) dari Abdullah bin Amr bin 
al-Ash r.a.
\item \textbf{Shahih}: HR. At-Tirmidzi (no. 3522), Ahmad (VI/302, 315) dari 
Ummu Salamah r.a., dan al-Hakim (I/525) dari an-Nawas bin Sam'an. 
Dishahihkan dan disepakati oleh adz-Dzahabi. Lihat juga \textit{Shah\^{i}h 
at-Tirmidzi} (III/171), no. 2792). Ummu Salamah berkata: "Doa itu adalah 
doa Nabi SAW. yang paling sering dibaca."\\\\
\end{enumerate}
\par
\noindent 
121------------------------------Doa agar Diberi Keteguhan Petunjuk yang Lurus
\begin{arabtext}
\noindent
al-ll_ahumma _tabbitni-y, wA^g`alni-y hAdiyaN mahdiyyaN.\\
\end{arabtext}
\noindent
\textbf{Artinya}:
\par
\indent
"Ya Allah, teguhkanlah diriku, jadikanlah diriku pemberi petunjuk dan 
diberi petunjuk (oleh-Mu)."{\scriptsize 1}\\
\begin{arabtext}
\noindent
al-ll_ahumma ahdini-y wasaddini-y, al-ll_ahumma 'inni-y 'as'aluka al-hudY 
wal-ssadAda.\\
\end{arabtext}
\noindent
\textbf{Artinya}:
\par
\indent
"Ya Allah, berilah petunjuk kepadaku, dan luruskanlah aku. Ya Allah, 
sungguh aku memohon petunjuk dan kelurusan kepada-Mu."{\scriptsize 2}\\
\par
\noindent
\textbf{Tingkatan Doa dan Sanad}:
\begin{enumerate}
\item Doa ini diambil dari doa Rasulullah SWT. untuk Jarir. 
\textbf{Shahih}: HR. Al-Bukhari (no. 6333) dan Muslim (no. 2476).
\item \textbf{Shahih}: HR. Muslim (no. 2725).\\\\
\end{enumerate}
\par
\noindent 
122-------------------------Doa agar Diberi Kekuatan Iman dan Berbagai Kebaikan
\begin{arabtext}
\noindent
al-ll_ahumma 'inni-y 'as'aluka 'iymAnaN lA yartaddu, wana`i-ymaN lA 
yanfadu, wamurAfaqaTa mu.hammadiN .sallY al-ll_ahu `ala-yhi wasallama fi-y 
'a`lY ^gannaTi al-_huldi.\\
\end{arabtext}
\noindent
\textbf{Artinya}:
\par
\indent
"Ya Allah, sesungguhnya aku memohon kepada-Mu iman yang tidak akan lepas, 
nikmat yang tidak akan habis, dan menyertai Muhammad SAW. di Surga yang 
paling tinggi selama-lamanya."{\scriptsize 1}\\
\begin{arabtext}
\noindent
al-ll_ahumma 'inni-y 'as'aluka min fa.dlika wara.hmatika, fa-'innahu lA 
yamlikuhA 'illA 'anta.\\
\end{arabtext}
\noindent
\textbf{Artinya}:
\par
\indent
“Ya Allah, sungguh aku memohon kepada-Mu karunia dan rahmat-Mu, karena 
tidak ada yang memilikinya kecuali hanya Engkau.”{\scriptsize 2}\\
\begin{arabtext}
\noindent
al-ll_ahumma 'inni-y 'as'aluka mina al-_ha-yri kullihi, `A^gilihi 
wa-^A^gilihi, mA `alimtu minhu wamA lam 'a`lam, wa'a`u-w_du bika mina 
al-^s^sarri kullihi, `A^gilihi wa-^A^gilihi, mA `alimtu minhu wamA lam 
'a`lam. al-ll_ahuma 'inni-y 'as'aluka min _ha-yri mA sa'alaka `abduka 
wanabiyyuka, wa'a`u-w_du bika min ^sarri mA `A_da bihi `abduka 
wanabiyyuka. al-ll_ahumma 'inni-y 'as'aluka al-^gannaTa, wamA qarraba 
'ila-yhA, min qawliN 'aw `amaliN, wa'a`u-w_du bika mina al-nnAri, wamA 
qarraba 'ila-yhA, min qawliN 'a-w `amaliN, wa'as'aluka 'an ta^g`ala kulla 
qa.dA'iN qa.da-ytahu li-y _hayraN.\\
\end{arabtext}
\noindent
\textbf{Artinya}:
\par
\indent
“Ya Allah, sesungguhnya aku  memohon kepada-Mu seluruh kebaikan, baik yang 
sekarang maupun yang akan datang, yang aku ketahui maupun yang tidak aku 
ketahui. Dan aku memohon perlindungan kepada-Mu dari seluruh kejahatan, 
baik yang sekarang maupun yang akan datang, baik yang kuketahui maupun yang
tidak kuketahui. Ya Allah, sesungguhnya aku memohon kebaikan yang diminta 
oleh hamba dan Nabi-Mu, dan aku pun berlindung kepada-Mu dari kejahatan 
yang hamba dan Nabi-Mu berlindung kepada-Mu darinya. Ya Allah, sesungguhnya
aku memohon kepada-Mu Surga dan apa-apa yang dapat mendekatkan kepadanya 
baik berupa ucapan maupun perbuatan. Dan aku juga berlindung kepada-Mu dari
Neraka dan apa-apa yang dapat mendekatkan kepadanya, baik berupa ucapan 
maupun perbuatan. Dan aku memohon kepada-Mu supaya  Engkau menjadikan 
seluruh ketetapan yang telah Engkau tetapkan bagiku merupakan suatu 
kebaikan.”{\scriptsize 3}\\
\par
\noindent
\textbf{Tingkatan Doa dan Sanad}:
\begin{enumerate}
\item \textbf{Shahih}: HR. Ahmad (I/400, 445-446, 454), Ibnu Hibban (no. 
2436) dari Ibnu Mas'ud r.a. Lihat \textit{Shah\^{i}h Maw\^{a}ridizh 
Zh\^{a}m-an} (no. 2065).
\item \textbf{Shahih}: HR. Abu Nu'aim dalam kitab \textit{Hilyatul Auliya'}
dan ath-Thabrani dalam \textit{al-Mu'jamul Kab\^{i}r} (X/178, no. 10379). 
Lihat \textit{Majma'uz Zaw\^{a}-id} (X/159), \textit{Shah\^{i}hul 
J\^{a}mi'} (no. 1278) serta \textit{Silsilah Ah\^{a}d\^{i}ts 
ash-Shah\^{i}hah} (no. 1543).
\item \textbf{Shahih}: HR. Ibnu Majah (no. 3846), Ibnu Hibban (no. 2413 - 
\textit{al-Maw\^{a}rid}), Ahmad (VI/134), al-Hakim (I/521-522), dan lafazh 
hadits tersebut adalah milik Ibnu Majah. Lihat kitab \textit{Shah\^{i}h 
Ibni Majah} (II/327, no. 3102) dan \textit{Silsilah Ah\^{a}d\^{i}ts 
ash-Shah\^{i}hah} (no. 1542).\\\\
\end{enumerate}
\par
\noindent 
123--------------------------Doa Memohon Petunjuk dan Ketakwaan
\begin{arabtext}
\noindent
al-ll_ahumma 'inni-y 'as'aluka al-hudY, wAl-ttuqY, wAl-`afAfa, wAl-.ginY.\\
\end{arabtext}
\noindent
\textbf{Artinya}:
\par
\indent
"Ya Allah, sesungguhnya aku memohon petunjuk, ketakwaan, kesucian 
(dijauhkan dari hal-hal yang tidak halal/tidak baik), dan kecukupan."\\
\par
\noindent
\textbf{Tingkatan Doa dan Sanad}: \textbf{Shahih}: HR. Muslim (no. 2721), 
at-Tirmidzi (no. 3489), Ahmad (I/416, 437), Ibnu Majah (no. 3832) dari Ibnu 
Mas'ud r.a.\\\\
\par
\noindent 
124------------------------Berlindung dari Sifat yang Jelek dan Mohon Dibersihkan Hati
\begin{arabtext}
\noindent
al-ll_ahumma 'inni-y 'a`u-w_du bika mina al-`a-^gzi, wAl-kasali, 
wAl-^gubni, wAl-bu_hli, wAl-harami, wa`a_dAbi al-qabri, al-ll_ahumma 
^Ati nafsi-y taqwAhA wazakkihA 'anta _ha-yru zakkAhA, 'anta waliyyuhA 
wama-wlAhA, al-ll_ahumma 'inni-y 'a`u-w_du bika min `ilmiN lA yanfa`u, 
wamin qalbiN lA ya_h^sa`u, wamin nafsiN lA ta^sba`u, waman da`waTiN lA 
yusta^gAbu lahA.\\
\end{arabtext}
\noindent
\textbf{Artinya}:
\par
\indent
"Ya Allah, sesungguhnya aku memohon perlindungan kepadamu dari kelemahan, 
kemalasan, sifat pengecut, kekikiran, pikun, dan dari adzab kubur. Ya 
Allah, berikanlah ketakwaan pada diriku dan sucikanlah ia, karena Engkaulah
sebaik-baik Rabb yang mensucikannya, Engkau Pelindung dan Pemeliharanya. Ya
Allah, sesungguhnya aku berlindung kepada-Mu dari ilmu yang tidak 
bermanfaat, hati yang tidak khusyu', nafsu yang tidak pernah puas, dan doa 
yang tidak dikabulkan."\\
\par
\noindent
\textbf{Tingkatan Doa dan Sanad}: \textbf{Shahih}: HR. Muslim (no. 2722) 
dan an-Nasai (VIII/269) dari Zaid bin al-Arqam r.a.\\\\
\par
\noindent 
125--------------------Doa Agar Diberi Ilmu Yang Bermanfaat dan Berlindung Dari Ilmu
Yang Tidak Bermanfaat
\begin{arabtext}
\noindent
al-ll_ahumma anfa`ni-y nimA `allamtani-y, wa-`allimni-y mA yanfa`uni-y, 
wazidni-y `ilmaN\\
\end{arabtext}
\noindent
\textbf{Artinya}:
\par
\indent
"Ya Allah, berilah manfaat bagiku atas apa yang Engkau ajarkan kepadaku, 
dan ajarkanlah kepadaku apa-apa yang bermanfaat bagiku, serta tambahkanlah 
ilmu kepadaku."{\scriptsize 1}\\
\begin{arabtext}
\noindent
al-ll_ahumma faqqihni-y fiy al-ddi-yni.\\
\end{arabtext}
\noindent
\textbf{Artinya}:
\par
\indent
"Ya Allah, berikanlah aku pemahaman dalam \textit{d\^{i}n} (agama 
Islam)."{\scriptsize 2}\\
\begin{arabtext}
\noindent
al-ll_ahumma 'inni-y 'a`u-w_dubika min qalbiN lA ya_h^sa`u, wa-min du`A'iN 
lA yusma`u, wa-min nafsiN lA ta^sba`u, wa-min `ilmiN lA yanfa`u, 'a`u-w_du 
bika min h_a'ulA'i al-'arba`i.\\
\end{arabtext}
\noindent
\textbf{Artinya}:
\par
\indent
"Ya Allah, sesungguhnya aku berlindung kepada-Mu dari hati yang tidak 
khusyu', doa yang tidak didengar, nafsu yang tidak pernah puas, dan dari 
ilmu yang tidak bermanfaat. Aku berlindung kepada-Mu dari keempat hal 
tersebut."{\scriptsize 3}\\
\begin{arabtext}
\noindent
al-ll_ahumma 'inni-y 'as'aluka `ilmaN nAfi`aN, wa-rizqaN .tayyibaN, 
wa-`amalaN mutaqabbalaN.\\
\end{arabtext}
\noindent
\textbf{Artinya}:
\par
\indent
"Ya Allah, sesungguhnya aku memohon kepada-Mu ilmu yang bermanfaat, rizki 
yang baik, dan amal yang diterima."{\scriptsize 4}\\
\begin{arabtext}
\noindent
al-ll_ahumma 'inni-y 'as'aluka `ilmaN nAfi`aN, wa-'a`u-wbika min `ilmiN lA 
yanfa`u. \\ \\
\end{arabtext}
\noindent
\textbf{Artinya}:
\par
\indent
"Ya Allah, sesunguhnya aku memohon kepada-Mu ilmu yang bermanfaat, dan aku 
berlindung kepada-Mu dari ilmu yang tidak bermanfaat."{\scriptsize 5}\\
\par
\noindent
\textbf{Tingkatan Doa dan Sanad}:
\begin{enumerate}
\item \textbf{Shahih}: HR. At-Tirmidzi (no. 3599), Ibnu Majah (no. 251, 
3833). Lihat \textit{Shah\^{i}h at-Tirmidzi} (III/185, no. 2845) dan 
\textit{Shah\^{i}h Ibni Majah} (I/47, no. 203) dari Abu Hurairah r.a. Lihat
juga \textit{Silsilah Ah\^{a}d\^{i}ts ash-Shah\^{i}hah} (no. 3151).
\item \textbf{Shahih}: HR. Al-Bukhari/\textit{Fathul B\^{a}ri} (I/244, no. 
143) dan Muslim (IV/1927, no. 2477) mengenai doa Nabi bagi Ibnu Abbas r.a. 
\item \textbf{Shahih}: HR. At-Tirmidzi (no. 3482), an-Nasai (VIII/254-255) 
dari Abdullah bin Amr, Abu Dawud (no. 1548), dan selainnya dari Abu 
Hurairah r.a. Lihat \textit{Shah\^{i}h al-J\^{a}mi'ish Shagh\^{i}r} (no. 
1297), \textit{Shah\^{i}h an-Nasai} (III/1113, no. 5053), dan 
\textit{Shah\^{i}h Sunan Abi Dawud} (no. 1384) terbitan Gharras.
\item \textbf{Shahih}: HR. Ibnu Majah (no. 925) dan ath-Thabrani dalam 
\textit{al-Mu'jamus Shagh\^{i}r} (I/260). Lihat \textit{Shah\^{i}h Ibni 
Majah} (I/152, no. 753).
\item \textbf{Shahih}: HR. Ibnu Majah (no. 3843), an-Nasai dalam 
\textit{Sunanul Kubra} (no. 7818), Ibnu Hibban (no. 82 - 
\textit{at-Ta'l\^{i}q\^{a}tul His\^{a}n}), Ibnu Abi Syaibah dalam 
\textit{al-Mushannaf} (no. 27127, 29610). Dan ini adalah lafazh an-Nasai 
dan Ibnu Hibban.\\\\
\end{enumerate}
\par
\noindent 
126-------------------------Doa supaya Terhindar dari Bahaya Syirik
\begin{arabtext}
\noindent
al-ll_ahumma 'innA na`u-w_du bika min 'an nu^srika bika ^sa-y'aN na`lamuhu, 
wanasta.gfiruka limA lA na`lamuhu.\\
\end{arabtext}
\noindent
\textbf{Artinya}:
\par
\indent
"Ya Allah, sungguh kami berlindung kepada-Mu dari menyekutukan-mu, sedang 
kami mengetahuinya dan kami memohon ampunan kepada-Mu atas apa yang kami 
tidak mengetahuinya."\\
\par
\noindent
\textbf{Tingkatan Doa dan Sanad}: \textbf{Shahih}: HR. Ahmad (IV/403) dan 
selainnya dari Abu Musa al-Asy'ari r.a. Lihat kitab \textit{Shah\^{i}h 
at-Targh\^{i}b wat Tarh\^{i}b} (I/121-122, no. 36), dan \textit{Shah\^{i}h 
al-Adabul Mufrad} (no. 551).\\\\
\par
\noindent 
127-------------------------Doa Berlindung dari Kesesatan
\begin{arabtext}
\noindent
al-ll_ahumma laka 'aslamtu, wabika ^Amantu, wa`ala-yka tawakkaltu, 
wa'i-la-yka 'anabtu wabika _hA.samtu, al-ll_ahumma 'inni-y 'a`u-w_du 
bi`izzatika lA 'il_aha 'illA 'anta 'an tu.dillani-y, 'anta al-.hayyu 
alla_di-y lA yamu-wtu, wAl-^ginnu wAl-'insu yamu-wtu-wna.\\
\end{arabtext}
\noindent
\textbf{Artinya}:
\par
\indent
"Ya Allah, kepada-Mulah aku berserah diri, dan kepada-Mulah aku beriman, 
serta kepada-Mulah aku bertawakal, kepada-Mu aku bertaubat dan dengan 
nama-Mu aku membela. Ya Allah, sesungguhnya aku berlindung dengan 
keperkasaan-Mu, tidak ada ilah yang berhak untuk diibadahi hamba dengan 
benar melainkan hanya Engkau, agar Engkau tidak menyesatkan diriku. 
Engkaulah yang Mahahidup dan yang tidak akan pernah mati, sedangkan jin dan
manusia semuanya akan mati."\\
\par
\noindent
\textbf{Tingkatan Doa dan Sanad}: \textbf{Shahih}: HR. Al-Bukhari (no. 
7383) dan Muslim (no. 2717) dari Ibnu Abbas r.a. Lafazh ini milik Muslim.\\\\
\par
\noindent 
128---------------------------Doa Malam Lailatul Qadar
\begin{arabtext}
\noindent
al-ll_ahumma 'innaka `afuwwuN, tu.hibbu al-`afwa, fA`fu `anni-y.\\
\end{arabtext}
\noindent
\textbf{Artinya}:
\par
\indent
"Ya Allah, sesungguhnya Engkau Maha Pemaaf, Engkau menyukai pemaafan. 
Karena itu, berilah maaf kepadaku."\\
\par
\noindent
\textbf{Tingkatan Doa dan Sanad}: \textbf{Shahih}: HR. At-Tirmidzi (no. 
3513), Ibnu Majah (no. 3850), Ahmad (VI/171), al-Hakim (I/530), an-Nasai 
dalam \textit{'Amalul Yaum wal Laila}h (no. 878). Lihat \textit{Shah\^{i}h 
at-Tirmidzi} (III/170, no. 2789) dan \textit{Silsilah Ah\^{a}d\^{i}ts 
ash-Shah\^{i}hah} (no. 3337).\\\\
\par
\noindent 
129-------------------------Doa Dimudahkan Beramal Shahih dan Dicintai Allah
\begin{arabtext}
\noindent
al-ll_ahumma 'inni-y 'as'aluka fi`la al-_ha-yrAti, watarka al-munkarAti, 
wa.hubba almasAki-yni, wa'anta.gfirali-y, watar.hamani-y, wa-'i_dA 'aradta 
fitnaTa qawmiN, fatawaffani-y .ga-yra maftu-wniN, wa'as'aluka .hubbaka, 
wa.hubba man yu.hibbuka, wa.hubba `amaliN yuqarribuni-y 'ilY .hubbika.\\
\end{arabtext}
\noindent
\textbf{Artinya}:
\par
\indent
"Ya Allah, sungguh aku memohon kepada-Mu supaya dapat melakukan berbagai 
perbuatan baik, meninggalkan semua perbuatan mungkar, mencintai orang-orang
miskin, dan agar Engkau mengampuni dan menyayangiku. Dan jika Engkau hendak
menimpakan suatu \textit{fitnah} (malapetaka) kepada suatu kaum, maka 
wafatkanlah aku dalam keadaan tidak terkena fitnah tersebut. Dan aku 
memohon akan rasa cinta kepada-Mu dan rasa cinta kepada tiap orang yang 
mencintai-Mu, juga rasa cinta kepada amal perbuatan yang dengannya dapat 
mendekatkan diriku ini sehingga lebih mencintai-Mu."\\
\par
\noindent
\textbf{Tingkatan Doa dan Sanad}: \textbf{Hasan Shahih}: HR. Ahmad dengan 
lafazhnya (V/243), at-Tirmidzi (no. 3235), dan al-Hakim (I/521).\\\\
\par
\noindent 
130---------------------Doa agar Menjadi Orang yang Banyak Berdzikir, Bersyukur, dan 
Taat
\begin{arabtext}
\noindent
rabbi 'a`inni-y walAtu`in `alayya, wAn.surni-y walA tan.sur `alayya, wAmkur 
li-y walA tamkur `alayya, wAhdini-y wayassiri al-hudY 'ilayya, wAn.surni-y 
`alY man ba.gY `alayya, rabbi a^g`alni-y laka ^sakkAraN, laka _dakkAraN, 
laka rahhAbaN, laka mi.twA`aN, 'ila-yka mu_hbitaN, laka 'awwAhaN muni-ybaN, 
rabbi taqabbal tawbati-y, wA.gsil .hawbati-y, wa'a^gib da`wati-y, wa_tabbit 
.hu^g^gati-y, wAhdi qalbi-y, wasaddid lisAni-y, wAslul sa_hiymaTa qalbi-y.
\\
\end{arabtext}
\noindent
\textbf{Artinya}:
\par
\indent
"Rabbku, tolonglah aku dan jangan Engkau tolong (orang yang hendak 
mencelakakan) atas diriku. Dan belalah aku dan jangan Engkau bela (orang 
yang akan mencelakakan) atas diriku. Perdayakanlah untukku dan jangan 
sampai aku diperdayai orang lain. Berilah aku petunjuk dan mudahkan ia 
untukku. Dan tolonglah aku atas orang yang menzhalimiku. Rabbku, jadikanlah
aku orang yang senantiasa bersyukur kepada-Mu, selalu berdzikir kepada-Mu, 
selalu takut kepada-Mu, selalu taat kepada-Mu, khusyu' patuh, banyak 
berdoa, serta bertaubat kepada-Mu. Rabbku, terimalah taubatku, bersihkan 
dosa-dosaku, perkenankanlah doaku, tetapkanlah hujjahku, beri petunjuk 
kepada hatiku, luruskanlah lidahku, dan hilangkanlah kejelekkan dalam 
hatiku."{\scriptsize 1}\\
\begin{arabtext}
\noindent
al-l_ahumma 'a`inni-y `alY _dikrika, wa^sukrika, wa.husni `ibAdatika.\\
\end{arabtext}
\noindent
\textbf{Artinya}:
\par
\indent
"Ya Allah, tolonglah aku untuk dapat berdzikir kepada-Mu, dapat bersyukur 
kepada-Mu, dan dapat beribadah dengan baik kepada-Mu."{\scriptsize 2}\\
\par
\noindent
\textbf{Tingkatan Doa dan Sanad}:
\begin{enumerate}
\item \textbf{Shahih}: HR. Ahmad (I/227)-lafazh ini miliknya-Abu Dawud (no. 
1510), at-Tirmidzi (no. 3551), Ibnu Majah (no. 3830), Ibnu Hibban (no. 993 
- \textit{at-Ta'liqatul His\^{a}n}), al-Hakim (I/519-520), dan yang 
lainnya. Dishahihkan oleh al-Hakim dan disepakati oleh adz-Dzahabi. Lihat 
\textit{Shah\^{i}h at-Tirmidzi} (III/178, no. 2816).
\item \textbf{Shahih}: HR. Abu Dawud (no. 1522), Ahmad (V/244-245, 247), 
an-Nasai (III/53), dan al-Hakim (I/273 dan III/273) dan dishahihkannya, 
juga disepakati oleh adz-Dzahabi. Nabi SAW. pernah berwasiat kepada Mu'adz 
r.a. agar dia mengucapkan dzikir tersebut pada setiap akhir shalatnya atau 
sesudah salam dari shalat wajib.\\\\
\end{enumerate}
\par
\noindent 
131--------------------------Memohon Akhlak yang Baik
\begin{arabtext}
\noindent
al-ll_ahumma 'a.hsanta _halqi-y fa'a.hsin _huluqi-y.\\
\end{arabtext}
\noindent
\textbf{Artinya}:
\par
\indent
"Ya Allah, sebagaimana Engkau  telah menciptakanku dengan baik, maka 
perbaiki pula akhlakku."\\
\par
\noindent
\textbf{Tingkatan Doa dan Sanad}: \textbf{Shahih}: HR. Ahmad (VI/68, 155; 
I/403) dan dishahihkan oleh al-Albani dalam \textit{Irw\^{a}-ul Ghal\^{i}l} 
(I/155, dibawah pembahasan hadits no. 74). Hadits doa ini bersifat 
\textit{mutlaq}, tidak terikat (tidak harus diucapkan) di depan cermin. 
Lihat kitab \textit{al-Kalmuth Thayyib} (hlm. 171).\\\\
\par
\noindent 
132--------------------------Doa Berlindung Dari Hutang dan Agar Dapat Melunasinya
\begin{arabtext}
\noindent
al-ll_ahumma 'inni-y 'a`u-w_du bika mina al-hammi wAl-.hazani, wAl-`a^gzi 
wAl-kasali, wAl-bu_hli, wAl-^gubni, wa.dala`i al-dda-yni, wa.galabaTi 
al-rri^gAli.\\
\end{arabtext}
\noindent
\textbf{Artinya}:
\par
\indent
"Ya Allah, sungguh aku  berlindung kepada-Mu dari kesusahan, kesedihan, 
kelemahan, kemalasan, sifat kikir, sifat pengecut, lilitan utang, dan 
dikuasai orang lain."{\scriptsize 1}\\
\begin{arabtext}
\noindent
al-ll_ahumma 'inni-y 'a`u-w_du bika min .galabaTi al-dda-yni, wa.galabaTi 
al-`aduwwi, wa^samAtaTi al-'a`dA'i.\\
\end{arabtext}
\noindent
\textbf{Artinya}:
\par
\indent
"Ya Allah, sesungguhnya aku berlindung kepada-Mu dari lilitan utang, 
tekanan musuh, dan kegembiraan para musuh."{\scriptsize 2}\\
\begin{arabtext}
\noindent
al-ll_ahumma akfini-y bi.halAlika `an .harAmika, wa'a.gnini-y bifa.dlika 
`amman siwAka.\\
\end{arabtext}
\noindent
\textbf{Artinya}:
\par
\indent
"Ya Allah, cukupilah aku dengan rizki-Mu yang halal (hingga terhindar) dari
yang haram. Cukupilah aku dengan karunia-mu (hingga aku tidak minta) kepada
selain-Mu."{\scriptsize 3}\\
\par
\noindent
\textbf{Tingkatan Doa dan Sanad}:
\begin{enumerate}
\item \textbf{Shahih}: HR. Al-Bukhari (no. 6363). Rasulullah SAW. sering 
memanjatkan doa ini. Lihat \textit{Fathul B\^{a}ri} (XI/173).
\item \textbf{Shahih}: HR. An-Nasai (VIII/265), Ahmad (II/173) dan al-Hakim
(I/531). Lihat \textit{Silsilah Ah\^{a}d\^{i}ts ash-Shah\^{i}hah} (no. 
1541).
\item \textbf{Hasan}: HR. At-Tirmidzi (no. 3563), Ahamd (I/153) dan 
al-Hakim (I/538) dari Ali bin Abi Thalib r.a.\\\\
\end{enumerate}
\par
\noindent 
133---------------------------Doa Menghadapi Musuh dan Orang yang Berkuasa
\begin{arabtext}
\noindent
al-ll_ahumma 'innA na^g`aluka fi-y nu.hu-wrihim wana`u-w_du bika min 
^suru-wrihim.\\
\end{arabtext}
\noindent
\textbf{Artinya}:
\par
\indent
"Ya Allah, sungguh kami menjadikan Engkau di leher mereka (agar kekuatan 
mereka tidak berdaya saat berhadapan dengan kami). Dan kami berlindung 
kepada-Mu dari kejelekan mereka."{\scriptsize 1}\\
\begin{arabtext}
\noindent
al-ll_ahumma 'anta `a.dudi-y, wa'anta na.si-yri-y, bika 'a.hu-wlu, wabika 
'a.su-wlu, wabika 'uqAtilu.\\
\end{arabtext}
\noindent
\textbf{Artinya}:
\par
\indent
"Ya Allah, Engkau adalah Penolongku. Engkau adalah Pembelaku. Dan dengan 
pertolongan-Mu aku bergerak, dengan bantuan-Mu aku menyergap, dengan 
pertolongan-Mu pula aku berperang."{\scriptsize 2}\\
\begin{arabtext}
\noindent
al-ll_ahumma munzila al-kitAbi sari-y`a al-.hisAbi, ihzimi al-'a.hzAba, 
al-ll_ahumma ahzimhum wazalzilhum.\\
\end{arabtext}
\noindent
\textbf{Artinya}:
\par
\indent
"Ya Allah, Yang menurunkan Kitab suci, Yang menghisab perbuatan manusia 
dengan cepat. Kalahkanlah golongan musuh. Ya Allah, cerai beraikanlah dan 
goncangkanlah mereka."{\scriptsize 3}\\
\begin{arabtext}
\noindent
( .hasbunA al-llahu wani`ma al-waki-ylu )\\
\end{arabtext}
\noindent
\textbf{Artinya}:
\par
\indent
\textit{"Cukuplah Allah menjadi Penolong kami. Dan Dia adalah sebaik-baik 
Pelindung."} (QS. Ali 'Imran [3]: 173).{\scriptsize 4}\\
\par
\noindent
\textbf{Tingkatan Doa dan Sanad}:
\begin{enumerate}
\item \textbf{Shahih}: HR. Abu Dawud (no. 1537), an-Nasai dalam 
\textit{'Amalul Yaum wal Lailah} (no. 606), dan al-Hakim (II/142). 
Dishahihkan oleh al-Hakim dan disepakati oleh adz-Dzahabi.
\item \textbf{Shahih}: HR. Abu Dawud (no. 2632) dan at-Tirmidzi (no. 3584) 
dari Anas r.a. Lihat \textit{Shah\^{i}h at-Tirmidzi} (III/183) dan 
\textit{al-Kalmuth Tahyyib} (hlm. 120, no. 126).     
\item \textbf{Shahih}: HR. Al-Bukhari (no. 2933, 4115), Muslim (no.  1742 
[21]), at-Tirmidzi (no. 1678) dan Ibnu Majah (no. 2796).
\item Kalimat ini diucapkan oleh Nabi Ibrahim a.s. ketika dilemparkan ke 
dalam api, dan juga diucapkan Nabi Muhammad SAW. ketika orang-orang 
berkata: "\textit{Sesungguhnya manusia telah mengumpulkan pasukan untuk 
menyerangmu}." (QS. Ali 'Imran [3]: 173). \textbf{Shahih}: HR. Al-Bukhari 
(no. 4563, 4564).\\\\
\end{enumerate}
\par
\noindent 
134--------------------------Doa Apabila Takut Dizhalimi Penguasa
\begin{arabtext}
\noindent
al-ll_ahumma rabba al-ssamA wAti al-ssab`i, warabba al-`ar^si al-`a.zi-ymi, 
kun li-y ^gAraN min fulAni bni fulAniN, wa'a.hzaabihi min _halA'iqika, 'an 
yafru.ta `alayya 'a.haduN minhum 'aw ya.t.gY, `azza ^gAruka, wa^galla 
_tanA'uka, walA 'il_aha 'illA 'anta.\\
\end{arabtext}
\noindent
\textbf{Artinya}:
\par
\indent
"Ya Allah, Rabb Penguasa langit yang tujuh, Rabb Penguasa Arsy yang agung. 
Jadilah Engkau pelindung bagiku dari Fulan bin Fulan, dan para kelompoknya 
dari makhluk-Mu. Jangan sampai ada seorang pun dari mereka menyakitiku 
ataupun melampaui batas terhadapku. Sungguh kuat perlindungan-Mu, dan 
sungguh agung puja-puji-Mu. Tidak ada ilah yang berhak diibadahi dengan 
benar kecuali Engkau."{\scriptsize 1}\\
\begin{arabtext}
\noindent
al-ll_ahu 'akbaru, al-ll_ahu 'a`azzu min _halqihi ^gami-y`aN, al-ll_ahu 
'a`azzu mimmA 'a_hAfu wa'a.h_daru, wa'a`u-w_du bi-al-ll_ahi alla_di-y lA 
'il_aha 'illA huwa, al-mumsiki al-ssamAwAti al-ssab`i 'an yaqa`na `alY 
al-'ar.di 'illA bi-'i_dnihi, min ^sarri `abdika fulAniN, wa^gunu-wdihi 
wa'atbA`ihi wa'a^syA`ihi, mina al-^ginni wAl-'insi, al-ll_ahumma kun li-y 
^gAraN min ^sarrihim, wa^galla _tanA'uka wa`azza ^gAruka, watabAraka 
asmuka, walA 'il_aha .gayruka.\\
\end{arabtext}
\noindent
\textbf{Artinya}:
\par
\indent
"Allah Mahabesar. Allha Mahaperkasa dari segala makhluk-Nya. Allah Maha 
perkasa dari apa yang kutakutkan dan kukhawatirkan. Aku berlindung kepada 
Allah, tiada Rabb yang haq selain Dia, yang menahan tujuh langit agar tidak
jatuh ke bumi kecuali dengan izin-Nya, dari kejahatan hamba-Mu Fulan, serta
dari bala tentaranya, pengikut dan para pendukungnya, dari jin serta 
manusia. Ya Allah, jadilah Engkau Pelindungku dari kejahatan mereka semua. 
Agunglah puji-Mu, dan kuatlah perlindungan-Mu dan Mahasuci \textit{asma} 
(nama)-Mu. Tidak ada ilah yang berhak diibadahi dengan benar melainkan 
Engkau."{\scriptsize 2} [\textbf{Dibaca 3x}).\\
\par
\noindent
\textbf{Tingkatan Doa dan Sanad}:
\begin{enumerate}
\item \textbf{Shahih}: HR. Al-Bukhari dalam \textit{al-Adabul Mufrad} (no. 
707). Dinyatakan shahih oleh Syaikh al-Albani dalam \textit{Shah\^{i}h 
al-Adabil Mufrad} (no. 545) dari Abdullah bin Mas'ud r.a.
\item \textbf{Shahih}: HR. Al-Bukhari dalam \textit{al-Adabul Mufrad} (no. 
708). Hadits di atas dishahihkan oleh Syaikh al-Albani dalam kitab 
\textit{Shah\^{i}h al-Adabil Mufrad} (no. 546) dari Abdullah bin Abbas r.a.\\\\
\end{enumerate}
\par
\noindent 
135--------------------------Bacaan bagi Orang yang Ragu dalam Beriman
\noindent
\begin{enumerate}
\item Bagi siapa saja yang ragu-ragu dalam beriman, maka hendaklah ia 
memohon perlindungan kepada Allah.{\scriptsize 1}
\item Berhenti dari keraguan.{\scriptsize 2}\\
\indent Hendaklah mengucapkan:\\
\begin{arabtext}
\noindent
^Amantu bi-al-ll_ahi warusulihi.\\
\end{arabtext}
\noindent
\textbf{Artinya}:
\par
\indent
"Aku beriman kepada Allah dan kepada (kebenaran) para Rasul (utusan)-Nya."
{\scriptsize 3}\\\\
\indent
Selain itu, hendaklah dia membaca firman-Nya: 
\begin{arabtext}
\noindent
huwa al-'awwalu wa-al-'a_hiru wAl-.z.z_ahiru wa-al-bA.tinu wahuwa bikulli 
^saY'iN `aliymuN.\\
\end{arabtext}
\noindent
\textbf{Artinya}:
\par
\indent
\textit{"Dialah yang awal (Allah telah ada sebelum segala sesuatu ada), 
yang akhir (disaat segala sesuatu telah hancur, Allah masih tetap kekal), 
yang zhahir (Dialah yang nyata, sebab banyak bukti yang menyatakan adanya 
Allah), yang bathin (tidak ada sesuatu yang bisa menghalangi-Nya. Allah 
lebih dekat kepada hamba-Nya daripada mereka kepada dirinya). Dialah Yang 
Maha Mengetahui atas segala sesuatu."} (QS. Al-Had\^{i}d [57]: 3).
{\scriptsize 4}\\
\end{enumerate}
\par
\noindent
\textbf{Tingkatan Doa dan Sanad}:
\begin{enumerate}
\item \textbf{Shahih}: HR. Al-Bukhari/\textit{Fathul B\^{a}ri} (VI/336) dan
Muslim (I/120).
\item \textbf{Shahih}: HR. Al-Bukhari/\textit{Fathul B\^{a}ri} (VI/336) dan
Muslim (I/120), pada Bab "Bay\^{a}nil Waswasah fil \^{I}m\^{a}n wa m\^{a} 
Yaq\^{u}luhu man Wajadaha".
\item \textbf{Shahih}: HR. Muslim (no. 134).
\item \textbf{Atsar Hasan}: Diriwayatkan oleh Abu Dawud (no. 5110), pada 
Bab "F\^{i} Raddil Waswasah" dari perkataan Ibnu Abbas r.a. Lihat 
\textit{Shah\^{i}h Abi Dawud} (III/962).\\\\
\end{enumerate}
\par
\noindent 
136----------------------Doa Penghilang Gangguan Syaitan tatkala Shalat atau Membaca 
Al-Qur'an
\begin{arabtext}
\noindent
'a`u-w_du bi-al-ll_ahi mina al-^s^sa-y.tAni al-rra^gi-ymi.\\
\end{arabtext}
\noindent
\textbf{Artinya}:
\par
\indent
"Aku berlindung kepada Allah dari godaan syaitan yang terkutuk."\\
Lalu meludahlah ke kirimu tiga kali.\\
\par
\noindent
\textbf{Tingkatan Doa dan Sanad}: \textbf{Shahih}: HR. Muslim (no. 2203).\\\\
\par
\noindent 
137------------------------------Doa Mengusir Syaitan
\par
\noindent
\begin{enumerate}
\item Minta perlindungan Allah dari syaitan (dengan membaca: 
\textit{A'\^{u}dzu bill\^{a}hi minasy syaith\^{a}nir raj\^{i}m})
{\scriptsize 1} atau:
\begin{arabtext}
\noindent
'a`u-w_du bi-al-ll_ahi al-ssami-y`i al-`ali-ymi mina al-^s^sa-y.tAni 
al-rra^gi-ymi, min hamzihi wanaf_hihi wanaf_tihi.\\
\end{arabtext}
\noindent
\textbf{Artinya}:
\par
\indent
"Aku berlindung kepada Allah Yang Maha Mendengar lagi Maha Mengetahui dari 
gangguan syaitan yang terkutuk, dari kegilaannya, kesombongannya, dan 
syairnya yang tercela."{\scriptsize 2}
\item Ketika dikumandangkan adzan untuk shalat.{\scriptsize 3}
\item Membaca dzikir tertentu yang sudah diterangkan dalam hadits dan 
membaca al-Qur-an, misalnya: dua ayat terakhir dari surah Al-Baqarah, 
dzikir waktu pagi dan sore, dan ayat Kursi. {\scriptsize 4}\\
\end{enumerate}
\par
\noindent
\textbf{Tingkatan Doa dan Sanad}:
\begin{enumerate}
\item Dasarnya ayat-ayat al-Qur-an (Al-A'raf ayat 200, Al-Mu'min\^{u}m ayat
97-98, Fushshilat ayat 36) dan hadits Nabi yang shahih.
\item \textbf{Shahih}: HR. Abu Dawud (no. 775), at-Tirmidzi (no. 242), dan 
selainnya. Lihat \textit{al-Kalimuth Thayyib} (no. 130) dan 
\textit{Irw\^{a}-ul Ghal\^{i}l} (no. 341, 342).
\item \textbf{Shahih}: HR. Al-Bukhari (no. 608)/\textit{Fathul B\^{a}ri} 
(II/85), dan Muslim (no. 388, 389 [16-19]).
\item HR. Muslim (no. 780 [212]).\\\\
\end{enumerate}
\par
\noindent 
138---------------------Doa Mohon Karunia Allah saat Mendengar Kokok Ayam, dan 
Berlindung kepada-Nya saat Mendengar Ringkikkan Keledai dan Lolongan Anjing
\begin{arabtext}
\noindent
'i_dA sami`tum .siyA.ha al-ddiyakaTi (min al-llayli) fAs'aluW al-ll_aha min
fa.dlihi, fa-'innahA ra'at malakaN, wa-'i_dA sami`tum nahi-yqa al-.himAri 
(mina al-llayli) fata`awwa_duW bi-al-ll_ahi mina al-^s^say.tAni, fa-'innahu
ra'aY ^say.tAnaN.\\
\end{arabtext}
\noindent
\textbf{Artinya}:
\par
\indent
"Apabila engkau mendengar ayam jago berkokok (pada waktu malam), mintalah 
anugerah kepada Allah, sesungguhnya ia melihat Malaikat. Tetapi apabila 
engkau mendengar keledai meringkik (pada waktu malam), mintalah 
perlindungan kepada Allah dari gangguan syaitan, sesungguhnya ia melihat 
syaitan." {\scriptsize 1}\\
\begin{arabtext}
\noindent
'i_dA sami`tum nubA.ha al-kilAbi wanahi-yqa al-.hami-yri bi-al-llayli 
fata`awwa_duW bi-al-ll_ahi mina (al-^s^say.tAni) fa-'innahunna yarayna mAlA
tarawna.\\
\end{arabtext}
\noindent
\textbf{Artinya}:
\par
\indent
"Jika kalian mendengar lolongan anjing dan ringkikan keledai pada malam 
hari, berlindunglah kepada Allah (dari syaitan), karena ia melihat apa yang
tidak dapat kalian lihat."{\scriptsize 2}\\
\par
\noindent
\textbf{Tingkatan Doa dan Sanad}:
\begin{enumerate}
\item \textbf{Shahih}: HR. Al-Bukhari (no. 3303)/\textit{Fathul B\^{a}ri} 
(VI/350), Muslim (no. 2729). Tambahan di dalam kurung diriwayatkan oleh 
al-Bukhari dalam \textit{al-Adabul Mufrad} (no. 1236). Lihat kitab 
\textit{Shah\^{i}h al-Adabil Mufrad} (no. 938) dan \textit{Silsilah 
al-Ah\^{a}d\^{i}ts ash-Shah\^{i}hah} (no. 3183). Hadits tersebut 
diriwayatkan dari Abu Hurairah r.a.
\item \textbf{Shahih}: HR. Abu Dawud (no. 5103), Ahmad (III/306, 355-356), 
dan Ibnus Sunni (no.311) dalam kitab \textit{'Amalul Yaum wal Lailah} dari 
Jabir bin Abdillah r.a. Lihat \textit{Shah\^{i}h al-Adabul Mufrad} (no. 
937).\\\\
\end{enumerate}
\par
\noindent 
139------------------------------Doa Kaffarat Thiyarah
\par
\indent
Dari Abdullah bin Amr r.a., dia berkata bahwa Rasulullah bersabda: "Barang 
siapa mengurungkan niatnya karena thiyarah, maka ia telah berbuat syirik." 
Para Sahabat bertanya: "Lantas, apakah tebusannya?" Beliau menjawab: 
"Hendaklah ia mengucapkan:\\
\begin{arabtext}
\noindent
al-ll_ahumma lA _ha-yra 'illA _ha-yruka walA .ta-yra 'illA .ta-yruka walA 
'il_aha .ga-yruka.\\
\end{arabtext}
\noindent
\textbf{Artinya}:
\par
\indent
'Ya Allah, tidak ada kebaikan kecuali kebaikan dari Engkau, tidaklah burung
itu (yang dijadikan objek tathayyur) melainkan makhluk-Mu dan tiada ilah 
yang berhak diibadahi dengan benar kecuali Engkau.'"{\scriptsize 1}\\
\par
\indent
Tathayyur termasuk adat Jahiliyyah. Mereka biasa berpatokan pada burung. 
Apabila melihat burung itu terbang ke arah kanan, maka mereka gembira dan 
meneruskan niat. Apabila ia terbang ke arah kiri, mereka pun menganggap ia 
pembawa sial dan menangguhkan niat. Bahkan, mereka sengaja menerbangkan 
burung untuk meramal nasib.
\par
\indent
Syariat yang \textit{hanif} (lurus) ini telah melarang segala bentuk 
tathayyur. Sebab, \textit{thair} (burung) tidak memiliki keistimewaan apa 
pun hingga geraknya dijadikan petunjuk untung atau rugi. Di dalam banyak 
hadits, Rasulullah SAW. menegaskan: "Tidak ada thiyarah!"\\
\par
\indent
Penegasan tersebut juga dinukil dari sejumlah Sahabat r.a.\\
\par
\indent
Bukti lain yang menguatkan riwayat yang menafikan hal ini adalah larangan 
Rasulullah terhadap \textit{thiyarah} dan \textit{syu'm} (kesialan) secara 
umum serta pujian dari beliau terhadap orang-orang yang menjauhi keduanya. 
Dinukilkan bahwa beliau berabda:\\
\begin{arabtext}
\noindent
yad_hulu al-^gannaTa min 'ummati-y sab`u-wna 'alfaN bi.ga-yri .hisAbiN, 
humu alla_di-yna lA yastar qu-wna, walA yata.tayyaru-wna, wa`alY rabbihim 
yatawakkalu-wna.\\
\end{arabtext}
\noindent
\textbf{Artinya}:
\par
\indent
"Tujuh puluh ribu orang dari umatku akan masuk Jannah tanpa hisab. Mereka 
adalah orang-orang yang tidak meminta diruqyah, tidak bertathayyur dan 
hanya bertawakal kepada Allah semata."{\scriptsize 2}\\
\par
\noindent
\textbf{Tingkatan Doa dan Sanad}:
\begin{enumerate}
\item \textbf{Shahih}: HR. Ahmad (II/220). Dishahihkan Syaikh Ahmad Syakir 
dalam \textit{ta'liq Musnad Ahmad} (no. 7045), dan oleh Syaikh Nashiruddin 
al-Albani dalam \textit{Silsilah Ah\^{a}d\^{i}ts ash-Shah\^{i}hah} (no. 
1065).
\item \textbf{Shahih}: HR. Al-Bukhari (no. 6472) dari Sahabat Ibnu Abbas 
r.a. Diriwayatkan dengan lafazh panjang oleh al-Bukhari (no. 5705, 5752) 
dan Muslim (no. 220) juga dari Ibnu Abbas r.a.\\\\
\end{enumerate}
\par
\noindent 
143------------------------Lafazh Shalawat dan Salam kepada Nabi yang Ringkas\\
\begin{arabtext}
.sallY al-ll_ahu `alayhi wasallama\\
`alayhi al-.s.salATu wAl-ssalAmu\\
al-ll_ahumma .salli wasallim `alayhi\\
\end{arabtext}
\par
\indent Imam an-Nawawi r.a. berkata: "Apablia seorang dari kalian 
bershalawat kepada Nabi, hendaklah dia menggabungkan antara shalawat dan 
salam, tidak boleh mengucapkan \textit{Shallallahu'alayhi} saja atau hanya 
mengucapkan \textit{'alayhissalam}.{\scriptsize 1}\\
\indent Ibnu Shalah r.a. berkata: "Sebaiknya penulis hadits dan para 
penuntut ilmu menulis shalawat serta salam kepada Nabi (dengan atau secara 
lengkap), dan saat menyebutnya jangan merasa bosan mengulang-ulangnya, 
karena amal yang demikian sangat besar manfaatnya yang segera diperoleh 
bagi mereka. Adapun bagi siapa saja yang lalai darinya, maka ia tercegah 
mendapat pahala besar, dan hendaklah dia tidak memotong dan tidak 
menyingkat shalawat ketika menuliskannya."{\scriptsize 2}\\
\indent Yang perlu kita perhatikan dalam hal ini, yakni mengucapkan 
shalawat kepada Nabi SAW, bahwa tidak boleh seseorang membuat 
shalawat-shalawat yang tidak dicontohkan oleh beliau. Sebab amalan tersebut
merupakan ibadah, sedangkan dasar ibadah dalam Islam adalah 
\textit{ittiba'} (mencontoh Rasulullah SAW.)\\
\indent \textbf{Di antara contoh shalawat yang diucapkan kaum Muslimin 
namun tidak ada contohnya dari Nabi kita atau bid'ah adalah shalawat 
al-Fatih dan yang lainnya.}\\
\indent \textbf{Adapun di antara contoh buku atau risalah yang berisikan 
shalawat bid'ah seperti Dal\^{a}-ilul Khair\^{a}t wa Syaw\^{a}riqul 
Anw\^{a}r F\^{i} Dzikris Shal\^{a}ti alan Nabiyyil Mukht\^{a}r}. 
{\scriptsize 3}\\\\
\noindent
\textbf{Sanad}:
\begin{enumerate}
\item Shah\^{i}h al-Adzk\^{a}r (I/325).
\item \textit{'Ulumul Had\^{i}ts} karya Ibnu Shalah (hlm. 124). Lihat 
nukilan tersebut dalam kitab \textit{al-B\^{a}'itsul Hatsits}: 
\textit{Syarh Ikhtishar 'Ul\^{u}mil Had\^{i}ts} karya Ibnu Katsit, dengan 
\textit{syarh} Ahmad Muhammad Syakir. Lihat pula kitab \textit{Fadhlush 
Shal\^{a}h 'alan Nabi} karya Syaikh Abdul Muhsin al-Abbad al-Badr (hlm. 
15).
\item Lihat \textit{Kutub allati Hadzdzara Minhal Ulama'} karya Syaikh 
Masyhur Hasan Alu Salman, serta kitab \textit{Fadhlush Shal\^{a}h 'alan} 
Nabi (hlm. 18-21) karya Syaikh Abdul Muhsin al-Abbad al-Badr.\\\\
\end{enumerate}
\par
\noindent 
144------------------------------Doa Bersin dan Menguap
\begin{arabtext}
\noindent
'i_dA `a.tasa 'a.hadukum falyaqul: al-.hamduli-ll_ahi, walyaqul lahu 
'a_hu-whu 'aw.sA.hibuhu: yar.hamuka al-ll_ahu, fa'i-_dA qala lahu: 
yar.hamuka al-ll_ahu falyaqul: yahdi-ykumu al-ll_ahu wayu.sli.hu bAlakum.\\
\end{arabtext}
\noindent
\textbf{Artinya}:
\par
\indent
"Apabila salah seorang di antara kalian bersin, hendaklah ia berkata: 
\textit{Alhamdulill\^{a}h} 'Segala puji bagi Allah,' lantas saudara atau 
temannya berkata: \textit{yarkhamukall\^{a}h} 'Semoga Allah memberikan 
rahmat kepada-Mu.' Apabila teman atau saudaranya berkata demikian, bacalah:
\textit{yahdiykumull\^{a}h wayushlikhu balakum} 'Semoga Allah memberi 
petunjuk kepadamu dan memperbaiki keadaanmu."{\scriptsize 1}\\
\begin{arabtext}
\noindent
'inna al-ll_aha yu.hibbu al-`u.tAsa wayakrahu al-tta_tA'uba, fa'i-_dA 
`a.tasa 'a.hadukum wa.hamida al-ll_aha kAna .haqqaN `alY kulli muslimiN 
sami`ahu 'an yaqu-wla lahu :  yar.hamuka al-ll_ahu, wa'ammA al-tta_tA'ubu 
fa'i-nnamA huwa min al-^s^say.tAni,  fa'i-_dA ta_tA'a ba 'a.hadukum 
falyaruddahu mA asta.tA`a, fa'i-nna 'a.hadakum 'i_dA ta_tA'aba .da.hika 
minhu al-^s^say.tAnu.\\
\end{arabtext}
\noindent
\textbf{Artinya}:
\par
\indent
"Sesungguhnya Allah menyukai bersin dan membenci menguap. Apabila salah 
seorang dari kalian bersin dan memuji Allah (mengucapkan 
\textit{Alhamdulill\^{a}h}), maka hendaklah setiap Muslim yang mendengarnya
berkata kepada orang yang bersin: '\textit{Yarhamukall\^{a}h} (artinya, 
semoga Allah merahmatimu).' Adapun menguap itu datangnya dari syaitan. Maka
apabila salah seorang dari kalian menguap, hendaklah ia berusaha untuk 
menahan semampunya. Sebab syaitan akan tertawa tatkala salah seorang dari 
kalian menguap." {\scriptsize 2}\\
\par
\indent
Dari Abu Musa al-Asy'ari r.a., dia bertutur: "Aku mendengar Rasulullah 
Shallallahu ‘alaihi wa sallam bersabda:\\
\begin{arabtext}
\noindent
'i_dA `a.tasa 'a.hadukum fa.hamida al-ll_aha, fa^sammitu-whu, fa'i-n lam 
ya.hmadi al-ll_aha, falA tu^sammitu-whu.\\
\end{arabtext}
\noindent
\textbf{Artinya}:
\par
\indent
'Jika alah seorang dari kalian bersin kemudian mengucapkan 
\textit{Alhamdulill\^{a}h}, hendaklah kalian membacakan \textit{tasymit} 
baginya (yaitu ucapan \textit{Yarhamukall\^{a}h}). Sedangkan jika dia tidak
mengucapkan \textit{Alhamdulill\^{a}h}, maka janganlah kamu membacakan 
\textit{tasymit} baginya.'"{\scriptsize 3}\\
\par
\noindent
\textbf{Tingkatan Doa dan Sanad}:
\begin{enumerate}
\item \textbf{Shahih}: HR. Al-Bukhari (no. 6224) dari Sahabat Abu Hurairah 
r.a.
\item HR. Al-Bukhari (no. 6226). Lihat \textit{Fathul B\^{a}ri} (X/611 no. 
6226).
\item \textbf{Shahih}: HR. Muslim (no. 2992).\\
\end{enumerate}
\par
\noindent 
146------------------------------Membaca Talbiyah
\begin{arabtext}
\noindent
labbayka Aal-ll_ahumma labbayka, labbayka lA ^sari-yka laka labbayka, 'inna 
al-.hamda wAl-nni`maTalaka wAl-mulka lA ^sari-yka laka.\\
\end{arabtext}
\noindent
\textbf{Artinya}:
\par
\indent
"Aku penuhi panggilan-Mu, ya Allah, aku penuhi panggilan-Mu. Aku penuhi 
pangilan-Mu, tiada sekutu bagi-Mu, aku penuhi panggilan-Mu. Sesungguhnya 
pujian dan nikmat adalah milik-Mu, begitu juga kerajaan, tidak ada sekutu 
bagi-Mu".\\
\par
\noindent
\textbf{Tingkatan Doa dan Sanad}: \textbf{Shahih}: HR. Al-Bukhari (no. 
1549), \textit{Fathul B\^{a}ri} (III/408), Muslim (no. 1184 [19]).\\\\
\par
\noindent 
147------------------------------Doa Melihat Ka'bah
\begin{arabtext}
\noindent
al-ll_ahumma 'anta al-ssalAmu, waminka al-ssalAmu, fa.hayyinA rabbanA 
bi-al-ssalAmi.\\
\end{arabtext}
\noindent
\textbf{Artinya}:
\par
\indent
"Ya Allah, Engkaulah Mahasejahtera, dari Engkau pula kesejahteraan, maka 
kekalkanlah kami, wahai Rabb kami, dalam kesejahteaan".\\\\
\par
\noindent
\textbf{Sanad}: HR. Al-Baihaqi (V/73). Lihat \textit{Man\^{a}sikul Hajji 
wal 'Umrah} (hlm. 20) karya Syaikh Muhammad Nashiruddin al-Albani.\\\\
\par
\noindent 
148-----------------------Mengerjakan Thawaf 7 kali Putaran dan 
Berdoa dengan Doa-doa yang Mudah
\par
\indent
Tidak dicontohkan oleh Nabi Shallallahu ‘alaihi wa sallam Dan para Sahabat 
beliau mengucapkan bacaan tertentu pada putaran pertama, kedua, ketiga, dan 
seterusnya sampai putaran terakhir.\\
\indent
Yang Nabi Shallallahu ‘alaihi wa sallam Contohkan adalah doa antara dua rukun 
pada setiap putaran. Selain itu, kita dianjurkan agar banyak berdzikir, 
membaca al-Quran dan doa, karena thawaf seperti shalat hanya saja dibolehkan 
bicara di dalamnya.\\
\indent
Dan, bacalah doa serta dzikir yang mudah ketika thawaf.\\\\
\par
\noindent 
149------------------------------Bertakbir Setiap Kali Sampai di Hajar Aswad
\begin{arabtext}
\noindent
.tAfa al-nnabiyyu .sallaY al-ll_ahu `ala-yhi wasallama bi-alba-yti `alY 
ba`i-yriN kullamA 'atY al-rrukna 'a^s^sAra `ila-yhi bi^sa-y'iN kAna `indahu 
wakabbara.\\
\end{arabtext}
\noindent
\textbf{Artinya}:
\par
\indent
"Nabi Shallallahu ‘alaihi wa sallam thawaf di Baitullah di atas 
(menunggangi) unta. Setiap datang ke Hajar Aswad (yakni sudut Ka'bah yang 
padanya terdapat Hajar Aswad), beliau memberi isyarat dengan sesuatu yang 
dipegangnya dan bertakbir."{\scriptsize 1}\\
\par
\indent
Yang demikian dilakukan setiap melewati Hajar Aswad. Apabila mampu 
menciumnya, hendaklah dia lakukan. Apabila tidak, cukup dengan disentuh. 
Dan apabila tidak mungkin disentuh, cukuplah berisyarat sambil bertakbir: 
"Allahu akbar". Dan dibolehkan juga membaca "Bismillahi allahu akbar" 
berdasarkan perbuatan Ibnu Umar r.a.{\scriptsize 2}\\
\par
\noindent
\textbf{Tingkatan Doa dan Sanad}:
\begin{enumerate}
\item \textbf{Shahih}: HR. Al-Bukhari (no. 1613). Yang dimaksud "sesuatu" 
adalah tongkat. Lihat \textit{Shah\^{i}h al-Bukhari} (no. 1607)
\item HR. Al-Baihaqi (V/79).\\\\
\end{enumerate}
\par
\noindent 
150------------------------------Doa Antara Rukun Yamani dan Hajar Aswad
\begin{arabtext}
\noindent
rabbana-^A -'a-atinA fiY al-ddunyA .hasanaTaN wafiY al-'a_hiraTi 
.hasanaTaN 
waqinA `a_dAba al-nnAri).\\
\end{arabtext}
\noindent
\textbf{Artinya}:
\par
\indent
"Wahai Rabb kami, berikanlah kami kebaikan di dunia dan juga kebaikan di 
akhirat, serta lindungilah kami dari siksa api Neraka."\\
\par
\indent
Setiap selesai tawaf tujuh putaran, disunnahkan shalat sunnah dua rakaat 
di belakang Maqam Ibrahim.\\
\indent
Adapun surah yang dibaca setelah Al-F\^{a}thihah pada rakaat pertama 
adalah Al-K\^{a}fir\^{u}n, sedang pada rakaat kedua adalah Al-Ikhl\^{a}sh 
berdasarkan hadits Jabir r.a.\\
\par
\noindent
\textbf{Tingkatan Doa dan Sanad}: \textbf{Hasan}: HR. Abu Dawud (no. 1892),
Ahmad (III/411), dan al-Baghawi dalam \textit{Syarhus Sunnah} (VII/128 no. 
1915) dari Abdullah bin as-Sa-ib r.a. Lihat \textit{Shah\^{i}h Abi Dawud} 
(I/354).\\\\
\par
\noindent 
152------------------------------Doa pada Hari Arafah
\par
\indent
Rasulullah SAW. bersabda: "Doa terbaik (yang mustajab) adalah pada hari 
Arafah, dan sebaik-baik apa yang aku dan para Nabi baca adalah:\\
\begin{arabtext}
\noindent
lA 'il_aha 'illA al-ll_ahu wa.hdahu lA ^sari-yka lahu, lahu al-mulku, 
walahu al-.hamdu wahuwa `alY kulli ^saY'iN qadi-yruN.\\
\end{arabtext}
\noindent
\textbf{Artinya}:
\par
\indent
'Tidak ada ilah yang berhak diibadahi dengan benar melainkan Allah Yang 
Maha Esa, tidak ada sekutu bagi-Nya. Bagi-Nya kerajaan dan pujian. Dialah 
Yang Mahakuasa atas segala sesuatu'".\\
\par
\noindent
\textbf{Tingkatan Doa dan Sanad}: \textbf{Hasan}: HR. At-Tirmidzi (no. 
3585); \textit{Shah\^{i}h at-Tirmidzi} (III/184). Lihat \textit{Silsilah 
Ah\^{a}d\^{i}hts ash-Shah\^{i}hah} (IV/6, no. 1503).\\\\
\par
\noindent 
153------------------------------Bacaan di Masy'aril Haram
\begin{arabtext}
\noindent
... rakiba .sallY al-ll_ahu `alayhi wasallama al-qa.swA'a .hattY 'atY 
al-ma^s`ara al-.harAma fAstaqbala al-qiblaTa fada`Ahu wakabbarahu 
wahallalahu wawa.h.hadahu falam yazal wAqifaN .hattY 'asfara ^giddaN 
fadafa`a qabla 'an ta.tlu`a al-^s^samsu ....\\
\end{arabtext}
\noindent
\textbf{Artinya}:
\par
\indent
"Nabi SAW. naik unta beliau yang bernama al-Qashwa hingga di Masy'aril 
Haram, lalu beliau menghadap Kiblat, berdoa, membaca takbir 
(\textit{All\^{a}hu Akbar}) dan tahlil (\textit{L\^{a} Il\^{a}ha 
Illall\^{a}h}) serta kalimat tauhid. Beliau pun terus berdoa hingga fajar 
menyingsing. Lantas beliau berangkat (ke Mina) sebelum Matahari terbit."
\\
\par
\noindent
\textbf{Tingkatan Doa dan Sanad}: \textbf{Shahih}: HR. Muslim (no. 1218).\\\\
\par
\noindent 
154------------------------------Bertakbir Setiap Melempar Jumrah
\begin{arabtext}
\noindent
'inna rasu-wla al-ll_ahi .sallaY al-ll_ahu `ala-yhi wasallama kAna 'i_dA 
ramY al-^gamraTa ... bisab`i .ha.sayAtiN, yukabbiru kullamA ramY 
bi.ha.sATiN, _tumma taqaddama 'amAmahA fawaqafa mustaqbila al-qiblaTi, 
rAfi`aN yadayhi yad`uw, wakAna yu.ti-ylu al-wuqu-wfa. _tumma ya'tiY 
al-^gamraTa al-_t_tAniyaTa fayarmi-yhA bisab`i .ha.sayAtiN yukabbiru 
kullamA ramY bi.ha.sATiN... fayaqifu mustaqbila al-qiblaTi rAfi`aN yadayhi
yad`uw. _tumma ya'tiy al-^gamraTa allatiy `inda al-`aqabaTi fayarmi-yhA 
bisab`i .ha.sayAtiN, yukabbiru `inda kulli .ha.sATiN, _tumma yan.sarifu 
walA yaqifu `indahA.\\
\end{arabtext}
\noindent
\textbf{Artinya}:
\par
\indent
"Sesungguhnya Rasulullah Shallallahu ‘alaihi wa sallam. melempar Jumratul 
Ula (jumrah pertama di dekat Masjid Khaif) dengan tujuh batu kerikil dan 
bertakbir setiap kali melemparnya. Kemudian beliau maju dan berdiri lama 
menghadap kiblat, lantas berdoa sambil mengangkat kedua tangan. Selanjutnya
beliau melakukan hal yang sama pada Jumratus Tsaniyah (jumrah kedua), lalu 
berdoa. Kemudian itu beliau melempar Jumratul Aqabah (jumrah ketiga) dengan
tujuh batu kerikil, bertakbir setiap kali melempar, lalu beliau langsung 
pergi dari situ dan tidak diam padanya (yakni tidak berdoa)."\\
\par
\noindent
\textbf{Tingkatan Doa dan Sanad} : \textbf{Shahih}: HR. Al-Bukhari (no. 
1753). Bab "Ad-Du'\^{a}' 'indal Jamrataini" - \textit{Fathul B\^{a}ri} 
(III/584)-dan Muslim (no. 1218).\\\\
\par
\noindent 
155------------------------------Doa Kaffaratul Majelis
\begin{arabtext}
\noindent
al-ll_ahumma aqsim lanA min _ha^syatika mA ta.hu-wlu bihi baynanA wabayna 
ma`A .si-yka, wamin .tA`atika mA tuballi.gunA bihi ^gannataka, wamina 
al-yaqi-yni mA tuhawwinu bihi `alaynA ma.sA'iba al-ddunyA, al-ll_ahumma 
matti`nA bi-'asmA`inA, wa'ab.sArinA, waquwwatinA mA 'a.hyaytanA, wA^g`alhu 
al-wAri_ta minnA, wA^g`al _ta'ranA `alY man .zalamanA, wAn.surnA `alY man
`AdAnA, walA ta^g`al mu.si-ybatanA fi-y di-yninA, walA ta^g`ali al-ddunyA 
'akbara hamminA, walA mabla.ga `ilminA, walA tusalli.t `alaynA man lA 
yar.hamunA.\\
\end{arabtext}
\noindent
\textbf{Artinya}:
\par
\indent
"Ya Allah, anugerahkanlah untuk kami rasa takut kepada-Mu, yang menghalangi
anatara kami dengan perbuatan maksiat kepada-Mu, dan (anugerahkanlah kepada
kami) ketaatan kepada-Mu yang akan menyampaikan kami ke Surga-Mu, dan 
(anugerahkanlah pula) keyakinan yang dapat menyebabkan ringannya bagi kami 
segala musibah di dunia ini. Ya Allah, anugerahkanlah kenikmatan kepada 
kami melalui pendengaran kami, pengelihatan kami dan dalam kekuatan kami 
selama kami masih hidup, serta jadikanlah ia sebagai warisan dari kami. Dan
jadikan ia balasan kami atas orang-orang yang menganiaya kami, dan 
tolonglah kami terhadap orang yang memusuhi kami, serta janganlah Engkau 
jadikan musibah ada dalam urusan agama kami, dan janganlah Engkau jadikan 
dunia ini sebagai cita-cita terbesar dan puncak dari ilmu kami, dan jangan 
Engkau jadikan orang-orang yang tidak mengasihi kami berkuasa atas 
kami."{\scriptsize 1}\\
\begin{arabtext}
\noindent
sub.hAnaka al-ll_ahumma wabi.hamdika, 'a^shadu 'an lA 'il_aha 'illA 'anta, 
'asta.gfiruka wa'atu-wbu 'ilayka.\\
\end{arabtext}
\noindent
\textbf{Artinya}:
\par
\indent
"Mahasuci Engkau, ya Allah, dan aku memuji-Mu. Aku bersaksi bahwa tidak ada
ilah yang berhak diibadahi dengan benar kecuali Engkau, serta aku meminta 
ampun dan bertaubat kepada-Mu."\\
\par
\indent
Rasulullah Shallallahu ‘alaihi wa sallam bersabda: "Barang siapa duduk 
dalam satu majelis, lalu ada kekeliruan dan banyak kesalahan, kemudian 
sebelum bangkit dari majelis itu ia mengucap: 
\textbf{'Subh\^{a}nakall\^{a}humma wabihamdika asyhadu all\^{a} 
Il\^{a}ha illa anta astaghfiruka wa at\^{u}bu ilaika'}, maka Allah 
akan menghapuskan kesalahannya yang terjadi di majelis tersebut." 
{\scriptsize 2}\\
\indent
Dari Aisyah r.a., dia berkata: "Setiap Rasulullah Shallallahu ‘alaihi wa 
sallam duduk di suatu tempat dan setiap melakukan shalat, beliau 
mengakhirinya dengan beberapa kalimat." Aisyah bertanya tentang beberapa 
kalimat tersebut." Beliau Shallallahu ‘alaihi wa sallam bersabda: "Ya, 
barang siapa yang berkata baik maka akan ditulis pada kebaikan itu (pahala 
bacaan kalimat ini), dan barang siapa yang berkata jelek maka kalimat 
inilah penghapusnya."\\
\indent
Kalimat yang dimaksudkan adalah: \\
\begin{arabtext}
\noindent
sub.hAnaka al-ll_ahumma wabi.hamdika, lA 'il_aha 'illA 'anta, 'asta.gfiruka
wa'atu-wbu 'ilayka.\\
\end{arabtext}
\noindent
\textbf{Artinya}:
\par
\indent
"Mahasuci Engkau ya Allah, aku memuji-Mu. Tidak ada ilah yang berhak 
diibadahi dengan benar selain Engkau, aku mohon ampun dan bertaubat 
kepada-Mu."{\scriptsize 3}\\
\begin{arabtext}
\noindent
al-ll_ahumma .salli wasallim `alY nabiyyinA mu.hammadiN wa`alY ^Alihi 
wa'a.s.hAbihi 'a^gma`i-yna, waman tabi`ahum bi-'i.hsAniN 'ilY yawmi 
al-ddi-yni.\\
\end{arabtext}
\noindent
\textbf{Artinya}:
\par
\indent
"Ya Allah, limpahkanlah shalawat dan salam kepada Nabi kami, Muhammad, 
serta kepada keluarga dan para Sahabat beliau secara keseluruhan, juga 
kepada orang-orang yang mengikuti mereka dengan baik sampai hari Kiamat 
kelak."\\
\par
\noindent
\textbf{Tingkatan Doa dan Sanad}:
\begin{enumerate}
\item \textbf{Shahih}: HR. At-Tirmidzi (no. 3502) al-Hakim (I/528) dan 
Ibnus Sunni dalam \textit{Amalul Yaum wal Lailah} (no. 446) dan an-Nasai 
dalam \textit{Amalul Yaum wal Lailah} (no. 4040, dari Abdullah bin Umar 
r.a. Hadits ini dishahihkan oleh al-Hakim dan disepakati oleh adz-Dzahabi. 
Abdullah bin Umar r.a. berkata: "Rasulullah Shallallahu ‘alaihi wa sallam 
seringkali mengucapkan doa ini bagi Sahabat-Sahabat beliau sebelum bangkit 
dari majelis." Lihat \textit{Shah\^{i}h at-Tirmidzi} (III/168, no. 2783) 
dan \textit{Shah\^{i}hul J\^{a}mi'} (no. 1268), \textit{Shah\^{i}h 
al-Kalimith Thayyib} (no. 226).
\item \textbf{Hasan shahih}: HR. At-Tirmidzi (no. 3433), an-Nasai dalam 
\textit{'Amalul Yaum wal Lailah} (no. 400), Ibnus Sunni dalam 
\textit{'Amalul Yaum wal Lailah} (no. 447), Ibnu Hibban (no. 
593-\textit{at-Ta'l\^{i}q\^{a}tul His\^{a}n}), dan al-Hakim (I/536-537) 
dari Abu Hurairah r.a. At-Tirmidzi berkata: "Hadits ini hasan shahih." 
Dishahihkan al-Hakim, dan disetujui adz-Dzahabi. Hadits ini ada 
\textit{syawahid} (penguat) juga dari Abu Barzah al-Aslami, Jubair bin 
Muth'im, dan Aisyah r.a.
\item \textbf{Shahih}: HR. An-Nasai (III/71-72) dan dalam \textit{'Amalul 
Yaum wal Lailah} (no. 403), serta Ahmad (VI/77). Lihat kitab \textit{Fathul
B\^{a}ri} (XIII/546), dan \textit{Silsilah Ah\^{a}d\^{i}ts 
ash-Shah\^{i}hah} (no. 3164).
\end{enumerate}
\end{document}